\documentclass[]{article}
\usepackage{lmodern}
\usepackage{amssymb,amsmath}
\usepackage{ifxetex,ifluatex}
\usepackage{fixltx2e} % provides \textsubscript
\ifnum 0\ifxetex 1\fi\ifluatex 1\fi=0 % if pdftex
  \usepackage[T1]{fontenc}
  \usepackage[utf8]{inputenc}
\else % if luatex or xelatex
  \ifxetex
    \usepackage{mathspec}
    \usepackage{xltxtra,xunicode}
  \else
    \usepackage{fontspec}
  \fi
  \defaultfontfeatures{Mapping=tex-text,Scale=MatchLowercase}
  \newcommand{\euro}{€}
\fi
% use upquote if available, for straight quotes in verbatim environments
\IfFileExists{upquote.sty}{\usepackage{upquote}}{}
% use microtype if available
\IfFileExists{microtype.sty}{\usepackage{microtype}}{}
\usepackage[margin=1in]{geometry}
\usepackage{color}
\usepackage{fancyvrb}
\newcommand{\VerbBar}{|}
\newcommand{\VERB}{\Verb[commandchars=\\\{\}]}
\DefineVerbatimEnvironment{Highlighting}{Verbatim}{commandchars=\\\{\}}
% Add ',fontsize=\small' for more characters per line
\newenvironment{Shaded}{}{}
\newcommand{\AlertTok}[1]{\textcolor[rgb]{1.00,0.00,0.00}{\textbf{#1}}}
\newcommand{\AnnotationTok}[1]{\textcolor[rgb]{0.38,0.63,0.69}{\textbf{\textit{#1}}}}
\newcommand{\AttributeTok}[1]{\textcolor[rgb]{0.49,0.56,0.16}{#1}}
\newcommand{\BaseNTok}[1]{\textcolor[rgb]{0.25,0.63,0.44}{#1}}
\newcommand{\BuiltInTok}[1]{#1}
\newcommand{\CharTok}[1]{\textcolor[rgb]{0.25,0.44,0.63}{#1}}
\newcommand{\CommentTok}[1]{\textcolor[rgb]{0.38,0.63,0.69}{\textit{#1}}}
\newcommand{\CommentVarTok}[1]{\textcolor[rgb]{0.38,0.63,0.69}{\textbf{\textit{#1}}}}
\newcommand{\ConstantTok}[1]{\textcolor[rgb]{0.53,0.00,0.00}{#1}}
\newcommand{\ControlFlowTok}[1]{\textcolor[rgb]{0.00,0.44,0.13}{\textbf{#1}}}
\newcommand{\DataTypeTok}[1]{\textcolor[rgb]{0.56,0.13,0.00}{#1}}
\newcommand{\DecValTok}[1]{\textcolor[rgb]{0.25,0.63,0.44}{#1}}
\newcommand{\DocumentationTok}[1]{\textcolor[rgb]{0.73,0.13,0.13}{\textit{#1}}}
\newcommand{\ErrorTok}[1]{\textcolor[rgb]{1.00,0.00,0.00}{\textbf{#1}}}
\newcommand{\ExtensionTok}[1]{#1}
\newcommand{\FloatTok}[1]{\textcolor[rgb]{0.25,0.63,0.44}{#1}}
\newcommand{\FunctionTok}[1]{\textcolor[rgb]{0.02,0.16,0.49}{#1}}
\newcommand{\ImportTok}[1]{#1}
\newcommand{\InformationTok}[1]{\textcolor[rgb]{0.38,0.63,0.69}{\textbf{\textit{#1}}}}
\newcommand{\KeywordTok}[1]{\textcolor[rgb]{0.00,0.44,0.13}{\textbf{#1}}}
\newcommand{\NormalTok}[1]{#1}
\newcommand{\OperatorTok}[1]{\textcolor[rgb]{0.40,0.40,0.40}{#1}}
\newcommand{\OtherTok}[1]{\textcolor[rgb]{0.00,0.44,0.13}{#1}}
\newcommand{\PreprocessorTok}[1]{\textcolor[rgb]{0.74,0.48,0.00}{#1}}
\newcommand{\RegionMarkerTok}[1]{#1}
\newcommand{\SpecialCharTok}[1]{\textcolor[rgb]{0.25,0.44,0.63}{#1}}
\newcommand{\SpecialStringTok}[1]{\textcolor[rgb]{0.73,0.40,0.53}{#1}}
\newcommand{\StringTok}[1]{\textcolor[rgb]{0.25,0.44,0.63}{#1}}
\newcommand{\VariableTok}[1]{\textcolor[rgb]{0.10,0.09,0.49}{#1}}
\newcommand{\VerbatimStringTok}[1]{\textcolor[rgb]{0.25,0.44,0.63}{#1}}
\newcommand{\WarningTok}[1]{\textcolor[rgb]{0.38,0.63,0.69}{\textbf{\textit{#1}}}}
\ifxetex
  \usepackage[setpagesize=false, % page size defined by xetex
              unicode=false, % unicode breaks when used with xetex
              xetex]{hyperref}
\else
  \usepackage[unicode=true]{hyperref}
\fi
\hypersetup{breaklinks=true,
            bookmarks=true,
            pdfauthor={Justin Le},
            pdftitle={mtl is Not a Monad Transformer Library},
            colorlinks=true,
            citecolor=blue,
            urlcolor=blue,
            linkcolor=magenta,
            pdfborder={0 0 0}}
\urlstyle{same}  % don't use monospace font for urls
% Make links footnotes instead of hotlinks:
\renewcommand{\href}[2]{#2\footnote{\url{#1}}}
\setlength{\parindent}{0pt}
\setlength{\parskip}{6pt plus 2pt minus 1pt}
\setlength{\emergencystretch}{3em}  % prevent overfull lines
\setcounter{secnumdepth}{0}

\title{mtl is Not a Monad Transformer Library}
\author{Justin Le}
\date{May 18, 2015}

\begin{document}
\maketitle

\emph{Originally posted on
\textbf{\href{https://blog.jle.im/entry/mtl-is-not-a-monad-transformer-library.html}{in
Code}}.}

\emph{mtl} is not a monad transformer library --- contrary to popular
conception. I believe that this commonly spread myth is due in part to some
rather peculiar branding choices (the name of the library) and in part to some
historical accidents (\emph{mtl} was, in the distant and pre-historic past,
indeed a monad transformer library).

What is \emph{mtl}? It is a library of \emph{interfaces} you can provide to your
own types, in the form of typeclasses. It abstracts over \emph{different design
patterns} for different types, in the form of typeclasses. Just like Functor
abstracts over ``things that can be fmapped''. \emph{mtl} provides typeclasses
abstracting over many useful patterns that many types satisfy --- patterns
involving different sorts of ``effects''.

\hypertarget{the-patterns}{%
\section{The Patterns}\label{the-patterns}}

\hypertarget{monaderror}{%
\subsection{MonadError}\label{monaderror}}

\texttt{MonadError} is a generic interface over things where you can throw
``errors'' of a specific type \texttt{e}, and ``catch'' them. It offers two
methods: \texttt{throwError\ ::\ e\ -\textgreater{}\ m\ a}, and
\texttt{catchError\ ::\ m\ a\ -\textgreater{}\ (e\ -\textgreater{}\ m\ a)\ -\textgreater{}\ m\ a},
which does what you'd expect from an error monad.

Now, we have a generic interface to work on \emph{all specific type
error-throwing Monads}. The \texttt{Either} type comes to mind as an obvious
candidate:

\begin{Shaded}
\begin{Highlighting}[]
\KeywordTok{instance} \DataTypeTok{MonadError}\NormalTok{ e (}\DataTypeTok{Either}\NormalTok{ e) }\KeywordTok{where}
\NormalTok{    throwError }\OtherTok{=} \DataTypeTok{Left}
\NormalTok{    catchError s f }\OtherTok{=} \KeywordTok{case}\NormalTok{ s }\KeywordTok{of}
                       \DataTypeTok{Right}\NormalTok{ \_ }\OtherTok{{-}>}\NormalTok{ s}
                       \DataTypeTok{Left}\NormalTok{ e  }\OtherTok{{-}>}\NormalTok{ f e}
\end{Highlighting}
\end{Shaded}

But there are definitely other instances possible. How about for \texttt{IO} and
\texttt{IOException}s, in specific?

\begin{Shaded}
\begin{Highlighting}[]
\KeywordTok{instance} \DataTypeTok{MonadError} \DataTypeTok{IOException} \DataTypeTok{IO} \KeywordTok{where}
\NormalTok{    throwError  }\OtherTok{=} \FunctionTok{ioError}
\NormalTok{    catchErrror }\OtherTok{=} \FunctionTok{catch}     \CommentTok{{-}{-} will not catch non{-}IOExceptions}
\end{Highlighting}
\end{Shaded}

This is great, because we can now write code \emph{generic} over \emph{all}
specific-type error things!

\hypertarget{error-behaviorfor-free}{%
\subsubsection{Error behavior\ldots for free!}\label{error-behaviorfor-free}}

If we're clever enough, we can actually imbue any arbitrary Monad \texttt{m}
with rudimentary, basic, ``dumb'' error handling by using the \texttt{ExceptT}
type. An \texttt{ExceptT\ e\ m} behaves \emph{just} like our original Monad
\texttt{m} in every way\ldots except now, we have access to rudimentary
implementations of side-channels of \texttt{throwError} and \texttt{catchError}.

This is pretty useful\ldots to be able to add short-circuiting error behavior to
any Monad we wanted. But remember, \texttt{ExceptT} is not the ``point'' of
\texttt{MonadError}. It's just one way to generate instances for free given a
Monad. The real power of \texttt{MonadError} is in the ability to write
generically over many Monads with some sort of ``error'' behavior, like
\texttt{Either} or \texttt{IO}.

\hypertarget{monadstate}{%
\subsection{MonadState}\label{monadstate}}

A \texttt{MonadState\ s\ m} is a Monad \texttt{m} where, during in the context
of \texttt{m}, you have access to a global state of type \texttt{s} that you can
modify.

You can ``get'' it with \texttt{get\ ::\ m\ s}. You can modify it with
\texttt{modify\ ::\ (s\ -\textgreater{}\ s)\ -\textgreater{}\ m\ ()}. You can
replace it with \texttt{put\ ::\ s\ -\textgreater{}\ m\ ()}.

There are a lot of types that can offer this type of interface. You might have,
for example, a type where ``getting'' the state comes from reading an
\texttt{IORef}, and ``putting'' it comes from writing to the \texttt{IORef}. Or
maybe the state can come from a a query to a database\ldots where \texttt{get}
queries a database in IO, and \texttt{put} writes to the database.

\texttt{MonadState}, as a typeclass, gives you the ability to \emph{write
generically over all Monads with state}. You can now write generically over
those database state things\ldots or those IORef state things\ldots or those web
query things\ldots or anything that cares to implement the interface!

\texttt{MonadState} says, ``the functions and actions I write can work for
\emph{all} Monads offering state I can modify!'' An action of type
\texttt{MonadState\ String\ m\ =\textgreater{}\ m\ Double} can create a
\texttt{Double} from \emph{any} monad offering some sort of \texttt{String}
state.

\hypertarget{statefor-free}{%
\subsubsection{State\ldots for free!}\label{statefor-free}}

Again, we can actually imbue any Monad \texttt{m} with some very rudimentary,
``dumb'' stateful interface, using a type called \texttt{StateT}. A
\texttt{StateT\ s\ m} behaves just like our monad \texttt{m} (be it \texttt{IO},
\texttt{Reader}, \texttt{ST}, \texttt{STM}\ldots), except now we have access to
a rudimentary state getting-and-putting mechanism on a state of type \texttt{s},
using a form of function composition. The implementation of the \texttt{StateT}
handles it under the hood.

Obviously, being able to add a rudimentary stateful interface on top of any
Monad is pretty useful. Very useful, in fact!

But remember, this isn't the \emph{point} of \texttt{MonadState}.
\texttt{MonadState} doesn't exist for \texttt{StateT}. \texttt{StateT} is just a
way to generate a free instance of \texttt{MonadState} if you just want to add
rudimentary statefulness to an existing Monad. But there are many instances of
\texttt{MonadState}\ldots really, \texttt{MonadState} has nothing to do with
\texttt{StateT} fundamentally, any more than \texttt{Monad} has to do with
\texttt{Maybe} fundamentally. And \texttt{MonadState} and \texttt{StateT} don't
even come from the same library!

\emph{mtl} offers a generic interface for working with all monads offering a
statey API.

\hypertarget{monadreader}{%
\subsection{MonadReader}\label{monadreader}}

\texttt{MonadReader} is more or less the same thing\ldots it offers a generic
interface to work on monads that have access to some sort of global, unchanging
``environment''. An example might be a Monad where you could work with command
line arguments, or environment variables, assuming they are read once and fixed
when things start up. You could access the command line arguments with
\texttt{ask}, and use them in your program.

\hypertarget{monadio}{%
\subsection{MonadIO}\label{monadio}}

This one is actually from \emph{transformers}, but it gives a nice picture. Any
\texttt{MonadIO\ m} is a \texttt{Monad} that allows you to embed and sequence in
any arbitrary IO action. This is pretty useful! In the
\emph{\href{http://hackage.haskell.org/package/persistent}{persistent}} database
library, for example --- the main ``database access type monad'' can sequence
actions that access databases \emph{and} arbitrary IO actions, as well. A lot of
resource managers and DSL's offer the ability to sequence IO in the middle of
all the other actions.

That's what \texttt{MonadIO} is for --- it allows you to write functions and
say, ``hey, my function is generic over \emph{all} things that can embed
IO\ldots anything that can embed IO can sequence my function/type''. The generic
``embedding'' action is
\texttt{liftIO\ ::\ MonadIO\ m\ =\textgreater{}\ IO\ a\ -\textgreater{}\ m\ a}.

\begin{center}\rule{0.5\linewidth}{\linethickness}\end{center}

\textbf{Aside}

You know\ldots ideally, all of these typeclasses would have laws, so we could
make conclusions and apply equational reasoning to generically written
functions.

Some of the laws are simple\ldots{}\texttt{liftIO} from \texttt{MonadIO} should
be a
\href{http://hackage.haskell.org/package/mmorph-1.0.4/docs/Control-Monad-Morph.html}{monad
morphism}. But the rest of them don't really have any well-established laws.
This is a bit of a shame, because we'd really like to be able to apply reasoning
to generic functions.

People have suggested \texttt{MonadState} have laws similar to how view/set/over
interact in the \emph{lens} laws. But as of now, most of we have in terms of our
capability of analyzing generic programs is rough heuristins/feelings about what
``should'' be right.

A bit un-ideal, but\ldots in practice, this ends up working not-so-badly :)

\begin{center}\rule{0.5\linewidth}{\linethickness}\end{center}

\hypertarget{not-a-monad-transformer-library}{%
\section{Not a Monad Transformer
Library}\label{not-a-monad-transformer-library}}

So, let's work together to dispel the myth that \emph{mtl} is a monad
transformer library. It really has nothing to do with monad transformers at
all\ldots any more than \texttt{Control.Monad} is an ``IO module'', or
\texttt{Control.Monoid} is a ``list module''. Transformers don't even come from
the \emph{mtl} library!

Together, we can overcome this myth. We can show people that we can live in a
world where we can combine effects, work generically in Monads with
\emph{multiple types of effects} by writing functions generic over many
different \emph{mtl} typeclasses at once! (\texttt{MonadState} +
\texttt{MonadIO}, maybe?)

We don't \emph{have to} reach for Monad transformers to work with combined
effects. We can write our own combined effects monads and just write the
instances\ldots or we can write generically and not even care about what Monad
we actually use in the end. We don't have to teach people to be afraid of monad
transformers as if they were the only way to get things done, and \emph{mtl} is
tied to them like a ball and chain.

\emph{mtl} is not a Monad transformer library. How liberating!

\hypertarget{signoff}{%
\section{Signoff}\label{signoff}}

Hi, thanks for reading! You can reach me via email at
\href{mailto:justin@jle.im}{\nolinkurl{justin@jle.im}}, or at twitter at
\href{https://twitter.com/mstk}{@mstk}! This post and all others are published
under the \href{https://creativecommons.org/licenses/by-nc-nd/3.0/}{CC-BY-NC-ND
3.0} license. Corrections and edits via pull request are welcome and encouraged
at \href{https://github.com/mstksg/inCode}{the source repository}.

If you feel inclined, or this post was particularly helpful for you, why not
consider \href{https://www.patreon.com/justinle/overview}{supporting me on
Patreon}, or a \href{bitcoin:3D7rmAYgbDnp4gp4rf22THsGt74fNucPDU}{BTC donation}?
:)

\end{document}
