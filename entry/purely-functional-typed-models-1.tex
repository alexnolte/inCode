\documentclass[]{article}
\usepackage{lmodern}
\usepackage{amssymb,amsmath}
\usepackage{ifxetex,ifluatex}
\usepackage{fixltx2e} % provides \textsubscript
\ifnum 0\ifxetex 1\fi\ifluatex 1\fi=0 % if pdftex
  \usepackage[T1]{fontenc}
  \usepackage[utf8]{inputenc}
\else % if luatex or xelatex
  \ifxetex
    \usepackage{mathspec}
    \usepackage{xltxtra,xunicode}
  \else
    \usepackage{fontspec}
  \fi
  \defaultfontfeatures{Mapping=tex-text,Scale=MatchLowercase}
  \newcommand{\euro}{€}
\fi
% use upquote if available, for straight quotes in verbatim environments
\IfFileExists{upquote.sty}{\usepackage{upquote}}{}
% use microtype if available
\IfFileExists{microtype.sty}{\usepackage{microtype}}{}
\usepackage[margin=1in]{geometry}
\usepackage{color}
\usepackage{fancyvrb}
\newcommand{\VerbBar}{|}
\newcommand{\VERB}{\Verb[commandchars=\\\{\}]}
\DefineVerbatimEnvironment{Highlighting}{Verbatim}{commandchars=\\\{\}}
% Add ',fontsize=\small' for more characters per line
\newenvironment{Shaded}{}{}
\newcommand{\AlertTok}[1]{\textcolor[rgb]{1.00,0.00,0.00}{\textbf{#1}}}
\newcommand{\AnnotationTok}[1]{\textcolor[rgb]{0.38,0.63,0.69}{\textbf{\textit{#1}}}}
\newcommand{\AttributeTok}[1]{\textcolor[rgb]{0.49,0.56,0.16}{#1}}
\newcommand{\BaseNTok}[1]{\textcolor[rgb]{0.25,0.63,0.44}{#1}}
\newcommand{\BuiltInTok}[1]{#1}
\newcommand{\CharTok}[1]{\textcolor[rgb]{0.25,0.44,0.63}{#1}}
\newcommand{\CommentTok}[1]{\textcolor[rgb]{0.38,0.63,0.69}{\textit{#1}}}
\newcommand{\CommentVarTok}[1]{\textcolor[rgb]{0.38,0.63,0.69}{\textbf{\textit{#1}}}}
\newcommand{\ConstantTok}[1]{\textcolor[rgb]{0.53,0.00,0.00}{#1}}
\newcommand{\ControlFlowTok}[1]{\textcolor[rgb]{0.00,0.44,0.13}{\textbf{#1}}}
\newcommand{\DataTypeTok}[1]{\textcolor[rgb]{0.56,0.13,0.00}{#1}}
\newcommand{\DecValTok}[1]{\textcolor[rgb]{0.25,0.63,0.44}{#1}}
\newcommand{\DocumentationTok}[1]{\textcolor[rgb]{0.73,0.13,0.13}{\textit{#1}}}
\newcommand{\ErrorTok}[1]{\textcolor[rgb]{1.00,0.00,0.00}{\textbf{#1}}}
\newcommand{\ExtensionTok}[1]{#1}
\newcommand{\FloatTok}[1]{\textcolor[rgb]{0.25,0.63,0.44}{#1}}
\newcommand{\FunctionTok}[1]{\textcolor[rgb]{0.02,0.16,0.49}{#1}}
\newcommand{\ImportTok}[1]{#1}
\newcommand{\InformationTok}[1]{\textcolor[rgb]{0.38,0.63,0.69}{\textbf{\textit{#1}}}}
\newcommand{\KeywordTok}[1]{\textcolor[rgb]{0.00,0.44,0.13}{\textbf{#1}}}
\newcommand{\NormalTok}[1]{#1}
\newcommand{\OperatorTok}[1]{\textcolor[rgb]{0.40,0.40,0.40}{#1}}
\newcommand{\OtherTok}[1]{\textcolor[rgb]{0.00,0.44,0.13}{#1}}
\newcommand{\PreprocessorTok}[1]{\textcolor[rgb]{0.74,0.48,0.00}{#1}}
\newcommand{\RegionMarkerTok}[1]{#1}
\newcommand{\SpecialCharTok}[1]{\textcolor[rgb]{0.25,0.44,0.63}{#1}}
\newcommand{\SpecialStringTok}[1]{\textcolor[rgb]{0.73,0.40,0.53}{#1}}
\newcommand{\StringTok}[1]{\textcolor[rgb]{0.25,0.44,0.63}{#1}}
\newcommand{\VariableTok}[1]{\textcolor[rgb]{0.10,0.09,0.49}{#1}}
\newcommand{\VerbatimStringTok}[1]{\textcolor[rgb]{0.25,0.44,0.63}{#1}}
\newcommand{\WarningTok}[1]{\textcolor[rgb]{0.38,0.63,0.69}{\textbf{\textit{#1}}}}
\usepackage{graphicx}
\makeatletter
\def\maxwidth{\ifdim\Gin@nat@width>\linewidth\linewidth\else\Gin@nat@width\fi}
\def\maxheight{\ifdim\Gin@nat@height>\textheight\textheight\else\Gin@nat@height\fi}
\makeatother
% Scale images if necessary, so that they will not overflow the page
% margins by default, and it is still possible to overwrite the defaults
% using explicit options in \includegraphics[width, height, ...]{}
\setkeys{Gin}{width=\maxwidth,height=\maxheight,keepaspectratio}
\ifxetex
  \usepackage[setpagesize=false, % page size defined by xetex
              unicode=false, % unicode breaks when used with xetex
              xetex]{hyperref}
\else
  \usepackage[unicode=true]{hyperref}
\fi
\hypersetup{breaklinks=true,
            bookmarks=true,
            pdfauthor={Justin Le},
            pdftitle={A Purely Functional Typed Approach to Trainable Models (Part 1)},
            colorlinks=true,
            citecolor=blue,
            urlcolor=blue,
            linkcolor=magenta,
            pdfborder={0 0 0}}
\urlstyle{same}  % don't use monospace font for urls
% Make links footnotes instead of hotlinks:
\renewcommand{\href}[2]{#2\footnote{\url{#1}}}
\setlength{\parindent}{0pt}
\setlength{\parskip}{6pt plus 2pt minus 1pt}
\setlength{\emergencystretch}{3em}  % prevent overfull lines
\setcounter{secnumdepth}{0}

\title{A Purely Functional Typed Approach to Trainable Models (Part 1)}
\author{Justin Le}
\date{May 14, 2018}

\begin{document}
\maketitle

\emph{Originally posted on
\textbf{\href{https://blog.jle.im/entry/purely-functional-typed-models-1.html}{in
Code}}.}

With the release of
\href{http://hackage.haskell.org/package/backprop}{backprop}, I've been
exploring the space of parameterized models of all sorts, from linear and
logistic regression and other statistical models to artificial neural networks,
feed-forward and recurrent (stateful). I wanted to see to what extent we can
really apply automatic differentiation and iterative gradient decent-based
training to all of these different models. Basically, I wanted to see how far we
can take \emph{differentiable programming} (a la
\href{https://www.facebook.com/yann.lecun/posts/10155003011462143}{Yann LeCun})
as a paradigm for writing trainable models.

Building on other writers, I'm starting to see a picture unifying all of these
models, painted in the language of purely typed functional programming. I'm
already applying these to models I'm using in real life and in my research, and
I thought I'd take some time to put my thoughts to writing in case anyone else
finds these illuminating or useful.

As a big picture, I really believe that a purely functional typed approach to
differentiable programming is \emph{the} way to move forward in the future for
models like artificial neural networks. In this light, the drawbacks of
object-oriented and imperative approaches becomes very apparent.

I'm not the first person to attempt to build a conceptual framework for these
types of models in a purely functional typed sense --
\href{http://colah.github.io/posts/2015-09-NN-Types-FP/}{Christopher Olah's
famous post} wrote a great piece in 2015 that this post heavily builds off of,
and is definitely worth a read! We'll be taking some of his ideas and seeing how
they work in real code!

This will be a three-part series, and the intended audience is people who have a
passing familiarity with statistical modeling or machine learning/deep learning.
The code in these posts is written in Haskell, using the
\href{http://hackage.haskell.org/package/backprop}{backprop} and
\href{http://hackage.haskell.org/package/hmatrix}{hmatrix} (with
\href{http://hackage.haskell.org/package/hmatrix-backprop}{hmatrix-backprop})
libraries, but the main themes and messages won't be \emph{about} haskell, but
rather about differentiable programming in a purely functional typed setting in
general. This isn't a Haskell post as much as it is an exploration, using
Haskell syntax/libraries to implement the points. The \emph{backprop} library is
roughly equivalent to \href{https://github.com/HIPS/autograd}{autograd} in
python, so all of the ideas apply there as well.

The source code for the written code in this module is available
\href{https://github.com/mstksg/inCode/tree/master/code-samples/functional-models/model.hs}{on
github}, if you want to follow along!

\hypertarget{essence-of-a-model}{%
\section{Essence of a Model}\label{essence-of-a-model}}

For the purpose of this post, a \emph{parameterized model} is a function from
some input ``question'' (predictor, independent variable) to some output
``answer'' (predictand, dependent variable)

Notationally, we might write it as a function:

{[} f\_p(x) =
y{]}(https://latex.codecogs.com/png.latex?\%0Af\_p\%28x\%29\%20\%3D\%20y\%0A "
f\_p(x) = y ")

The important thing is that, for every choice of \emph{parameterization}
\includegraphics{https://latex.codecogs.com/png.latex?p}, we get a
\emph{different function}
\includegraphics{https://latex.codecogs.com/png.latex?f_p\%28x\%29}.

For example, you might want to write a model that, when given an email, outputs
whether or not that email is spam.

The parameterization \emph{p} is some piece of data that we tweak to produce a
different \includegraphics{https://latex.codecogs.com/png.latex?f_p\%28x\%29}.
So, ``training'' (or ``learning'', or ``estimating'') a model is a process of
picking the \includegraphics{https://latex.codecogs.com/png.latex?p} that gives
the ``correct'' function
\includegraphics{https://latex.codecogs.com/png.latex?f_p\%28x\%29} --- that is,
the function that accurately predicts spam or whatever thing you are trying to
predict.

For example, for \href{https://en.wikipedia.org/wiki/Linear_regression}{linear
regression}, you are trying to ``fit'' your
\includegraphics{https://latex.codecogs.com/png.latex?\%28x\%2C\%20y\%29} data
points to some function
\includegraphics{https://latex.codecogs.com/png.latex?f\%28x\%29\%20\%3D\%20\%5Cbeta\%20\%2B\%20\%5Calpha\%20x}.
The \emph{parameters} are
\includegraphics{https://latex.codecogs.com/png.latex?\%5Calpha} and
\includegraphics{https://latex.codecogs.com/png.latex?\%5Cbeta}, the
\emph{input} is \includegraphics{https://latex.codecogs.com/png.latex?x}, and
the \emph{output} is
\includegraphics{https://latex.codecogs.com/png.latex?\%5Cbeta\%20\%2B\%20\%5Calpha\%20x}.

As it so happens, a
\includegraphics{https://latex.codecogs.com/png.latex?f_p\%28x\%29} is really
just a ``partially applied''
\includegraphics{https://latex.codecogs.com/png.latex?f\%28p\%2Cx\%29}.
Imagining that function, it has type:\footnote{Those familiar with Haskell
  idioms might recognize this type as being essentially
  \texttt{a\ -\textgreater{}\ Reader\ p\ b} (or
  \texttt{Kleisli\ (Reader\ p)\ a\ b}) which roughly represents the notion of
  ``A function from \texttt{a} to \texttt{b} with an `environment' of type
  \texttt{p}''.}

{[} f : (P \textbackslash times A) \textbackslash rightarrow
B{]}(https://latex.codecogs.com/png.latex?\%0Af\%20\%3A\%20\%28P\%20\%5Ctimes\%20A\%29\%20\%5Crightarrow\%20B\%0A
" f : (P \times A) \rightarrow B ")

If we \href{https://en.wikipedia.org/wiki/Currying}{curry} this, we get the
original model representation we talked about:

{[} f : P \textbackslash rightarrow (A \textbackslash rightarrow
B){]}(https://latex.codecogs.com/png.latex?\%0Af\%20\%3A\%20P\%20\%5Crightarrow\%20\%28A\%20\%5Crightarrow\%20B\%29\%0A
" f : P \rightarrow (A \rightarrow B) ")

\hypertarget{optimizing-models-with-observations}{%
\subsection{Optimizing Models with
Observations}\label{optimizing-models-with-observations}}

Something interesting happens if we flip the script. What if, instead of
\includegraphics{https://latex.codecogs.com/png.latex?f_p\%28x\%29}, we talked
about \includegraphics{https://latex.codecogs.com/png.latex?f_x\%28p\%29}? That
is, we fix the input and vary the parameter, and see what type of outputs we get
for the same output while we vary the parameter?

If we have an ``expected output'' for our input, then one thing we can do is
look at \includegraphics{https://latex.codecogs.com/png.latex?f_x\%28p\%29} and
see when the result is close to
\includegraphics{https://latex.codecogs.com/png.latex?y_x} (the expected output
of our model when given
\includegraphics{https://latex.codecogs.com/png.latex?x}).

In fact, we can turn this into an optimization problem by trying to pick
\includegraphics{https://latex.codecogs.com/png.latex?p} that minimizes the
difference between
\includegraphics{https://latex.codecogs.com/png.latex?f_x\%28p\%29} and
\includegraphics{https://latex.codecogs.com/png.latex?y_x}. We can say that our
model with parameter \includegraphics{https://latex.codecogs.com/png.latex?p}
predicts \includegraphics{https://latex.codecogs.com/png.latex?y_x} the best
when we minimize:

{[} (f\_x(p) -
y\_x)\^{}2{]}(https://latex.codecogs.com/png.latex?\%0A\%28f\_x\%28p\%29\%20-\%20y\_x\%29\%5E2\%0A
" (f\_x(p) - y\_x)\^{}2 ")

If we minimize the squared error between the result of picking the parameter and
the expected result, we find the best parameters for that given input!

In general, picking the best parameter for the model involves picking the
\includegraphics{https://latex.codecogs.com/png.latex?p} that minimizes the
relationship

{[} \textbackslash text\{loss\}(y\_x,
f\_x(p)){]}(https://latex.codecogs.com/png.latex?\%0A\%5Ctext\%7Bloss\%7D\%28y\_x\%2C\%20f\_x\%28p\%29\%29\%0A
" \text{loss}(y\_x, f\_x(p)) ")

Where
\includegraphics{https://latex.codecogs.com/png.latex?\%5Ctext\%7Bloss\%7D\%20\%3A\%20B\%20\%5Ctimes\%20B\%20\%5Crightarrow\%20\%5Cmathbb\%7BR\%7D}
gives a measure of ``how badly'' the model result differs from the expected
target. Common loss functions include squared error, cross-entropy, etc.

This gives us a supervised way to train any model: if we have enough
observations
(\includegraphics{https://latex.codecogs.com/png.latex?\%28x\%2C\%20y_x\%29}
pairs) we can just pick a
\includegraphics{https://latex.codecogs.com/png.latex?p} that does its best to
make the loss between all observations as small as possible.

\hypertarget{stochastic-gradient-descent}{%
\subsection{Stochastic Gradient Descent}\label{stochastic-gradient-descent}}

If our model is a \emph{differentiable function}, then we have a nice tool we
can use: \emph{stochastic gradient descent} (SGD).

That is, we can always calculate the \emph{gradient} of the loss function with
respect to our parameters. This gives us the direction we can ``nudge'' our
parameters to make the loss bigger or smaller.

That is, if we get the \emph{gradient} of the loss with respect to
\includegraphics{https://latex.codecogs.com/png.latex?p}
(\includegraphics{https://latex.codecogs.com/png.latex?\%5Cnabla_p\%20\%5Ctext\%7Bloss\%7D\%28f_x\%28p\%29\%2C\%20y_x\%29}),
we now have a nice iterative way to ``train'' our model:

\begin{enumerate}
\def\labelenumi{\arabic{enumi}.}
\tightlist
\item
  Start with an initial guess at the parameter
\item
  Look at a random
  \includegraphics{https://latex.codecogs.com/png.latex?\%28x\%2C\%20y_x\%29}
  observation pair.
\item
  Compute the gradient
  \includegraphics{https://latex.codecogs.com/png.latex?\%5Cnabla_p\%20\%5Ctext\%7Bloss\%7D\%28f_x\%28p\%29\%2C\%20y_x\%29}
  of our current \includegraphics{https://latex.codecogs.com/png.latex?p}, which
  tells us a direction we can ``nudge''
  \includegraphics{https://latex.codecogs.com/png.latex?p} in to make the loss
  smaller.
\item
  Nudge \includegraphics{https://latex.codecogs.com/png.latex?p} in that
  direction
\item
  Repeat from \#2 until satisfied
\end{enumerate}

With every new observation, we see how we can nudge the parameter to make the
model more accurate, and then we perform that nudge. At the end of it all, we
wind up just the right \texttt{p} to model the relationship between our
observation pairs.

\hypertarget{functional-implementation}{%
\section{Functional Implementation}\label{functional-implementation}}

What I described naturally lends to a functional implementation. That's because,
in this light, a model is nothing more than a curried function (a function
returning a function). A model that is trainable using SGD is simply a
differentiable function.

Using the \emph{\href{http://hackage.haskell.org/package/backprop}{backprop}}
library, we can write these differentiable functions as normal functions.

Let's pick a type for our models. A model from type \texttt{a} to type
\texttt{b} with parameter \texttt{p} can be written as the type synonym

\begin{Shaded}
\begin{Highlighting}[]
\KeywordTok{type} \DataTypeTok{Model}\NormalTok{ p a b }\OtherTok{=}\NormalTok{ p }\OtherTok{{-}>}\NormalTok{ a }\OtherTok{{-}>}\NormalTok{ b}
\end{Highlighting}
\end{Shaded}

Not normally differentiable, but we can make it a differentiable function by
having it work with \texttt{BVar\ z\ p} and \texttt{BVar\ z\ a} (\texttt{BVar}s
containing those values) instead:

\begin{Shaded}
\begin{Highlighting}[]
\CommentTok{{-}{-} source: https://github.com/mstksg/inCode/tree/master/code{-}samples/functional{-}models/model.hs\#L52{-}L55}

\KeywordTok{type} \DataTypeTok{Model}\NormalTok{ p a b }\OtherTok{=} \KeywordTok{forall}\NormalTok{ z}\OperatorTok{.} \DataTypeTok{Reifies}\NormalTok{ z }\DataTypeTok{W}
                \OtherTok{=>} \DataTypeTok{BVar}\NormalTok{ z p}
                \OtherTok{{-}>} \DataTypeTok{BVar}\NormalTok{ z a}
                \OtherTok{{-}>} \DataTypeTok{BVar}\NormalTok{ z b}
\end{Highlighting}
\end{Shaded}

This is a RankN \emph{type synonym}, which is saying that a
\texttt{Model\ p\ a\ b} is just a type synonym for a differentiable
\texttt{BVar\ z\ p\ -\textgreater{}\ BVar\ z\ a\ -\textgreater{}\ BVar\ z\ b}.
The \texttt{Reifies\ z\ W} is just a constraint that allows for backpropagation
of \texttt{BVar}s.

We can write a simple linear regression model:

{[} f\_\{\textbackslash alpha, \textbackslash beta\}(x) = \textbackslash beta x
+
\textbackslash alpha{]}(https://latex.codecogs.com/png.latex?\%0Af\_\%7B\%5Calpha\%2C\%20\%5Cbeta\%7D\%28x\%29\%20\%3D\%20\%5Cbeta\%20x\%20\%2B\%20\%5Calpha\%0A
" f\_\{\alpha, \beta\}(x) = \beta x + \alpha ")

\begin{Shaded}
\begin{Highlighting}[]
\CommentTok{{-}{-} source: https://github.com/mstksg/inCode/tree/master/code{-}samples/functional{-}models/model.hs\#L48{-}L369}

\KeywordTok{data}\NormalTok{ a }\OperatorTok{:\&}\NormalTok{ b }\OtherTok{=} \OperatorTok{!}\NormalTok{a }\OperatorTok{:\&} \OperatorTok{!}\NormalTok{b}

\KeywordTok{pattern}\OtherTok{ (:\&\&) ::}\NormalTok{ (}\DataTypeTok{Backprop}\NormalTok{ a, }\DataTypeTok{Backprop}\NormalTok{ b, }\DataTypeTok{Reifies}\NormalTok{ z }\DataTypeTok{W}\NormalTok{)}
              \OtherTok{=>} \DataTypeTok{BVar}\NormalTok{ z a }\OtherTok{{-}>} \DataTypeTok{BVar}\NormalTok{ z b }\OtherTok{{-}>} \DataTypeTok{BVar}\NormalTok{ z (a }\OperatorTok{:\&}\NormalTok{ b)}

\OtherTok{linReg ::} \DataTypeTok{Model}\NormalTok{ (}\DataTypeTok{Double} \OperatorTok{:\&} \DataTypeTok{Double}\NormalTok{) }\DataTypeTok{Double} \DataTypeTok{Double}
\NormalTok{linReg (a }\OperatorTok{:\&\&}\NormalTok{ b) x }\OtherTok{=}\NormalTok{ b }\OperatorTok{*}\NormalTok{ x }\OperatorTok{+}\NormalTok{ a}
\end{Highlighting}
\end{Shaded}

A couple things going on here to help us do things smoothly:

\begin{itemize}
\item
  We define a custom tuple data type \texttt{:\&}; backprop works with normal
  tuples, but using a custom tuple with a \texttt{Num} instance will come in
  handy later for training models.
\item
  We define a pattern synonym \texttt{:\&\&} that lets us ``pattern match out''
  \texttt{BVar}s of that tuple type. So if we have a
  \texttt{BVar\ z\ (a\ :\&\ b)} (a \texttt{BVar} containing a tuple), then
  matching on \texttt{(x\ :\&\&\ y)} will give us \texttt{x\ ::\ BVar\ z\ a} and
  \texttt{y\ ::\ BVar\ z\ b}.
\item
  With that, we define \texttt{linReg}, whose parameters are a
  \texttt{Double\ :\&\ Double}, a tuple the two parameters \texttt{a} and
  \texttt{b}. After pattern matching out the contents, we just write the linear
  regression formula --- \texttt{b\ *\ x\ +\ a}. We can use normal numeric
  operations like \texttt{*} and \texttt{+} because \texttt{BVar}s have a
  \texttt{Num} instance.
\end{itemize}

We can \emph{run} \texttt{linReg} using \texttt{evalBP2}:

\begin{Shaded}
\begin{Highlighting}[]
\NormalTok{ghci}\OperatorTok{>}\NormalTok{ evalBP2 linReg (}\FloatTok{0.3} \OperatorTok{:\&}\NormalTok{ (}\OperatorTok{{-}}\FloatTok{0.1}\NormalTok{)) }\DecValTok{5}
\OperatorTok{{-}}\FloatTok{0.2}        \CommentTok{{-}{-} ({-}0.1) * 5 + 0.3}
\end{Highlighting}
\end{Shaded}

But the neat thing is that we can also get the gradient of the parameters, too,
if we identify a loss function:\footnote{Note that this is only sound as a loss
  function for a single ``scalar value'', like \texttt{Double} or a one-vector.
  In general, we'd have this take a loss function as a parameter.}

{[} \textbackslash nabla\_p (f(p, x) -
y\_x)\^{}2{]}(https://latex.codecogs.com/png.latex?\%0A\%5Cnabla\_p\%20\%28f\%28p\%2C\%20x\%29\%20-\%20y\_x\%29\%5E2\%0A
" \nabla\_p (f(p, x) - y\_x)\^{}2 ")

\begin{Shaded}
\begin{Highlighting}[]
\CommentTok{{-}{-} source: https://github.com/mstksg/inCode/tree/master/code{-}samples/functional{-}models/model.hs\#L60{-}L68}

\NormalTok{squaredErrorGrad}
\OtherTok{    ::}\NormalTok{ (}\DataTypeTok{Backprop}\NormalTok{ p, }\DataTypeTok{Backprop}\NormalTok{ b, }\DataTypeTok{Num}\NormalTok{ b)}
    \OtherTok{=>} \DataTypeTok{Model}\NormalTok{ p a b      }\CommentTok{{-}{-} \^{} Model}
    \OtherTok{{-}>}\NormalTok{ a                }\CommentTok{{-}{-} \^{} Observed input}
    \OtherTok{{-}>}\NormalTok{ b                }\CommentTok{{-}{-} \^{} Observed output}
    \OtherTok{{-}>}\NormalTok{ p                }\CommentTok{{-}{-} \^{} Parameter guess}
    \OtherTok{{-}>}\NormalTok{ p                }\CommentTok{{-}{-} \^{} Gradient}
\NormalTok{squaredErrorGrad f x targ }\OtherTok{=}\NormalTok{ gradBP }\OperatorTok{$}\NormalTok{ \textbackslash{}p }\OtherTok{{-}>}
\NormalTok{    (f p (auto x) }\OperatorTok{{-}}\NormalTok{ auto targ) }\OperatorTok{\^{}} \DecValTok{2}
\end{Highlighting}
\end{Shaded}

We use \texttt{auto\ ::\ a\ -\textgreater{}\ BVar\ z\ a}, to lift a normal value
to a \texttt{BVar} holding that value, since our model \texttt{f} takes
\texttt{BVar}s.

And finally, we can train it using stochastic gradient descent, with just a
simple fold over all observations:

\begin{Shaded}
\begin{Highlighting}[]
\CommentTok{{-}{-} source: https://github.com/mstksg/inCode/tree/master/code{-}samples/functional{-}models/model.hs\#L70{-}L76}

\NormalTok{trainModel}
\OtherTok{    ::}\NormalTok{ (}\DataTypeTok{Fractional}\NormalTok{ p, }\DataTypeTok{Backprop}\NormalTok{ p, }\DataTypeTok{Num}\NormalTok{ b, }\DataTypeTok{Backprop}\NormalTok{ b)}
    \OtherTok{=>} \DataTypeTok{Model}\NormalTok{ p a b      }\CommentTok{{-}{-} \^{} model to train}
    \OtherTok{{-}>}\NormalTok{ p                }\CommentTok{{-}{-} \^{} initial parameter guess}
    \OtherTok{{-}>}\NormalTok{ [(a,b)]          }\CommentTok{{-}{-} \^{} list of observations}
    \OtherTok{{-}>}\NormalTok{ p                }\CommentTok{{-}{-} \^{} updated parameter guess}
\NormalTok{trainModel f }\OtherTok{=}\NormalTok{ foldl\textquotesingle{} }\OperatorTok{$}\NormalTok{ \textbackslash{}p (x,y) }\OtherTok{{-}>}\NormalTok{ p }\OperatorTok{{-}} \FloatTok{0.1} \OperatorTok{*}\NormalTok{ squaredErrorGrad f x y p}
\end{Highlighting}
\end{Shaded}

For convenience, we can define a \texttt{Random} instance for our tuple type
using the \emph{\href{http://hackage.haskell.org/package/random}{random}}
library and make a wrapper that uses \texttt{IO} to generate a random initial
parameter:

\begin{Shaded}
\begin{Highlighting}[]
\CommentTok{{-}{-} source: https://github.com/mstksg/inCode/tree/master/code{-}samples/functional{-}models/model.hs\#L78{-}L85}

\NormalTok{trainModelIO}
\OtherTok{    ::}\NormalTok{ (}\DataTypeTok{Fractional}\NormalTok{ p, }\DataTypeTok{Backprop}\NormalTok{ p, }\DataTypeTok{Num}\NormalTok{ b, }\DataTypeTok{Backprop}\NormalTok{ b, }\DataTypeTok{Random}\NormalTok{ p)}
    \OtherTok{=>} \DataTypeTok{Model}\NormalTok{ p a b      }\CommentTok{{-}{-} \^{} model to train}
    \OtherTok{{-}>}\NormalTok{ [(a,b)]          }\CommentTok{{-}{-} \^{} list of observations}
    \OtherTok{{-}>} \DataTypeTok{IO}\NormalTok{ p             }\CommentTok{{-}{-} \^{} parameter guess}
\NormalTok{trainModelIO m xs }\OtherTok{=} \KeywordTok{do}
\NormalTok{    p0 }\OtherTok{<{-}}\NormalTok{ (}\OperatorTok{/} \DecValTok{10}\NormalTok{) }\OperatorTok{.} \FunctionTok{subtract} \FloatTok{0.5} \OperatorTok{<$>}\NormalTok{ randomIO}
    \FunctionTok{return} \OperatorTok{$}\NormalTok{ trainModel m p0 xs}
\end{Highlighting}
\end{Shaded}

Let's train our linear regression model to fit the points \texttt{(1,1)},
\texttt{(2,3)}, \texttt{(3,5)}, \texttt{(4,7)}, and \texttt{(5,9)}! This should
follow
\includegraphics{https://latex.codecogs.com/png.latex?f\%28x\%29\%20\%3D\%202\%20x\%20-\%201},
or
\includegraphics{https://latex.codecogs.com/png.latex?\%5Calpha\%20\%3D\%20-1\%2C\%5C\%2C\%20\%5Cbeta\%20\%3D\%202}:

\begin{Shaded}
\begin{Highlighting}[]
\NormalTok{ghci}\OperatorTok{>}\NormalTok{ samps }\OtherTok{=}\NormalTok{ [(}\DecValTok{1}\NormalTok{,}\DecValTok{1}\NormalTok{),(}\DecValTok{2}\NormalTok{,}\DecValTok{3}\NormalTok{),(}\DecValTok{3}\NormalTok{,}\DecValTok{5}\NormalTok{),(}\DecValTok{4}\NormalTok{,}\DecValTok{7}\NormalTok{),(}\DecValTok{5}\NormalTok{,}\DecValTok{9}\NormalTok{)]}
\NormalTok{ghci}\OperatorTok{>}\NormalTok{ trainModelIO linReg }\OperatorTok{$} \FunctionTok{take} \DecValTok{5000}\NormalTok{ (}\FunctionTok{cycle}\NormalTok{ samps)}
\NormalTok{(}\OperatorTok{{-}}\FloatTok{1.0000000000000024}\NormalTok{) }\OperatorTok{:\&} \FloatTok{2.0000000000000036}
\CommentTok{{-}{-} roughly:}
\NormalTok{(}\OperatorTok{{-}}\FloatTok{1.0}\NormalTok{) }\OperatorTok{:\&} \FloatTok{2.0}
\end{Highlighting}
\end{Shaded}

Neat --- after going through all of those observations a thousand times, the
model nudges itself all the way to the right parameters to fit our model!

The important takeaway is that all we specified was the \emph{function} of the
model itself. The training part all follows automatically.

\hypertarget{feed-forward-neural-network}{%
\subsection{Feed-forward Neural Network}\label{feed-forward-neural-network}}

Here's another example: a fully-connected feed-forward neural network layer.

We can start with a single layer. The model here will also take two parameters
(a weight matrix and a bias vector), take in a vector, and output a vector.

\begin{Shaded}
\begin{Highlighting}[]
\KeywordTok{import} \DataTypeTok{Numeric.LinearAlgebra.Static.Backprop}
\CommentTok{{-}{-} source: https://github.com/mstksg/inCode/tree/master/code{-}samples/functional{-}models/model.hs\#L92{-}L103}

\OtherTok{logistic ::} \DataTypeTok{Floating}\NormalTok{ a }\OtherTok{=>}\NormalTok{ a }\OtherTok{{-}>}\NormalTok{ a}
\NormalTok{logistic x }\OtherTok{=} \DecValTok{1} \OperatorTok{/}\NormalTok{ (}\DecValTok{1} \OperatorTok{+} \FunctionTok{exp}\NormalTok{ (}\OperatorTok{{-}}\NormalTok{x))}

\NormalTok{feedForwardLog}
\OtherTok{    ::}\NormalTok{ (}\DataTypeTok{KnownNat}\NormalTok{ i, }\DataTypeTok{KnownNat}\NormalTok{ o)}
    \OtherTok{=>} \DataTypeTok{Model}\NormalTok{ (}\DataTypeTok{L}\NormalTok{ o i }\OperatorTok{:\&} \DataTypeTok{R}\NormalTok{ o) (}\DataTypeTok{R}\NormalTok{ i) (}\DataTypeTok{R}\NormalTok{ o)}
\NormalTok{feedForwardLog (w }\OperatorTok{:\&\&}\NormalTok{ b) x }\OtherTok{=}\NormalTok{ logistic (w }\OperatorTok{\#>}\NormalTok{ x }\OperatorTok{+}\NormalTok{ b)}
\end{Highlighting}
\end{Shaded}

Here we use the \texttt{L\ n\ m} (an n-by-m matrix) and \texttt{R\ n} (an
n-vector) types from the \emph{hmatrix} library, and \texttt{\#\textgreater{}}
for backprop-aware matrix-vector multiplication.

Let's try training a model to learn the simple
\href{https://en.wikipedia.org/wiki/Logical_conjunction}{logical ``AND''}:

\begin{Shaded}
\begin{Highlighting}[]
\NormalTok{ghci}\OperatorTok{>} \KeywordTok{import} \KeywordTok{qualified} \DataTypeTok{Numeric.LinearAlgebra.Static} \KeywordTok{as} \DataTypeTok{H}
\NormalTok{ghci}\OperatorTok{>}\NormalTok{ samps }\OtherTok{=}\NormalTok{ [(H.vec2 }\DecValTok{0} \DecValTok{0}\NormalTok{, }\DecValTok{0}\NormalTok{), (H.vec2 }\DecValTok{1} \DecValTok{0}\NormalTok{, }\DecValTok{0}\NormalTok{), (H.vec2 }\DecValTok{0} \DecValTok{1}\NormalTok{, }\DecValTok{0}\NormalTok{), (H.vec2 }\DecValTok{1} \DecValTok{1}\NormalTok{, }\DecValTok{1}\NormalTok{)]}
\NormalTok{ghci}\OperatorTok{>}\NormalTok{ trained }\OtherTok{<{-}}\NormalTok{ trainModelIO feedForwardLog }\OperatorTok{$} \FunctionTok{take} \DecValTok{10000}\NormalTok{ (}\FunctionTok{cycle}\NormalTok{ samps)}
\end{Highlighting}
\end{Shaded}

We have our trained parameters! Let's see if they actually model ``AND''?

\begin{Shaded}
\begin{Highlighting}[]
\NormalTok{ghci}\OperatorTok{>}\NormalTok{ evalBP2 feedForwardLog trained (H.vec2 }\DecValTok{0} \DecValTok{0}\NormalTok{)}
\NormalTok{(}\FloatTok{7.468471910660985e{-}5}\OtherTok{ ::} \DataTypeTok{R} \DecValTok{1}\NormalTok{)       }\CommentTok{{-}{-} 0.0}
\NormalTok{ghci}\OperatorTok{>}\NormalTok{ evalBP2 feedForwardLog trained (H.vec2 }\DecValTok{1} \DecValTok{0}\NormalTok{)}
\NormalTok{(}\FloatTok{3.816205998697482e{-}2}\OtherTok{ ::} \DataTypeTok{R} \DecValTok{1}\NormalTok{)       }\CommentTok{{-}{-} 0.0}
\NormalTok{ghci}\OperatorTok{>}\NormalTok{ evalBP2 feedForwardLog trained (H.vec2 }\DecValTok{0} \DecValTok{1}\NormalTok{)}
\NormalTok{(}\FloatTok{3.817490115313559e{-}2}\OtherTok{ ::} \DataTypeTok{R} \DecValTok{1}\NormalTok{)       }\CommentTok{{-}{-} 0.0}
\NormalTok{ghci}\OperatorTok{>}\NormalTok{ evalBP2 feedForwardLog trained (H.vec2 }\DecValTok{1} \DecValTok{1}\NormalTok{)}
\NormalTok{(}\FloatTok{0.9547178031665701}\OtherTok{ ::} \DataTypeTok{R} \DecValTok{1}\NormalTok{)         }\CommentTok{{-}{-} 1.0}
\end{Highlighting}
\end{Shaded}

Close enough for me!

If we inspect the arrived-at parameters, we can peek into the neural network's
brain:

\begin{Shaded}
\begin{Highlighting}[]
\NormalTok{ghci}\OperatorTok{>}\NormalTok{ trained}
\NormalTok{(matrix}
\NormalTok{ [ }\FloatTok{4.652034474187562}\NormalTok{, }\FloatTok{4.65355702367007}\NormalTok{ ]}\OtherTok{ ::} \DataTypeTok{L} \DecValTok{1} \DecValTok{2}\NormalTok{) }\OperatorTok{:\&}\NormalTok{ (}\OperatorTok{{-}}\FloatTok{7.073724083776028}\OtherTok{ ::} \DataTypeTok{R} \DecValTok{1}\NormalTok{)}
\end{Highlighting}
\end{Shaded}

It seems like there is a heavy negative bias, and that each of the inputs makes
some contribution that is slightly more than half of the negative bias; the end
goal is that one of the inputs alone makes no dent, but only if both inputs are
``on'', the output can overcome the negative bias.

The network was able to arrive that this configuration just by exploring the
gradient of our differentiable function!

\hypertarget{functional-composition}{%
\subsection{Functional composition}\label{functional-composition}}

Because our functions are simply just \emph{normal functions}, we can create
new, complex models from simpler ones using just functional composition.

For example, we can map the result of a model to create a new model. Here, we
compose \texttt{linReg\ ab} (linear regression with parameter \texttt{ab}) with
the logistic function to create a
\emph{\href{https://en.wikipedia.org/wiki/Logistic_regression}{logistic
regression}} model.

\begin{Shaded}
\begin{Highlighting}[]
\CommentTok{{-}{-} source: https://github.com/mstksg/inCode/tree/master/code{-}samples/functional{-}models/model.hs\#L117{-}L118}

\OtherTok{logReg ::} \DataTypeTok{Model}\NormalTok{ (}\DataTypeTok{Double} \OperatorTok{:\&} \DataTypeTok{Double}\NormalTok{) }\DataTypeTok{Double} \DataTypeTok{Double}
\NormalTok{logReg ab }\OtherTok{=}\NormalTok{ logistic }\OperatorTok{.}\NormalTok{ linReg ab}
\end{Highlighting}
\end{Shaded}

Here, we use function composition \texttt{(.)}, one of the most common
combinators in Haskell, saying that \texttt{(f\ .\ g)\ x\ =\ f\ (g\ x)}.

We could have even written our \texttt{feedForwardLog} without its activation
function:

\begin{Shaded}
\begin{Highlighting}[]
\CommentTok{{-}{-} source: https://github.com/mstksg/inCode/tree/master/code{-}samples/functional{-}models/model.hs\#L95{-}L98}

\NormalTok{feedForward}
\OtherTok{    ::}\NormalTok{ (}\DataTypeTok{KnownNat}\NormalTok{ i, }\DataTypeTok{KnownNat}\NormalTok{ o)}
    \OtherTok{=>} \DataTypeTok{Model}\NormalTok{ (}\DataTypeTok{L}\NormalTok{ o i }\OperatorTok{:\&} \DataTypeTok{R}\NormalTok{ o) (}\DataTypeTok{R}\NormalTok{ i) (}\DataTypeTok{R}\NormalTok{ o)}
\NormalTok{feedForward (w }\OperatorTok{:\&\&}\NormalTok{ b) x }\OtherTok{=}\NormalTok{ w }\OperatorTok{\#>}\NormalTok{ x }\OperatorTok{+}\NormalTok{ b}
\end{Highlighting}
\end{Shaded}

And now we can swap out activation functions using simple function composition:

\begin{Shaded}
\begin{Highlighting}[]
\CommentTok{{-}{-} source: https://github.com/mstksg/inCode/tree/master/code{-}samples/functional{-}models/model.hs\#L120{-}L123}

\NormalTok{feedForwardLog\textquotesingle{}}
\OtherTok{    ::}\NormalTok{ (}\DataTypeTok{KnownNat}\NormalTok{ i, }\DataTypeTok{KnownNat}\NormalTok{ o)}
    \OtherTok{=>} \DataTypeTok{Model}\NormalTok{ (}\DataTypeTok{L}\NormalTok{ o i }\OperatorTok{:\&} \DataTypeTok{R}\NormalTok{ o) (}\DataTypeTok{R}\NormalTok{ i) (}\DataTypeTok{R}\NormalTok{ o)}
\NormalTok{feedForwardLog\textquotesingle{} wb }\OtherTok{=}\NormalTok{ logistic }\OperatorTok{.}\NormalTok{ feedForward wb}
\end{Highlighting}
\end{Shaded}

Maybe even a \href{https://en.wikipedia.org/wiki/Softmax_function}{softmax}
classifier!

\begin{Shaded}
\begin{Highlighting}[]
\CommentTok{{-}{-} source: https://github.com/mstksg/inCode/tree/master/code{-}samples/functional{-}models/model.hs\#L125{-}L133}

\OtherTok{softMax ::}\NormalTok{ (}\DataTypeTok{Reifies}\NormalTok{ z }\DataTypeTok{W}\NormalTok{, }\DataTypeTok{KnownNat}\NormalTok{ n) }\OtherTok{=>} \DataTypeTok{BVar}\NormalTok{ z (}\DataTypeTok{R}\NormalTok{ n) }\OtherTok{{-}>} \DataTypeTok{BVar}\NormalTok{ z (}\DataTypeTok{R}\NormalTok{ n)}
\NormalTok{softMax x }\OtherTok{=}\NormalTok{ konst (}\DecValTok{1} \OperatorTok{/}\NormalTok{ sumElements expx) }\OperatorTok{*}\NormalTok{ expx}
  \KeywordTok{where}
\NormalTok{    expx }\OtherTok{=} \FunctionTok{exp}\NormalTok{ x}

\NormalTok{feedForwardSoftMax}
\OtherTok{    ::}\NormalTok{ (}\DataTypeTok{KnownNat}\NormalTok{ i, }\DataTypeTok{KnownNat}\NormalTok{ o)}
    \OtherTok{=>} \DataTypeTok{Model}\NormalTok{ (}\DataTypeTok{L}\NormalTok{ o i }\OperatorTok{:\&} \DataTypeTok{R}\NormalTok{ o) (}\DataTypeTok{R}\NormalTok{ i) (}\DataTypeTok{R}\NormalTok{ o)}
\NormalTok{feedForwardSoftMax wb }\OtherTok{=}\NormalTok{ softMax }\OperatorTok{.}\NormalTok{ feedForward wb}
\end{Highlighting}
\end{Shaded}

We can even write a function to \emph{compose} two models, keeping their two
original parameters separate:

\begin{Shaded}
\begin{Highlighting}[]
\CommentTok{{-}{-} source: https://github.com/mstksg/inCode/tree/master/code{-}samples/functional{-}models/model.hs\#L135{-}L141}

\NormalTok{(}\OperatorTok{<\textasciitilde{}}\NormalTok{)}
\OtherTok{    ::}\NormalTok{ (}\DataTypeTok{Backprop}\NormalTok{ p, }\DataTypeTok{Backprop}\NormalTok{ q)}
    \OtherTok{=>} \DataTypeTok{Model}\NormalTok{  p       b c}
    \OtherTok{{-}>} \DataTypeTok{Model}\NormalTok{       q  a b}
    \OtherTok{{-}>} \DataTypeTok{Model}\NormalTok{ (p }\OperatorTok{:\&}\NormalTok{ q) a c}
\NormalTok{(f }\OperatorTok{<\textasciitilde{}}\NormalTok{ g) (p }\OperatorTok{:\&\&}\NormalTok{ q) }\OtherTok{=}\NormalTok{ f p }\OperatorTok{.}\NormalTok{ g q}
\KeywordTok{infixr} \DecValTok{8} \OperatorTok{<\textasciitilde{}}
\end{Highlighting}
\end{Shaded}

And now we have a way to chain models! Maybe even make a multiple-layer neural
network? Let's see if we can get a two-layer model to learn
\href{https://en.wikipedia.org/wiki/Exclusive_or}{XOR} \ldots{}

Our model is two feed-forward layers with logistic activation functions, with 4
hidden layer units:

\begin{Shaded}
\begin{Highlighting}[]
\NormalTok{ghci}\OperatorTok{>} \KeywordTok{let}\OtherTok{ twoLayer ::} \DataTypeTok{Model}\NormalTok{ \_ (}\DataTypeTok{R} \DecValTok{2}\NormalTok{) (}\DataTypeTok{R} \DecValTok{1}\NormalTok{)}
\NormalTok{          twoLayer }\OtherTok{=}\NormalTok{ feedForwardLog\textquotesingle{} }\OperatorTok{@}\DecValTok{4} \OperatorTok{<\textasciitilde{}}\NormalTok{ feedForwardLog\textquotesingle{}}
\end{Highlighting}
\end{Shaded}

Note we use type application syntax (the \texttt{@}) to specify the input/output
dimensions of \texttt{feedForwardLog\textquotesingle{}} to set our hidden layer
size; when we write \texttt{feedForwardLog\textquotesingle{}\ @4}, it means to
set the \texttt{i} type variable to \texttt{4}. We also use \texttt{\_} type
wildcard syntax because we want to just let the compiler infer the type of the
model parameter for us instead of explicitly writing it out ourselves.

We can train it on sample points:

\begin{Shaded}
\begin{Highlighting}[]
\NormalTok{ghci}\OperatorTok{>} \KeywordTok{let}\OtherTok{ samps ::}\NormalTok{ [(}\DataTypeTok{R} \DecValTok{2}\NormalTok{, }\DataTypeTok{R} \DecValTok{1}\NormalTok{)]}
\NormalTok{          samps }\OtherTok{=}\NormalTok{ [(H.vec2 }\DecValTok{0} \DecValTok{0}\NormalTok{, }\DecValTok{0}\NormalTok{), (H.vec2 }\DecValTok{1} \DecValTok{0}\NormalTok{, }\DecValTok{1}\NormalTok{), (H.vec2 }\DecValTok{0} \DecValTok{1}\NormalTok{, }\DecValTok{1}\NormalTok{), (H.vec2 }\DecValTok{1} \DecValTok{1}\NormalTok{, }\DecValTok{0}\NormalTok{)]}
\NormalTok{ghci}\OperatorTok{>}\NormalTok{ trained }\OtherTok{<{-}}\NormalTok{ trainModelIO twoLayer }\OperatorTok{$} \FunctionTok{take} \DecValTok{50000}\NormalTok{ (}\FunctionTok{cycle}\NormalTok{ samps)}
\end{Highlighting}
\end{Shaded}

Trained. Now, does it model ``XOR''?

\begin{Shaded}
\begin{Highlighting}[]
\NormalTok{ghci}\OperatorTok{>}\NormalTok{ evalBP2 twoLayer trained (H.vec2 }\DecValTok{0} \DecValTok{0}\NormalTok{)}
\NormalTok{(}\FloatTok{3.0812844350410647e{-}2}\OtherTok{ ::} \DataTypeTok{R} \DecValTok{1}\NormalTok{)          }\CommentTok{{-}{-} 0.0}
\NormalTok{ghci}\OperatorTok{>}\NormalTok{ evalBP2 twoLayer trained (H.vec2 }\DecValTok{1} \DecValTok{0}\NormalTok{)}
\NormalTok{(}\FloatTok{0.959153369985914}\OtherTok{ ::} \DataTypeTok{R} \DecValTok{1}\NormalTok{)              }\CommentTok{{-}{-} 1.0}
\NormalTok{ghci}\OperatorTok{>}\NormalTok{ evalBP2 twoLayer trained (H.vec2 }\DecValTok{0} \DecValTok{1}\NormalTok{)}
\NormalTok{(}\FloatTok{0.9834757090696419}\OtherTok{ ::} \DataTypeTok{R} \DecValTok{1}\NormalTok{)             }\CommentTok{{-}{-} 1.0}
\NormalTok{ghci}\OperatorTok{>}\NormalTok{ evalBP2 twoLayer trained (H.vec2 }\DecValTok{1} \DecValTok{1}\NormalTok{)}
\NormalTok{(}\FloatTok{3.6846467867668035e{-}2}\OtherTok{ ::} \DataTypeTok{R} \DecValTok{1}\NormalTok{)          }\CommentTok{{-}{-} 0.0}
\end{Highlighting}
\end{Shaded}

Not bad!

\hypertarget{just-functions}{%
\section{Just Functions}\label{just-functions}}

We just built a working neural network using normal function composition and
simple combinators. No need for any objects or mutability or fancy explicit
graphs. Just pure, typed functions! Why would you ever bring anything imperative
into this?

You can build a lot with just these tools alone. By using primitive models and
the various combinators, you can create autoencoders, nonlinear regressions,
convolutional neural networks, multi-layered neural networks, generative
adversarial networks\ldots you can create complex ``graphs'' of networks that
fork and re-combine with themselves.

The nice thing is that these are all just regular (Rank-2) functions,
so\ldots you have two models? Just compose their functions like normal
functions!

It is tempting to look at something like

\begin{Shaded}
\begin{Highlighting}[]
\NormalTok{feedForwardLog }\OperatorTok{@}\DecValTok{4} \OperatorTok{<\textasciitilde{}}\NormalTok{ feedForwardLog}
\end{Highlighting}
\end{Shaded}

and think of it as some sort of abstract, opaque data type with magic inside.
After all, ``layers'' are ``data'', right? But, at the end of the day, it's all
just:

\begin{Shaded}
\begin{Highlighting}[]
\NormalTok{\textbackslash{}(p }\OperatorTok{:\&\&}\NormalTok{ q) }\OtherTok{{-}>}\NormalTok{ feedForwardLog }\OperatorTok{@}\DecValTok{4}\NormalTok{ p }\OperatorTok{.}\NormalTok{ feedForwardLog q}
\end{Highlighting}
\end{Shaded}

Just normal function composition -- we're really just defining the
\emph{function} itself, and \emph{backprop} turns that function into a trainable
model.

In the past I've talked about
\href{https://blog.jle.im/entry/practical-dependent-types-in-haskell-1.html}{layers
as data}, and neural network libraries like
\href{http://hackage.haskell.org/package/grenade-0.1.0}{grenade} let you
manipulate neural network layers in a composable way. My previous attempts at
neural networks like \href{https://github.com/mstksg/tensor-ops}{tensor-ops}
also force a similar structure of composition of data types. Frameworks like
\emph{\href{https://www.tensorflow.org/}{tensorflow}} and
\emph{\href{http://caffe.berkeleyvision.org/}{caffe}} also treat
\href{https://docs.google.com/presentation/d/1UeKXVgRvvxg9OUdh_UiC5G71UMscNPlvArsWER41PsU/edit\#slide=id.gc2fcdcce7_216_264}{layer
as data}. However, I feel this is a bit limiting.

You are forced to ``compose'' your layers in only the ways that the API of the
data type gives you. You have to use the data type's ``function composition''
functions, or its special ``mapping'' functions\ldots and for weird things like
forking compositions like
\texttt{\textbackslash{}x\ -\textgreater{}\ f\ (g\ x)\ (h\ x)} you have to learn
how the data type offers such an API.

However, such a crazy composition here is ``trivial'' -- it's all just normal
functions, so you can just literally write out code like
\texttt{\textbackslash{}x\ -\textgreater{}\ f\ (g\ x)\ (h\ x)} (or something
very close). You don't have to learn any rules of special ``layer'' data types.
Layers aren't matrices or ``data'' --- they're functions. Not just abstractly,
but literally. All your models are! And, with differentiable programming, they
become \emph{trainable functions}.

\hypertarget{what-makes-it-tick}{%
\subsection{What Makes It Tick}\label{what-makes-it-tick}}

My overall thesis of this series is about four essential properties of executing
effective differentiable programing-based models. All of these things, I feel,
have to come together seamlessly to make this all work.

\begin{enumerate}
\def\labelenumi{\arabic{enumi}.}
\item
  \emph{Functional programming}, allowing us to write higher-order functions and
  combinators that take functions and return functions.

  This is the entire crux of this approach, and lets us not only draw from
  mathematical models directly, but also combine and reshape models in arbitrary
  ways just by using normal function composition and application, instead of
  being forced into a rigid compositional model.

  We were able to chain, fork, recombine simple model primitives to make
  \emph{new} models by just writing normal higher-order functions. In fact, as
  we will see in the upcoming posts, we can actually re-use higher order
  functions like \texttt{foldl} and \texttt{map} that are already commonly used
  in functional programming.

  In the upcoming posts, we will take this principle to the extreme. We'll
  define more combinators like \texttt{(\textless{}\textasciitilde{})} and see
  how many models we think are ``fundamental'' (like recurrent neural networks,
  autoregressive models) really are just combinators applied to even simpler
  models.

  The role of these combinators is not \emph{essential}, but \emph{helpful} ---
  we could always fall back on normal function composition, but higher-order
  functions and combinators let us encapsulate certain repeating design patterns
  and transformations.
\item
  \emph{Differentiable} programs, allowing us to write normal functions and have
  them be automatically differentiable for gradient descent.

  I'm not sure at this point if this is best when supported at the
  language/compiler level, or at the library level. Whatever it is, though, the
  combination of differentiable programming with higher-order functions and
  other functional programming fundamentals is what makes this particularly
  powerful.
\item
  \emph{Purely} functional programming. If \emph{any} of these functions were
  side-effecting and impure functions, the correspondence between functions and
  mathematical models completely falls apart. This is something we often take
  for granted when writing Haskell, but in other languages, without purity, no
  model is sound. If we are writing in a non-pure language, we have to consider
  this as an explicit assumption.
\item
  A \emph{strong expressive static type system with type inference} makes this
  all reasonable to work with.

  A lot of the combinators in this approach (like
  \texttt{(\textless{}\textasciitilde{})}) manipulate the \emph{type} of model
  parameters, and if we use a lot of them, it becomes either impossible or
  unfeasible to manage it all in our heads. Without the help of a compiler, it
  would be impossible to sanely write complex programs. Having a statically type
  system with \emph{type inference} allows the compiler to keep track of these
  for us and manage parameter shapes, and lets us ask questions about the
  parameters that our models have at compile-time.

  For example, note how in our \texttt{twoLayer} definition, we left a type
  wildcard so the compiler can fill in the type for us.

  We'll also see in later posts that if we pick the types of our combinators
  correctly, the compiler can sometimes basically write our code for us.

  In addition, having to think about types forces us to think, ahead of time,
  about how types interact. This thought process itself often yields important
  insight.
\end{enumerate}

In the
\href{https://blog.jle.im/entry/purely-functional-typed-models-2.html}{next
post}, we will explore how to reap the surprising benefits of this purely
functional typed style when applying it to stateful and recurrent models.

\hypertarget{signoff}{%
\section{Signoff}\label{signoff}}

Hi, thanks for reading! You can reach me via email at
\href{mailto:justin@jle.im}{\nolinkurl{justin@jle.im}}, or at twitter at
\href{https://twitter.com/mstk}{@mstk}! This post and all others are published
under the \href{https://creativecommons.org/licenses/by-nc-nd/3.0/}{CC-BY-NC-ND
3.0} license. Corrections and edits via pull request are welcome and encouraged
at \href{https://github.com/mstksg/inCode}{the source repository}.

If you feel inclined, or this post was particularly helpful for you, why not
consider \href{https://www.patreon.com/justinle/overview}{supporting me on
Patreon}, or a \href{bitcoin:3D7rmAYgbDnp4gp4rf22THsGt74fNucPDU}{BTC donation}?
:)

\end{document}
