\documentclass[]{article}
\usepackage{lmodern}
\usepackage{amssymb,amsmath}
\usepackage{ifxetex,ifluatex}
\usepackage{fixltx2e} % provides \textsubscript
\ifnum 0\ifxetex 1\fi\ifluatex 1\fi=0 % if pdftex
  \usepackage[T1]{fontenc}
  \usepackage[utf8]{inputenc}
\else % if luatex or xelatex
  \ifxetex
    \usepackage{mathspec}
    \usepackage{xltxtra,xunicode}
  \else
    \usepackage{fontspec}
  \fi
  \defaultfontfeatures{Mapping=tex-text,Scale=MatchLowercase}
  \newcommand{\euro}{€}
\fi
% use upquote if available, for straight quotes in verbatim environments
\IfFileExists{upquote.sty}{\usepackage{upquote}}{}
% use microtype if available
\IfFileExists{microtype.sty}{\usepackage{microtype}}{}
\usepackage[margin=1in]{geometry}
\usepackage{color}
\usepackage{fancyvrb}
\newcommand{\VerbBar}{|}
\newcommand{\VERB}{\Verb[commandchars=\\\{\}]}
\DefineVerbatimEnvironment{Highlighting}{Verbatim}{commandchars=\\\{\}}
% Add ',fontsize=\small' for more characters per line
\newenvironment{Shaded}{}{}
\newcommand{\AlertTok}[1]{\textcolor[rgb]{1.00,0.00,0.00}{\textbf{#1}}}
\newcommand{\AnnotationTok}[1]{\textcolor[rgb]{0.38,0.63,0.69}{\textbf{\textit{#1}}}}
\newcommand{\AttributeTok}[1]{\textcolor[rgb]{0.49,0.56,0.16}{#1}}
\newcommand{\BaseNTok}[1]{\textcolor[rgb]{0.25,0.63,0.44}{#1}}
\newcommand{\BuiltInTok}[1]{#1}
\newcommand{\CharTok}[1]{\textcolor[rgb]{0.25,0.44,0.63}{#1}}
\newcommand{\CommentTok}[1]{\textcolor[rgb]{0.38,0.63,0.69}{\textit{#1}}}
\newcommand{\CommentVarTok}[1]{\textcolor[rgb]{0.38,0.63,0.69}{\textbf{\textit{#1}}}}
\newcommand{\ConstantTok}[1]{\textcolor[rgb]{0.53,0.00,0.00}{#1}}
\newcommand{\ControlFlowTok}[1]{\textcolor[rgb]{0.00,0.44,0.13}{\textbf{#1}}}
\newcommand{\DataTypeTok}[1]{\textcolor[rgb]{0.56,0.13,0.00}{#1}}
\newcommand{\DecValTok}[1]{\textcolor[rgb]{0.25,0.63,0.44}{#1}}
\newcommand{\DocumentationTok}[1]{\textcolor[rgb]{0.73,0.13,0.13}{\textit{#1}}}
\newcommand{\ErrorTok}[1]{\textcolor[rgb]{1.00,0.00,0.00}{\textbf{#1}}}
\newcommand{\ExtensionTok}[1]{#1}
\newcommand{\FloatTok}[1]{\textcolor[rgb]{0.25,0.63,0.44}{#1}}
\newcommand{\FunctionTok}[1]{\textcolor[rgb]{0.02,0.16,0.49}{#1}}
\newcommand{\ImportTok}[1]{#1}
\newcommand{\InformationTok}[1]{\textcolor[rgb]{0.38,0.63,0.69}{\textbf{\textit{#1}}}}
\newcommand{\KeywordTok}[1]{\textcolor[rgb]{0.00,0.44,0.13}{\textbf{#1}}}
\newcommand{\NormalTok}[1]{#1}
\newcommand{\OperatorTok}[1]{\textcolor[rgb]{0.40,0.40,0.40}{#1}}
\newcommand{\OtherTok}[1]{\textcolor[rgb]{0.00,0.44,0.13}{#1}}
\newcommand{\PreprocessorTok}[1]{\textcolor[rgb]{0.74,0.48,0.00}{#1}}
\newcommand{\RegionMarkerTok}[1]{#1}
\newcommand{\SpecialCharTok}[1]{\textcolor[rgb]{0.25,0.44,0.63}{#1}}
\newcommand{\SpecialStringTok}[1]{\textcolor[rgb]{0.73,0.40,0.53}{#1}}
\newcommand{\StringTok}[1]{\textcolor[rgb]{0.25,0.44,0.63}{#1}}
\newcommand{\VariableTok}[1]{\textcolor[rgb]{0.10,0.09,0.49}{#1}}
\newcommand{\VerbatimStringTok}[1]{\textcolor[rgb]{0.25,0.44,0.63}{#1}}
\newcommand{\WarningTok}[1]{\textcolor[rgb]{0.38,0.63,0.69}{\textbf{\textit{#1}}}}
\ifxetex
  \usepackage[setpagesize=false, % page size defined by xetex
              unicode=false, % unicode breaks when used with xetex
              xetex]{hyperref}
\else
  \usepackage[unicode=true]{hyperref}
\fi
\hypersetup{breaklinks=true,
            bookmarks=true,
            pdfauthor={Justin Le},
            pdftitle={Setting up a dead-simple TCP/IP service using servant},
            colorlinks=true,
            citecolor=blue,
            urlcolor=blue,
            linkcolor=magenta,
            pdfborder={0 0 0}}
\urlstyle{same}  % don't use monospace font for urls
% Make links footnotes instead of hotlinks:
\renewcommand{\href}[2]{#2\footnote{\url{#1}}}
\setlength{\parindent}{0pt}
\setlength{\parskip}{6pt plus 2pt minus 1pt}
\setlength{\emergencystretch}{3em}  % prevent overfull lines
\setcounter{secnumdepth}{0}

\title{Setting up a dead-simple TCP/IP service using servant}
\author{Justin Le}

\begin{document}
\maketitle

\emph{Originally posted on
\textbf{\href{https://blog.jle.im/entry/servant-services.html}{in Code}}.}

In my time I've written a lot of throwaway binary TCP/IP services (servers and
services you can interact with over an internet connection, through command line
interface or GUI). For me, this involves designing a protocol from scratch every
time with varying levels of hand-rolled authentication and error detection (Send
this byte for this command, this byte for this other command, etc.). Once I
design the protocol, I then have to write both the client and the server ---
something I usually do from scratch over the raw TCP streams.

This process was fun (and informative) the first few times I did it, but
spinning it up from scratch again every time discouraged me from doing it very
often. However, thankfully, with the
\emph{\href{https://hackage.haskell.org/package/servant}{servant}} haskell
library, writing a TCP server/client pair for a TCP service becomes dead-simple
--- the barrier for creating one fades away that designing/writing a service
becomes a tool that I reach for immediately in a lot of cases without second
thought.

\emph{servant} is usually advertised as a tool for writing web servers, web
applications, and REST APIs, but it's easily adapted to write non-web things as
well (especially with the help of
\emph{\href{https://hackage.haskell.org/package/servant-client}{servant-client}}
and \emph{\href{https://hackage.haskell.org/package/servant-cli}{servant-cli}}).
Let's dive in and write a simple TCP/IP service (a todo list manager) to see how
straightforward the process is!

\hypertarget{todo-api}{%
\section{Todo API}\label{todo-api}}

As an example, we'll work through building one of my favorite self-contained
mini-app projects, a \href{http://todomvc.com/}{Todo list manager a la
todo-mvc}. Our service will provide functionality for:

\begin{enumerate}
\def\labelenumi{\arabic{enumi}.}
\tightlist
\item
  Viewing all tasks and their status
\item
  Adding a new task
\item
  Setting a task's completion status
\item
  Deleting a task
\item
  Pruning all completed tasks
\end{enumerate}

To facilitate doing this over an API, we'll assign each task a task ID when it
comes in, and so commands 3 and 4 will require a task ID.

To formally specify our API:

\begin{enumerate}
\def\labelenumi{\arabic{enumi}.}
\tightlist
\item
  \texttt{view}: View all tasks by their ID, status, and description. Optionally
  be able to filter for only incomplete tasks.
\item
  \texttt{add}: Given a new task description, insert a new uncompleted task.
  Return the ID of the new task.
\item
  \texttt{set}: Given a task ID and an updated status, update the task's status.
\item
  \texttt{delete}: Given a task ID, delete the task.
\item
  \texttt{prune}: Remove all completed tasks. Returns all the task IDs that
  where deleted.
\end{enumerate}

We can state this using servant's type level DSL, using an \texttt{IntMap} to
represent the current tasks and an \texttt{IntSet} to represent a set of task
IDs.

\begin{Shaded}
\begin{Highlighting}[]
\CommentTok{{-}{-} source: https://github.com/mstksg/inCode/tree/master/code{-}samples/servant{-}services/Api.hs}

\OtherTok{\{{-}\# LANGUAGE DeriveGeneric \#{-}\}}
\OtherTok{\{{-}\# LANGUAGE TypeInType    \#{-}\}}
\OtherTok{\{{-}\# LANGUAGE TypeOperators \#{-}\}}

\KeywordTok{module} \DataTypeTok{Api} \KeywordTok{where}

\KeywordTok{import}           \DataTypeTok{Data.Aeson}
\KeywordTok{import}           \DataTypeTok{Data.IntMap}\NormalTok{ (}\DataTypeTok{IntMap}\NormalTok{)}
\KeywordTok{import}           \DataTypeTok{Data.IntSet}\NormalTok{ (}\DataTypeTok{IntSet}\NormalTok{)}
\KeywordTok{import}           \DataTypeTok{Data.Proxy}
\KeywordTok{import}           \DataTypeTok{Data.Text}\NormalTok{ (}\DataTypeTok{Text}\NormalTok{)}
\KeywordTok{import}           \DataTypeTok{GHC.Generics}
\KeywordTok{import}           \DataTypeTok{Servant.API}

\KeywordTok{data} \DataTypeTok{Task} \OtherTok{=} \DataTypeTok{Task}
\NormalTok{    \{}\OtherTok{ taskStatus ::} \DataTypeTok{Bool}
\NormalTok{    ,}\OtherTok{ taskDesc   ::} \DataTypeTok{Text}
\NormalTok{    \}}
  \KeywordTok{deriving}\NormalTok{ (}\DataTypeTok{Show}\NormalTok{, }\DataTypeTok{Generic}\NormalTok{)}
\KeywordTok{instance} \DataTypeTok{ToJSON}   \DataTypeTok{Task}
\KeywordTok{instance} \DataTypeTok{FromJSON} \DataTypeTok{Task}

\KeywordTok{type} \DataTypeTok{TodoApi} \OtherTok{=}
      \StringTok{"list"}   \OperatorTok{:>} \DataTypeTok{QueryFlag} \StringTok{"filtered"}                      \OperatorTok{:>} \DataTypeTok{Get}\NormalTok{  \textquotesingle{}[}\DataTypeTok{JSON}\NormalTok{] (}\DataTypeTok{IntMap} \DataTypeTok{Task}\NormalTok{)}
 \OperatorTok{:<|>} \StringTok{"add"}    \OperatorTok{:>} \DataTypeTok{QueryParam\textquotesingle{}}\NormalTok{ \textquotesingle{}[}\DataTypeTok{Required}\NormalTok{] }\StringTok{"desc"} \DataTypeTok{Text}       \OperatorTok{:>} \DataTypeTok{Post}\NormalTok{ \textquotesingle{}[}\DataTypeTok{JSON}\NormalTok{] }\DataTypeTok{Int}
 \OperatorTok{:<|>} \StringTok{"set"}    \OperatorTok{:>} \DataTypeTok{Capture} \StringTok{"id"} \DataTypeTok{Int} \OperatorTok{:>} \DataTypeTok{Capture} \StringTok{"status"} \DataTypeTok{Bool} \OperatorTok{:>} \DataTypeTok{Post}\NormalTok{ \textquotesingle{}[}\DataTypeTok{JSON}\NormalTok{] ()}
 \OperatorTok{:<|>} \StringTok{"delete"} \OperatorTok{:>} \DataTypeTok{Capture} \StringTok{"id"} \DataTypeTok{Int}                          \OperatorTok{:>} \DataTypeTok{Post}\NormalTok{ \textquotesingle{}[}\DataTypeTok{JSON}\NormalTok{] ()}
 \OperatorTok{:<|>} \StringTok{"prune"}                                               \OperatorTok{:>} \DataTypeTok{Post}\NormalTok{ \textquotesingle{}[}\DataTypeTok{JSON}\NormalTok{] }\DataTypeTok{IntSet}

\OtherTok{todoApi ::} \DataTypeTok{Proxy} \DataTypeTok{TodoApi}
\NormalTok{todoApi }\OtherTok{=} \DataTypeTok{Proxy}
\end{Highlighting}
\end{Shaded}

\hypertarget{signoff}{%
\section{Signoff}\label{signoff}}

Hi, thanks for reading! You can reach me via email at
\href{mailto:justin@jle.im}{\nolinkurl{justin@jle.im}}, or at twitter at
\href{https://twitter.com/mstk}{@mstk}! This post and all others are published
under the \href{https://creativecommons.org/licenses/by-nc-nd/3.0/}{CC-BY-NC-ND
3.0} license. Corrections and edits via pull request are welcome and encouraged
at \href{https://github.com/mstksg/inCode}{the source repository}.

If you feel inclined, or this post was particularly helpful for you, why not
consider \href{https://www.patreon.com/justinle/overview}{supporting me on
Patreon}, or a \href{bitcoin:3D7rmAYgbDnp4gp4rf22THsGt74fNucPDU}{BTC donation}?
:)

\end{document}
