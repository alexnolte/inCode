\documentclass[]{article}
\usepackage{lmodern}
\usepackage{amssymb,amsmath}
\usepackage{ifxetex,ifluatex}
\usepackage{fixltx2e} % provides \textsubscript
\ifnum 0\ifxetex 1\fi\ifluatex 1\fi=0 % if pdftex
  \usepackage[T1]{fontenc}
  \usepackage[utf8]{inputenc}
\else % if luatex or xelatex
  \ifxetex
    \usepackage{mathspec}
    \usepackage{xltxtra,xunicode}
  \else
    \usepackage{fontspec}
  \fi
  \defaultfontfeatures{Mapping=tex-text,Scale=MatchLowercase}
  \newcommand{\euro}{€}
\fi
% use upquote if available, for straight quotes in verbatim environments
\IfFileExists{upquote.sty}{\usepackage{upquote}}{}
% use microtype if available
\IfFileExists{microtype.sty}{\usepackage{microtype}}{}
\usepackage[margin=1in]{geometry}
\usepackage{color}
\usepackage{fancyvrb}
\newcommand{\VerbBar}{|}
\newcommand{\VERB}{\Verb[commandchars=\\\{\}]}
\DefineVerbatimEnvironment{Highlighting}{Verbatim}{commandchars=\\\{\}}
% Add ',fontsize=\small' for more characters per line
\newenvironment{Shaded}{}{}
\newcommand{\AlertTok}[1]{\textcolor[rgb]{1.00,0.00,0.00}{\textbf{#1}}}
\newcommand{\AnnotationTok}[1]{\textcolor[rgb]{0.38,0.63,0.69}{\textbf{\textit{#1}}}}
\newcommand{\AttributeTok}[1]{\textcolor[rgb]{0.49,0.56,0.16}{#1}}
\newcommand{\BaseNTok}[1]{\textcolor[rgb]{0.25,0.63,0.44}{#1}}
\newcommand{\BuiltInTok}[1]{#1}
\newcommand{\CharTok}[1]{\textcolor[rgb]{0.25,0.44,0.63}{#1}}
\newcommand{\CommentTok}[1]{\textcolor[rgb]{0.38,0.63,0.69}{\textit{#1}}}
\newcommand{\CommentVarTok}[1]{\textcolor[rgb]{0.38,0.63,0.69}{\textbf{\textit{#1}}}}
\newcommand{\ConstantTok}[1]{\textcolor[rgb]{0.53,0.00,0.00}{#1}}
\newcommand{\ControlFlowTok}[1]{\textcolor[rgb]{0.00,0.44,0.13}{\textbf{#1}}}
\newcommand{\DataTypeTok}[1]{\textcolor[rgb]{0.56,0.13,0.00}{#1}}
\newcommand{\DecValTok}[1]{\textcolor[rgb]{0.25,0.63,0.44}{#1}}
\newcommand{\DocumentationTok}[1]{\textcolor[rgb]{0.73,0.13,0.13}{\textit{#1}}}
\newcommand{\ErrorTok}[1]{\textcolor[rgb]{1.00,0.00,0.00}{\textbf{#1}}}
\newcommand{\ExtensionTok}[1]{#1}
\newcommand{\FloatTok}[1]{\textcolor[rgb]{0.25,0.63,0.44}{#1}}
\newcommand{\FunctionTok}[1]{\textcolor[rgb]{0.02,0.16,0.49}{#1}}
\newcommand{\ImportTok}[1]{#1}
\newcommand{\InformationTok}[1]{\textcolor[rgb]{0.38,0.63,0.69}{\textbf{\textit{#1}}}}
\newcommand{\KeywordTok}[1]{\textcolor[rgb]{0.00,0.44,0.13}{\textbf{#1}}}
\newcommand{\NormalTok}[1]{#1}
\newcommand{\OperatorTok}[1]{\textcolor[rgb]{0.40,0.40,0.40}{#1}}
\newcommand{\OtherTok}[1]{\textcolor[rgb]{0.00,0.44,0.13}{#1}}
\newcommand{\PreprocessorTok}[1]{\textcolor[rgb]{0.74,0.48,0.00}{#1}}
\newcommand{\RegionMarkerTok}[1]{#1}
\newcommand{\SpecialCharTok}[1]{\textcolor[rgb]{0.25,0.44,0.63}{#1}}
\newcommand{\SpecialStringTok}[1]{\textcolor[rgb]{0.73,0.40,0.53}{#1}}
\newcommand{\StringTok}[1]{\textcolor[rgb]{0.25,0.44,0.63}{#1}}
\newcommand{\VariableTok}[1]{\textcolor[rgb]{0.10,0.09,0.49}{#1}}
\newcommand{\VerbatimStringTok}[1]{\textcolor[rgb]{0.25,0.44,0.63}{#1}}
\newcommand{\WarningTok}[1]{\textcolor[rgb]{0.38,0.63,0.69}{\textbf{\textit{#1}}}}
\ifxetex
  \usepackage[setpagesize=false, % page size defined by xetex
              unicode=false, % unicode breaks when used with xetex
              xetex]{hyperref}
\else
  \usepackage[unicode=true]{hyperref}
\fi
\hypersetup{breaklinks=true,
            bookmarks=true,
            pdfauthor={Justin Le},
            pdftitle={Applicative Regular Expressions using the Free Alternative},
            colorlinks=true,
            citecolor=blue,
            urlcolor=blue,
            linkcolor=magenta,
            pdfborder={0 0 0}}
\urlstyle{same}  % don't use monospace font for urls
% Make links footnotes instead of hotlinks:
\renewcommand{\href}[2]{#2\footnote{\url{#1}}}
\setlength{\parindent}{0pt}
\setlength{\parskip}{6pt plus 2pt minus 1pt}
\setlength{\emergencystretch}{3em}  % prevent overfull lines
\setcounter{secnumdepth}{0}

\title{Applicative Regular Expressions using the Free Alternative}
\author{Justin Le}

\begin{document}
\maketitle

\emph{Originally posted on
\textbf{\href{https://blog.jle.im/entry/free-alternative-regexp.html}{in
Code}}.}

Today, we're going to implement applicative regular expressions and parsers (in
the style of the
\href{https://hackage.haskell.org/package/regex-applicative}{regex-applicative}
library) using free structures!

Free structures are some of my favorite tools in Haskell, and I've actually
written a few posts about them before, including
\href{https://blog.jle.im/entry/alchemical-groups.html}{this one using free
groups}, \href{https://blog.jle.im/entry/interpreters-a-la-carte-duet.html}{this
one on a free monad variation}, and
\href{https://blog.jle.im/entry/const-applicative-and-monoids.html}{this one on
a ``free'' applicative on a monoid}.

Regular expressions (and parsers) are ubiquitous in computer science and
programming, and I hope that demonstrating that they are pretty straightforward
to implement using free structures will help you see the value in free
structures without getting too bogged down in the details!

All of the code in this post is
\href{https://github.com/mstksg/inCode/tree/master/code-samples/misc/regexp.hs}{available
online} as a ``stack executable''. When you run it (\texttt{./regexp.hs}),
you'll load up a \emph{ghci} session with all of the definitions in scope, so
you can play around with the functions and types :)

\hypertarget{regular-languages}{%
\section{Regular Languages}\label{regular-languages}}

A \emph{regular expression} is something that defines a \emph{regular language}.
\href{https://en.wikipedia.org/wiki/Regular_expression\#Formal_language_theory}{Formally},
it consists of the following primitives:

\begin{enumerate}
\def\labelenumi{\arabic{enumi}.}
\tightlist
\item
  The empty set, which always fails to match.
\item
  The empty string, which always succeeds matching the empty string.
\item
  The literal character, denoting a single matching character
\end{enumerate}

And the following operations:

\begin{enumerate}
\def\labelenumi{\arabic{enumi}.}
\tightlist
\item
  Concatenation: \texttt{RS}, sequence one after the other. A set product.
\item
  Alternation: \texttt{R\textbar{}S}, one or the other. A set union.
\item
  Kleene Star: \texttt{R*}, the repetition of \texttt{R} one or more times.
\end{enumerate}

And that's \emph{all} that's in a regular expression. Nothing more, nothing
less. From these basic tools, you can derive the rest of the regexp operations
--- for example, \texttt{a+} can be expressed as \texttt{aa*}, and categories
like \texttt{\textbackslash{}w} can be expressed as alternations of valid
characters.

\hypertarget{alternative}{%
\subsection{Alternative}\label{alternative}}

Looking at this, does this look a little familiar? It reminds me a lot of the
\texttt{Alternative} typeclass. If a functor \texttt{f} has an
\texttt{Alternative} instance, it means that it has:

\begin{enumerate}
\def\labelenumi{\arabic{enumi}.}
\tightlist
\item
  \texttt{empty}, the failing operation
\item
  \texttt{pure\ x}, the always-succeeding operation (from the
  \texttt{Applicative} class)
\item
  \texttt{\textless{}*\textgreater{}}, the sequencing operation (from the
  \texttt{Applicative} class)
\item
  \texttt{\textless{}\textbar{}\textgreater{}}, the alternating operation
\item
  \texttt{many}, the ``one or more'' operation.
\end{enumerate}

This\ldots{}looks a lot like the construction of a regular language, doesn't it?
It's almost as if \texttt{Alternative} has almost \emph{exactly} what we need.
The only thing missing is the literal character primitive.

If you're unfamiliar with \texttt{Alternative}, the
\href{https://wiki.haskell.org/Typeclassopedia}{typeclassopedia} has a good
step-by-step introduction. But for the purposes of this article, it's basically
just a ``double monoid'', with two ``combining'' actions
\texttt{\textless{}*\textgreater{}} and
\texttt{\textless{}\textbar{}\textgreater{}}, which roughly correspond to
\texttt{*} and \texttt{+} in the integers. It's basically pretty much nothing
more than 1-5 in the list above, and some distributivity laws.

So, one way we can look at regular expressions is ``The entire
\texttt{Alternative} interface, plus a character primitive''. \emph{But!}
There's another way of looking at this, that leads us directly to free
structures.

Instead of seeing things as ``\texttt{Alternative} with a character primitive'',
we can look at it as \emph{a character primitive enriched with a Alternative
instance}.

\hypertarget{free}{%
\section{Free}\label{free}}

Let's write this out. Our character primitive will be:

\begin{Shaded}
\begin{Highlighting}[]
\CommentTok{-- source: https://github.com/mstksg/inCode/tree/master/code-samples/misc/regexp.hs#L16-L17}

\KeywordTok{data} \DataTypeTok{Prim}\NormalTok{ a }\FunctionTok{=} \DataTypeTok{Prim} \DataTypeTok{Char}\NormalTok{ a}
  \KeywordTok{deriving} \DataTypeTok{Functor}
\end{Highlighting}
\end{Shaded}

Note that because we're working with functors, applicatives, alternatives, etc.,
all of our regular expressions can have an associated ``result''. This is
because our regexp values will have a type parameter (which is required for
\texttt{Functor}, \texttt{Applicative}, and \texttt{Alternative}). We can choose
to ignore this type parameter, of course, but we can also have some fun by using
it to represent a ``result'' that a regexp match will be interpreted as. This is
similar to the idea of ``capturing'' in industrial regexp applications.

Here, the value
\texttt{Prim\ \textquotesingle{}a\textquotesingle{}\ 1\ ::\ Prim\ Int} will
represent a primitive that matches on the character \texttt{a}, interpreting it
with a result of \texttt{1}.

And now\ldots{}we give it \texttt{Alternative} structure using the \emph{Free
Alternative}, from the
\emph{\href{https://hackage.haskell.org/package/free}{free}} package:

\begin{Shaded}
\begin{Highlighting}[]
\KeywordTok{import} \DataTypeTok{Control.Alternative.Free}

\CommentTok{-- source: https://github.com/mstksg/inCode/tree/master/code-samples/misc/regexp.hs#L19-L19}

\KeywordTok{type} \DataTypeTok{RegExp} \FunctionTok{=} \DataTypeTok{Alt} \DataTypeTok{Prim}
\end{Highlighting}
\end{Shaded}

And that's it! That's our entire regular expression type! By giving a
\texttt{Alt} a \texttt{Functor}, we get all of the operations of
\texttt{Applicative} and \texttt{Alternative} over our base. That's because we
have \texttt{instance\ Applicative\ (Alt\ f)} and
\texttt{instance\ Alternative\ (Alt\ f)}. We now have:

\begin{enumerate}
\def\labelenumi{\arabic{enumi}.}
\tightlist
\item
  The empty set, coming from \texttt{empty} from \texttt{Alternative}
\item
  The empty string, coming from \texttt{pure} from \texttt{Applicative}
\item
  The character primitive, coming from the underlying functor \texttt{Prim} that
  we are enhancing
\item
  The concatenation operation, from \texttt{\textless{}*\textgreater{}}, from
  \texttt{Applicative}.
\item
  The alternating operation, from \texttt{\textless{}\textbar{}\textgreater{}},
  from \texttt{Alternative}.
\item
  The kleene star, from \texttt{many}, from \texttt{Alternative}.
\end{enumerate}

All of these additions to our primitive come ``for free''!

Essentially, what a free structure gives us is the structure of the abstraction
(\texttt{Alternative}, here) automatically for our base type, and \emph{nothing
else}.

Remember that regular expressions have these operations, \emph{and nothing else}
--- no more, no less. That's exactly what the free Alternative gives us: these
these operations and the primitive. No more, no less.

After adding some convenient wrappers\ldots{}we're done here!

\begin{Shaded}
\begin{Highlighting}[]
\CommentTok{-- source: https://github.com/mstksg/inCode/tree/master/code-samples/misc/regexp.hs#L21-L32}

\CommentTok{-- | charAs: Parse a given character as a given constant result.}
\OtherTok{charAs ::} \DataTypeTok{Char} \OtherTok{->}\NormalTok{ a }\OtherTok{->} \DataTypeTok{RegExp}\NormalTok{ a}
\NormalTok{charAs c x }\FunctionTok{=}\NormalTok{ liftAlt (}\DataTypeTok{Prim}\NormalTok{ c x)     }\CommentTok{-- liftAlt lets us use the underlying}
                                    \CommentTok{-- functor Prim in RegExp}

\CommentTok{-- | char: Parse a given character as itself.}
\OtherTok{char ::} \DataTypeTok{Char} \OtherTok{->} \DataTypeTok{RegExp} \DataTypeTok{Char}
\NormalTok{char c }\FunctionTok{=}\NormalTok{ charAs c c}

\CommentTok{-- | string: Parse a given string as itself.}
\OtherTok{string ::} \DataTypeTok{String} \OtherTok{->} \DataTypeTok{RegExp} \DataTypeTok{String}
\NormalTok{string }\FunctionTok{=} \FunctionTok{traverse}\NormalTok{ char              }\CommentTok{-- neat, huh}
\end{Highlighting}
\end{Shaded}

\hypertarget{examples}{%
\subsection{Examples}\label{examples}}

Let's try it out! Let's match on \texttt{(a\textbar{}b)(cd)*e} and return
\texttt{()}:

\begin{Shaded}
\begin{Highlighting}[]
\CommentTok{-- source: https://github.com/mstksg/inCode/tree/master/code-samples/misc/regexp.hs#L34-L37}

\OtherTok{testRegExp_ ::} \DataTypeTok{RegExp}\NormalTok{ ()}
\NormalTok{testRegExp_ }\FunctionTok{=}\NormalTok{ void }\FunctionTok{$}\NormalTok{ (char }\CharTok{'a'} \FunctionTok{<|>}\NormalTok{ char }\CharTok{'b'}\NormalTok{)}
                  \FunctionTok{*>}\NormalTok{ many (string }\StringTok{"cd"}\NormalTok{)}
                  \FunctionTok{*>}\NormalTok{ char }\CharTok{'e'}
\end{Highlighting}
\end{Shaded}

\texttt{void} from \emph{Data.Functor} discards the results, since we only care
if it matches or not. But we use \texttt{\textless{}\textbar{}\textgreater{}}
and \texttt{*\textgreater{}} and \texttt{many} exactly how we'd expect to
concatenate and alternate things with \texttt{Applicative} and
\texttt{Alternative}.

Or maybe more interesting (but slightly more complicated), let's match on the
same pattern and return how many \texttt{cd}s are repeated

\begin{Shaded}
\begin{Highlighting}[]
\CommentTok{-- source: https://github.com/mstksg/inCode/tree/master/code-samples/misc/regexp.hs#L39-L42}

\OtherTok{testRegExp ::} \DataTypeTok{RegExp} \DataTypeTok{Int}
\NormalTok{testRegExp }\FunctionTok{=}\NormalTok{ (char }\CharTok{'a'} \FunctionTok{<|>}\NormalTok{ char }\CharTok{'b'}\NormalTok{)}
          \FunctionTok{*>}\NormalTok{ (}\FunctionTok{length} \FunctionTok{<$>}\NormalTok{ many (string }\StringTok{"cd"}\NormalTok{))}
          \FunctionTok{<*}\NormalTok{ char }\CharTok{'e'}
\end{Highlighting}
\end{Shaded}

This one does require a little more finesse with \texttt{*\textgreater{}} and
\texttt{\textless{}*}: the arrows point towards which result to ``keep''. And
since \texttt{many\ (string\ "cd")\ ::\ RegExp\ {[}String{]}} (it returns a
list, with an item for each repetition), we can \texttt{fmap\ length} to get the
\texttt{Int} result of ``how many repetitions''.

However, we can also turn on \emph{-XApplicativeDo} and write it using do
notation, which requires a little less thought:

\begin{Shaded}
\begin{Highlighting}[]
\CommentTok{-- source: https://github.com/mstksg/inCode/tree/master/code-samples/misc/regexp.hs#L44-L49}

\OtherTok{testRegExpDo ::} \DataTypeTok{RegExp} \DataTypeTok{Int}
\NormalTok{testRegExpDo }\FunctionTok{=} \KeywordTok{do}
\NormalTok{    char }\CharTok{'a'} \FunctionTok{<|>}\NormalTok{ char }\CharTok{'b'}
\NormalTok{    cds }\OtherTok{<-}\NormalTok{ many (string }\StringTok{"cd"}\NormalTok{)}
\NormalTok{    char }\CharTok{'e'}
    \FunctionTok{pure}\NormalTok{ (}\FunctionTok{length}\NormalTok{ cds)}
\end{Highlighting}
\end{Shaded}

It's all a little bit like how we often use ``captures'' in regular expressions
to access a \emph{specific part} of a match. Here's an example in ruby:

\begin{Shaded}
\begin{Highlighting}[]
\NormalTok{irb> }\OtherTok{/(a|b)((cd)*)e/}\NormalTok{.match(}\StringTok{"acdcdcdcde"}\NormalTok{)[}\DecValTok{2}\NormalTok{]}
\NormalTok{=> }\StringTok{"cdcdcdcd"}
\end{Highlighting}
\end{Shaded}

except we also include a ``post-processing'' to get the length of the number of
repetitions.

\hypertarget{parsing}{%
\section{Parsing}\label{parsing}}

Okay, so all we did was define a data structure that supports character
matching, concatenation, alternation, and starring. Big whoop. What we really
want to do is use it to parse things, right? How does the Free Alternative help
us with \emph{that}?

Well, it does a lot, actually. Let's look at two ways of doing this!

\hypertarget{offloading-alternative}{%
\subsection{Offloading Alternative}\label{offloading-alternative}}

\hypertarget{what-is-freeness}{%
\subsubsection{What is Freeness?}\label{what-is-freeness}}

The ``canonical'' way of using a free structure is by using its ``folding''
operation into a concrete structure with the proper instances. For example,
\texttt{foldMap} will turn a free monoid into a value of any monoid instance:

\begin{Shaded}
\begin{Highlighting}[]
\FunctionTok{foldMap}\OtherTok{ ::} \DataTypeTok{Monoid}\NormalTok{ m }\OtherTok{=>}\NormalTok{ (a }\OtherTok{->}\NormalTok{ m) }\OtherTok{->}\NormalTok{ ([a] }\OtherTok{->}\NormalTok{ m)}
\end{Highlighting}
\end{Shaded}

\texttt{foldMap} lifts an \texttt{a\ -\textgreater{}\ m} into a
\texttt{{[}a{]}\ -\textgreater{}\ m} (or,
\texttt{FreeMonoid\ a\ -\textgreater{}\ m}), with a concrete monoid \texttt{m}.
The general idea is that using a free structure can ``defer'' the concretization
from between the time of construction to the time of use.

For example, we can construct value in the free monoid made from integers:

\begin{Shaded}
\begin{Highlighting}[]
\CommentTok{-- | Lift the "primitive" `Int` into a value in the free monoid on `Int`.}
\CommentTok{--}
\CommentTok{-- Analogous to `liftAlt` from earlier.}
\OtherTok{liftFM ::} \DataTypeTok{Int} \OtherTok{->}\NormalTok{ [}\DataTypeTok{Int}\NormalTok{]}
\NormalTok{liftFM x }\FunctionTok{=}\NormalTok{ [x]}

\OtherTok{myMon ::}\NormalTok{ [}\DataTypeTok{Int}\NormalTok{]}
\NormalTok{myMon }\FunctionTok{=}\NormalTok{ liftFM }\DecValTok{1} \FunctionTok{<>}\NormalTok{ liftFM }\DecValTok{2} \FunctionTok{<>}\NormalTok{ liftFM }\DecValTok{3} \FunctionTok{<>}\NormalTok{ liftFM }\DecValTok{4}
\end{Highlighting}
\end{Shaded}

And now we can decide how we want to interpret
\texttt{\textless{}\textgreater{}} --- should it be \texttt{+}?

\begin{Shaded}
\begin{Highlighting}[]
\NormalTok{ghci}\FunctionTok{>} \FunctionTok{foldMap} \DataTypeTok{Sum}\NormalTok{ myMon}
\DataTypeTok{Sum} \DecValTok{10}              \CommentTok{-- 1 + 2 + 3 + 4}
\end{Highlighting}
\end{Shaded}

Or should it be \texttt{*}?

\begin{Shaded}
\begin{Highlighting}[]
\NormalTok{ghci}\FunctionTok{>} \FunctionTok{foldMap} \DataTypeTok{Product}\NormalTok{ myMon}
\DataTypeTok{Product} \DecValTok{24}          \CommentTok{-- 1 * 2 * 3 * 4}
\end{Highlighting}
\end{Shaded}

The idea is that we can ``defer'' the choice of concrete \texttt{Monoid} that
\texttt{\textless{}\textgreater{}} is interpreted under by first pushing 1, 2,
3, and 4 into a free monoid value. The free monoid on \texttt{Int} gives
\emph{exactly enough structure} to \texttt{Int} to do this job: no more, no
less.

To use \texttt{foldMap}, we say ``how to handle the base type'', and it lets us
handle the free structure in its entirety by offloading the behavior of
\texttt{\textless{}\textgreater{}} to the concrete monoid.

\hypertarget{interpreting-in-state}{%
\subsubsection{Interpreting in State}\label{interpreting-in-state}}

In practice, getting a result from a free structure is often about finding (or
creating) the right concrete \texttt{Alternative} that gives us the behavior we
want. In this case, we're in luck. There's a concrete \texttt{Alternative}
instance that works just the way we want: \texttt{StateT\ String\ Maybe}:

\begin{itemize}
\tightlist
\item
  Its \texttt{\textless{}*\textgreater{}} works by sequencing changes in state;
  in this case, we'll consider the state as ``characters yet to be parsed'', so
  sequential parsing fits perfectly with \texttt{\textless{}*\textgreater{}}.
  That's because combining regexps sequentially can be thought of as statefully
  chomping down on a string.
\item
  Its \texttt{\textless{}\textbar{}\textgreater{}} works by backtracking and
  trying again if it runs into a failure. It saves the state of the last
  successful point and resets to it on failure. This is exactly how we want
  regexp alternation \texttt{R\textbar{}S} to behave.
\end{itemize}

The ``folding'' operation of the free alternative is called \texttt{runAlt}:

\begin{Shaded}
\begin{Highlighting}[]
\OtherTok{runAlt ::} \DataTypeTok{Alternative}\NormalTok{ f}
       \OtherTok{=>}\NormalTok{ (}\KeywordTok{forall}\NormalTok{ b}\FunctionTok{.}\NormalTok{ p b }\OtherTok{->}\NormalTok{ f b)}
       \OtherTok{->} \DataTypeTok{Alt}\NormalTok{ p a}
       \OtherTok{->}\NormalTok{ f a}
\end{Highlighting}
\end{Shaded}

And in the case of \texttt{RegExp}, we have:

\begin{Shaded}
\begin{Highlighting}[]
\OtherTok{runAlt ::} \DataTypeTok{Alternative}\NormalTok{ f}
       \OtherTok{=>}\NormalTok{ (}\KeywordTok{forall}\NormalTok{ b}\FunctionTok{.} \DataTypeTok{Prim}\NormalTok{ b }\OtherTok{->}\NormalTok{ f b)}
       \OtherTok{->} \DataTypeTok{RegExp}\NormalTok{ a}
       \OtherTok{->}\NormalTok{ f a}
\end{Highlighting}
\end{Shaded}

If you're unfamiliar with the RankN type (the \texttt{forall\ b.} stuff),
there's a
\href{https://ocharles.org.uk/guest-posts/2014-12-18-rank-n-types.html}{nice
introduction here}. But basically, you just need to provide \texttt{runAlt} with
a function that can handle a \texttt{Prim\ b} for \emph{any} \texttt{b} (and not
just a specific one like \texttt{Int} or \texttt{Bool}).

So, like \texttt{foldMap}, we need to say ``how to handle our base type''. Here,
we have to answer ``How do we handle \texttt{Prim}?''

\begin{Shaded}
\begin{Highlighting}[]
\CommentTok{-- source: https://github.com/mstksg/inCode/tree/master/code-samples/misc/regexp.hs#L51-L56}

\OtherTok{processPrim ::} \DataTypeTok{Prim}\NormalTok{ a }\OtherTok{->} \DataTypeTok{StateT} \DataTypeTok{String} \DataTypeTok{Maybe}\NormalTok{ a}
\NormalTok{processPrim (}\DataTypeTok{Prim}\NormalTok{ c x) }\FunctionTok{=} \KeywordTok{do}
\NormalTok{    d}\FunctionTok{:}\NormalTok{ds }\OtherTok{<-}\NormalTok{ get}
\NormalTok{    guard (c }\FunctionTok{==}\NormalTok{ d)}
\NormalTok{    put ds}
    \FunctionTok{pure}\NormalTok{ x}
\end{Highlighting}
\end{Shaded}

This lets us interpret a \texttt{Prim} as a \texttt{StateT\ String\ Maybe}
action where the state is the ``string left to be be processed''. Remember, a
\texttt{Prim\ a} contains the character we want to match on, and the \texttt{a}
value we want it to be interpreted as. To process a \texttt{Prim}, we:

\begin{enumerate}
\def\labelenumi{\arabic{enumi}.}
\tightlist
\item
  Get the state's (the string left to be parsed) head and tail, using
  \texttt{get}. If this match fails, backtrack.
\item
  Using \texttt{guard}, backtrack unless the head matches what the \texttt{Prim}
  expects.
\item
  Set the state to be the original tail, using \texttt{put}. This is because we
  parsed the head already, so now the ``string left to be parsed'' is just the
  original tail.
\item
  The result is what the \texttt{Prim} says it should be.
\end{enumerate}

We can use this to write a function that matches the \texttt{RegExp} on a
prefix. We need to run the state action (using \texttt{evalStateT}) on the
string we want to parse:

\begin{Shaded}
\begin{Highlighting}[]
\CommentTok{-- source: https://github.com/mstksg/inCode/tree/master/code-samples/misc/regexp.hs#L58-L59}

\OtherTok{matchPrefix ::} \DataTypeTok{RegExp}\NormalTok{ a }\OtherTok{->} \DataTypeTok{String} \OtherTok{->} \DataTypeTok{Maybe}\NormalTok{ a}
\NormalTok{matchPrefix re }\FunctionTok{=}\NormalTok{ evalStateT (runAlt processPrim re)}
\end{Highlighting}
\end{Shaded}

And that's it! Our first solution:

\begin{Shaded}
\begin{Highlighting}[]
\NormalTok{ghci}\FunctionTok{>}\NormalTok{ matchPrefix testRegexp_ }\StringTok{"acdcdcde"}
\DataTypeTok{Just}\NormalTok{ ()}
\NormalTok{ghci}\FunctionTok{>}\NormalTok{ matchPrefix testRegexp_ }\StringTok{"acdcdcdx"}
\DataTypeTok{Nothing}
\NormalTok{ghci}\FunctionTok{>}\NormalTok{ matchPrefix testRegexp }\StringTok{"acdcdcde"}
\DataTypeTok{Just} \DecValTok{3}
\NormalTok{ghci}\FunctionTok{>}\NormalTok{ matchPrefix testRegexp }\StringTok{"bcdcdcdcdcdcdcde"}
\DataTypeTok{Just} \DecValTok{7}
\end{Highlighting}
\end{Shaded}

\hypertarget{wait-what-just-happened}{%
\subsubsection{Wait, what just happened?}\label{wait-what-just-happened}}

Okay, so that might have happened a little quicker than you expected. One minute
we were writing our primitive, and the next we had already finished. Here's the
entirety of the code, in a few lines of Haskell:

\begin{Shaded}
\begin{Highlighting}[]
\KeywordTok{type} \DataTypeTok{RegExp} \FunctionTok{=} \DataTypeTok{Alt} \DataTypeTok{Prim}

\OtherTok{matchPrefix ::} \DataTypeTok{RegExp}\NormalTok{ a }\OtherTok{->} \DataTypeTok{String} \OtherTok{->} \DataTypeTok{Maybe}\NormalTok{ a}
\NormalTok{matchPrefix re }\FunctionTok{=}\NormalTok{ evalStateT (runAlt processPrim re)}
  \KeywordTok{where}
\NormalTok{    processPrim (}\DataTypeTok{Prim}\NormalTok{ c x) }\FunctionTok{=} \KeywordTok{do}
\NormalTok{      d}\FunctionTok{:}\NormalTok{ds }\OtherTok{<-}\NormalTok{ get}
\NormalTok{      guard (c }\FunctionTok{==}\NormalTok{ d)}
\NormalTok{      put ds}
      \FunctionTok{pure}\NormalTok{ x}
\end{Highlighting}
\end{Shaded}

And now we have a fully functioning regexp parser? What happened?

From a high-level view, remember that \texttt{Alt\ Prim} has, in its structure,
\texttt{pure}, \texttt{empty}, \texttt{Prim},
\texttt{\textless{}*\textgreater{}},
\texttt{\textless{}\textbar{}\textgreater{}}, and \texttt{many}\footnote{A
  caveat exists here for \texttt{many}. More on this later!}.

Essentially, what \texttt{runAlt} does is that it uses a given concrete
\texttt{Alternative} (here, \texttt{StateT\ String\ Maybe}) to get the behavior
of \texttt{pure}, \texttt{empty}, \texttt{\textless{}*\textgreater{}},
\texttt{\textless{}\textbar{}\textgreater{}}, and \texttt{many}. But! As we can
see from that list, \texttt{StateT} does \emph{not} have a built-in behavior for
\texttt{Prim}. And so, that's where \texttt{processPrim} comes in.

\begin{itemize}
\tightlist
\item
  For \texttt{Prim}, \texttt{runAlt} uses \texttt{processPrim}.
\item
  For \texttt{pure}, \texttt{empty}, \texttt{\textless{}*\textgreater{}},
  \texttt{\textless{}\textbar{}\textgreater{}}, and \texttt{many},
  \texttt{runAlt} uses \texttt{StateT\ String\ Maybe}'s \texttt{Alternative}
  instance.
\end{itemize}

So, really, 83\% of the work was done for us by \texttt{StateT}'s
\texttt{Alternative} instance, and the other 17\% is in \texttt{processPrim}.

Admittedly, this \emph{does} feel a little disappointing, or at least
anticlimactic. This makes us wonder: why even use \texttt{Alt} in the first
place? Why not just have \texttt{type\ RegExp\ =\ StateT\ String\ Maybe} and
write an appropriate
\texttt{char\ ::\ Char\ -\textgreater{}\ StateT\ String\ Maybe\ Char}? If
\texttt{StateT} does all of the work anyway, why even bother with \texttt{Alt},
the free Alternative?

One major advantage we get from using \texttt{Alt} is that \texttt{StateT}
is\ldots{}pretty powerful. It's actually \emph{stupid} powerful. It can
represent a lot of things\ldots{}most troubling, it can represent things that
\emph{are not regular expressions}. For example, something as simple as
\texttt{put\ "hello"\ ::\ StateT\ String\ Maybe\ ()} does not correspond to
\emph{any} regular expression.

So, while we can say that \texttt{Alt\ Prim} corresponds to ``regular
expressions, nothing less and nothing more'', we \emph{cannot} say the same
about \texttt{StateT\ String\ Maybe}.

\texttt{Alt\ Prim} contains a ``perfect fit'' representation of a regular
expression data type. Everything it can express is a regular expression, and
there is nothing it can express that \emph{isn't} a regular
expression.\footnote{Note that there are some caveats that should be noted here,
  due to laziness in Haskell. We will go deeper into this later.}

Here, we can think of \texttt{StateT} as the context that we use to
\emph{interpret} a \texttt{RegExp} as a \emph{parser}. But, there might be
\emph{other} ways we want to work with a \texttt{RegExp}. For example, we might
want to inspect it and ``print'' it out for inspection. This is something we
can't do with \texttt{StateT}.

We can't say that \texttt{StateT\ String\ Maybe} ``is'' a regular expression ---
only that it can represent a parser based on a regular expression. But we
\emph{can} say that about \texttt{Alt\ Prim}.

\hypertarget{using-the-free-structure-directly}{%
\subsection{Using the Free structure
directly}\label{using-the-free-structure-directly}}

Alright, that's great and all. But what if we didn't want to offload 83\% of the
behavior to a type that has already been written for us. Is there a way we can
directly use the structure of \texttt{Alt} itself to write our parser?

This is analogous to asking, what if we wanted to actually write a function on a
list (by pattern matching on \texttt{:} and \texttt{{[}{]}}) instead of always
using \texttt{foldMap}? Can we directly operate on the structure of the list
instead of using \texttt{foldMap} with some concrete monoid?

I'm glad you asked! Let's look at the definition of the free alternative:

\begin{Shaded}
\begin{Highlighting}[]
\KeywordTok{newtype} \DataTypeTok{Alt}\NormalTok{ f a }\FunctionTok{=} \DataTypeTok{Alt}\NormalTok{ \{}\OtherTok{ alternatives ::}\NormalTok{ [}\DataTypeTok{AltF}\NormalTok{ f a] \}}

\KeywordTok{data} \DataTypeTok{AltF}\NormalTok{ f a }\FunctionTok{=} \KeywordTok{forall}\NormalTok{ r}\FunctionTok{.} \DataTypeTok{Ap}\NormalTok{ (f r) (}\DataTypeTok{Alt}\NormalTok{ f (r }\OtherTok{->}\NormalTok{ a))}
              \FunctionTok{|}           \DataTypeTok{Pure}\NormalTok{ a}
\end{Highlighting}
\end{Shaded}

It's a mutually recursive type, so it might be a little confusing. One way to
understand \texttt{Alt} is that \texttt{Alt\ xs} contains a \emph{list of
alternatives}, or a list of \texttt{\textless{}\textbar{}\textgreater{}}s. And
each of those alternatives is an \texttt{AltF}, which is a \emph{sequence of
\texttt{f\ a}s} (as a chain of function applications).

You can essentially think of \texttt{AltF\ f\ a} as a linked list
\texttt{{[}f\ r{]}}, except with a different \texttt{r} for each item.
\texttt{Ap} is cons (\texttt{:}), containing the \texttt{f\ r}, and
\texttt{Pure} is nil (\texttt{{[}{]}}). The \texttt{forall\ r.} here is
\emph{-XExistentialQuantification}, and is what lets us have a different
intermediate type for each item in our chain.

All in all, \texttt{Alt\ f} is like a list (\texttt{Alt} list) of lists
(\texttt{AltF} chains), which take turn alternating between alternative lists
and application sequences. A list of chains of lists of chains of lists of
chains \ldots{}

In the big picture, you can think of \texttt{Alt\ f} as a ``normalized'' form of
successive or nested \texttt{\textless{}*\textgreater{}} and
\texttt{\textless{}\textbar{}\textgreater{}}s, similar to how \texttt{{[}a{]}}
is a ``normalized'' form of successive \texttt{\textless{}\textgreater{}}s.

Ultimately we want to write a
\texttt{RegExp\ a\ -\textgreater{}\ String\ -\textgreater{}\ Maybe\ a}, which
parses a string based on a \texttt{RegExp}. We can do this using the most basic
of all Haskell function-writing techniques: pattern matching on each case, and
handling each case.

First, the top-level \texttt{Alt} case. When faced with a list of chains, we can
try to parse each one. The result is the first success.

\begin{Shaded}
\begin{Highlighting}[]
\CommentTok{-- source: https://github.com/mstksg/inCode/tree/master/code-samples/misc/regexp.hs#L61-L62}

\OtherTok{matchAlts ::} \DataTypeTok{RegExp}\NormalTok{ a }\OtherTok{->} \DataTypeTok{String} \OtherTok{->} \DataTypeTok{Maybe}\NormalTok{ a}
\NormalTok{matchAlts (}\DataTypeTok{Alt}\NormalTok{ ls) xs }\FunctionTok{=}\NormalTok{ asum [ matchChain l xs }\FunctionTok{|}\NormalTok{ l }\OtherTok{<-}\NormalTok{ ls  ]}
\end{Highlighting}
\end{Shaded}

Here, \texttt{asum\ ::\ {[}Maybe\ a{]}\ -\textgreater{}\ Maybe\ a} finds the
first \texttt{Just} (success) in a list of attempts.

Now, we need to handle the chain case. To do this, we can pattern match on each
constructor, and handle each case.

\begin{Shaded}
\begin{Highlighting}[]
\OtherTok{matchChain ::} \DataTypeTok{AltF} \DataTypeTok{Prim}\NormalTok{ a }\OtherTok{->} \DataTypeTok{String} \OtherTok{->} \DataTypeTok{Maybe}\NormalTok{ a}
\NormalTok{matchChain (}\DataTypeTok{Ap}\NormalTok{ (}\DataTypeTok{Prim}\NormalTok{ c x) next) cs }\FunctionTok{=}\NormalTok{ _}
\NormalTok{matchChain (}\DataTypeTok{Pure}\NormalTok{ x)             cs }\FunctionTok{=}\NormalTok{ _}
\end{Highlighting}
\end{Shaded}

From here, it's mostly ``type tetris''! We just continually ask GHC what goes in
what holes (and what types need to change) until we get something that
typechecks.

In the end of the very mechanical process, we get:

\begin{Shaded}
\begin{Highlighting}[]
\CommentTok{-- source: https://github.com/mstksg/inCode/tree/master/code-samples/misc/regexp.hs#L64-L69}

\OtherTok{matchChain ::} \DataTypeTok{AltF} \DataTypeTok{Prim}\NormalTok{ a }\OtherTok{->} \DataTypeTok{String} \OtherTok{->} \DataTypeTok{Maybe}\NormalTok{ a}
\NormalTok{matchChain (}\DataTypeTok{Ap}\NormalTok{ (}\DataTypeTok{Prim}\NormalTok{ c x) next) cs }\FunctionTok{=} \KeywordTok{case}\NormalTok{ cs }\KeywordTok{of}
\NormalTok{    []  }\OtherTok{->} \DataTypeTok{Nothing}
\NormalTok{    d}\FunctionTok{:}\NormalTok{ds }\FunctionTok{|}\NormalTok{ c }\FunctionTok{==}\NormalTok{ d    }\OtherTok{->}\NormalTok{ matchAlts ((}\FunctionTok{$}\NormalTok{ x) }\FunctionTok{<$>}\NormalTok{ next) ds}
         \FunctionTok{|} \FunctionTok{otherwise} \OtherTok{->} \DataTypeTok{Nothing}
\NormalTok{matchChain (}\DataTypeTok{Pure}\NormalTok{ x)             _      }\FunctionTok{=} \DataTypeTok{Just}\NormalTok{ x}
\end{Highlighting}
\end{Shaded}

\begin{enumerate}
\def\labelenumi{\arabic{enumi}.}
\item
  If it's \texttt{Ap} (analogous to cons, \texttt{:}), it means we're in the
  middle of the chain.

  \begin{itemize}
  \tightlist
  \item
    If the input string is empty, then we fail to match.
  \item
    Otherwise, here's the interesting thing. We have the \texttt{Prim\ r} with
    the character we want to match, the first letter in the string, and
    \texttt{next\ ::\ RegExp\ (r\ -\textgreater{}\ a)}, the next \texttt{RegExp}
    in our sequential parsing chain.

    \begin{itemize}
    \tightlist
    \item
      If the match is a success, we continue down the chain, to \texttt{next}.
      We just need to massage the types a bit to make it all work out.
    \item
      Otherwise, it's a failure. We're done here.
    \end{itemize}
  \end{itemize}
\item
  If it's \texttt{Pure\ x} (analogous to nil, \texttt{{[}{]}}), it means we're
  at the end of the chain. We return the result in \texttt{Just}.
\end{enumerate}

In the end though, you don't really need to understand any of this \emph{too}
deeply in order to write this. Sure, it's nice to understand what \texttt{Ap},
\texttt{Pure}, \texttt{AltF}, etc. really ``mean''. But, we don't have to ---
the types take care of all of it for you :)

That should be good enough to implement another prefix parser:

\begin{Shaded}
\begin{Highlighting}[]
\NormalTok{ghci}\FunctionTok{>}\NormalTok{ matchAlts testRegexp_ }\StringTok{"acdcdcde"}
\DataTypeTok{Just}\NormalTok{ ()}
\NormalTok{ghci}\FunctionTok{>}\NormalTok{ matchAlts testRegexp_ }\StringTok{"acdcdcdx"}
\DataTypeTok{Nothing}
\NormalTok{ghci}\FunctionTok{>}\NormalTok{ matchAlts testRegexp }\StringTok{"acdcdcde"}
\DataTypeTok{Just} \DecValTok{3}
\NormalTok{ghci}\FunctionTok{>}\NormalTok{ matchAlts testRegexp }\StringTok{"bcdcdcdcdcdcdcde"}
\DataTypeTok{Just} \DecValTok{7}
\end{Highlighting}
\end{Shaded}

\hypertarget{what-did-we-do}{%
\subsection{What did we do?}\label{what-did-we-do}}

The two attempts here can be compared to using lists via \texttt{foldMap}
vs.~using lists via pattern matching.

Because lists act as a free monoid, \emph{any list function} can be written
using \texttt{foldMap} and not directly pattern matching. If this seems
unbelievable to you, try finding a function that can't --- you might be
surprised!

However, because lists are an algebraic data type, \emph{any list function} can
be written using straight up pattern matching on \texttt{:} and \texttt{{[}{]}}.

One nice thing about list is that, no matter how you assemble it, it always ends
up as a series of conses and nil. We say that the free monoid is
\emph{normalizing}. That is,
\texttt{{[}1,2,3{]}\ \textless{}\textgreater{}\ {[}4{]}} has the same
representation as
\texttt{{[}1{]}\ \textless{}\textgreater{}\ {[}2,3{]}\ \textless{}\textgreater{}\ {[}4{]}.}
When we pattern match on \texttt{:} and \texttt{{[}{]}}, we can't distinguish
between those two original methods of creation.

\texttt{Alt} is normalizing as well. An example of a possible variant that is
\emph{not} normalizing is:

\begin{Shaded}
\begin{Highlighting}[]
\KeywordTok{data} \DataTypeTok{RegExp}\NormalTok{ a }\FunctionTok{=} \DataTypeTok{Empty}
              \FunctionTok{|} \DataTypeTok{Pure}\NormalTok{ a}
              \FunctionTok{|} \DataTypeTok{Prim} \DataTypeTok{Char}\NormalTok{ a}
              \FunctionTok{|} \KeywordTok{forall}\NormalTok{ r}\FunctionTok{.} \DataTypeTok{Seq}\NormalTok{ (}\DataTypeTok{RegExp}\NormalTok{ r) (}\DataTypeTok{RegExp}\NormalTok{ (r }\OtherTok{->}\NormalTok{ a))}
              \FunctionTok{|} \DataTypeTok{Union}\NormalTok{ (}\DataTypeTok{RegExp}\NormalTok{ a) (}\DataTypeTok{RegExp}\NormalTok{ a)}
              \FunctionTok{|} \DataTypeTok{Many}\NormalTok{ (}\DataTypeTok{RegExp}\NormalTok{ a)}
\end{Highlighting}
\end{Shaded}

This is how we \emph{might} have written \texttt{RegExp}, if we didn't know
about the free alternative. However, this representation is not normalizing,
because we have two \texttt{RegExp\ a} values that represent the same regexp:

\begin{Shaded}
\begin{Highlighting}[]
\CommentTok{-- | a|(b|c)}
\OtherTok{abc1 ::} \DataTypeTok{RegExp} \DataTypeTok{Int}
\NormalTok{abc1 }\FunctionTok{=} \DataTypeTok{Prim} \CharTok{'a'} \DecValTok{1} \OtherTok{`Union`}\NormalTok{ (}\DataTypeTok{Prim} \CharTok{'b'} \DecValTok{2} \OtherTok{`Union`} \DataTypeTok{Prim} \CharTok{'c'} \DecValTok{3}\NormalTok{)}

\CommentTok{-- | (a|b)|c}
\OtherTok{abc2 ::} \DataTypeTok{RegExp} \DataTypeTok{Int}
\NormalTok{abc2 }\FunctionTok{=}\NormalTok{ (}\DataTypeTok{Prim} \CharTok{'a'} \DecValTok{1} \OtherTok{`Union`} \DataTypeTok{Prim} \CharTok{'b'} \DecValTok{2}\NormalTok{) }\OtherTok{`Union`} \DataTypeTok{Prim} \CharTok{'c'} \DecValTok{3}
\end{Highlighting}
\end{Shaded}

These two match the same thing. But they have different representations. This
representation \emph{not} normalizing, since the same regexp can be expressed in
two different ways.

\texttt{Alt\ Prim} is better because if two regexps are the same\ldots{}they
will correspond to the same \texttt{Alt\ Prim}. It forces each value to exist in
a ``canonical'' normalized form.

This means that when we eventually write our parsing function
\texttt{matchAlts}, \emph{we aren't allowed to care} about ``how'' the regexps
are made. We aren't \emph{allowed} to distinguish between
\texttt{(a\textbar{}b)\textbar{}c} and \texttt{a\textbar{}(b\textbar{}c)}. The
normalizing property of \texttt{Alt} means that we are forced to treat both of
those the \emph{exact} same way. It \emph{forces} us to obey the laws about the
structures of regular languages.

It's easy to imagine a bug that might occur if we accidentally treated
\texttt{(a\textbar{}b)\textbar{}c} differently than
\texttt{a\textbar{}(b\textbar{}c)} --- and indeed it sounds like an easy bug to
accidentally make if we are using the non-normalizing representation.

Using \texttt{Alt} instead of rolling our own regexp expression not only
enforces our integrity, but it also eliminates a huge class of potential bugs.

However, it should be noted that, while \texttt{Alt\ f} is strongly normalizing
with respect to Alternative structure, \texttt{Alt\ Prim} isn't strongly
normalizing with respect to \emph{regular expressions} structure in every single
case. For example, \texttt{Alt\ Prim} will still treat \texttt{a\textbar{}a} as
different from \texttt{a}. This is mostly because \texttt{Alt} has to be
``agnostic'' to \texttt{f}. But, like with all structural ``type safety'', I
always follow this rule of thumb: \emph{A lot of safety is better than no
safety}. This method can't eliminate \emph{all} bugs arising from this angle,
but it can eliminate a whoooole lot.

\begin{center}\rule{0.5\linewidth}{\linethickness}\end{center}

\textbf{Aside}

Before we wrap things up, let's take some time to clarify a subtle point. Feel
free to skip this whole section if you don't care about the fact that these
aren't identical to the mathematical formalism of regular languages.

While we can \emph{use} \texttt{RegExp} just like a regular expression, the
formal concept of regular expressions is actually slightly different, due to one
pesky thing: \textbf{laziness}.

We really \emph{shouldn't} be too surprised, since laziness actually throws a
wrench into a \emph{lot} of Haskell abstractions that are based on math. For
example, laziness is the reason that lists aren't ``true'' mathematical free
monoids.

The reason is that because of laziness and unbounded recursion, we can create an
``infinite'' regular language:
\texttt{a\textbar{}aa\textbar{}aaa\textbar{}aaaa\textbar{}aaaaa\textbar{}aaaaaa\textbar{}...},
forever and ever. However, infinite regular expressions aren't allowed in the
mathematical version. Unfortunately, in Haskell, there is no way to ``turn off''
recursion: we're stuck with it.

Even more unfortunately, this is actually how the \texttt{Alt} encoding of the
free alternative above implements \texttt{many}. \texttt{a*} is implemented as
\texttt{\textbar{}a\textbar{}aa\textbar{}aaa\textbar{}aaaa\textbar{}aaaaa\textbar{}...},
infinitely. So the representation actually \emph{relies} on laziness and
infinite recursion to do its job.

For the purposes we talked about in this post, this doesn't matter. However,
this does create serious issues if we want to write a
\href{https://en.wikipedia.org/wiki/Nondeterministic_finite_automaton}{non-deterministic
finite automata based parser} (NFA), which is the de facto standard for
implementing fast regexp parsers. We can only really compile to an NFA if we
have a \emph{finite} value, which means anything using \texttt{many} is out of
the question.

However, not all hope is lost. We can actually use the \emph{final encoding} of
\texttt{Alt}, from \emph{Control.Alternative.Free.Final}, to gain a
\texttt{many} that is non-recursive.\footnote{As of now, this actually doesn't
  work. But it will when my \href{https://github.com/ekmett/free/pull/188}{pull
  request} goes through :)}

Using the final encoding means we loose the ``pattern match'' method, and can
only use the \texttt{runAlt} method. However, we can off-load to
\texttt{Alternative} instances that have non-recursive \texttt{many} (like the
\texttt{RE} type from \emph{regex-applicative}) that allows us to generate an
NFA parser.

\begin{center}\rule{0.5\linewidth}{\linethickness}\end{center}

\hypertarget{wrapping-up}{%
\section{Wrapping Up}\label{wrapping-up}}

As a cherry on top, we can write our final function to find matches anywhere
inside a string by using \texttt{tails} (which gives us all prefixes in a
string) and \texttt{mapMaybe} (which maps \texttt{matchPrefix} on every prefix
and keeps the successes)

\begin{Shaded}
\begin{Highlighting}[]
\CommentTok{-- source: https://github.com/mstksg/inCode/tree/master/code-samples/misc/regexp.hs#L71-L72}

\OtherTok{matches ::} \DataTypeTok{RegExp}\NormalTok{ a }\OtherTok{->} \DataTypeTok{String} \OtherTok{->}\NormalTok{ [a]}
\NormalTok{matches l }\FunctionTok{=}\NormalTok{ mapMaybe (matchPrefix l) }\FunctionTok{.}\NormalTok{ tails}
\end{Highlighting}
\end{Shaded}

Hopefully from this, you can see the value of free structures :)

\begin{itemize}
\tightlist
\item
  Given some base primitive, they give you exactly the structure you need --- no
  more, no less.
\item
  They let you work with your type safely, and then unsafely ``run'' it inside
  different useful contexts.
\item
  They are normalizing, so you are not allowed to make ``illegal'' distinctions.
  This eliminates a class of bugs where you accidentally treat two cases
  differently when they are meant to be treated the same way.
\end{itemize}

Our journey from going to regular languages to \texttt{Alt\ Prim} was to
recognize that the structures involved in regular expressions matched an
\texttt{Alternative} interface ``plus'' some extra primitives --- and then
shifting our perspective to enriching a primitive with \texttt{Alternative}
energy.

Where can we go from here? First, try playing around with the
\href{https://github.com/mstksg/inCode/tree/master/code-samples/misc/regexp.hs}{sample
code}. One easy addition would be to add other types of primitives:

\begin{Shaded}
\begin{Highlighting}[]
\KeywordTok{data} \DataTypeTok{Prim}\NormalTok{ a }\FunctionTok{=} \DataTypeTok{Prim} \DataTypeTok{Char}\NormalTok{ a}
            \FunctionTok{|} \DataTypeTok{Letter}\NormalTok{ a}
            \FunctionTok{|} \DataTypeTok{Wildcard}\NormalTok{ a}
\end{Highlighting}
\end{Shaded}

so we can support a lot of the basic character classes that many implementations
of regular expressions support. Try this out in the
\href{https://github.com/mstksg/inCode/tree/master/code-samples/misc/regexp.hs}{sample
code} as an exercise!

I hope that after working through this example, you will begin to start
recognizing opportunities for using free structures everywhere you look! Once
you start, it's hard to stop :)

\hypertarget{signoff}{%
\section{Signoff}\label{signoff}}

Hi, thanks for reading! You can reach me via email at
\href{mailto:justin@jle.im}{\nolinkurl{justin@jle.im}}, or at twitter at
\href{https://twitter.com/mstk}{@mstk}! This post and all others are published
under the \href{https://creativecommons.org/licenses/by-nc-nd/3.0/}{CC-BY-NC-ND
3.0} license. Corrections and edits via pull request are welcome and encouraged
at \href{https://github.com/mstksg/inCode}{the source repository}.

If you feel inclined, or this post was particularly helpful for you, why not
consider \href{https://www.patreon.com/justinle/overview}{supporting me on
Patreon}, or a \href{bitcoin:3D7rmAYgbDnp4gp4rf22THsGt74fNucPDU}{BTC donation}?
:)

\end{document}
