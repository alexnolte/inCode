\documentclass[]{article}
\usepackage{lmodern}
\usepackage{amssymb,amsmath}
\usepackage{ifxetex,ifluatex}
\usepackage{fixltx2e} % provides \textsubscript
\ifnum 0\ifxetex 1\fi\ifluatex 1\fi=0 % if pdftex
  \usepackage[T1]{fontenc}
  \usepackage[utf8]{inputenc}
\else % if luatex or xelatex
  \ifxetex
    \usepackage{mathspec}
    \usepackage{xltxtra,xunicode}
  \else
    \usepackage{fontspec}
  \fi
  \defaultfontfeatures{Mapping=tex-text,Scale=MatchLowercase}
  \newcommand{\euro}{€}
\fi
% use upquote if available, for straight quotes in verbatim environments
\IfFileExists{upquote.sty}{\usepackage{upquote}}{}
% use microtype if available
\IfFileExists{microtype.sty}{\usepackage{microtype}}{}
\usepackage[margin=1in]{geometry}
\usepackage{color}
\usepackage{fancyvrb}
\newcommand{\VerbBar}{|}
\newcommand{\VERB}{\Verb[commandchars=\\\{\}]}
\DefineVerbatimEnvironment{Highlighting}{Verbatim}{commandchars=\\\{\}}
% Add ',fontsize=\small' for more characters per line
\newenvironment{Shaded}{}{}
\newcommand{\AlertTok}[1]{\textcolor[rgb]{1.00,0.00,0.00}{\textbf{#1}}}
\newcommand{\AnnotationTok}[1]{\textcolor[rgb]{0.38,0.63,0.69}{\textbf{\textit{#1}}}}
\newcommand{\AttributeTok}[1]{\textcolor[rgb]{0.49,0.56,0.16}{#1}}
\newcommand{\BaseNTok}[1]{\textcolor[rgb]{0.25,0.63,0.44}{#1}}
\newcommand{\BuiltInTok}[1]{#1}
\newcommand{\CharTok}[1]{\textcolor[rgb]{0.25,0.44,0.63}{#1}}
\newcommand{\CommentTok}[1]{\textcolor[rgb]{0.38,0.63,0.69}{\textit{#1}}}
\newcommand{\CommentVarTok}[1]{\textcolor[rgb]{0.38,0.63,0.69}{\textbf{\textit{#1}}}}
\newcommand{\ConstantTok}[1]{\textcolor[rgb]{0.53,0.00,0.00}{#1}}
\newcommand{\ControlFlowTok}[1]{\textcolor[rgb]{0.00,0.44,0.13}{\textbf{#1}}}
\newcommand{\DataTypeTok}[1]{\textcolor[rgb]{0.56,0.13,0.00}{#1}}
\newcommand{\DecValTok}[1]{\textcolor[rgb]{0.25,0.63,0.44}{#1}}
\newcommand{\DocumentationTok}[1]{\textcolor[rgb]{0.73,0.13,0.13}{\textit{#1}}}
\newcommand{\ErrorTok}[1]{\textcolor[rgb]{1.00,0.00,0.00}{\textbf{#1}}}
\newcommand{\ExtensionTok}[1]{#1}
\newcommand{\FloatTok}[1]{\textcolor[rgb]{0.25,0.63,0.44}{#1}}
\newcommand{\FunctionTok}[1]{\textcolor[rgb]{0.02,0.16,0.49}{#1}}
\newcommand{\ImportTok}[1]{#1}
\newcommand{\InformationTok}[1]{\textcolor[rgb]{0.38,0.63,0.69}{\textbf{\textit{#1}}}}
\newcommand{\KeywordTok}[1]{\textcolor[rgb]{0.00,0.44,0.13}{\textbf{#1}}}
\newcommand{\NormalTok}[1]{#1}
\newcommand{\OperatorTok}[1]{\textcolor[rgb]{0.40,0.40,0.40}{#1}}
\newcommand{\OtherTok}[1]{\textcolor[rgb]{0.00,0.44,0.13}{#1}}
\newcommand{\PreprocessorTok}[1]{\textcolor[rgb]{0.74,0.48,0.00}{#1}}
\newcommand{\RegionMarkerTok}[1]{#1}
\newcommand{\SpecialCharTok}[1]{\textcolor[rgb]{0.25,0.44,0.63}{#1}}
\newcommand{\SpecialStringTok}[1]{\textcolor[rgb]{0.73,0.40,0.53}{#1}}
\newcommand{\StringTok}[1]{\textcolor[rgb]{0.25,0.44,0.63}{#1}}
\newcommand{\VariableTok}[1]{\textcolor[rgb]{0.10,0.09,0.49}{#1}}
\newcommand{\VerbatimStringTok}[1]{\textcolor[rgb]{0.25,0.44,0.63}{#1}}
\newcommand{\WarningTok}[1]{\textcolor[rgb]{0.38,0.63,0.69}{\textbf{\textit{#1}}}}
\ifxetex
  \usepackage[setpagesize=false, % page size defined by xetex
              unicode=false, % unicode breaks when used with xetex
              xetex]{hyperref}
\else
  \usepackage[unicode=true]{hyperref}
\fi
\hypersetup{breaklinks=true,
            bookmarks=true,
            pdfauthor={Justin Le},
            pdftitle={Type-safe Tic Tac Toe (Part 1)},
            colorlinks=true,
            citecolor=blue,
            urlcolor=blue,
            linkcolor=magenta,
            pdfborder={0 0 0}}
\urlstyle{same}  % don't use monospace font for urls
% Make links footnotes instead of hotlinks:
\renewcommand{\href}[2]{#2\footnote{\url{#1}}}
\setlength{\parindent}{0pt}
\setlength{\parskip}{6pt plus 2pt minus 1pt}
\setlength{\emergencystretch}{3em}  % prevent overfull lines
\setcounter{secnumdepth}{0}

\title{Type-safe Tic Tac Toe (Part 1)}
\author{Justin Le}

\begin{document}
\maketitle

\emph{Originally posted on
\textbf{\href{https://blog.jle.im/entry/type-safe-tic-tac-toe-1.html}{in
Code}}.}

One problem with adoption of dependent types in everyday programming, I think,
is that most examples out there are sort of small and self-contained. There
aren't \emph{too} many larger-scale examples out there showing how dependent
types can permeate your whole program to make everything more robust and
error-free.

So, this series will be implementing a type-safe \emph{tic tac toe} game (a
medium-scale Haskell app) that can be played both on the console (using
Haskeline) and in the browser (using Miso), using some custom built AI. We will:

\begin{enumerate}
\def\labelenumi{\arabic{enumi}.}
\tightlist
\item
  Build up our core game engine, talking about what it really means to be type
  safe
\item
  Use our type-safe engine to build type-safe controllers (AI, GUI)
\end{enumerate}

This series will also be a mini-tutorial on the
\emph{\href{https://hackage.haskell.org/package/decidable}{decidable}} package
that I just recently released :) We will also be heavily using the
\emph{\href{https://hackage.haskell.org/package/singletons}{singletons}}
library. Where relevant, I will explain singletons concepts in brief. If you
want a more in-depth introduction to the \emph{singletons} library, however,
check out my
\href{https://blog.jle.im/entries/series/+introduction-to-singletons.html}{Introduction
to Singletons} series!

\hypertarget{type-safety}{%
\section{Type-Safety}\label{type-safety}}

First off, we should ask the question: what does it mean to be type-safe?

?????

\hypertarget{the-specification}{%
\section{The Specification}\label{the-specification}}

We're going to create a type that represents a \emph{valid} game state. The goal
is to make a GADT where you can only construct values whose types represent
\emph{valid} game states. If we have a value of this type, then we know that the
game state must be valid.

A good way to start with this is by thinking of \emph{induction rules} for
defining a valid state.

We'll say that there are two parts of a game state:

\begin{enumerate}
\def\labelenumi{\arabic{enumi}.}
\tightlist
\item
  The current board
\item
  The current player
\end{enumerate}

and that there are two ways of ``constructing'' a valid state:

\begin{enumerate}
\def\labelenumi{\arabic{enumi}.}
\item
  The empty board with player X is a valid state.
\item
  If we have:

  \begin{itemize}
  \tightlist
  \item
    A valid state with board \emph{b} and current player \emph{p}
  \item
    The game is still in play
  \item
    We can add a valid move by player \emph{p} to board \emph{b}
  \end{itemize}

  Then the result of this move represents a new valid board \emph{b}, with
  swapped player \emph{p}.
\end{enumerate}

This is a denotative way to talk about what it means for a state to be valid.

Note that our ``type safety'' is only as strong as the specification we just
wrote. Type safety using dependent types isn't omnipotent, and it can't read
your mind. However, there is a nice assurance that as long as your
\emph{specification} is right, your program will work as expected. And hey, it's
a step up from the untyped case, where you can have a specification wrong, but
implement it incorrectly. With ``type-safety'', you cut out one huge area where
bugs come from: the implementation.

Alright, let's do this!

\hypertarget{valid-state}{%
\section{Valid State}\label{valid-state}}

First, we'll define the types we need to specify our state:

\begin{Shaded}
\begin{Highlighting}[]
\FunctionTok{$}\NormalTok{(singletons [d|}
\NormalTok{-- source: https://github.com/mstksg/inCode/tree/master/code-samples/ttt/Part1.hs#L37-L51}

\NormalTok{  data Piece = PX | PO}
\NormalTok{    deriving (Eq, Ord)}
  
\NormalTok{  type Board = [[Maybe Piece]]}

\NormalTok{  emptyBoard :: Board}
\NormalTok{  emptyBoard = [[Nothing, Nothing, Nothing]}
\NormalTok{               ,[Nothing, Nothing, Nothing]}
\NormalTok{               ,[Nothing, Nothing, Nothing]}
\NormalTok{               ]}

\NormalTok{  altP :: Piece -> Piece}
\NormalTok{  altP PX = PO}
\NormalTok{  altP PO = PX}
\NormalTok{  |])}

  \KeywordTok{type} \DataTypeTok{Board} \FunctionTok{=}\NormalTok{ [[}\DataTypeTok{Maybe} \DataTypeTok{Piece}\NormalTok{]]}

\OtherTok{  emptyBoard ::} \DataTypeTok{Board}
\NormalTok{  emptyBoard }\FunctionTok{=}\NormalTok{ [[}\DataTypeTok{Nothing}\NormalTok{, }\DataTypeTok{Nothing}\NormalTok{, }\DataTypeTok{Nothing}\NormalTok{]}
\NormalTok{               ,[}\DataTypeTok{Nothing}\NormalTok{, }\DataTypeTok{Nothing}\NormalTok{, }\DataTypeTok{Nothing}\NormalTok{]}
\NormalTok{               ,[}\DataTypeTok{Nothing}\NormalTok{, }\DataTypeTok{Nothing}\NormalTok{, }\DataTypeTok{Nothing}\NormalTok{]}
\NormalTok{               ]}

\OtherTok{  altP ::} \DataTypeTok{Piece} \OtherTok{->} \DataTypeTok{Piece}
\NormalTok{  altP }\DataTypeTok{PX} \FunctionTok{=} \DataTypeTok{PO}
\NormalTok{  altP }\DataTypeTok{PO} \FunctionTok{=} \DataTypeTok{PX}
  \FunctionTok{|}\NormalTok{])}
  \FunctionTok{|}\NormalTok{])}
\end{Highlighting}
\end{Shaded}

A \texttt{Piece} will also represent our player -- either \texttt{PX} or
\texttt{PO}. Our \texttt{Board} will be a list of lists of
\texttt{Maybe\ Piece}. If the spot contains \texttt{Nothing}, the spot is
unplayed; if the spot is \texttt{Just\ p}, then it means the spot has been
played by \texttt{p}.

And some values and functions we need to talk about empty boards and state
transformations:

\begin{Shaded}
\begin{Highlighting}[]
\FunctionTok{$}\NormalTok{(singletons [d|}
\NormalTok{-- source: https://github.com/mstksg/inCode/tree/master/code-samples/ttt/Part1.hs#L42-L51}

\NormalTok{  emptyBoard :: Board}
\NormalTok{  emptyBoard = [[Nothing, Nothing, Nothing]}
\NormalTok{               ,[Nothing, Nothing, Nothing]}
\NormalTok{               ,[Nothing, Nothing, Nothing]}
\NormalTok{               ]}

\NormalTok{  altP :: Piece -> Piece}
\NormalTok{  altP PX = PO}
\NormalTok{  altP PO = PX}
\NormalTok{  |])}

\OtherTok{  altP ::} \DataTypeTok{Piece} \OtherTok{->} \DataTypeTok{Piece}
\NormalTok{  altP }\DataTypeTok{PX} \FunctionTok{=} \DataTypeTok{PO}
\NormalTok{  altP }\DataTypeTok{PO} \FunctionTok{=} \DataTypeTok{PX}
  \FunctionTok{|}\NormalTok{])}
  \FunctionTok{|}\NormalTok{])}
\end{Highlighting}
\end{Shaded}

Let's just throw in a quick proof as a sanity check:

\begin{Shaded}
\begin{Highlighting}[]
\CommentTok{-- source: https://github.com/mstksg/inCode/tree/master/code-samples/ttt/Part1.hs#L59-L61}

\OtherTok{altP_cyclic ::} \DataTypeTok{Sing}\NormalTok{ p }\OtherTok{->} \DataTypeTok{AltP}\NormalTok{ (}\DataTypeTok{AltP}\NormalTok{ p) }\FunctionTok{:~:}\NormalTok{ p}
\NormalTok{altP_cyclic }\DataTypeTok{SPX} \FunctionTok{=} \DataTypeTok{Refl} \FunctionTok{@}\NormalTok{'}\DataTypeTok{PX}
\NormalTok{altP_cyclic }\DataTypeTok{SPO} \FunctionTok{=} \DataTypeTok{Refl} \FunctionTok{@}\NormalTok{'}\DataTypeTok{PO}
\end{Highlighting}
\end{Shaded}

With that in mind, we can write our valid state constructor. We'll do that with
two helper types that we will implement later. First, we'll use the
\href{https://hackage.haskell.org/package/decidable}{decidable} library to talk
about the kind of a \emph{type-level predicate}.

\begin{Shaded}
\begin{Highlighting}[]
\KeywordTok{data} \DataTypeTok{InPlay}\OtherTok{ ::} \DataTypeTok{Predicate} \DataTypeTok{Board}
\end{Highlighting}
\end{Shaded}

\texttt{InPlay} is a predicate that a given board is in-play; a value of type
\texttt{InPlay\ @@\ b} is a witness or proof that a board is in play.

We also need to define a type for a valid update by a given player onto a given
board:

\begin{Shaded}
\begin{Highlighting}[]
\KeywordTok{data} \DataTypeTok{Update}\OtherTok{ ::} \DataTypeTok{Piece} \OtherTok{->} \DataTypeTok{Board} \OtherTok{->} \DataTypeTok{Board} \OtherTok{->} \DataTypeTok{Type}
\end{Highlighting}
\end{Shaded}

A value of type \texttt{Update\ p\ b1\ b2} will represent a valid update to
board \texttt{b1} by player \texttt{p} to create a board \texttt{b2}.

And finally, our valid state constructor:

\begin{Shaded}
\begin{Highlighting}[]
\KeywordTok{data} \DataTypeTok{GameState}\OtherTok{ ::} \DataTypeTok{Piece} \OtherTok{->} \DataTypeTok{Board} \OtherTok{->} \DataTypeTok{Type} \KeywordTok{where}
    \CommentTok{-- | The empty board is a valid state}
    \DataTypeTok{GSStart}
\OtherTok{        ::} \DataTypeTok{GameState}\NormalTok{ '}\DataTypeTok{PX} \DataTypeTok{EmptyBoard}
    \CommentTok{-- | We can also construct a valid game state if we have:}
    \DataTypeTok{GSUpdate}
\OtherTok{        ::}\NormalTok{ forall p b1 b2}\FunctionTok{.}\NormalTok{ ()}
        \OtherTok{=>} \DataTypeTok{InPlay}          \FunctionTok{@@}\NormalTok{ b1     }\CommentTok{-- ^ a proof that b1 is in play}
        \OtherTok{->} \DataTypeTok{Update}\NormalTok{    p        b1 b2  }\CommentTok{-- ^ a valid update}
        \OtherTok{->} \DataTypeTok{GameState}\NormalTok{ p        b1     }\CommentTok{-- ^ a proof that p, b1 are a valid state}
        \CommentTok{-- ---------------------------- then}
        \OtherTok{->} \DataTypeTok{GameState}\NormalTok{ (}\DataTypeTok{AltP}\NormalTok{ p)    b2  }\CommentTok{-- ^ AltP p, b2 is a valid satte}
\end{Highlighting}
\end{Shaded}

And that's it --- a verified-correct representation of a game state, directly
transcribed from our plain-language denotative specification.

Now we just need to talk about \texttt{InPlay} and \texttt{Update}. In
particular, we need:

\begin{enumerate}
\def\labelenumi{\arabic{enumi}.}
\tightlist
\item
  A definition of \texttt{Update}, and a way to turn user-input into a valid
  \texttt{Update} (or reject it if it isn't valid).
\item
  A definition of \texttt{InPlay}, and a way to decide whether or not a given
  board \texttt{b} is \texttt{InPlay}. This is something that the appropriately
  named \emph{\href{https://hackage.haskell.org/package/decidable}{decidable}}
  library will help us with.
\end{enumerate}

\hypertarget{update}{%
\section{Update}\label{update}}

Let's go about what thinking about what defines a valid update. Remember, the
kind we wanted was:

\begin{Shaded}
\begin{Highlighting}[]
\KeywordTok{data} \DataTypeTok{Update}\OtherTok{ ::} \DataTypeTok{Piece} \OtherTok{->} \DataTypeTok{Board} \OtherTok{->} \DataTypeTok{Board} \OtherTok{->} \DataTypeTok{Type}
\end{Highlighting}
\end{Shaded}

An \texttt{Update\ p\ b1\ b2} will be a valid update of \texttt{b1} by player
\texttt{p} to produce \texttt{b2}. So, we need to:

\begin{enumerate}
\def\labelenumi{\arabic{enumi}.}
\tightlist
\item
  Produce \texttt{b2} from \texttt{b1}
\item
  Be sure that the move is valid --- namely, that it is placed in a clean spot
  so that it doesn't overwrite any previous moves.
\end{enumerate}

Producing \texttt{b2} from \texttt{b1} is simple enough as a type family. In
fact, we can just use the
\emph{\href{https://hackage.haskell.org/package/lens-typelevel}{lens-typelevel}}
library to update our nested list:

\begin{Shaded}
\begin{Highlighting}[]
\FunctionTok{$}\NormalTok{(singletonsOnly [d|}
\NormalTok{  placeBoard :: N -> N -> Piece -> Board -> Board}
\NormalTok{  placeBoard i j p = set (ixList i . ixList j) (Just p)}
\NormalTok{  |])}
\end{Highlighting}
\end{Shaded}

This is just lenses --- \texttt{set\ l\ x} is a function that sets the field
specified by \texttt{l} to \texttt{x}. Here, we set the jth item of the ith list
to be \texttt{Just\ p}. That means we can now produce \texttt{b2} from
\texttt{b1} -- it's just \texttt{PlaceBoard\ i\ j\ p\ b1}.

Here, \texttt{N} is the peano nat type (a lot of libraries define it, but it's
also defined as a uility in \emph{lens-typelevel}). It's essentially
\texttt{{[}(){]}} (which makes it useful as an index type), or:

\begin{Shaded}
\begin{Highlighting}[]
\KeywordTok{data} \DataTypeTok{N} \FunctionTok{=} \DataTypeTok{Z} \FunctionTok{|} \DataTypeTok{S} \DataTypeTok{N}
\end{Highlighting}
\end{Shaded}

A natural number is either zero, or the successor of another natural number.
\texttt{S\ (S\ Z)}, for example, would represent 2.

The trickier part is making sure that the spot at \emph{(i, j)} isn't already
taken. For that, we'll introduce a common helper type to say \emph{what} the
piece at spot \emph{(i, j)} is:

\begin{Shaded}
\begin{Highlighting}[]
\KeywordTok{data} \DataTypeTok{Coord}\OtherTok{ ::}\NormalTok{ (}\DataTypeTok{N}\NormalTok{, }\DataTypeTok{N}\NormalTok{) }\OtherTok{->}\NormalTok{ [[k]] }\OtherTok{->}\NormalTok{ k }\OtherTok{->} \DataTypeTok{Type}
\end{Highlighting}
\end{Shaded}

A \texttt{Coord\ \textquotesingle{}(i,\ j)\ xss\ x} is a data type that
specifies that the jth item in the ith list in \texttt{b} is \texttt{p}.

And we require \texttt{Update} to only be constructable if the spot at \emph{(i,
j)} is \texttt{Nothing}:

\begin{Shaded}
\begin{Highlighting}[]
\KeywordTok{data} \DataTypeTok{Update}\OtherTok{ ::} \DataTypeTok{Piece} \OtherTok{->} \DataTypeTok{Board} \OtherTok{->} \DataTypeTok{Board} \OtherTok{->} \DataTypeTok{Type} \KeywordTok{where}
    \DataTypeTok{MkUpdate}\OtherTok{ ::}\NormalTok{ forall i j p b}\FunctionTok{.}\NormalTok{ ()}
             \OtherTok{=>} \DataTypeTok{Coord}\NormalTok{ '(i, j) b '}\DataTypeTok{Nothing}
             \OtherTok{->} \DataTypeTok{Update}\NormalTok{ p b (}\DataTypeTok{PlaceBoard}\NormalTok{ i j p b)}
\end{Highlighting}
\end{Shaded}

\texttt{Update} is now defined so that, for \texttt{Update\ p\ b1\ b2},
\texttt{b2} is the update via placement of a piece \texttt{p} at some position
in \texttt{b1}, where the placement does not overwrite a previous piece. Note
that our \texttt{MkUpdate} constructor only has four ``free'' variables,
\texttt{i}, \texttt{j}, \texttt{p}, and \texttt{b}. If we use \texttt{MkUpdate},
it means that the ``final board'' is fully determined from only \texttt{i},
\texttt{j}, \texttt{p}, and \texttt{b}.

\hypertarget{coord}{%
\subsection{Coord}\label{coord}}

Now we need to define \texttt{Coord}. We're going to do that in terms of a
simpler type that is essentially the same for normal lists --- a type:

\begin{Shaded}
\begin{Highlighting}[]
\KeywordTok{data} \DataTypeTok{Sel}\OtherTok{ ::} \DataTypeTok{N} \OtherTok{->}\NormalTok{ [k] }\OtherTok{->}\NormalTok{ k }\OtherTok{->} \DataTypeTok{Type}
\end{Highlighting}
\end{Shaded}

A value of type \texttt{Sel\ n\ xs\ x} says that the nth item in \texttt{xs} is
\texttt{x}.

We can define this type inductively, similar to the common
\href{http://hackage.haskell.org/package/type-combinators-0.2.4.3/docs/Data-Type-Index.html}{\texttt{Index}}
data type. We can mention our induction rules:

\begin{enumerate}
\def\labelenumi{\arabic{enumi}.}
\tightlist
\item
  The first item in a list as at index 0 (\texttt{Z})
\item
  If an item is at index \texttt{n} in list \texttt{as}, then it is also at
  index \texttt{S\ n} in list \texttt{b\ \textquotesingle{}:\ as}.
\end{enumerate}

\begin{Shaded}
\begin{Highlighting}[]
\KeywordTok{data} \DataTypeTok{Sel}\OtherTok{ ::} \DataTypeTok{N} \OtherTok{->}\NormalTok{ [k] }\OtherTok{->}\NormalTok{ k }\OtherTok{->} \DataTypeTok{Type} \KeywordTok{where}
    \DataTypeTok{SelZ}\OtherTok{ ::} \DataTypeTok{Sel}\NormalTok{ '}\DataTypeTok{Z}\NormalTok{ (a '}\FunctionTok{:}\NormalTok{ as) a}
    \DataTypeTok{SelS}\OtherTok{ ::} \DataTypeTok{Sel}\NormalTok{ n  as        a }\OtherTok{->} \DataTypeTok{Sel}\NormalTok{ ('}\DataTypeTok{S}\NormalTok{ n) (b '}\FunctionTok{:}\NormalTok{ bs) a}
\end{Highlighting}
\end{Shaded}

For example, for the type-level list \texttt{\textquotesingle{}{[}10,5,2,8{]}},
we can make values:

\begin{Shaded}
\begin{Highlighting}[]
\DataTypeTok{SelZ}\OtherTok{             ::} \DataTypeTok{Sel}\NormalTok{         '}\DataTypeTok{Z}\NormalTok{   '[}\DecValTok{10}\NormalTok{,}\DecValTok{5}\NormalTok{,}\DecValTok{2}\NormalTok{,}\DecValTok{8}\NormalTok{] }\DecValTok{10}
\DataTypeTok{SelS} \DataTypeTok{SelZ}\OtherTok{        ::} \DataTypeTok{Sel}\NormalTok{     ('}\DataTypeTok{S}\NormalTok{ '}\DataTypeTok{Z}\NormalTok{)  '[}\DecValTok{10}\NormalTok{,}\DecValTok{5}\NormalTok{,}\DecValTok{2}\NormalTok{,}\DecValTok{8}\NormalTok{] }\DecValTok{5}
\DataTypeTok{SelS}\NormalTok{ (}\DataTypeTok{SelS} \DataTypeTok{SelZ}\NormalTok{)}\OtherTok{ ::} \DataTypeTok{Sel}\NormalTok{ ('}\DataTypeTok{S}\NormalTok{ ('}\DataTypeTok{S}\NormalTok{ '}\DataTypeTok{Z}\NormalTok{)) '[}\DecValTok{10}\NormalTok{,}\DecValTok{5}\NormalTok{,}\DecValTok{2}\NormalTok{,}\DecValTok{8}\NormalTok{] }\DecValTok{2}
\end{Highlighting}
\end{Shaded}

etc.

We can then use this to define \texttt{Coord}:

\begin{Shaded}
\begin{Highlighting}[]
\KeywordTok{data} \DataTypeTok{Coord}\OtherTok{ ::}\NormalTok{ (}\DataTypeTok{N}\NormalTok{, }\DataTypeTok{N}\NormalTok{) }\OtherTok{->}\NormalTok{ [[k]] }\OtherTok{->}\NormalTok{ k }\OtherTok{->} \DataTypeTok{Type} \KeywordTok{where}
\OtherTok{    (:$:) ::}\NormalTok{ forall i j rows row piece}\FunctionTok{.}\NormalTok{ ()}
          \OtherTok{=>} \DataTypeTok{Sel}\NormalTok{ i rows row}
          \OtherTok{->} \DataTypeTok{Sel}\NormalTok{ j row  piece}
          \OtherTok{->} \DataTypeTok{Coord}\NormalTok{ '(i, j) rows piece}
\end{Highlighting}
\end{Shaded}

A \texttt{Coord\ \textquotesingle{}(i,\ j)\ rows\ piece} contains a selection
into the ith list in \texttt{rows}, to get \texttt{row}, and a selection into
the jth item in \texttt{row}, to get \texttt{piece}.

\hypertarget{trying-it-out}{%
\subsection{Trying it out}\label{trying-it-out}}

That's it! Let's see if we can generate some sensible \texttt{Update}s, and
maybe even play a sample game.

We'll start with the \texttt{EmptyBoard}, and let's add a piece by \texttt{PX}
at the middle spot, index (1,1). This means we want
\texttt{SelS\ SelZ\ :\$:\ SelS\ SelZ} (a \texttt{Coord} with two indexes into
spots 1 and 1) applied to \texttt{MkUpdate}. We'll use \emph{-XTypeApplications}
to specify the type variables \texttt{p} and \texttt{b}:

\begin{Shaded}
\begin{Highlighting}[]
\NormalTok{ghci}\FunctionTok{>} \FunctionTok{:}\NormalTok{t }\DataTypeTok{MkUpdate} \FunctionTok{@}\NormalTok{_ }\FunctionTok{@}\NormalTok{_ }\FunctionTok{@}\NormalTok{'}\DataTypeTok{PX} \FunctionTok{@}\DataTypeTok{EmptyBoard}\NormalTok{ (}\DataTypeTok{SelS} \DataTypeTok{SelZ} \FunctionTok{:$:} \DataTypeTok{SelS} \DataTypeTok{SelZ}\NormalTok{)}
\DataTypeTok{Update}
\NormalTok{  '}\DataTypeTok{PX}
\NormalTok{  '[ '[ '}\DataTypeTok{Nothing}\NormalTok{, '}\DataTypeTok{Nothing}\NormalTok{ , '}\DataTypeTok{Nothing}\NormalTok{],}
\NormalTok{     '[ '}\DataTypeTok{Nothing}\NormalTok{, '}\DataTypeTok{Nothing}\NormalTok{ , '}\DataTypeTok{Nothing}\NormalTok{],}
\NormalTok{     '[ '}\DataTypeTok{Nothing}\NormalTok{, '}\DataTypeTok{Nothing}\NormalTok{ , '}\DataTypeTok{Nothing}\NormalTok{]}
\NormalTok{   ]}
\NormalTok{  '[ '[ '}\DataTypeTok{Nothing}\NormalTok{, '}\DataTypeTok{Nothing}\NormalTok{ , '}\DataTypeTok{Nothing}\NormalTok{],}
\NormalTok{     '[ '}\DataTypeTok{Nothing}\NormalTok{, '}\DataTypeTok{Just}\NormalTok{ '}\DataTypeTok{PX}\NormalTok{, '}\DataTypeTok{Nothing}\NormalTok{],}
\NormalTok{     '[ '}\DataTypeTok{Nothing}\NormalTok{, '}\DataTypeTok{Nothing}\NormalTok{ , '}\DataTypeTok{Nothing}\NormalTok{]}
\NormalTok{  ]}
\end{Highlighting}
\end{Shaded}

Nice! This update produces exactly he board expected.

Let's see if we can see if this prevents us from creating an illegal board.
We'll take the result board and see if we can place a \texttt{PO} piece there:

\begin{Shaded}
\begin{Highlighting}[]
\NormalTok{ghci}\FunctionTok{>} \KeywordTok{let} \DataTypeTok{NewBoard} \FunctionTok{=}\NormalTok{ '[ '[ '}\DataTypeTok{Nothing}\NormalTok{, '}\DataTypeTok{Nothing}\NormalTok{ , '}\DataTypeTok{Nothing}\NormalTok{ ]}
\NormalTok{                      , '[ '}\DataTypeTok{Nothing}\NormalTok{, '}\DataTypeTok{Just}\NormalTok{ '}\DataTypeTok{PX}\NormalTok{, '}\DataTypeTok{Nothing}\NormalTok{ ]}
\NormalTok{                      , '[ '}\DataTypeTok{Nothing}\NormalTok{, '}\DataTypeTok{Nothing}\NormalTok{ , '}\DataTypeTok{Nothing}\NormalTok{ ]}
\NormalTok{                      ]}
\NormalTok{ghci}\FunctionTok{>} \FunctionTok{:}\NormalTok{k }\DataTypeTok{MkUpdate} \FunctionTok{@}\NormalTok{_ }\FunctionTok{@}\NormalTok{_ }\FunctionTok{@}\NormalTok{'}\DataTypeTok{PO} \FunctionTok{@}\DataTypeTok{NewBoard}\NormalTok{ (}\DataTypeTok{SelS} \DataTypeTok{SelZ} \FunctionTok{:$:} \DataTypeTok{SelS} \DataTypeTok{SelZ}\NormalTok{)}
\NormalTok{    • }\DataTypeTok{Couldn't}\NormalTok{ match }\KeywordTok{type}\NormalTok{ ‘'}\DataTypeTok{Nothing}\NormalTok{’ with ‘'}\DataTypeTok{Just}\NormalTok{ '}\DataTypeTok{PX}\NormalTok{’}
\end{Highlighting}
\end{Shaded}

Right! That's because \texttt{SelS\ SelZ\ :\&:\ SelS\ SellZ}, applied to
\texttt{NewBoard}, gives
\texttt{Coord\ \textquotesingle{}(\textquotesingle{}S\ \textquotesingle{}Z,\ \textquotesingle{}S\ \textquotesingle{}Z)\ NewBoard\ (\textquotesingle{}Just\ \textquotesingle{}PX)}.
However, in order to be used with \texttt{MkUpdate}, the final field has to be
\texttt{\textquotesingle{}Nothing}, not
\texttt{\textquotesingle{}Just\ \textquotesingle{}PX}. So, type error.

We can also use this with \texttt{GameState}, using \texttt{undefined} as a
placeholder for our \texttt{InPlay} witness, since we haven't defined it yet:

\begin{Shaded}
\begin{Highlighting}[]
\NormalTok{ghci}\FunctionTok{>} \FunctionTok{:}\NormalTok{t }\DataTypeTok{GSStart}
\DataTypeTok{GameState}
\NormalTok{    '}\DataTypeTok{PX}
\NormalTok{    '[ '[ '}\DataTypeTok{Nothing}\NormalTok{, '}\DataTypeTok{Nothing}\NormalTok{, '}\DataTypeTok{Nothing}\NormalTok{]}
\NormalTok{     , '[ '}\DataTypeTok{Nothing}\NormalTok{, '}\DataTypeTok{Nothing}\NormalTok{, '}\DataTypeTok{Nothing}\NormalTok{]}
\NormalTok{     , '[ '}\DataTypeTok{Nothing}\NormalTok{, '}\DataTypeTok{Nothing}\NormalTok{, '}\DataTypeTok{Nothing}\NormalTok{]}
\NormalTok{     ]}

\NormalTok{ghci}\FunctionTok{>} \KeywordTok{let}\NormalTok{ s1 }\FunctionTok{=} \DataTypeTok{GSUpdate}\NormalTok{ undefined (}\DataTypeTok{MkUpdate}\NormalTok{ (}\DataTypeTok{SelS} \DataTypeTok{SelZ} \FunctionTok{:$:} \DataTypeTok{SelS} \DataTypeTok{SelZ}\NormalTok{))}
             \FunctionTok{$} \DataTypeTok{GSStart}
\NormalTok{ghci}\FunctionTok{>} \FunctionTok{:}\NormalTok{t s1}
\DataTypeTok{GameState}
\NormalTok{    '}\DataTypeTok{PO}         \CommentTok{-- Player switches correclty!}
\NormalTok{    '[ '[ '}\DataTypeTok{Nothing}\NormalTok{, '}\DataTypeTok{Nothing}\NormalTok{ , '}\DataTypeTok{Nothing}\NormalTok{]}
\NormalTok{     , '[ '}\DataTypeTok{Nothing}\NormalTok{, '}\DataTypeTok{Just}\NormalTok{ '}\DataTypeTok{PX}\NormalTok{, '}\DataTypeTok{Nothing}\NormalTok{]}
\NormalTok{     , '[ '}\DataTypeTok{Nothing}\NormalTok{, '}\DataTypeTok{Nothing}\NormalTok{ , '}\DataTypeTok{Nothing}\NormalTok{]}
\NormalTok{     ]}

\NormalTok{ghci}\FunctionTok{>} \KeywordTok{let}\NormalTok{ s2 }\FunctionTok{=} \DataTypeTok{GSUpdate}\NormalTok{ undefined (}\DataTypeTok{MkUpdate}\NormalTok{ (}\DataTypeTok{SelZ} \FunctionTok{:$:} \DataTypeTok{SelZ}\NormalTok{))}
             \FunctionTok{$}\NormalTok{ s1}
\NormalTok{ghci}\FunctionTok{>} \FunctionTok{:}\NormalTok{t s2}
\DataTypeTok{GameState}
\NormalTok{    '}\DataTypeTok{PX}
\NormalTok{    '[ '[ '}\DataTypeTok{Just}\NormalTok{ '}\DataTypeTok{PO}\NormalTok{, '}\DataTypeTok{Nothing}\NormalTok{ , '}\DataTypeTok{Nothing}\NormalTok{]}
\NormalTok{     , '[ '}\DataTypeTok{Nothing}\NormalTok{ , '}\DataTypeTok{Just}\NormalTok{ '}\DataTypeTok{PX}\NormalTok{, '}\DataTypeTok{Nothing}\NormalTok{]}
\NormalTok{     , '[ '}\DataTypeTok{Nothing}\NormalTok{ , '}\DataTypeTok{Nothing}\NormalTok{ , '}\DataTypeTok{Nothing}\NormalTok{]}
\NormalTok{     ]}
\NormalTok{ghci}\FunctionTok{>} \KeywordTok{let}\NormalTok{ s3 }\FunctionTok{=} \DataTypeTok{GSUpdate}\NormalTok{ undefined (}\DataTypeTok{MkUpdate}\NormalTok{ (}\DataTypeTok{SelZ} \FunctionTok{:$:} \DataTypeTok{SelZ}\NormalTok{))}
             \FunctionTok{$}\NormalTok{ s2}
\NormalTok{    • }\DataTypeTok{Couldn't}\NormalTok{ match }\KeywordTok{type}\NormalTok{ ‘'}\DataTypeTok{Nothing}\NormalTok{’ with ‘'}\DataTypeTok{Just}\NormalTok{ '}\DataTypeTok{PO}\NormalTok{’}
\end{Highlighting}
\end{Shaded}

We see that \texttt{GSUpdate} appropriately alternates the player each turn,
keeps track of our updated boards, and also cannot be constructed if we give an
illegal move (that is, if we try to place a piece where a piece has already been
placed). Note that type inference allows us to not have to manually specify the
board at each point in time.

\hypertarget{creating-updates-from-user-input}{%
\subsection{Creating Updates from User
Input}\label{creating-updates-from-user-input}}

Yay!

\hypertarget{signoff}{%
\section{Signoff}\label{signoff}}

Hi, thanks for reading! You can reach me via email at
\href{mailto:justin@jle.im}{\nolinkurl{justin@jle.im}}, or at twitter at
\href{https://twitter.com/mstk}{@mstk}! This post and all others are published
under the \href{https://creativecommons.org/licenses/by-nc-nd/3.0/}{CC-BY-NC-ND
3.0} license. Corrections and edits via pull request are welcome and encouraged
at \href{https://github.com/mstksg/inCode}{the source repository}.

If you feel inclined, or this post was particularly helpful for you, why not
consider \href{https://www.patreon.com/justinle/overview}{supporting me on
Patreon}, or a \href{bitcoin:3D7rmAYgbDnp4gp4rf22THsGt74fNucPDU}{BTC donation}?
:)

\end{document}
