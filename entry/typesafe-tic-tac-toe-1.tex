\documentclass[]{article}
\usepackage{lmodern}
\usepackage{amssymb,amsmath}
\usepackage{ifxetex,ifluatex}
\usepackage{fixltx2e} % provides \textsubscript
\ifnum 0\ifxetex 1\fi\ifluatex 1\fi=0 % if pdftex
  \usepackage[T1]{fontenc}
  \usepackage[utf8]{inputenc}
\else % if luatex or xelatex
  \ifxetex
    \usepackage{mathspec}
    \usepackage{xltxtra,xunicode}
  \else
    \usepackage{fontspec}
  \fi
  \defaultfontfeatures{Mapping=tex-text,Scale=MatchLowercase}
  \newcommand{\euro}{€}
\fi
% use upquote if available, for straight quotes in verbatim environments
\IfFileExists{upquote.sty}{\usepackage{upquote}}{}
% use microtype if available
\IfFileExists{microtype.sty}{\usepackage{microtype}}{}
\usepackage[margin=1in]{geometry}
\usepackage{color}
\usepackage{fancyvrb}
\newcommand{\VerbBar}{|}
\newcommand{\VERB}{\Verb[commandchars=\\\{\}]}
\DefineVerbatimEnvironment{Highlighting}{Verbatim}{commandchars=\\\{\}}
% Add ',fontsize=\small' for more characters per line
\newenvironment{Shaded}{}{}
\newcommand{\AlertTok}[1]{\textcolor[rgb]{1.00,0.00,0.00}{\textbf{#1}}}
\newcommand{\AnnotationTok}[1]{\textcolor[rgb]{0.38,0.63,0.69}{\textbf{\textit{#1}}}}
\newcommand{\AttributeTok}[1]{\textcolor[rgb]{0.49,0.56,0.16}{#1}}
\newcommand{\BaseNTok}[1]{\textcolor[rgb]{0.25,0.63,0.44}{#1}}
\newcommand{\BuiltInTok}[1]{#1}
\newcommand{\CharTok}[1]{\textcolor[rgb]{0.25,0.44,0.63}{#1}}
\newcommand{\CommentTok}[1]{\textcolor[rgb]{0.38,0.63,0.69}{\textit{#1}}}
\newcommand{\CommentVarTok}[1]{\textcolor[rgb]{0.38,0.63,0.69}{\textbf{\textit{#1}}}}
\newcommand{\ConstantTok}[1]{\textcolor[rgb]{0.53,0.00,0.00}{#1}}
\newcommand{\ControlFlowTok}[1]{\textcolor[rgb]{0.00,0.44,0.13}{\textbf{#1}}}
\newcommand{\DataTypeTok}[1]{\textcolor[rgb]{0.56,0.13,0.00}{#1}}
\newcommand{\DecValTok}[1]{\textcolor[rgb]{0.25,0.63,0.44}{#1}}
\newcommand{\DocumentationTok}[1]{\textcolor[rgb]{0.73,0.13,0.13}{\textit{#1}}}
\newcommand{\ErrorTok}[1]{\textcolor[rgb]{1.00,0.00,0.00}{\textbf{#1}}}
\newcommand{\ExtensionTok}[1]{#1}
\newcommand{\FloatTok}[1]{\textcolor[rgb]{0.25,0.63,0.44}{#1}}
\newcommand{\FunctionTok}[1]{\textcolor[rgb]{0.02,0.16,0.49}{#1}}
\newcommand{\ImportTok}[1]{#1}
\newcommand{\InformationTok}[1]{\textcolor[rgb]{0.38,0.63,0.69}{\textbf{\textit{#1}}}}
\newcommand{\KeywordTok}[1]{\textcolor[rgb]{0.00,0.44,0.13}{\textbf{#1}}}
\newcommand{\NormalTok}[1]{#1}
\newcommand{\OperatorTok}[1]{\textcolor[rgb]{0.40,0.40,0.40}{#1}}
\newcommand{\OtherTok}[1]{\textcolor[rgb]{0.00,0.44,0.13}{#1}}
\newcommand{\PreprocessorTok}[1]{\textcolor[rgb]{0.74,0.48,0.00}{#1}}
\newcommand{\RegionMarkerTok}[1]{#1}
\newcommand{\SpecialCharTok}[1]{\textcolor[rgb]{0.25,0.44,0.63}{#1}}
\newcommand{\SpecialStringTok}[1]{\textcolor[rgb]{0.73,0.40,0.53}{#1}}
\newcommand{\StringTok}[1]{\textcolor[rgb]{0.25,0.44,0.63}{#1}}
\newcommand{\VariableTok}[1]{\textcolor[rgb]{0.10,0.09,0.49}{#1}}
\newcommand{\VerbatimStringTok}[1]{\textcolor[rgb]{0.25,0.44,0.63}{#1}}
\newcommand{\WarningTok}[1]{\textcolor[rgb]{0.38,0.63,0.69}{\textbf{\textit{#1}}}}
\ifxetex
  \usepackage[setpagesize=false, % page size defined by xetex
              unicode=false, % unicode breaks when used with xetex
              xetex]{hyperref}
\else
  \usepackage[unicode=true]{hyperref}
\fi
\hypersetup{breaklinks=true,
            bookmarks=true,
            pdfauthor={Justin Le},
            pdftitle={Type-safe Tic Tac Toe (Part 1)},
            colorlinks=true,
            citecolor=blue,
            urlcolor=blue,
            linkcolor=magenta,
            pdfborder={0 0 0}}
\urlstyle{same}  % don't use monospace font for urls
% Make links footnotes instead of hotlinks:
\renewcommand{\href}[2]{#2\footnote{\url{#1}}}
\setlength{\parindent}{0pt}
\setlength{\parskip}{6pt plus 2pt minus 1pt}
\setlength{\emergencystretch}{3em}  % prevent overfull lines
\setcounter{secnumdepth}{0}

\title{Type-safe Tic Tac Toe (Part 1)}
\author{Justin Le}

\begin{document}
\maketitle

\emph{Originally posted on
\textbf{\href{https://blog.jle.im/entry/typesafe-tic-tac-toe-1.html}{in Code}}.}

One problem with adoption of dependent types in everyday programming, I think,
is that most examples out there are sort of small and self-contained. There
aren't \emph{too} many larger-scale examples out there showing how dependent
types can permeate your whole program to make everything more robust and
error-free.

So, this series will be implementing a type-safe \emph{tic tac toe} game (a
medium-scale Haskell app) that can be played both on the console (using
Haskeline) and in the browser (using Miso), using some custom built AI. We will:

\begin{enumerate}
\def\labelenumi{\arabic{enumi}.}
\tightlist
\item
  Build up our core game engine, talking about what it really means to be type
  safe
\item
  Use our type-safe engine to build type-safe controllers (AI, GUI)
\end{enumerate}

This series will also be a mini-tutorial on the
\emph{\href{https://hackage.haskell.org/package/decidable}{decidable}} package
that I just recently released :) We will also be heavily using the
\emph{\href{https://hackage.haskell.org/package/singletons}{singletons}}
library. I strongly strongly recommend reading my
\href{https://blog.jle.im/entries/series/+introduction-to-singletons.html}{Introduction
to Singletons} series (and doing the exercises), if you are new to the
singletons library. However, I will do my best to explain singletons concepts in
brief as they come up.

\hypertarget{type-safety}{%
\section{Type-Safety}\label{type-safety}}

First off, we should ask the question: what does it mean to be type-safe?

?????

\hypertarget{the-specification}{%
\section{The Specification}\label{the-specification}}

We're going to create a type that represents a \emph{valid} game state. The goal
is to make a GADT where you can only construct values whose types represent
\emph{valid} game states. If we have a value of this type, then we know that the
game state must be valid.

A good way to start with this is by thinking of \emph{induction rules} for
defining a valid state.

We'll say that there are two parts of a game state:

\begin{enumerate}
\def\labelenumi{\arabic{enumi}.}
\tightlist
\item
  The current board
\item
  The current player
\end{enumerate}

and that there are two ways of ``constructing'' a valid state:

\begin{enumerate}
\def\labelenumi{\arabic{enumi}.}
\item
  The empty board with player X is a valid state.
\item
  If we have:

  \begin{itemize}
  \tightlist
  \item
    A valid state with board \emph{b} and current player \emph{p}
  \item
    The game is still in play
  \item
    We can add a valid move by player \emph{p} to board \emph{b}
  \end{itemize}

  Then the result of this move represents a new valid board \emph{b}, with
  swapped player \emph{p}.
\end{enumerate}

This is a denotative way to talk about what it means for a state to be valid.

Note that our ``type safety'' is only as strong as the specification we just
wrote. Type safety using dependent types isn't omnipotent, and it can't read
your mind. However, there is a nice assurance that as long as your
\emph{specification} is right, your program will work as expected. And hey, it's
a step up from the untyped case, where you can have a specification wrong, but
implement it incorrectly. With ``type-safety'', you cut out one huge area where
bugs come from: the implementation.

Alright, let's do this!

\hypertarget{valid-state}{%
\section{Valid State}\label{valid-state}}

First, we'll define the types we need to specify our state:

\begin{Shaded}
\begin{Highlighting}[]
\FunctionTok{$}\NormalTok{(singletons [d|}
\NormalTok{  data Piece = PX | PO}
\NormalTok{    deriving (Eq, Ord)}

\NormalTok{  type Board = [[Maybe Piece]]}
\NormalTok{  |])}
\end{Highlighting}
\end{Shaded}

A \texttt{Piece} will also represent our player -- either \texttt{PX} or
\texttt{PO}. Our \texttt{Board} will be a list of lists of
\texttt{Maybe\ Piece}. If the spot contains \texttt{Nothing}, the spot is
unplayed; if the spot is \texttt{Just\ p}, then it means the spot has been
played by \texttt{p}.

And some values and functions we need to talk about empty boards and state
transformations:

\begin{Shaded}
\begin{Highlighting}[]
\FunctionTok{$}\NormalTok{(singletons [d|}
\NormalTok{  emptyBoard :: Board}
\NormalTok{  emptyBoard = [ [Nothing, Nothing, Nothing]}
\NormalTok{               , [Nothing, Nothing, Nothing]}
\NormalTok{               , [Nothing, Nothing, Nothing]}
\NormalTok{               ]}

\NormalTok{  altP :: Piece -> Piece}
\NormalTok{  altP PX = PO}
\NormalTok{  altP PO = PX}
\NormalTok{  |])}
\end{Highlighting}
\end{Shaded}

Let's just throw in a quick proof as a sanity check:

\begin{Shaded}
\begin{Highlighting}[]
\CommentTok{-- source: https://github.com/mstksg/inCode/tree/master/code-samples/ttt/Part1.hs#L63-L65}

\OtherTok{altP_cyclic ::} \DataTypeTok{Sing}\NormalTok{ p }\OtherTok{->} \DataTypeTok{AltP}\NormalTok{ (}\DataTypeTok{AltP}\NormalTok{ p) }\FunctionTok{:~:}\NormalTok{ p}
\NormalTok{altP_cyclic }\DataTypeTok{SPX} \FunctionTok{=} \DataTypeTok{Refl} \FunctionTok{@}\NormalTok{'}\DataTypeTok{PX}
\NormalTok{altP_cyclic }\DataTypeTok{SPO} \FunctionTok{=} \DataTypeTok{Refl} \FunctionTok{@}\NormalTok{'}\DataTypeTok{PO}
\end{Highlighting}
\end{Shaded}

With that in mind, we can write our valid state constructor. We'll do that with
two helper types that we will implement later. First, we'll use the
\href{https://hackage.haskell.org/package/decidable}{decidable} library to talk
about the kind of a \emph{type-level predicate}.

\begin{Shaded}
\begin{Highlighting}[]
\CommentTok{-- source: https://github.com/mstksg/inCode/tree/master/code-samples/ttt/Part1.hs#L67-L67}

\KeywordTok{data} \DataTypeTok{InPlay}\OtherTok{ ::} \DataTypeTok{Predicate} \DataTypeTok{Board}
\end{Highlighting}
\end{Shaded}

\texttt{InPlay} is a predicate that a given board is in-play; a value of type
\texttt{InPlay\ @@\ b} is a witness or proof that a board is in play.

We also need to define a type for a valid update by a given player onto a given
board:

\begin{Shaded}
\begin{Highlighting}[]
\CommentTok{-- source: https://github.com/mstksg/inCode/tree/master/code-samples/ttt/Part1.hs#L82-L82}

\KeywordTok{data} \DataTypeTok{Update}\OtherTok{ ::} \DataTypeTok{Piece} \OtherTok{->} \DataTypeTok{Board} \OtherTok{->} \DataTypeTok{Board} \OtherTok{->} \DataTypeTok{Type} \KeywordTok{where}
\end{Highlighting}
\end{Shaded}

A value of type \texttt{Update\ p\ b1\ b2} will represent a valid update to
board \texttt{b1} by player \texttt{p} to create a board \texttt{b2}.

And finally, our valid state constructor:

\begin{Shaded}
\begin{Highlighting}[]
\CommentTok{-- source: https://github.com/mstksg/inCode/tree/master/code-samples/ttt/Part1.hs#L69-L80}

\KeywordTok{data} \DataTypeTok{GameState}\OtherTok{ ::} \DataTypeTok{Piece} \OtherTok{->} \DataTypeTok{Board} \OtherTok{->} \DataTypeTok{Type} \KeywordTok{where}
    \CommentTok{-- | The empty board is a valid state}
    \DataTypeTok{GSStart}
\OtherTok{        ::} \DataTypeTok{GameState}\NormalTok{ '}\DataTypeTok{PX} \DataTypeTok{EmptyBoard}
    \CommentTok{-- | We can also construct a valid game state if we have:}
    \DataTypeTok{GSUpdate}
\OtherTok{        ::}\NormalTok{ forall p b1 b2}\FunctionTok{.}\NormalTok{ ()}
        \OtherTok{=>} \DataTypeTok{InPlay}          \FunctionTok{@@}\NormalTok{ b1     }\CommentTok{-- ^ a proof that b1 is in play}
        \OtherTok{->} \DataTypeTok{Update}\NormalTok{    p        b1 b2  }\CommentTok{-- ^ a valid update}
        \OtherTok{->} \DataTypeTok{GameState}\NormalTok{ p        b1     }\CommentTok{-- ^ a proof that p, b1 are a valid state}
        \CommentTok{-- ---------------------------- then}
        \OtherTok{->} \DataTypeTok{GameState}\NormalTok{ (}\DataTypeTok{AltP}\NormalTok{ p)    b2  }\CommentTok{-- ^ `AltP p`, b2 is a valid satte}
\end{Highlighting}
\end{Shaded}

And that's it --- a verified-correct representation of a game state, directly
transcribed from our plain-language denotative specification.

Now we just need to talk about \texttt{InPlay} and \texttt{Update}. In
particular, we need:

\begin{enumerate}
\def\labelenumi{\arabic{enumi}.}
\tightlist
\item
  A definition of \texttt{Update}, and a way to turn user-input into a valid
  \texttt{Update} (or reject it if it isn't valid).
\item
  A definition of \texttt{InPlay}, and a way to decide whether or not a given
  board \texttt{b} is \texttt{InPlay}. This is something that the appropriately
  named \emph{\href{https://hackage.haskell.org/package/decidable}{decidable}}
  library will help us with.
\end{enumerate}

\hypertarget{update}{%
\subsection{Update}\label{update}}

Let's go about what thinking about what defines a valid update. Remember, the
kind we wanted was:

\begin{Shaded}
\begin{Highlighting}[]
\CommentTok{-- source: https://github.com/mstksg/inCode/tree/master/code-samples/ttt/Part1.hs#L82-L82}

\KeywordTok{data} \DataTypeTok{Update}\OtherTok{ ::} \DataTypeTok{Piece} \OtherTok{->} \DataTypeTok{Board} \OtherTok{->} \DataTypeTok{Board} \OtherTok{->} \DataTypeTok{Type} \KeywordTok{where}
\end{Highlighting}
\end{Shaded}

An \texttt{Update\ p\ b1\ b2} will be a valid update of \texttt{b1} by player
\texttt{p} to produce \texttt{b2}. So, we need to:

\begin{enumerate}
\def\labelenumi{\arabic{enumi}.}
\tightlist
\item
  Produce \texttt{b2} from \texttt{b1}
\item
  Be sure that the move is valid --- namely, that it is placed in a clean spot
  so that it doesn't overwrite any previous moves.
\end{enumerate}

Producing \texttt{b2} from \texttt{b1} is simple enough as a type family. In
fact, we can just use the
\emph{\href{https://hackage.haskell.org/package/lens-typelevel}{lens-typelevel}}
library to update our nested list:

\begin{Shaded}
\begin{Highlighting}[]
\FunctionTok{$}\NormalTok{(singletonsOnly [d|}
\NormalTok{  placeBoard :: N -> N -> Piece -> Board -> Board}
\NormalTok{  placeBoard i j p = set (ixList i . ixList j) (Just p)}
\NormalTok{  |])}
\end{Highlighting}
\end{Shaded}

This is just lenses --- \texttt{set\ l\ x} is a function that sets the field
specified by \texttt{l} to \texttt{x}. Here, we set the jth item of the ith list
to be \texttt{Just\ p}. That means we can now produce \texttt{b2} from
\texttt{b1} -- it's just \texttt{PlaceBoard\ i\ j\ p\ b1}.

Here, \texttt{N} is the peano nat type (a lot of libraries define it, but it's
also defined as a uility in \emph{lens-typelevel}). It's essentially
\texttt{{[}(){]}} (which makes it useful as an index type), or:

\begin{Shaded}
\begin{Highlighting}[]
\KeywordTok{data} \DataTypeTok{N} \FunctionTok{=} \DataTypeTok{Z} \FunctionTok{|} \DataTypeTok{S} \DataTypeTok{N}
\end{Highlighting}
\end{Shaded}

A natural number is either zero, or the successor of another natural number.
\texttt{S\ (S\ Z)}, for example, would represent 2.

The trickier part is making sure that the spot at \emph{(i, j)} isn't already
taken. For that, we'll introduce a common helper type to say \emph{what} the
piece at spot \emph{(i, j)} is:

\begin{Shaded}
\begin{Highlighting}[]
\CommentTok{-- source: https://github.com/mstksg/inCode/tree/master/code-samples/ttt/Part1.hs#L96-L96}

\KeywordTok{data} \DataTypeTok{Coord}\OtherTok{ ::}\NormalTok{ (}\DataTypeTok{N}\NormalTok{, }\DataTypeTok{N}\NormalTok{) }\OtherTok{->}\NormalTok{ [[k]] }\OtherTok{->}\NormalTok{ k }\OtherTok{->} \DataTypeTok{Type} \KeywordTok{where}
\end{Highlighting}
\end{Shaded}

A \texttt{Coord\ \textquotesingle{}(i,\ j)\ xss\ x} is a data type that
specifies that the jth item in the ith list in \texttt{b} is \texttt{p}.

And we require \texttt{Update} to only be constructable if the spot at \emph{(i,
j)} is \texttt{Nothing}:

\begin{Shaded}
\begin{Highlighting}[]
\CommentTok{-- source: https://github.com/mstksg/inCode/tree/master/code-samples/ttt/Part1.hs#L82-L87}

\KeywordTok{data} \DataTypeTok{Update}\OtherTok{ ::} \DataTypeTok{Piece} \OtherTok{->} \DataTypeTok{Board} \OtherTok{->} \DataTypeTok{Board} \OtherTok{->} \DataTypeTok{Type} \KeywordTok{where}
    \DataTypeTok{MkUpdate}
\OtherTok{        ::}\NormalTok{ forall i j p b}\FunctionTok{.}\NormalTok{ ()}
        \OtherTok{=>} \DataTypeTok{Coord}\NormalTok{ '(i, j) b '}\DataTypeTok{Nothing}         \CommentTok{-- ^ If the item at (i, j) in b is Nothing}
        \CommentTok{-- ------------------------------------- then}
        \OtherTok{->} \DataTypeTok{Update}\NormalTok{ p b (}\DataTypeTok{PlaceBoard}\NormalTok{ i j p b)  }\CommentTok{-- ^ Placing `Just p` at i, j is a valid update}
\end{Highlighting}
\end{Shaded}

\texttt{Update} is now defined so that, for \texttt{Update\ p\ b1\ b2},
\texttt{b2} is the update via placement of a piece \texttt{p} at some position
in \texttt{b1}, where the placement does not overwrite a previous piece. Note
that our \texttt{MkUpdate} constructor only has four ``free'' variables,
\texttt{i}, \texttt{j}, \texttt{p}, and \texttt{b}. If we use \texttt{MkUpdate},
it means that the ``final board'' is fully determined from only \texttt{i},
\texttt{j}, \texttt{p}, and \texttt{b}.

\hypertarget{coord}{%
\subsubsection{Coord}\label{coord}}

Now we need to define \texttt{Coord}. We're going to do that in terms of a
simpler type that is essentially the same for normal lists --- a type:

\begin{Shaded}
\begin{Highlighting}[]
\CommentTok{-- source: https://github.com/mstksg/inCode/tree/master/code-samples/ttt/Part1.hs#L89-L89}

\KeywordTok{data} \DataTypeTok{Sel}\OtherTok{ ::} \DataTypeTok{N} \OtherTok{->}\NormalTok{ [k] }\OtherTok{->}\NormalTok{ k }\OtherTok{->} \DataTypeTok{Type} \KeywordTok{where}
\end{Highlighting}
\end{Shaded}

A value of type \texttt{Sel\ n\ xs\ x} says that the nth item in \texttt{xs} is
\texttt{x}.

We can define this type inductively, similar to the common
\href{http://hackage.haskell.org/package/type-combinators-0.2.4.3/docs/Data-Type-Index.html}{\texttt{Index}}
data type. We can mention our induction rules:

\begin{enumerate}
\def\labelenumi{\arabic{enumi}.}
\tightlist
\item
  The first item in a list as at index 0 (\texttt{Z})
\item
  If an item is at index \texttt{n} in list \texttt{as}, then it is also at
  index \texttt{S\ n} in list \texttt{b\ \textquotesingle{}:\ as}.
\end{enumerate}

\begin{Shaded}
\begin{Highlighting}[]
\CommentTok{-- source: https://github.com/mstksg/inCode/tree/master/code-samples/ttt/Part1.hs#L89-L94}

\KeywordTok{data} \DataTypeTok{Sel}\OtherTok{ ::} \DataTypeTok{N} \OtherTok{->}\NormalTok{ [k] }\OtherTok{->}\NormalTok{ k }\OtherTok{->} \DataTypeTok{Type} \KeywordTok{where}
    \CommentTok{-- | The first item in a list is at index ''Z'}
    \DataTypeTok{SelZ}\OtherTok{ ::} \DataTypeTok{Sel}\NormalTok{ '}\DataTypeTok{Z}\NormalTok{ (a '}\FunctionTok{:}\NormalTok{ as) a}
    \DataTypeTok{SelS}\OtherTok{ ::} \DataTypeTok{Sel}\NormalTok{     n        as  a  }\CommentTok{-- ^ If item `a` is at index `n` in list `as`}
         \CommentTok{-- ---------------------------- then}
         \OtherTok{->} \DataTypeTok{Sel}\NormalTok{ ('}\DataTypeTok{S}\NormalTok{ n) (b '}\FunctionTok{:}\NormalTok{ as) a  }\CommentTok{-- ^ Item `a` is at index `S n` in list `b : as`}
\end{Highlighting}
\end{Shaded}

For example, for the type-level list \texttt{\textquotesingle{}{[}10,5,2,8{]}},
we can make values:

\begin{Shaded}
\begin{Highlighting}[]
\DataTypeTok{SelZ}\OtherTok{             ::} \DataTypeTok{Sel}\NormalTok{         '}\DataTypeTok{Z}\NormalTok{   '[}\DecValTok{10}\NormalTok{,}\DecValTok{5}\NormalTok{,}\DecValTok{2}\NormalTok{,}\DecValTok{8}\NormalTok{] }\DecValTok{10}
\DataTypeTok{SelS} \DataTypeTok{SelZ}\OtherTok{        ::} \DataTypeTok{Sel}\NormalTok{     ('}\DataTypeTok{S}\NormalTok{ '}\DataTypeTok{Z}\NormalTok{)  '[}\DecValTok{10}\NormalTok{,}\DecValTok{5}\NormalTok{,}\DecValTok{2}\NormalTok{,}\DecValTok{8}\NormalTok{] }\DecValTok{5}
\DataTypeTok{SelS}\NormalTok{ (}\DataTypeTok{SelS} \DataTypeTok{SelZ}\NormalTok{)}\OtherTok{ ::} \DataTypeTok{Sel}\NormalTok{ ('}\DataTypeTok{S}\NormalTok{ ('}\DataTypeTok{S}\NormalTok{ '}\DataTypeTok{Z}\NormalTok{)) '[}\DecValTok{10}\NormalTok{,}\DecValTok{5}\NormalTok{,}\DecValTok{2}\NormalTok{,}\DecValTok{8}\NormalTok{] }\DecValTok{2}
\end{Highlighting}
\end{Shaded}

etc.

We can then use this to define \texttt{Coord}:

\begin{Shaded}
\begin{Highlighting}[]
\CommentTok{-- source: https://github.com/mstksg/inCode/tree/master/code-samples/ttt/Part1.hs#L96-L101}

\KeywordTok{data} \DataTypeTok{Coord}\OtherTok{ ::}\NormalTok{ (}\DataTypeTok{N}\NormalTok{, }\DataTypeTok{N}\NormalTok{) }\OtherTok{->}\NormalTok{ [[k]] }\OtherTok{->}\NormalTok{ k }\OtherTok{->} \DataTypeTok{Type} \KeywordTok{where}
\OtherTok{    (:$:) ::}\NormalTok{ forall i j rows row p}\FunctionTok{.}\NormalTok{ ()}
          \OtherTok{=>} \DataTypeTok{Sel}\NormalTok{ i rows row         }\CommentTok{-- ^ If the ith list in `rows` is `row`}
          \OtherTok{->} \DataTypeTok{Sel}\NormalTok{ j row  p           }\CommentTok{-- ^ And the jth item in `row` is `p`}
          \CommentTok{-- --------------------------- then}
          \OtherTok{->} \DataTypeTok{Coord}\NormalTok{ '(i, j) rows p   }\CommentTok{-- ^ The item at (i, j) is `p`}
\end{Highlighting}
\end{Shaded}

A \texttt{Coord\ \textquotesingle{}(i,\ j)\ rows\ piece} contains a selection
into the ith list in \texttt{rows}, to get \texttt{row}, and a selection into
the jth item in \texttt{row}, to get \texttt{piece}.

\hypertarget{trying-it-out}{%
\subsection{Trying it out}\label{trying-it-out}}

That's it! Let's see if we can generate some sensible \texttt{Update}s, and
maybe even play a sample game.

We'll start with the \texttt{EmptyBoard}, and let's add a piece by \texttt{PX}
at the middle spot, index (1,1). This means we want
\texttt{SelS\ SelZ\ :\$:\ SelS\ SelZ} (a \texttt{Coord} with two indexes into
spots 1 and 1) applied to \texttt{MkUpdate}. We'll use \emph{-XTypeApplications}
to specify the type variables \texttt{p} and \texttt{b}:

\begin{Shaded}
\begin{Highlighting}[]
\NormalTok{ghci}\FunctionTok{>} \FunctionTok{:}\NormalTok{t }\DataTypeTok{MkUpdate} \FunctionTok{@}\NormalTok{_ }\FunctionTok{@}\NormalTok{_ }\FunctionTok{@}\NormalTok{'}\DataTypeTok{PX} \FunctionTok{@}\DataTypeTok{EmptyBoard}\NormalTok{ (}\DataTypeTok{SelS} \DataTypeTok{SelZ} \FunctionTok{:$:} \DataTypeTok{SelS} \DataTypeTok{SelZ}\NormalTok{)}
\DataTypeTok{Update}
\NormalTok{  '}\DataTypeTok{PX}
\NormalTok{  '[ '[ '}\DataTypeTok{Nothing}\NormalTok{, '}\DataTypeTok{Nothing}\NormalTok{ , '}\DataTypeTok{Nothing}\NormalTok{],}
\NormalTok{     '[ '}\DataTypeTok{Nothing}\NormalTok{, '}\DataTypeTok{Nothing}\NormalTok{ , '}\DataTypeTok{Nothing}\NormalTok{],}
\NormalTok{     '[ '}\DataTypeTok{Nothing}\NormalTok{, '}\DataTypeTok{Nothing}\NormalTok{ , '}\DataTypeTok{Nothing}\NormalTok{]}
\NormalTok{   ]}
\NormalTok{  '[ '[ '}\DataTypeTok{Nothing}\NormalTok{, '}\DataTypeTok{Nothing}\NormalTok{ , '}\DataTypeTok{Nothing}\NormalTok{],}
\NormalTok{     '[ '}\DataTypeTok{Nothing}\NormalTok{, '}\DataTypeTok{Just}\NormalTok{ '}\DataTypeTok{PX}\NormalTok{, '}\DataTypeTok{Nothing}\NormalTok{],}
\NormalTok{     '[ '}\DataTypeTok{Nothing}\NormalTok{, '}\DataTypeTok{Nothing}\NormalTok{ , '}\DataTypeTok{Nothing}\NormalTok{]}
\NormalTok{  ]}
\end{Highlighting}
\end{Shaded}

Nice! This update produces exactly he board expected.

Let's see if we can see if this prevents us from creating an illegal board.
We'll take the result board and see if we can place a \texttt{PO} piece there:

\begin{Shaded}
\begin{Highlighting}[]
\NormalTok{ghci}\FunctionTok{>} \KeywordTok{let} \DataTypeTok{NewBoard} \FunctionTok{=}\NormalTok{ '[ '[ '}\DataTypeTok{Nothing}\NormalTok{, '}\DataTypeTok{Nothing}\NormalTok{ , '}\DataTypeTok{Nothing}\NormalTok{ ]}
\NormalTok{                      , '[ '}\DataTypeTok{Nothing}\NormalTok{, '}\DataTypeTok{Just}\NormalTok{ '}\DataTypeTok{PX}\NormalTok{, '}\DataTypeTok{Nothing}\NormalTok{ ]}
\NormalTok{                      , '[ '}\DataTypeTok{Nothing}\NormalTok{, '}\DataTypeTok{Nothing}\NormalTok{ , '}\DataTypeTok{Nothing}\NormalTok{ ]}
\NormalTok{                      ]}
\NormalTok{ghci}\FunctionTok{>} \FunctionTok{:}\NormalTok{k }\DataTypeTok{MkUpdate} \FunctionTok{@}\NormalTok{_ }\FunctionTok{@}\NormalTok{_ }\FunctionTok{@}\NormalTok{'}\DataTypeTok{PO} \FunctionTok{@}\DataTypeTok{NewBoard}\NormalTok{ (}\DataTypeTok{SelS} \DataTypeTok{SelZ} \FunctionTok{:$:} \DataTypeTok{SelS} \DataTypeTok{SelZ}\NormalTok{)}
\NormalTok{    • }\DataTypeTok{Couldn't}\NormalTok{ match }\KeywordTok{type}\NormalTok{ ‘'}\DataTypeTok{Nothing}\NormalTok{’ with ‘'}\DataTypeTok{Just}\NormalTok{ '}\DataTypeTok{PX}\NormalTok{’}
\end{Highlighting}
\end{Shaded}

Right! That's because \texttt{SelS\ SelZ\ :\&:\ SelS\ SellZ}, applied to
\texttt{NewBoard}, gives
\texttt{Coord\ \textquotesingle{}(\textquotesingle{}S\ \textquotesingle{}Z,\ \textquotesingle{}S\ \textquotesingle{}Z)\ NewBoard\ (\textquotesingle{}Just\ \textquotesingle{}PX)}.
However, in order to be used with \texttt{MkUpdate}, the final field has to be
\texttt{\textquotesingle{}Nothing}, not
\texttt{\textquotesingle{}Just\ \textquotesingle{}PX}. So, type error.

\hypertarget{type-safe-play}{%
\subsection{Type-safe Play}\label{type-safe-play}}

At the end of this all, we finally have enough to write a truly type-safe
\texttt{play} function that allows us to play a round of our game!

\begin{Shaded}
\begin{Highlighting}[]
\CommentTok{-- source: https://github.com/mstksg/inCode/tree/master/code-samples/ttt/Part1.hs#L103-L109}

\NormalTok{play}
\OtherTok{    ::}\NormalTok{ forall i j p b}\FunctionTok{.}\NormalTok{ ()}
    \OtherTok{=>} \DataTypeTok{InPlay} \FunctionTok{@@}\NormalTok{ b}
    \OtherTok{->} \DataTypeTok{Coord}\NormalTok{ '(i, j) b '}\DataTypeTok{Nothing}
    \OtherTok{->} \DataTypeTok{GameState}\NormalTok{ p b}
    \OtherTok{->} \DataTypeTok{GameState}\NormalTok{ (}\DataTypeTok{AltP}\NormalTok{ p) (}\DataTypeTok{PlaceBoard}\NormalTok{ i j p b)}
\NormalTok{play r c }\FunctionTok{=} \DataTypeTok{GSUpdate}\NormalTok{ r (}\DataTypeTok{MkUpdate}\NormalTok{ c)}
\end{Highlighting}
\end{Shaded}

\texttt{play} is basically the entirety of our game engine! (Minus defining
\texttt{InPlay}, which we will take care of later). It'll take our new move and
a proof that the game is still in play, and return a updated new game state. Our
entire game is done, and type-safe! It's impossible to play a game in an
incorrect way! (once we define \texttt{InPlay}).

Let's try out a few rounds in ghci, using \texttt{undefined} instead of a proper
\texttt{InPlay} for now:

\begin{Shaded}
\begin{Highlighting}[]
\NormalTok{ghci}\FunctionTok{>}\NormalTok{ g1 }\FunctionTok{=}\NormalTok{ play undefined (}\DataTypeTok{SelS} \DataTypeTok{SelZ} \FunctionTok{:$:} \DataTypeTok{SelS} \DataTypeTok{SelZ}\NormalTok{) }\DataTypeTok{GSStart}   \CommentTok{-- X plays (1,1)}
\NormalTok{ghci}\FunctionTok{>} \FunctionTok{:}\NormalTok{t g1}
\DataTypeTok{GameState}\NormalTok{ '}\DataTypeTok{PO}
\NormalTok{    '[ '[ '}\DataTypeTok{Nothing}\NormalTok{, '}\DataTypeTok{Nothing}\NormalTok{ , '}\DataTypeTok{Nothing}\NormalTok{]}
\NormalTok{     , '[ '}\DataTypeTok{Nothing}\NormalTok{, '}\DataTypeTok{Just}\NormalTok{ '}\DataTypeTok{PX}\NormalTok{, '}\DataTypeTok{Nothing}\NormalTok{]}
\NormalTok{     , '[ '}\DataTypeTok{Nothing}\NormalTok{, '}\DataTypeTok{Nothing}\NormalTok{ , '}\DataTypeTok{Nothing}\NormalTok{]}
\NormalTok{     ]}

\NormalTok{ghci}\FunctionTok{>}\NormalTok{ g2 }\FunctionTok{=}\NormalTok{ play undefined (}\DataTypeTok{SelZ} \FunctionTok{:$:} \DataTypeTok{SelS} \DataTypeTok{SelZ}\NormalTok{) g1   }\CommentTok{-- O plays (0,1)}
\NormalTok{ghci}\FunctionTok{>} \FunctionTok{:}\NormalTok{t g2}
\DataTypeTok{GameState}\NormalTok{ '}\DataTypeTok{PX}
\NormalTok{    '[ '[ '}\DataTypeTok{Nothing}\NormalTok{, '}\DataTypeTok{Just}\NormalTok{ '}\DataTypeTok{PO}\NormalTok{, '}\DataTypeTok{Nothing}\NormalTok{]}
\NormalTok{     , '[ '}\DataTypeTok{Nothing}\NormalTok{, '}\DataTypeTok{Just}\NormalTok{ '}\DataTypeTok{PX}\NormalTok{, '}\DataTypeTok{Nothing}\NormalTok{]}
\NormalTok{     , '[ '}\DataTypeTok{Nothing}\NormalTok{, '}\DataTypeTok{Nothing}\NormalTok{ , '}\DataTypeTok{Nothing}\NormalTok{]}
\NormalTok{     ]}

\NormalTok{ghci}\FunctionTok{>}\NormalTok{ g3 }\FunctionTok{=}\NormalTok{ play undefined (}\DataTypeTok{SelZ} \FunctionTok{:$:} \DataTypeTok{SelS} \DataTypeTok{SelZ}\NormalTok{) g2   }\CommentTok{-- X plays (1,0)}
\NormalTok{ghci}\FunctionTok{>} \FunctionTok{:}\NormalTok{t g3}
\DataTypeTok{GameState}\NormalTok{ '}\DataTypeTok{PO}
\NormalTok{    '[ '[ '}\DataTypeTok{Nothing}\NormalTok{ , '}\DataTypeTok{Just}\NormalTok{ '}\DataTypeTok{PO}\NormalTok{, '}\DataTypeTok{Nothing}\NormalTok{]}
\NormalTok{     , '[ '}\DataTypeTok{Just}\NormalTok{ '}\DataTypeTok{PX}\NormalTok{, '}\DataTypeTok{Just}\NormalTok{ '}\DataTypeTok{PX}\NormalTok{, '}\DataTypeTok{Nothing}\NormalTok{]}
\NormalTok{     , '[ '}\DataTypeTok{Nothing}\NormalTok{ , '}\DataTypeTok{Nothing}\NormalTok{ , '}\DataTypeTok{Nothing}\NormalTok{]}
\NormalTok{     ]}

\NormalTok{ghci}\FunctionTok{>}\NormalTok{ g4 }\FunctionTok{=}\NormalTok{ play undefined (}\DataTypeTok{SelS} \DataTypeTok{SelZ} \FunctionTok{:$:} \DataTypeTok{SelS} \DataTypeTok{SelZ}\NormalTok{) g3   }\CommentTok{-- O plays (1,1)}
\NormalTok{    • }\DataTypeTok{Couldn't}\NormalTok{ match }\KeywordTok{type}\NormalTok{ ‘'}\DataTypeTok{Just}\NormalTok{ '}\DataTypeTok{PX}\NormalTok{’ with ‘'}\DataTypeTok{Nothing}\NormalTok{’}

\NormalTok{ghci}\FunctionTok{>}\NormalTok{ g4 }\FunctionTok{=}\NormalTok{ play undefined (}\DataTypeTok{SelS}\NormalTok{ (}\DataTypeTok{SelS}\NormalTok{ (}\DataTypeTok{SelS} \DataTypeTok{SelZ}\NormalTok{)) }\FunctionTok{:$:} \DataTypeTok{SelZ}\NormalTok{) g3  }\CommentTok{-- O plays (3,0)}
\NormalTok{    • }\DataTypeTok{Couldn't}\NormalTok{ match }\KeywordTok{type}\NormalTok{ ‘'[]’ with ‘'}\DataTypeTok{Nothing}\NormalTok{ '}\FunctionTok{:}\NormalTok{ as’}
\end{Highlighting}
\end{Shaded}

\texttt{play} enforces:

\begin{enumerate}
\def\labelenumi{\arabic{enumi}.}
\tightlist
\item
  Turns are always alternating X, then O
\item
  We cannot place a piece in a previously-played spot
\item
  We cannot place a piece out-of-bounds.
\end{enumerate}

\hypertarget{decision-functions-and-views}{%
\section{Decision Functions and Views}\label{decision-functions-and-views}}

This seems nice, but we're forgetting an important part. \texttt{play} requires
us to only give valid inputs, and enforces that the inputs are valid. However,
how do we \emph{create} valid inputs, in a way that satisfies \texttt{play}?

As we'll see, this is one of the core problems that dependently typed
programming gives us tools to solve.

At this point, we've reached the important part of any ``type-safe''
application: \emph{decision functions} and dependent \emph{views}.
\emph{Decision functions} let you slowly refine your more general values (types)
into more specific valid types. \emph{Views} let you sort out your our values
into more ``useful'' perspectives.

We're going to allow for users to pick to move at any natural number pair
(\texttt{(N,\ N)}), but only \emph{some} of those natural numbers can become
valid updates. In particular, we only allow an \texttt{Update} to be made if
\texttt{(N,\ N)} represent valid updates.

What are two ways this can go wrong? Well, if we allow the user to enter any two
natural numbers, here are all of the potential outcomes:

\begin{enumerate}
\def\labelenumi{\arabic{enumi}.}
\tightlist
\item
  We might get a coordinate that is out of bounds in x
\item
  We might get a coordinate that is in bounds in x, but out of bounds in y
\item
  We might get a coordinate that is in bounds in x, in bounds in y, but
  referencing a position that has already been played.
\item
  We might get a coordinate that is in bounds in x, in bounds in y, and
  references a blank position. This is the only ``success'' case.
\end{enumerate}

Note that we could also just have a ``success or nor success'' situation, but,
because we might want to provide feedback to the user, it is helpful to not be
``\href{https://twitter.com/cattheory/status/887760004622757890}{decision-blind}''
(a cousin of
\href{https://existentialtype.wordpress.com/2011/03/15/boolean-blindness/}{boolean
blindness}).

We'll call these potential ``views'' out of \texttt{(N,\ N)} with respect to
some board \texttt{b}. Let's create a data type representing all of these
possibilities (using \texttt{OutOfBounds} as a placeholder predicate for an
out-of-bounds coordinate):

\begin{Shaded}
\begin{Highlighting}[]
\CommentTok{-- | Placeholder predicate if a given number `n` is out of bounds for a given}
\CommentTok{-- list.  Predicate is from the 'decidable' library}
\KeywordTok{data} \DataTypeTok{OutOfBounds}\OtherTok{ n ::} \DataTypeTok{Predicate}\NormalTok{ [k]}
\CommentTok{-- source: https://github.com/mstksg/inCode/tree/master/code-samples/ttt/Part1.hs#L117-L127}

\KeywordTok{data} \DataTypeTok{Pick}\OtherTok{ ::}\NormalTok{ (}\DataTypeTok{N}\NormalTok{, }\DataTypeTok{N}\NormalTok{, }\DataTypeTok{Board}\NormalTok{) }\OtherTok{->} \DataTypeTok{Type} \KeywordTok{where}
    \CommentTok{-- | We are out of bounds in x}
    \DataTypeTok{PickOoBX}\OtherTok{   ::} \DataTypeTok{OutOfBounds}\NormalTok{ i }\FunctionTok{@@}\NormalTok{ b                         }\OtherTok{->} \DataTypeTok{Pick}\NormalTok{ '(i, j, b)}
    \CommentTok{-- | We are in-bounds in x, but out of bounds in y}
    \DataTypeTok{PickOoBY}\OtherTok{   ::} \DataTypeTok{Sel}\NormalTok{ i b row        }\OtherTok{->} \DataTypeTok{OutOfBounds}\NormalTok{ j }\FunctionTok{@@}\NormalTok{ row }\OtherTok{->} \DataTypeTok{Pick}\NormalTok{ '(i, j, b)}
    \CommentTok{-- | We are in-bounds in x, in-bounds in y, but spot is taken by `p`.}
    \CommentTok{-- We include `Sing p` in this constructor to potentially provide}
    \CommentTok{-- feedback to the user on what piece is already in the spot.}
    \DataTypeTok{PickPlayed}\OtherTok{ ::} \DataTypeTok{Coord}\NormalTok{ '(i, j) b ('}\DataTypeTok{Just}\NormalTok{ p) }\OtherTok{->} \DataTypeTok{Sing}\NormalTok{ p        }\OtherTok{->} \DataTypeTok{Pick}\NormalTok{ '(i, j, b)}
    \CommentTok{-- | We are in-bounds in x, in-bounds in y, and spot is clear}
    \DataTypeTok{PickValid}\OtherTok{  ::} \DataTypeTok{Coord}\NormalTok{ '(i, j) b '}\DataTypeTok{Nothing}                   \OtherTok{->} \DataTypeTok{Pick}\NormalTok{ '(i, j, b)}
\end{Highlighting}
\end{Shaded}

(A value of type \texttt{OutOfBounds\ n\ @@\ xs} is a witness that \texttt{xs}
satisfies the \texttt{OutOfBounds\ n} --- that is, \texttt{n} is out of bounds
of \texttt{xs}. More on this later when we talk about the \emph{decidable}
library!)

So, if we have an \texttt{(N,\ N,\ Board)}, we should be able to categorize it
into one of each of these potential views.

\hypertarget{proving-functions}{%
\subsection{Proving functions}\label{proving-functions}}

This is the job of a ``decision function''; in this case, actually, a ``proving
function'', or a ``viewing function''. We need to be able to write a function:

\begin{Shaded}
\begin{Highlighting}[]
\OtherTok{pick ::} \DataTypeTok{Sing}\NormalTok{ '(i, j, b)}
     \OtherTok{->} \DataTypeTok{Pick}\NormalTok{ '(i, j, b)}
\end{Highlighting}
\end{Shaded}

That is, given any coordinate and board, we should be able to \emph{totally}
categorize it to one of the four categories (think of them like four
perspectives/classifications), without exception.

This can be considered the boundary between the unsafe and the safe world. And,
to me, this is the ``hard part'' about dependently typed programming :)

Now, let's write \texttt{pick}. If we want to take any \texttt{i}, \texttt{j},
and \texttt{b}, and turn it into a valid \texttt{Pick}, remember that a valid
\texttt{Pick\ \textquotesingle{}(i,\ j,\ b)} contains a
\texttt{Coord\ \textquotesingle{}(i,\ j)\ b\ \textquotesingle{}Nothing}, which
contains a \texttt{Sel\ i\ b\ row} and a
\texttt{Sel\ j\ row\ \textquotesingle{}Nothing}. So we need to ``convert'' some
\texttt{i}, \texttt{j} into a \texttt{Sel\ i\ b\ row} and
\texttt{Sel\ j\ row\ \textquotesingle{}Nothing}.

Essentially, we want a function:

\begin{Shaded}
\begin{Highlighting}[]
\OtherTok{sel ::} \DataTypeTok{Sing}\NormalTok{ i}
    \OtherTok{->} \DataTypeTok{Sing}\NormalTok{ xs}
    \OtherTok{->} \DataTypeTok{Sel}\NormalTok{ i xs }\FunctionTok{??????}
\end{Highlighting}
\end{Shaded}

where \texttt{????} is whatever value is in \texttt{xs} at index \texttt{i}.
It's something we might not know directly from the input types, necessarily,
because it might not even exist (the list might be too short).

\hypertarget{existential-return-types-and-dependent-pairs}{%
\subsection{Existential return types and Dependent
Pairs}\label{existential-return-types-and-dependent-pairs}}

This pattern --- where we don't know the type of something until after we
receive the function inputs --- is something you might recognize as an
\emph{existential type}, implementable using a \emph{dependent pair} (or
dependent sum).

We could write our own dependent pair from scratch, but this is a good
opportunity to practice using \emph{singletons} library's versatile ``anonymous
dependent pair'' type, \texttt{Σ} (or \texttt{Sigma}), from
\emph{Data.Singletons.Sigma}.

\begin{Shaded}
\begin{Highlighting}[]
\KeywordTok{data} \DataTypeTok{Sigma}\OtherTok{ k ::} \DataTypeTok{Predicate}\NormalTok{ k }\OtherTok{->} \DataTypeTok{Type} \KeywordTok{where}
\OtherTok{    (:&:) ::} \DataTypeTok{Sing}\NormalTok{ x }\OtherTok{->}\NormalTok{ (p }\FunctionTok{@@}\NormalTok{ x) }\OtherTok{->} \DataTypeTok{Sigma}\NormalTok{ k f}

\KeywordTok{type}\NormalTok{ Σ k }\FunctionTok{=} \DataTypeTok{Sigma}\NormalTok{ k}
\end{Highlighting}
\end{Shaded}

A value of type \texttt{Sigma\ k\ p} contains an \texttt{p\ @@\ x} (a witness
that \texttt{p} satisfies \texttt{x}), existentially \emph{hiding} the
\texttt{x\ ::\ k}, and also \texttt{x} itself (as \texttt{Sing\ x}; remember
that \texttt{Sing\ x} is essentially a value-level representation of type
\texttt{x}).

You can think of \texttt{Sigma\ k\ p} as a proof that \texttt{p} is satisfied
for \emph{some} \texttt{x\ ::\ k}, but we can't know which \texttt{x} it is
until you pattern match. If you pattern match, you'll get both \texttt{Sing\ x}
(to find out the \texttt{x}), and the \texttt{p\ @@\ x} (the witness that
\texttt{p} is satisfied by \texttt{x}).

We can use this to return a \texttt{Sel\ n\ xs\ ????}, \emph{hiding} the
\texttt{???}.

Now for some plumbing: We can't directly give \texttt{Sel\ n\ xs} to
\texttt{Sigma} (because it expects a \texttt{Predicate\ k}, not a
\texttt{k\ -\textgreater{}\ Type}), but we can turn a type constructor into a
\texttt{Predicate} using \texttt{TyPred}, a convenient combinator from the
\emph{decidable} library:

\begin{Shaded}
\begin{Highlighting}[]
\DataTypeTok{TyPred}\OtherTok{ ::}\NormalTok{ (k }\OtherTok{->} \DataTypeTok{Type}\NormalTok{) }\OtherTok{->} \DataTypeTok{Predicate}\NormalTok{ k}
\end{Highlighting}
\end{Shaded}

\begin{Shaded}
\begin{Highlighting}[]
\NormalTok{ghci}\FunctionTok{>} \FunctionTok{:}\NormalTok{k }\DataTypeTok{Pick}
\NormalTok{(}\DataTypeTok{N}\NormalTok{, }\DataTypeTok{N}\NormalTok{, }\DataTypeTok{Board}\NormalTok{) }\OtherTok{->} \DataTypeTok{Type}

\NormalTok{ghci}\FunctionTok{>} \FunctionTok{:}\NormalTok{k }\DataTypeTok{TyPred} \DataTypeTok{Pick}
\DataTypeTok{Predicate}\NormalTok{ (}\DataTypeTok{N}\NormalTok{, }\DataTypeTok{N}\NormalTok{, }\DataTypeTok{Board}\NormalTok{)}

\NormalTok{ghci}\FunctionTok{>} \FunctionTok{:}\NormalTok{k }\DataTypeTok{Sel}\NormalTok{ '}\DataTypeTok{Z} \DataTypeTok{EmptyBoard}
\NormalTok{[}\DataTypeTok{Maybe} \DataTypeTok{Piece}\NormalTok{] }\OtherTok{->} \DataTypeTok{Type}

\NormalTok{ghci}\FunctionTok{>} \FunctionTok{:}\NormalTok{k }\DataTypeTok{TyPred}\NormalTok{ (}\DataTypeTok{Sel}\NormalTok{ '}\DataTypeTok{Z} \DataTypeTok{EmptyBoard}\NormalTok{)}
\DataTypeTok{Predicate}\NormalTok{ [}\DataTypeTok{Maybe} \DataTypeTok{Piece}\NormalTok{]}
\end{Highlighting}
\end{Shaded}

Let's make sure this type works like we expect it to. We want a
\texttt{Σ\ k\ (TyPred\ (Sel\ n\ xs))} to contain the \texttt{x} at position
\texttt{n} in \texttt{xs}, \emph{and} the \texttt{Sel} into that position.

To make things a little less verbose, we can define a type synonym for
\texttt{Σ\ k\ (TyPred\ (Sel\ n\ xs))}, \texttt{SelFound\ n\ xs}:

\begin{Shaded}
\begin{Highlighting}[]
\KeywordTok{type} \DataTypeTok{SelFound}\NormalTok{ n (}\OtherTok{xs ::}\NormalTok{ [k]) }\FunctionTok{=}\NormalTok{ Σ k (}\DataTypeTok{TyPred}\NormalTok{ (}\DataTypeTok{Sel}\NormalTok{ n xs))}
\end{Highlighting}
\end{Shaded}

Or, as practice, maybe we can treat \texttt{SelFound\ n} as a predicate on a
list? The predicate \texttt{SelFound\ n} will be satisfied if list \texttt{xs}
has some item \texttt{x} at index \texttt{n}!

\begin{Shaded}
\begin{Highlighting}[]
\CommentTok{-- source: https://github.com/mstksg/inCode/tree/master/code-samples/ttt/Part1.hs#L111-L112}

\KeywordTok{data} \DataTypeTok{SelFound}\OtherTok{ ::} \DataTypeTok{N} \OtherTok{->} \DataTypeTok{Predicate}\NormalTok{ [k]}
\KeywordTok{type} \KeywordTok{instance} \DataTypeTok{Apply}\NormalTok{ (}\DataTypeTok{SelFound}\NormalTok{ n) (}\OtherTok{xs ::}\NormalTok{ [k]) }\FunctionTok{=}\NormalTok{ Σ k (}\DataTypeTok{TyPred}\NormalTok{ (}\DataTypeTok{Sel}\NormalTok{ n xs))}
\end{Highlighting}
\end{Shaded}

We can check to make sure this works, by checking the type of witnesses of
\texttt{SelFound\ \textquotesingle{}Z\ @@\ \textquotesingle{}{[}\ \textquotesingle{}True,\ \textquotesingle{}False\ {]}}:

\begin{Shaded}
\begin{Highlighting}[]
\NormalTok{ghci}\FunctionTok{>} \FunctionTok{:}\NormalTok{kind}\FunctionTok{!} \DataTypeTok{SelFound}\NormalTok{ '}\DataTypeTok{Z} \FunctionTok{@@}\NormalTok{ '[ '}\DataTypeTok{True}\NormalTok{, '}\DataTypeTok{False}\NormalTok{ ]}
\NormalTok{Σ }\DataTypeTok{Bool}\NormalTok{ (}\DataTypeTok{TyPred} \DataTypeTok{Sel}\NormalTok{ ('}\DataTypeTok{Z}\NormalTok{ '[ '}\DataTypeTok{True}\NormalTok{, '}\DataTypeTok{False}\NormalTok{ ]))}
\end{Highlighting}
\end{Shaded}

Now let's make some sample witnesses of predicate \texttt{SelFound\ n} to ensure
we are thinking about things correctly:

\begin{Shaded}
\begin{Highlighting}[]
\CommentTok{-- source: https://github.com/mstksg/inCode/tree/master/code-samples/ttt/Part1.hs#L174-L176}

\OtherTok{selFoundTest1 ::} \DataTypeTok{SelFound}\NormalTok{ '}\DataTypeTok{Z} \FunctionTok{@@}\NormalTok{ '[ '}\DataTypeTok{True}\NormalTok{, '}\DataTypeTok{False}\NormalTok{ ]}
\NormalTok{selFoundTest1 }\FunctionTok{=} \DataTypeTok{STrue} \FunctionTok{:&:} \DataTypeTok{SelZ}
                       \CommentTok{-- ^ Sel 'Z '[ 'True, 'False ] 'True}
\end{Highlighting}
\end{Shaded}

Note that \texttt{SFalse\ :\&:\ SelZ} would be a type error, because the second
half \texttt{SelZ} would be
\texttt{Sel\ ::\ \textquotesingle{}Z\ \textquotesingle{}{[}\ \textquotesingle{}True,\ \textquotesingle{}False\ {]}\ \textquotesingle{}True}
(because \texttt{\textquotesingle{}True} is the 0th item in the list), so we
have to have the first half match \texttt{\textquotesingle{}True}, with
\texttt{STrue}.

We can write a witness for
\texttt{SelFound\ (\textquotesingle{}S\ \textquotesingle{}Z)\ @@\ \textquotesingle{}{[}\ \textquotesingle{}True,\ \textquotesingle{}False\ {]}},
as well, by giving the value of the list at index 1,
\texttt{\textquotesingle{}False}:

\begin{Shaded}
\begin{Highlighting}[]
\CommentTok{-- source: https://github.com/mstksg/inCode/tree/master/code-samples/ttt/Part1.hs#L178-L180}

\OtherTok{selFoundTest2 ::} \DataTypeTok{SelFound}\NormalTok{ ('}\DataTypeTok{S}\NormalTok{ '}\DataTypeTok{Z}\NormalTok{) }\FunctionTok{@@}\NormalTok{ '[ '}\DataTypeTok{True}\NormalTok{, '}\DataTypeTok{False}\NormalTok{ ]}
\NormalTok{selFoundTest2 }\FunctionTok{=} \DataTypeTok{SFalse} \FunctionTok{:&:} \DataTypeTok{SelS} \DataTypeTok{SelZ}
                        \CommentTok{-- ^ Sel ('S 'Z) '[ 'True, 'False ] 'False}
\end{Highlighting}
\end{Shaded}

Before moving on, I strongly recommend trying to write some values of type
\texttt{SelFound\ n\ @@\ xs} for different \texttt{n\ ::\ N} and
\texttt{xs\ ::\ {[}a{]}}, to see what type-checks and what doesn't. It'll help
you get a feel of the types we are working with, which might be more advanced
than types you might encounter in everyday Haskell programming. Remember that
you can load up all of the definitions in this post into a ghci session by
downloading
\href{https://github.com/mstksg/inCode/tree/master/code-samples/ttt/Part1.hs}{the
source code} and executing it on the command line, \texttt{./Part1.hs}.

Now, we now have enough tools to write the type of the function we would like:

\begin{Shaded}
\begin{Highlighting}[]
\NormalTok{selFound}
\OtherTok{    ::} \DataTypeTok{Sing}\NormalTok{ n}
    \OtherTok{->} \DataTypeTok{Sing}\NormalTok{ xs}
    \OtherTok{->} \DataTypeTok{SelFound}\NormalTok{ n }\FunctionTok{@@}\NormalTok{ xs}
\end{Highlighting}
\end{Shaded}

Remember, a \texttt{SelFound\ n\ @@\ xs} contains both a \texttt{Sel\ n\ xs\ x}
\emph{and} a \texttt{Sing\ x}: a selection into the \texttt{i}th item in
\texttt{xs} (the \texttt{Sel\ n\ xs\ x}), and also the item itself (the
\texttt{Sing\ x}).

We can start writing this, but the type system will soon show you where you run
into problems. And that's one of the best things about type systems! They help
you realize when you're trying to do something that doesn't make sense.

\begin{Shaded}
\begin{Highlighting}[]
\NormalTok{selFound}
\OtherTok{    ::} \DataTypeTok{Sing}\NormalTok{ n}
    \OtherTok{->} \DataTypeTok{Sing}\NormalTok{ xs}
    \OtherTok{->} \DataTypeTok{SelFound}\NormalTok{ n }\FunctionTok{@@}\NormalTok{ xs}
\NormalTok{selFound }\FunctionTok{=}\NormalTok{ \textbackslash{}}\KeywordTok{case}
    \DataTypeTok{SZ} \OtherTok{->}\NormalTok{ \textbackslash{}}\KeywordTok{case}
      \DataTypeTok{SNil} \OtherTok{->}\NormalTok{ _ }\FunctionTok{:&:}\NormalTok{ _}
\end{Highlighting}
\end{Shaded}

Things start out pretty standard. We want to match on all potential constructors
of \texttt{N} and \texttt{{[}a{]}}: \texttt{N} has \texttt{Z} and \texttt{S}, so
we match on singleton constructors \texttt{SZ} and \texttt{SS}; \texttt{{[}a{]}}
has \texttt{{[}{]}} and \texttt{(:)}, so we match on singleton constructors
\texttt{SNil} and \texttt{SCons}.

If you ask ghc what goes in the typed holes, it'll say that we need a
\texttt{Sing\ x} and a
\texttt{Sel\ \textquotesingle{}Z\ \textquotesingle{}{[}{]}\ x} (which is because
matching on \texttt{SZ} tells us we are in \texttt{\textquotesingle{}Z}, and
matching on \texttt{SNil} tells us we are in \texttt{\textquotesingle{}{[}{]}}).
And this\ldots{}is a problem.

Remember that the \texttt{x} is supposed to be the \texttt{n}th item in
\texttt{xs}. Here, in this pattern match branch, we want the zeroth (first) item
in \texttt{{[}{]}}. This doesn't exist! That's because there is no item in
\texttt{{[}{]}}, so there is nothing we can put for the \texttt{Sing\ x}.

There's also nothing we could put for the \texttt{Sel} (the right hand side of
\texttt{:\&:}), since there is no constructor of \texttt{Sel} that returns a
\texttt{Sel\ n\ \textquotesingle{}{[}{]}} (the constructors of \texttt{Sel} all
return \texttt{x\ \textquotesingle{}:\ xs}, never \texttt{Nil}).

So, this branch is impossible to fulfil. We know now that we made a large
conceptual error (aren't types great?)

The problem? Well, indexing into item \texttt{n} in list \texttt{xs} \emph{might
not always succeed}. We might try to index into an empty list, so we can't ever
get a result!

\hypertarget{decision-functions}{%
\subsection{Decision Functions}\label{decision-functions}}

What we need is not a \emph{proving function}, but, rather, a \emph{decision}
function. A decision function for a predicate \texttt{P} is a function:

\begin{Shaded}
\begin{Highlighting}[]
\OtherTok{decidePred ::} \DataTypeTok{Sing}\NormalTok{ x}
           \OtherTok{->} \DataTypeTok{Decision}\NormalTok{ (}\DataTypeTok{P} \FunctionTok{@@}\NormalTok{ x)}
\end{Highlighting}
\end{Shaded}

That is, instead of producing a \texttt{P\ @@\ x} directly, we produce a
\texttt{Decision\ (P\ @@\ x)}. Here, \texttt{Decision} is:

\begin{Shaded}
\begin{Highlighting}[]
\KeywordTok{data} \DataTypeTok{Decision}\NormalTok{ a}
    \FunctionTok{=} \DataTypeTok{Proved}\NormalTok{     a                }\CommentTok{-- ^ `a` is provably true}
    \FunctionTok{|} \DataTypeTok{Disproved}\NormalTok{ (a }\OtherTok{->} \DataTypeTok{Void}\NormalTok{)       }\CommentTok{-- ^ `a` is provably false}

\CommentTok{-- | The type with no constructors.  If we have a function `a -> Void`, it must}
\CommentTok{-- mean that no value of type `a` exists.}
\KeywordTok{data} \DataTypeTok{Void}
\end{Highlighting}
\end{Shaded}

A decision function means that, for any \texttt{x}, we can say that either
\texttt{P\ @@\ x} can be proven true or can be proven false. See
\href{https://blog.jle.im/entry/introduction-to-singletons-3.html\#decision}{this
section} for a deeper discussion on why \texttt{Decision} has both the
\texttt{Proved} and \texttt{Disproved} branch. Essentially, it prevents us from
just returning ``disproved'' without proving it (so we can be sure that our
decision function is ``correct'' and not just cheating), and, in the long term,
we keep track of ``provably false'' because we can use it later to build other
useful decision functions and proving functions.

We use decision functions when we want to ``conditionally prove'' something ---
it might be true, or it might not be (but definitely one or the other). It might
exist, or it might not. We can construct the view, or we can't. Whatever the
perspective, it's always one or the other.

\texttt{selFound} fits this category: for a \texttt{Sel\ n\ xs\ ????}, either
\texttt{n} is ``in bounds'' of \texttt{xs} (and we can prove this with the item
\texttt{x} in \texttt{xs}), or \texttt{n} is ``out of bounds''. Either we get
the \texttt{x} out of \texttt{xs} at slot \texttt{n}, or we prove that no
possible \texttt{x} exists in \texttt{xs} at slot \texttt{n}.

\hypertarget{deciding-selfound}{%
\subsubsection{Deciding SelFound}\label{deciding-selfound}}

Enough talk, let's get to it!

Let's write our first dependently typed function. We start the same way --- by
looking at every possible constructor of \texttt{N} and \texttt{{[}a{]}}:

\begin{Shaded}
\begin{Highlighting}[]
\NormalTok{selFound}
\OtherTok{    ::} \DataTypeTok{Sing}\NormalTok{ n}
    \OtherTok{->} \DataTypeTok{Sing}\NormalTok{ xs}
    \OtherTok{->} \DataTypeTok{Decision}\NormalTok{ (}\DataTypeTok{SelFound}\NormalTok{ n }\FunctionTok{@@}\NormalTok{ xs)}
\NormalTok{selFound }\FunctionTok{=}\NormalTok{ \textbackslash{}}\KeywordTok{case}
    \DataTypeTok{SZ} \OtherTok{->}\NormalTok{ \textbackslash{}}\KeywordTok{case}
      \DataTypeTok{SNil}         \OtherTok{->}\NormalTok{ _   }\CommentTok{-- n is 'Z, xs is '[]}
\NormalTok{      x }\OtherTok{`SCons`}\NormalTok{ xs }\OtherTok{->}\NormalTok{ _   }\CommentTok{-- n is 'Z, xs is (x ': xs)}
    \DataTypeTok{SS}\NormalTok{ n }\OtherTok{->}\NormalTok{ \textbackslash{}}\KeywordTok{case}
      \DataTypeTok{SNil}         \OtherTok{->}\NormalTok{ _   }\CommentTok{-- n is ('S n), xs is '[]}
\NormalTok{      x }\OtherTok{`SCons`}\NormalTok{ xs }\OtherTok{->}\NormalTok{ _   }\CommentTok{-- n is ('S n), xs is (x ': xs)}
\end{Highlighting}
\end{Shaded}

Okay, four cases. Initially daunting, but we can just handle this one by one.
Normally we can just fill in the blanks with the ``right'' responses, but, for
learning's sake, let's split these branches into four helper functions --- one
for each case.

\begin{Shaded}
\begin{Highlighting}[]
\CommentTok{-- source: https://github.com/mstksg/inCode/tree/master/code-samples/ttt/Part1.hs#L129-L164}

\NormalTok{selFound}
\OtherTok{    ::} \DataTypeTok{Sing}\NormalTok{ n}
    \OtherTok{->} \DataTypeTok{Sing}\NormalTok{ xs}
    \OtherTok{->} \DataTypeTok{Decision}\NormalTok{ (}\DataTypeTok{SelFound}\NormalTok{ n }\FunctionTok{@@}\NormalTok{ xs)}
\NormalTok{selFound }\FunctionTok{=}\NormalTok{ \textbackslash{}}\KeywordTok{case}
    \DataTypeTok{SZ} \OtherTok{->}\NormalTok{ \textbackslash{}}\KeywordTok{case}
      \DataTypeTok{SNil}         \OtherTok{->}\NormalTok{ selFound_znil}
\NormalTok{      x }\OtherTok{`SCons`}\NormalTok{ xs }\OtherTok{->}\NormalTok{ selFound_zcons x xs}
    \DataTypeTok{SS}\NormalTok{ n }\OtherTok{->}\NormalTok{ \textbackslash{}}\KeywordTok{case}
      \DataTypeTok{SNil}         \OtherTok{->}\NormalTok{ selFound_snil n}
\NormalTok{      x }\OtherTok{`SCons`}\NormalTok{ xs }\OtherTok{->}\NormalTok{ selFound_scons n x xs}

\NormalTok{selFound_znil}
\OtherTok{    ::} \DataTypeTok{Decision}\NormalTok{ (}\DataTypeTok{SelFound}\NormalTok{ '}\DataTypeTok{Z} \FunctionTok{@@}\NormalTok{ '[])}

\NormalTok{selFound_zcons}
\OtherTok{    ::} \DataTypeTok{Sing}\NormalTok{ x}
    \OtherTok{->} \DataTypeTok{Sing}\NormalTok{ xs}
    \OtherTok{->} \DataTypeTok{Decision}\NormalTok{ (}\DataTypeTok{SelFound}\NormalTok{ '}\DataTypeTok{Z} \FunctionTok{@@}\NormalTok{ (x '}\FunctionTok{:}\NormalTok{ xs))}

\NormalTok{selFound_snil}
\OtherTok{    ::} \DataTypeTok{Sing}\NormalTok{ n}
    \OtherTok{->} \DataTypeTok{Decision}\NormalTok{ (}\DataTypeTok{SelFound}\NormalTok{ ('}\DataTypeTok{S}\NormalTok{ n) }\FunctionTok{@@}\NormalTok{ '[])}

\NormalTok{selFound_scons}
\OtherTok{    ::} \DataTypeTok{Sing}\NormalTok{ n}
    \OtherTok{->} \DataTypeTok{Sing}\NormalTok{ x}
    \OtherTok{->} \DataTypeTok{Sing}\NormalTok{ xs}
    \OtherTok{->} \DataTypeTok{Decision}\NormalTok{ (}\DataTypeTok{SelFound}\NormalTok{ ('}\DataTypeTok{S}\NormalTok{ n) }\FunctionTok{@@}\NormalTok{ (x '}\FunctionTok{:}\NormalTok{ xs))}
\end{Highlighting}
\end{Shaded}

\begin{enumerate}
\def\labelenumi{\arabic{enumi}.}
\item
  For the first branch, we have \texttt{\textquotesingle{}Z} and
  \texttt{\textquotesingle{}{[}{]}}. This should be false, because there is no
  item in the zeroth position in \texttt{{[}{]}}. But, also, there is no way to
  construct the \texttt{Sel} necessary for the witness, since there is no
  constructor for \texttt{Sel} that gives \texttt{\textquotesingle{}{[}{]}}.

  We can witness this by using a \emph{total} helper function,
  \texttt{noEmptySel\ ::\ Sel\ n\ \textquotesingle{}{[}{]}\ a\ -\textgreater{}\ Void},
  which is a ``disproof'' of the fact that an empty \texttt{Sel} can exist. It's
  a disproof because, if we \emph{had} such a \texttt{Sel}, we could produce
  \texttt{Void} with it\ldots{}but \texttt{Void} has no constructors. So no such
  \texttt{Sel} must exist!

  We implement it by pattern matching on all potential patterns (using the
  \emph{-XLambdaCase} extension):

\begin{Shaded}
\begin{Highlighting}[]
\CommentTok{-- source: https://github.com/mstksg/inCode/tree/master/code-samples/ttt/Part1.hs#L141-L143}

\OtherTok{noEmptySel ::} \DataTypeTok{Sel}\NormalTok{ n '[] a }\OtherTok{->} \DataTypeTok{Void}
\NormalTok{noEmptySel }\FunctionTok{=}\NormalTok{ \textbackslash{}}\KeywordTok{case}\NormalTok{ \{\}}
            \CommentTok{-- ^ we handle all 0 of the valid patterns for Sel n '[] a}
\end{Highlighting}
\end{Shaded}

  \texttt{noEmptySel} succesfuly implements
  \texttt{Sel\ n\ \textquotesingle{}{[}{]}\ as} by succesfully matching on every
  legal constructor that could produce
  \texttt{Sel\ n\ \textquotesingle{}{[}{]}\ as}. But, because there are no
  constructors for \texttt{Sel} that produce
  \texttt{Sel\ n\ \textquotesingle{}{[}{]}\ as} (we just have \texttt{SelZ} and
  \texttt{SelS}, which both produce non-empty \texttt{Sel}s), that means we have
  to handle all \emph{zero} legal constructors. Once we handle all zero legal
  constructors, we're done! (Remember to enable
  \emph{-Werror=incomplete-patterns} to be sure! GHC will then reject the
  program if there is a pattern we do not handle)

  So we can write this as \texttt{Disproved}, which takes a
  \texttt{SelFound\ \textquotesingle{}Z\ @@\ \textquotesingle{}{[}{]}\ -\textgreater{}\ Void}:

\begin{Shaded}
\begin{Highlighting}[]
\CommentTok{-- source: https://github.com/mstksg/inCode/tree/master/code-samples/ttt/Part1.hs#L145-L147}

\NormalTok{selFound_znil}
\OtherTok{    ::} \DataTypeTok{Decision}\NormalTok{ (}\DataTypeTok{SelFound}\NormalTok{ '}\DataTypeTok{Z} \FunctionTok{@@}\NormalTok{ '[])}
\NormalTok{selFound_znil }\FunctionTok{=} \DataTypeTok{Disproved} \FunctionTok{$}\NormalTok{ \textbackslash{}(_ }\FunctionTok{:&:}\NormalTok{ s) }\OtherTok{->}\NormalTok{ noEmptySel s}
\end{Highlighting}
\end{Shaded}

  Armed with the \texttt{Sel\ \textquotesingle{}Z\ \textquotesingle{}{[}{]}\ as}
  that is inside the
  \texttt{SelFound\ \textquotesingle{}Z\ @@\ \textquotesingle{}{[}{]}}, we can
  use \texttt{noEmptySel} to produce the \texttt{Void}. We succefully disprove
  the fact that there is any item that can be found in
  \texttt{\textquotesingle{}{[}{]}}, by providing a function
  \texttt{SelFound\ \textquotesingle{}Z\ @@\ \textquotesingle{}{[}{]}\ -\textgreater{}\ Void}.
\item
  For the second branch, we have \texttt{\textquotesingle{}Z} and
  \texttt{(x\ \textquotesingle{}:\ xs)}. We want to prove that there exists an
  item at position \texttt{\textquotesingle{}Z} in the list
  \texttt{x\ \textquotesingle{}:\ xs}. The answer is \emph{yes}, there does, and
  that item is \texttt{x}, and the \texttt{Sel} is \texttt{SelZ}!

\begin{Shaded}
\begin{Highlighting}[]
\CommentTok{-- source: https://github.com/mstksg/inCode/tree/master/code-samples/ttt/Part1.hs#L149-L153}

\NormalTok{selFound_zcons}
\OtherTok{    ::} \DataTypeTok{Sing}\NormalTok{ x}
    \OtherTok{->} \DataTypeTok{Sing}\NormalTok{ xs}
    \OtherTok{->} \DataTypeTok{Decision}\NormalTok{ (}\DataTypeTok{SelFound}\NormalTok{ '}\DataTypeTok{Z} \FunctionTok{@@}\NormalTok{ (x '}\FunctionTok{:}\NormalTok{ xs))}
\NormalTok{selFound_zcons x _ }\FunctionTok{=} \DataTypeTok{Proved}\NormalTok{ (x }\FunctionTok{:&:} \DataTypeTok{SelZ}\NormalTok{)}
\end{Highlighting}
\end{Shaded}
\item
  For the third branch, we have \texttt{\textquotesingle{}S\ n} and
  \texttt{\textquotesingle{}{[}{]}}. Again, this should be false, because there
  is no item in the \texttt{\textquotesingle{}S\ n} position in
  \texttt{\textquotesingle{}{[}{]}}. We should be able to use the same strategy
  for the first branch, by re-using our helper function \texttt{noEmptySel}:

\begin{Shaded}
\begin{Highlighting}[]
\CommentTok{-- source: https://github.com/mstksg/inCode/tree/master/code-samples/ttt/Part1.hs#L155-L158}

\NormalTok{selFound_snil}
\OtherTok{    ::} \DataTypeTok{Sing}\NormalTok{ n}
    \OtherTok{->} \DataTypeTok{Decision}\NormalTok{ (}\DataTypeTok{SelFound}\NormalTok{ ('}\DataTypeTok{S}\NormalTok{ n) }\FunctionTok{@@}\NormalTok{ '[])}
\NormalTok{selFound_snil _ }\FunctionTok{=} \DataTypeTok{Disproved} \FunctionTok{$}\NormalTok{ \textbackslash{}(_ }\FunctionTok{:&:}\NormalTok{ s) }\OtherTok{->}\NormalTok{ noEmptySel s}
\end{Highlighting}
\end{Shaded}
\item
  The fourth branch is the most interesting one. We have
  \texttt{\textquotesingle{}S\ n} and \texttt{(x\ \textquotesingle{}:\ xs)}. How
  do we know if the list \texttt{x\ \textquotesingle{}:\ xs} has an item in the
  \texttt{\textquotesingle{}S\ n} spot?

  Well, we can check if the list \texttt{xs} has an item in its \texttt{n} spot.

  \begin{itemize}
  \item
    If it does, then call that item \texttt{y}, and we know that
    \texttt{x\ \textquotesingle{}:\ xs} has \texttt{y} in its
    \texttt{\textquotesingle{}S\ n} spot.
  \item
    If it doesn't, then we can't have an item at \texttt{\textquotesingle{}S\ n}
    spot in \texttt{x\ \textquotesingle{}:\ xs} either! To show why, we can do a
    proof by contradiction.

    Suppose there \emph{was} an item \texttt{y} at the
    \texttt{\textquotesingle{}S\ n} spot in \texttt{x\ \textquotesingle{}:\ xs}.
    If so, then that means that there would be an item \texttt{y} in the
    \texttt{n} spot in \texttt{xs}. However, this was found to be false.
    Therefore, we cannot have an item in the \texttt{\textquotesingle{}S\ n}
    spot in \texttt{x\ \textquotesingle{}:\ xs}.

    This is a situation where having a disproof in the \texttt{Disproved} branch
    is useful --- we use them to build more complex disproofs from simple ones.
  \end{itemize}

\begin{Shaded}
\begin{Highlighting}[]
\CommentTok{-- source: https://github.com/mstksg/inCode/tree/master/code-samples/ttt/Part1.hs#L160-L172}

\NormalTok{selFound_scons}
\OtherTok{    ::} \DataTypeTok{Sing}\NormalTok{ n}
    \OtherTok{->} \DataTypeTok{Sing}\NormalTok{ x}
    \OtherTok{->} \DataTypeTok{Sing}\NormalTok{ xs}
    \OtherTok{->} \DataTypeTok{Decision}\NormalTok{ (}\DataTypeTok{SelFound}\NormalTok{ ('}\DataTypeTok{S}\NormalTok{ n) }\FunctionTok{@@}\NormalTok{ (x '}\FunctionTok{:}\NormalTok{ xs))}
\NormalTok{selFound_scons n _ xs }\FunctionTok{=} \KeywordTok{case}\NormalTok{ selFound n xs }\KeywordTok{of}
    \DataTypeTok{Proved}\NormalTok{ (y }\FunctionTok{:&:}\NormalTok{ s) }\OtherTok{->}       \CommentTok{-- if xs has y in its n spot}
      \DataTypeTok{Proved}\NormalTok{ (y }\FunctionTok{:&:} \DataTypeTok{SelS}\NormalTok{ s)   }\CommentTok{-- then (x : xs) has y in its (S n) spot}
    \DataTypeTok{Disproved}\NormalTok{ v      }\OtherTok{->} \DataTypeTok{Disproved} \FunctionTok{$} \CommentTok{-- v is a disproof that an item is in n spot in xs}
\NormalTok{      \textbackslash{}(y }\FunctionTok{:&:}\NormalTok{ s) }\OtherTok{->}      \CommentTok{-- suppose we had item y in (S n) spot in (x : xs)}
        \KeywordTok{case}\NormalTok{ s }\KeywordTok{of}
          \DataTypeTok{SelS}\NormalTok{ s' }\OtherTok{->}     \CommentTok{-- this would mean that item y is in n spot in xs}
\NormalTok{            v (y }\FunctionTok{:&:}\NormalTok{ s') }\CommentTok{-- however, v disproves this.}
\end{Highlighting}
\end{Shaded}

  If you have problems understanding this, try playing around with typed holes
  in GHC, or trying to guess what types everything has in the implementation
  above, until you can figure out what is happening when.
\end{enumerate}

\hypertarget{proving-pick}{%
\subsection{Proving Pick}\label{proving-pick}}

Now that we can decide \texttt{SelFound}, let's finally prove \texttt{Pick}.

\begin{Shaded}
\begin{Highlighting}[]
\NormalTok{pick}
\OtherTok{    ::}\NormalTok{ forall i j b}\FunctionTok{.}\NormalTok{ ()}
    \OtherTok{=>} \DataTypeTok{Sing}\NormalTok{ i}
    \OtherTok{->} \DataTypeTok{Sing}\NormalTok{ j}
    \OtherTok{->} \DataTypeTok{Sing}\NormalTok{ b}
    \OtherTok{->} \DataTypeTok{Pick}\NormalTok{ '(i, j, b)}
\NormalTok{pick i j b }\FunctionTok{=}
\end{Highlighting}
\end{Shaded}

Remember, the goal is to try to prove we have a valid pick. We want to create
something with the \texttt{PickValid} constructor if we can:

\begin{Shaded}
\begin{Highlighting}[]
\DataTypeTok{PickValid}\OtherTok{  ::} \DataTypeTok{Coord}\NormalTok{ '(i, j) b '}\DataTypeTok{Nothing} \OtherTok{->} \DataTypeTok{Pick}\NormalTok{ '(i, j, b)}

\OtherTok{(:$:) ::} \DataTypeTok{Sel}\NormalTok{ i rows row}
      \OtherTok{->} \DataTypeTok{Sel}\NormalTok{ j row  p}
      \OtherTok{->} \DataTypeTok{Coord}\NormalTok{ '(i, j) rows p}
\end{Highlighting}
\end{Shaded}

So we need a
\texttt{Coord\ \textquotesingle{}(i,\ j)\ b\ \textquotesingle{}Nothing}, which
means we need a \texttt{Sel\ i\ b\ row} and a
\texttt{Sel\ j\ row\ \textquotesingle{}Nothing}. Let's use our decision
functions we wrote to get these! In particular, we can use
\texttt{selFound\ i\ b} to get our \texttt{Sel\ i\ b\ row} and \texttt{row}, and
then use \texttt{selFound\ j\ row} to get our \texttt{Sel\ j\ row\ piece} and
\texttt{piece}!

Note that we also might need to handle the cases where the picks are
out-of-bounds, which requires an \texttt{OuOfBounds\ n} predicate.

\begin{Shaded}
\begin{Highlighting}[]
\DataTypeTok{PickOoBX}\OtherTok{   ::} \DataTypeTok{OutOfBounds}\NormalTok{ i }\FunctionTok{@@}\NormalTok{ b }\OtherTok{->} \DataTypeTok{Pick}\NormalTok{ '(i, j, b)}
\end{Highlighting}
\end{Shaded}

We never defined \texttt{OutOfBounds\ n}, so let's do it now. We know that
\texttt{SelFound\ n} is the predicate where \texttt{SelFound\ n\ @@\ xs} is
satisfied when index \texttt{n} is ``in bounds'' of \texttt{xs}. So,
\texttt{OutOfBounds\ n} is the predicate that we \emph{can't} produce
\texttt{SelFound\ n}; a witness of \texttt{OutOfBounds\ n\ @@\ xs} is a proof
that we can't make \texttt{SelFound\ n\ @@\ xs}:

\begin{Shaded}
\begin{Highlighting}[]
\KeywordTok{data} \DataTypeTok{OutOfBounds}\OtherTok{ ::} \DataTypeTok{N} \OtherTok{->} \DataTypeTok{Predicate}\NormalTok{ [k]}
\KeywordTok{type} \KeywordTok{instance} \DataTypeTok{Apply}\NormalTok{ (}\DataTypeTok{OutOfBounds}\NormalTok{ n) xs }\FunctionTok{=} \DataTypeTok{SelFound}\NormalTok{ n }\FunctionTok{@@}\NormalTok{ xs }\OtherTok{->} \DataTypeTok{Void}
\end{Highlighting}
\end{Shaded}

Remember that \texttt{a\ -\textgreater{}\ Void} is a witness that \texttt{a}
cannot exist, since if it could exist, we could create a value of type
\texttt{Void}, but no such value exists. So if we had a
\texttt{SelFound\ n\ @@\ xs\ -\textgreater{}\ Void}, it means that no such
\texttt{SelFound\ n\ @@\ xs} could exist. It means that \texttt{n} must
\emph{not} be in bounds of \texttt{xs}.

Actually, we could take advantage of some of the combinators that the
\emph{decidable} library provides to define \texttt{OutOfBounds} in a cleaner
way:

\begin{Shaded}
\begin{Highlighting}[]
\CommentTok{-- | Provided by decidable; it's the negation of a predicate}
\KeywordTok{data} \DataTypeTok{Not}\OtherTok{ ::} \DataTypeTok{Predicate}\NormalTok{ k }\OtherTok{->} \DataTypeTok{Predicate}\NormalTok{ k}
\KeywordTok{type} \KeywordTok{instance} \DataTypeTok{Apply}\NormalTok{ (}\DataTypeTok{Not}\NormalTok{ p) x }\FunctionTok{=}\NormalTok{ p }\FunctionTok{@@}\NormalTok{ x }\OtherTok{->} \DataTypeTok{Void}
\CommentTok{-- source: https://github.com/mstksg/inCode/tree/master/code-samples/ttt/Part1.hs#L114-L114}

\KeywordTok{type} \DataTypeTok{OutOfBounds}\NormalTok{ n }\FunctionTok{=} \DataTypeTok{Not}\NormalTok{ (}\DataTypeTok{SelFound}\NormalTok{ n)}
\end{Highlighting}
\end{Shaded}

Alright, now that everything is defined, let's start writing our viewing
function for \texttt{Pick}. Recall again the definition of \texttt{Pick}:

\begin{Shaded}
\begin{Highlighting}[]
\CommentTok{-- source: https://github.com/mstksg/inCode/tree/master/code-samples/ttt/Part1.hs#L117-L127}

\KeywordTok{data} \DataTypeTok{Pick}\OtherTok{ ::}\NormalTok{ (}\DataTypeTok{N}\NormalTok{, }\DataTypeTok{N}\NormalTok{, }\DataTypeTok{Board}\NormalTok{) }\OtherTok{->} \DataTypeTok{Type} \KeywordTok{where}
    \CommentTok{-- | We are out of bounds in x}
    \DataTypeTok{PickOoBX}\OtherTok{   ::} \DataTypeTok{OutOfBounds}\NormalTok{ i }\FunctionTok{@@}\NormalTok{ b                         }\OtherTok{->} \DataTypeTok{Pick}\NormalTok{ '(i, j, b)}
    \CommentTok{-- | We are in-bounds in x, but out of bounds in y}
    \DataTypeTok{PickOoBY}\OtherTok{   ::} \DataTypeTok{Sel}\NormalTok{ i b row        }\OtherTok{->} \DataTypeTok{OutOfBounds}\NormalTok{ j }\FunctionTok{@@}\NormalTok{ row }\OtherTok{->} \DataTypeTok{Pick}\NormalTok{ '(i, j, b)}
    \CommentTok{-- | We are in-bounds in x, in-bounds in y, but spot is taken by `p`.}
    \CommentTok{-- We include `Sing p` in this constructor to potentially provide}
    \CommentTok{-- feedback to the user on what piece is already in the spot.}
    \DataTypeTok{PickPlayed}\OtherTok{ ::} \DataTypeTok{Coord}\NormalTok{ '(i, j) b ('}\DataTypeTok{Just}\NormalTok{ p) }\OtherTok{->} \DataTypeTok{Sing}\NormalTok{ p        }\OtherTok{->} \DataTypeTok{Pick}\NormalTok{ '(i, j, b)}
    \CommentTok{-- | We are in-bounds in x, in-bounds in y, and spot is clear}
    \DataTypeTok{PickValid}\OtherTok{  ::} \DataTypeTok{Coord}\NormalTok{ '(i, j) b '}\DataTypeTok{Nothing}                   \OtherTok{->} \DataTypeTok{Pick}\NormalTok{ '(i, j, b)}
\end{Highlighting}
\end{Shaded}

And let's start writing. First, we'll use our decision functions
\texttt{selFound} to \emph{try} to produce the \texttt{Sel}s necessary for the
\texttt{PickValid} branch:

\begin{Shaded}
\begin{Highlighting}[]
\NormalTok{pick}
\OtherTok{    ::}\NormalTok{ forall i j b}\FunctionTok{.}\NormalTok{ ()}
    \OtherTok{=>} \DataTypeTok{Sing}\NormalTok{ i}
    \OtherTok{->} \DataTypeTok{Sing}\NormalTok{ j}
    \OtherTok{->} \DataTypeTok{Sing}\NormalTok{ b}
    \OtherTok{->} \DataTypeTok{Pick}\NormalTok{ '(i, j, b)}
\NormalTok{pick i j b }\FunctionTok{=} \KeywordTok{case}\NormalTok{ selFound i b }\KeywordTok{of}
    \DataTypeTok{Proved}\NormalTok{ (row }\FunctionTok{:&:}\NormalTok{ selX) }\OtherTok{->} \KeywordTok{case}\NormalTok{ selFound j row }\KeywordTok{of}
      \DataTypeTok{Proved}\NormalTok{ (p }\FunctionTok{:&:}\NormalTok{ selY) }\OtherTok{->}
        \KeywordTok{let}\NormalTok{ c }\FunctionTok{=}\NormalTok{ selX }\FunctionTok{:$:}\NormalTok{ selY}
        \KeywordTok{in}  \FunctionTok{????}
\end{Highlighting}
\end{Shaded}

In the case where both \texttt{i} and \texttt{j} are in-bounds, what do we do?
Well, we can create the \texttt{Coord} from the \texttt{selX} and \texttt{selY}.
Can't we just give this to \texttt{PickValid}?

\begin{Shaded}
\begin{Highlighting}[]
\NormalTok{pick}
\OtherTok{    ::}\NormalTok{ forall i j b}\FunctionTok{.}\NormalTok{ ()}
    \OtherTok{=>} \DataTypeTok{Sing}\NormalTok{ i}
    \OtherTok{->} \DataTypeTok{Sing}\NormalTok{ j}
    \OtherTok{->} \DataTypeTok{Sing}\NormalTok{ b}
    \OtherTok{->} \DataTypeTok{Pick}\NormalTok{ '(i, j, b)}
\NormalTok{pick i j b }\FunctionTok{=} \KeywordTok{case}\NormalTok{ selFound i b }\KeywordTok{of}
    \DataTypeTok{Proved}\NormalTok{ (row }\FunctionTok{:&:}\NormalTok{ selX) }\OtherTok{->} \KeywordTok{case}\NormalTok{ selFound j row }\KeywordTok{of}
      \DataTypeTok{Proved}\NormalTok{ (p }\FunctionTok{:&:}\NormalTok{ selY) }\OtherTok{->}
        \KeywordTok{let}\NormalTok{ c }\FunctionTok{=}\NormalTok{ selX }\FunctionTok{:$:}\NormalTok{ selY}
        \KeywordTok{in}  \DataTypeTok{PickValid}\NormalTok{ c}
\end{Highlighting}
\end{Shaded}

Ah, but this is a compiler error. It's hard to see why without a trusty compiler
to help us. But that's the best part about using dependent types --- the
compilers are here to help us write our programs!

The compiler basically tells us that \texttt{c} is supposed to point to
\texttt{\textquotesingle{}Nothing}\ldots{}but right now it's pointing to some
type variable that we don't know is \texttt{\textquotesingle{}Nothing} or
\texttt{\textquotesingle{}Just}. So \texttt{c} doesn't \emph{necessarily} work,
because it \emph{might} be pointing to \texttt{\textquotesingle{}Just}, and not
\texttt{\textquotesingle{}Nothing}. We haven't satisfied \texttt{PickValid} yet,
because we have to be certain that it is \texttt{\textquotesingle{}Nothing} and
not \texttt{\textquotesingle{}Just}

Just to clarify what's going on, let's give types to the names above:

\begin{Shaded}
\begin{Highlighting}[]
\OtherTok{b    ::} \DataTypeTok{Sing}\NormalTok{ (}\OtherTok{b   ::}\NormalTok{ [[}\DataTypeTok{Maybe} \DataTypeTok{Piece}\NormalTok{]])}
\OtherTok{row  ::} \DataTypeTok{Sing}\NormalTok{ (}\OtherTok{row ::}\NormalTok{  [}\DataTypeTok{Maybe} \DataTypeTok{Piece}\NormalTok{] )}
\OtherTok{selX ::} \DataTypeTok{Sel}\NormalTok{ i b row}
\OtherTok{p    ::} \DataTypeTok{Sing}\NormalTok{ (}\OtherTok{p   ::}   \DataTypeTok{Maybe} \DataTypeTok{Piece}\NormalTok{  )}
\OtherTok{selY ::} \DataTypeTok{Sel}\NormalTok{ j row p}
\OtherTok{c    ::} \DataTypeTok{Coord}\NormalTok{ '(i, j) b p}
\end{Highlighting}
\end{Shaded}

\texttt{row} above is the \texttt{Sing} that comes attached with all \texttt{Σ}
constructors, which is why we can give it to \texttt{selFound\ j}, which expects
a singleton of the list.

So, now we have \texttt{Coord\ \textquotesingle{}(i,\ j)\ b\ p}. We know that
\texttt{i} and \texttt{j} are in-bounds. But, we need to know that \texttt{p} is
\texttt{\textquotesingle{}Nothing} before we can use it with \texttt{PickValid}.
To do that, we can pattern match on \texttt{p}, because it's the singleton that
comes with the \texttt{Σ} constructor:

\begin{Shaded}
\begin{Highlighting}[]
\NormalTok{pick}
\OtherTok{    ::}\NormalTok{ forall i j b}\FunctionTok{.}\NormalTok{ ()}
    \OtherTok{=>} \DataTypeTok{Sing}\NormalTok{ i}
    \OtherTok{->} \DataTypeTok{Sing}\NormalTok{ j}
    \OtherTok{->} \DataTypeTok{Sing}\NormalTok{ b}
    \OtherTok{->} \DataTypeTok{Pick}\NormalTok{ '(i, j, b)}
\NormalTok{pick i j b }\FunctionTok{=} \KeywordTok{case}\NormalTok{ selFound i b }\KeywordTok{of}
    \DataTypeTok{Proved}\NormalTok{ (row }\FunctionTok{:&:}\NormalTok{ selX) }\OtherTok{->} \KeywordTok{case}\NormalTok{ selFound j row }\KeywordTok{of}
      \DataTypeTok{Proved}\NormalTok{ (p }\FunctionTok{:&:}\NormalTok{ selY) }\OtherTok{->}
        \KeywordTok{let}\NormalTok{ c }\FunctionTok{=}\NormalTok{ selX }\FunctionTok{:$:}\NormalTok{ selY}
        \KeywordTok{in}  \KeywordTok{case}\NormalTok{ p }\KeywordTok{of}
              \DataTypeTok{SNothing} \OtherTok{->} \DataTypeTok{PickValid}\NormalTok{   c     }\CommentTok{-- p is 'Nothing}
              \DataTypeTok{SJust}\NormalTok{ q  }\OtherTok{->} \DataTypeTok{PickPlayed}\NormalTok{  c q   }\CommentTok{-- p is 'Just q}
\end{Highlighting}
\end{Shaded}

Finally, knowing that \texttt{p} is \texttt{\textquotesingle{}Nothing}, we can
create \texttt{PickValid}!

As a bonus, if we know that \texttt{p} is \texttt{\textquotesingle{}Just\ p}, we
can create \texttt{PickPlayed}, which is the constructor for an in-bounds pick
but pointing to a spot that is already occupied by piece
\texttt{p\textquotesingle{}}.

\begin{Shaded}
\begin{Highlighting}[]
\DataTypeTok{PickPlayed}\OtherTok{ ::} \DataTypeTok{Coord}\NormalTok{ '(i, j) b ('}\DataTypeTok{Just}\NormalTok{ p)}
           \OtherTok{->} \DataTypeTok{Sing}\NormalTok{ p}
           \OtherTok{->} \DataTypeTok{Pick}\NormalTok{ '(i, j, b)}
\end{Highlighting}
\end{Shaded}

We now have to deal with the situations where things are out of bounds.

\begin{Shaded}
\begin{Highlighting}[]
\DataTypeTok{PickOoBX}\OtherTok{ ::} \DataTypeTok{OutOfBounds}\NormalTok{ i }\FunctionTok{@@}\NormalTok{ b}
         \OtherTok{->} \DataTypeTok{Pick}\NormalTok{ '(i, j, b)}
\DataTypeTok{PickOoBY}\OtherTok{ ::} \DataTypeTok{Sel}\NormalTok{ i b row}
         \OtherTok{->} \DataTypeTok{OutOfBounds}\NormalTok{ j }\FunctionTok{@@}\NormalTok{ row}
         \OtherTok{->} \DataTypeTok{Pick}\NormalTok{ '(i, j, b)}
\end{Highlighting}
\end{Shaded}

\texttt{PickOoBX} requires an \texttt{OutOfBounds\ i\ @@\ b}, which we defined
as \texttt{SelFound\ i\ @@\ b\ -\textgreater{}\ Void}. Well, that's
\emph{exactly} what the \texttt{Disproved} constructor contains, that
\texttt{selFound\ i\ b} returns! And \texttt{PickOoBY} requires an
\texttt{OutOfBounds\ j\ @@\ row}, which we defined as
\texttt{SelFound\ j\ @@\ row\ -\textgreater{}\ Void}. And that's \emph{exactly}
what the \texttt{Disproved} constructor of \texttt{selFound\ j\ row} returns.

\begin{Shaded}
\begin{Highlighting}[]
\CommentTok{-- source: https://github.com/mstksg/inCode/tree/master/code-samples/ttt/Part1.hs#L182-L200}

\NormalTok{pick}
\OtherTok{    ::}\NormalTok{ forall i j b}\FunctionTok{.}\NormalTok{ ()}
    \OtherTok{=>} \DataTypeTok{Sing}\NormalTok{ i}
    \OtherTok{->} \DataTypeTok{Sing}\NormalTok{ j}
    \OtherTok{->} \DataTypeTok{Sing}\NormalTok{ b}
    \OtherTok{->} \DataTypeTok{Pick}\NormalTok{ '(i, j, b)}
\NormalTok{pick i j b }\FunctionTok{=} \KeywordTok{case}\NormalTok{ selFound i b }\KeywordTok{of}
    \DataTypeTok{Proved}\NormalTok{ (row }\FunctionTok{:&:}\NormalTok{ selX) }\OtherTok{->} \KeywordTok{case}\NormalTok{ selFound j row }\KeywordTok{of}
      \DataTypeTok{Proved}\NormalTok{ (p }\FunctionTok{:&:}\NormalTok{ selY) }\OtherTok{->}
        \KeywordTok{let}\NormalTok{ c }\FunctionTok{=}\NormalTok{ selX }\FunctionTok{:$:}\NormalTok{ selY}
        \KeywordTok{in}  \KeywordTok{case}\NormalTok{ p }\KeywordTok{of}
              \DataTypeTok{SNothing} \OtherTok{->} \DataTypeTok{PickValid}\NormalTok{   c     }\CommentTok{-- p is 'Nothing}
              \DataTypeTok{SJust}\NormalTok{ q  }\OtherTok{->} \DataTypeTok{PickPlayed}\NormalTok{  c q   }\CommentTok{-- p is 'Just q}
      \DataTypeTok{Disproved}\NormalTok{ vY }\OtherTok{->} \DataTypeTok{PickOoBY}\NormalTok{ selX vY    }\CommentTok{-- vY :: SelFound j @@ row -> Void}
                                          \CommentTok{-- vY :: Not (SelFound j) @@ row}
                                          \CommentTok{-- vY :: OutOfBounds j @@ row}
    \DataTypeTok{Disproved}\NormalTok{ vX }\OtherTok{->} \DataTypeTok{PickOoBX}\NormalTok{ vX   }\CommentTok{-- vX :: SelFound i @@ b   -> Void}
                                  \CommentTok{-- vX :: Not (SelFound i) @@ b}
                                  \CommentTok{-- vX :: OutOfBounds i @@ b}
\end{Highlighting}
\end{Shaded}

And that's it!

\hypertarget{play-ball}{%
\section{Play Ball}\label{play-ball}}

Bringing it all together, we can write a simple function to take user input and
\emph{play} it.

First, some utility functions to get user input and print out boards:

\begin{Shaded}
\begin{Highlighting}[]
\CommentTok{-- source: https://github.com/mstksg/inCode/tree/master/code-samples/ttt/Part1.hs#L202-L221}

\OtherTok{intToN ::} \DataTypeTok{Int} \OtherTok{->} \DataTypeTok{Maybe} \DataTypeTok{N}
\NormalTok{intToN n }\FunctionTok{=} \KeywordTok{case}\NormalTok{ compare n }\DecValTok{0} \KeywordTok{of}
    \DataTypeTok{LT} \OtherTok{->} \DataTypeTok{Nothing}
    \DataTypeTok{EQ} \OtherTok{->} \DataTypeTok{Just} \DataTypeTok{Z}
    \DataTypeTok{GT} \OtherTok{->} \DataTypeTok{S} \FunctionTok{<$>}\NormalTok{ intToN (n }\FunctionTok{-} \DecValTok{1}\NormalTok{)}

\OtherTok{getN ::} \DataTypeTok{String} \OtherTok{->} \DataTypeTok{IO} \DataTypeTok{N}
\NormalTok{getN prompt }\FunctionTok{=} \KeywordTok{do}
\NormalTok{    putStrLn }\FunctionTok{$} \StringTok{"Enter non-negative integer "} \FunctionTok{++}\NormalTok{ prompt }\FunctionTok{++} \StringTok{":"}
\NormalTok{    res }\OtherTok{<-}\NormalTok{ getLine}
    \KeywordTok{case}\NormalTok{ intToN }\FunctionTok{=<<}\NormalTok{ readMaybe res }\KeywordTok{of}
      \DataTypeTok{Nothing} \OtherTok{->}\NormalTok{ putStrLn }\StringTok{"Bad."} \FunctionTok{>>}\NormalTok{ getN prompt}
      \DataTypeTok{Just}\NormalTok{ n  }\OtherTok{->}\NormalTok{ pure n}

\OtherTok{printBoard ::} \DataTypeTok{Board} \OtherTok{->} \DataTypeTok{IO}\NormalTok{ ()}
\NormalTok{printBoard }\FunctionTok{=}\NormalTok{ mapM_ }\FunctionTok{$}\NormalTok{ putStrLn }\FunctionTok{.}\NormalTok{ intercalate }\StringTok{"|"} \FunctionTok{.}\NormalTok{ map showPiece}
  \KeywordTok{where}
\NormalTok{    showPiece }\DataTypeTok{Nothing}   \FunctionTok{=} \StringTok{" _ "}
\NormalTok{    showPiece (}\DataTypeTok{Just} \DataTypeTok{PX}\NormalTok{) }\FunctionTok{=} \StringTok{" X "}
\NormalTok{    showPiece (}\DataTypeTok{Just} \DataTypeTok{PO}\NormalTok{) }\FunctionTok{=} \StringTok{" O "}
\end{Highlighting}
\end{Shaded}

And here is the logic for getting user input, viewing it using \texttt{pick},
and updating the \texttt{GameState}:

\begin{Shaded}
\begin{Highlighting}[]
\CommentTok{-- source: https://github.com/mstksg/inCode/tree/master/code-samples/ttt/Part1.hs#L226-L250}

\NormalTok{simplePlayIO'}
\OtherTok{    ::} \DataTypeTok{Sing}\NormalTok{ p}
    \OtherTok{->} \DataTypeTok{Sing}\NormalTok{ b}
    \OtherTok{->} \DataTypeTok{GameState}\NormalTok{ p b}
    \OtherTok{->} \DataTypeTok{IO}\NormalTok{ ()}
\NormalTok{simplePlayIO' p b gs }\FunctionTok{=} \KeywordTok{do}
\NormalTok{    printBoard }\FunctionTok{$} \DataTypeTok{FromSing}\NormalTok{ b}
    \DataTypeTok{FromSing}\NormalTok{ i }\OtherTok{<-}\NormalTok{ getN }\StringTok{"for row"}
    \DataTypeTok{FromSing}\NormalTok{ j }\OtherTok{<-}\NormalTok{ getN }\StringTok{"for column"}
    \KeywordTok{case}\NormalTok{ pick i j b }\KeywordTok{of}
      \DataTypeTok{PickOoBX}\NormalTok{ _ }\OtherTok{->} \KeywordTok{do}
\NormalTok{        putStrLn }\StringTok{"Out of bounds in rows.  Try again."}
\NormalTok{        simplePlayIO' p b gs}
      \DataTypeTok{PickOoBY}\NormalTok{ _ _ }\OtherTok{->} \KeywordTok{do}
\NormalTok{        putStrLn }\StringTok{"Out of bounds in cols.  Try again."}
\NormalTok{        simplePlayIO' p b gs}
      \DataTypeTok{PickPlayed}\NormalTok{ _ q }\OtherTok{->} \KeywordTok{do}
\NormalTok{        putStrLn }\FunctionTok{$} \StringTok{"Already played by "} \FunctionTok{++}\NormalTok{ show (fromSing q) }\FunctionTok{++} \StringTok{". Try again."}
\NormalTok{        simplePlayIO' p b gs}
      \DataTypeTok{PickValid}\NormalTok{ c }\OtherTok{->} \KeywordTok{do}
\NormalTok{        putStrLn }\StringTok{"Success!"}
        \KeywordTok{let}\NormalTok{ p'  }\FunctionTok{=}\NormalTok{ sAltP p                 }\CommentTok{-- update player (enforced by `play`)}
\NormalTok{            b'  }\FunctionTok{=}\NormalTok{ sPlaceBoard i j p b     }\CommentTok{-- update board  (enforced by `play`)}
\NormalTok{            gs' }\FunctionTok{=}\NormalTok{ play undefined c gs     }\CommentTok{-- update game state}
\NormalTok{        simplePlayIO' p' b' gs'}
\end{Highlighting}
\end{Shaded}

We use the \texttt{FromSing\ ::\ Sing\ (x\ ::\ a)\ -\textgreater{}\ a} pattern
synonym here to jump between the value-level and type-level with our values.
First we use it as a \texttt{Sing\ (b\ ::\ Board)\ -\textgreater{}\ Board} to
give us the \texttt{Board} that \texttt{printBoard} demands. Then we use it as a
constructor to ``get'' a \texttt{Sing\ (i\ ::\ N)} from the value-level
\texttt{N} that \texttt{getN} returns. If \texttt{FromSing\ x\ ::\ a}, then
\texttt{x} is the singleton of \texttt{FromSing\ x}. That is,
\texttt{True\ ==\ FromSing\ STrue}, and \texttt{S\ Z\ ==\ FromSing\ (SS\ SZ)}.

Note that the type of \texttt{play} enforces that we modify \texttt{p} and
\texttt{b} according to nothing other than exactly what the type of a new board
game demands.

And to start it off, we give \texttt{simplePlayIO\textquotesingle{}} an initial
state:

\begin{Shaded}
\begin{Highlighting}[]
\CommentTok{-- source: https://github.com/mstksg/inCode/tree/master/code-samples/ttt/Part1.hs#L223-L224}

\OtherTok{simplePlayIO ::} \DataTypeTok{IO}\NormalTok{ ()}
\NormalTok{simplePlayIO }\FunctionTok{=}\NormalTok{ simplePlayIO' }\DataTypeTok{SPX}\NormalTok{ sEmptyBoard }\DataTypeTok{GSStart}
\end{Highlighting}
\end{Shaded}

This isn't too bad! A type-safe tic-tac-toe that enforces that:

\begin{enumerate}
\def\labelenumi{\arabic{enumi}.}
\tightlist
\item
  Players alternate
\item
  You can't place a piece not on the board
\item
  You can't place a piece over an existing piece
\end{enumerate}

\begin{verbatim}
ghci> simplePlayIO
 _ | _ | _
 _ | _ | _
 _ | _ | _
Enter non-negative integer for row:
10
Enter non-negative integer for column:
100
Out of bounds in rows.  Try again.
 _ | _ | _
 _ | _ | _
 _ | _ | _
Enter non-negative integer for row:
0
Enter non-negative integer for column:
0
Success!
 X | _ | _
 _ | _ | _
 _ | _ | _
Enter non-negative integer for row:
1
Enter non-negative integer for column:
1
Success!
 X | _ | _
 _ | O | _
 _ | _ | _
Enter non-negative integer for row:
1
Enter non-negative integer for column:
1
Already played by PO. Try again.
 X | _ | _
 _ | O | _
 _ | _ | _
Enter non-negative integer for row:
^C
\end{verbatim}

However, this still lets you play on \emph{after} a game has already won. To
prevent this, we must finally start implementing \texttt{InPlay}.

\hypertarget{signoff}{%
\section{Signoff}\label{signoff}}

Hi, thanks for reading! You can reach me via email at
\href{mailto:justin@jle.im}{\nolinkurl{justin@jle.im}}, or at twitter at
\href{https://twitter.com/mstk}{@mstk}! This post and all others are published
under the \href{https://creativecommons.org/licenses/by-nc-nd/3.0/}{CC-BY-NC-ND
3.0} license. Corrections and edits via pull request are welcome and encouraged
at \href{https://github.com/mstksg/inCode}{the source repository}.

If you feel inclined, or this post was particularly helpful for you, why not
consider \href{https://www.patreon.com/justinle/overview}{supporting me on
Patreon}, or a \href{bitcoin:3D7rmAYgbDnp4gp4rf22THsGt74fNucPDU}{BTC donation}?
:)

\end{document}
