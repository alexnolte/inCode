\documentclass[]{article}
\usepackage{lmodern}
\usepackage{amssymb,amsmath}
\usepackage{ifxetex,ifluatex}
\usepackage{fixltx2e} % provides \textsubscript
\ifnum 0\ifxetex 1\fi\ifluatex 1\fi=0 % if pdftex
  \usepackage[T1]{fontenc}
  \usepackage[utf8]{inputenc}
\else % if luatex or xelatex
  \ifxetex
    \usepackage{mathspec}
    \usepackage{xltxtra,xunicode}
  \else
    \usepackage{fontspec}
  \fi
  \defaultfontfeatures{Mapping=tex-text,Scale=MatchLowercase}
  \newcommand{\euro}{€}
\fi
% use upquote if available, for straight quotes in verbatim environments
\IfFileExists{upquote.sty}{\usepackage{upquote}}{}
% use microtype if available
\IfFileExists{microtype.sty}{\usepackage{microtype}}{}
\usepackage[margin=1in]{geometry}
\usepackage{color}
\usepackage{fancyvrb}
\newcommand{\VerbBar}{|}
\newcommand{\VERB}{\Verb[commandchars=\\\{\}]}
\DefineVerbatimEnvironment{Highlighting}{Verbatim}{commandchars=\\\{\}}
% Add ',fontsize=\small' for more characters per line
\newenvironment{Shaded}{}{}
\newcommand{\AlertTok}[1]{\textcolor[rgb]{1.00,0.00,0.00}{\textbf{#1}}}
\newcommand{\AnnotationTok}[1]{\textcolor[rgb]{0.38,0.63,0.69}{\textbf{\textit{#1}}}}
\newcommand{\AttributeTok}[1]{\textcolor[rgb]{0.49,0.56,0.16}{#1}}
\newcommand{\BaseNTok}[1]{\textcolor[rgb]{0.25,0.63,0.44}{#1}}
\newcommand{\BuiltInTok}[1]{#1}
\newcommand{\CharTok}[1]{\textcolor[rgb]{0.25,0.44,0.63}{#1}}
\newcommand{\CommentTok}[1]{\textcolor[rgb]{0.38,0.63,0.69}{\textit{#1}}}
\newcommand{\CommentVarTok}[1]{\textcolor[rgb]{0.38,0.63,0.69}{\textbf{\textit{#1}}}}
\newcommand{\ConstantTok}[1]{\textcolor[rgb]{0.53,0.00,0.00}{#1}}
\newcommand{\ControlFlowTok}[1]{\textcolor[rgb]{0.00,0.44,0.13}{\textbf{#1}}}
\newcommand{\DataTypeTok}[1]{\textcolor[rgb]{0.56,0.13,0.00}{#1}}
\newcommand{\DecValTok}[1]{\textcolor[rgb]{0.25,0.63,0.44}{#1}}
\newcommand{\DocumentationTok}[1]{\textcolor[rgb]{0.73,0.13,0.13}{\textit{#1}}}
\newcommand{\ErrorTok}[1]{\textcolor[rgb]{1.00,0.00,0.00}{\textbf{#1}}}
\newcommand{\ExtensionTok}[1]{#1}
\newcommand{\FloatTok}[1]{\textcolor[rgb]{0.25,0.63,0.44}{#1}}
\newcommand{\FunctionTok}[1]{\textcolor[rgb]{0.02,0.16,0.49}{#1}}
\newcommand{\ImportTok}[1]{#1}
\newcommand{\InformationTok}[1]{\textcolor[rgb]{0.38,0.63,0.69}{\textbf{\textit{#1}}}}
\newcommand{\KeywordTok}[1]{\textcolor[rgb]{0.00,0.44,0.13}{\textbf{#1}}}
\newcommand{\NormalTok}[1]{#1}
\newcommand{\OperatorTok}[1]{\textcolor[rgb]{0.40,0.40,0.40}{#1}}
\newcommand{\OtherTok}[1]{\textcolor[rgb]{0.00,0.44,0.13}{#1}}
\newcommand{\PreprocessorTok}[1]{\textcolor[rgb]{0.74,0.48,0.00}{#1}}
\newcommand{\RegionMarkerTok}[1]{#1}
\newcommand{\SpecialCharTok}[1]{\textcolor[rgb]{0.25,0.44,0.63}{#1}}
\newcommand{\SpecialStringTok}[1]{\textcolor[rgb]{0.73,0.40,0.53}{#1}}
\newcommand{\StringTok}[1]{\textcolor[rgb]{0.25,0.44,0.63}{#1}}
\newcommand{\VariableTok}[1]{\textcolor[rgb]{0.10,0.09,0.49}{#1}}
\newcommand{\VerbatimStringTok}[1]{\textcolor[rgb]{0.25,0.44,0.63}{#1}}
\newcommand{\WarningTok}[1]{\textcolor[rgb]{0.38,0.63,0.69}{\textbf{\textit{#1}}}}
\ifxetex
  \usepackage[setpagesize=false, % page size defined by xetex
              unicode=false, % unicode breaks when used with xetex
              xetex]{hyperref}
\else
  \usepackage[unicode=true]{hyperref}
\fi
\hypersetup{breaklinks=true,
            bookmarks=true,
            pdfauthor={Justin Le},
            pdftitle={Type-safe Tic Tac Toe (Part 1)},
            colorlinks=true,
            citecolor=blue,
            urlcolor=blue,
            linkcolor=magenta,
            pdfborder={0 0 0}}
\urlstyle{same}  % don't use monospace font for urls
% Make links footnotes instead of hotlinks:
\renewcommand{\href}[2]{#2\footnote{\url{#1}}}
\setlength{\parindent}{0pt}
\setlength{\parskip}{6pt plus 2pt minus 1pt}
\setlength{\emergencystretch}{3em}  % prevent overfull lines
\setcounter{secnumdepth}{0}

\title{Type-safe Tic Tac Toe (Part 1)}
\author{Justin Le}

\begin{document}
\maketitle

\emph{Originally posted on
\textbf{\href{https://blog.jle.im/entry/typesafe-tic-tac-toe-1.html}{in Code}}.}

One problem with adoption of dependent types in everyday programming, I think,
is that most examples out there are sort of small and self-contained. There
aren't \emph{too} many larger-scale examples out there showing how dependent
types can permeate your whole program to make everything more robust and
error-free.

So, this series will be implementing a type-safe \emph{tic tac toe} game (a
medium-scale Haskell app) that can be played both on the console (using
Haskeline) and in the browser (using Miso), using some custom built AI. We will:

\begin{enumerate}
\def\labelenumi{\arabic{enumi}.}
\tightlist
\item
  Build up our core game engine, talking about what it really means to be type
  safe
\item
  Use our type-safe engine to build type-safe controllers (AI, GUI)
\end{enumerate}

This series will also be a mini-tutorial on the
\emph{\href{https://hackage.haskell.org/package/decidable}{decidable}} package
that I just recently released :) We will also be heavily using the
\emph{\href{https://hackage.haskell.org/package/singletons}{singletons}}
library. Where relevant, I will explain singletons concepts in brief. If you
want a more in-depth introduction to the \emph{singletons} library, however,
check out my
\href{https://blog.jle.im/entries/series/+introduction-to-singletons.html}{Introduction
to Singletons} series!

\hypertarget{type-safety}{%
\section{Type-Safety}\label{type-safety}}

First off, we should ask the question: what does it mean to be type-safe?

?????

\hypertarget{the-specification}{%
\section{The Specification}\label{the-specification}}

We're going to create a type that represents a \emph{valid} game state. The goal
is to make a GADT where you can only construct values whose types represent
\emph{valid} game states. If we have a value of this type, then we know that the
game state must be valid.

A good way to start with this is by thinking of \emph{induction rules} for
defining a valid state.

We'll say that there are two parts of a game state:

\begin{enumerate}
\def\labelenumi{\arabic{enumi}.}
\tightlist
\item
  The current board
\item
  The current player
\end{enumerate}

and that there are two ways of ``constructing'' a valid state:

\begin{enumerate}
\def\labelenumi{\arabic{enumi}.}
\item
  The empty board with player X is a valid state.
\item
  If we have:

  \begin{itemize}
  \tightlist
  \item
    A valid state with board \emph{b} and current player \emph{p}
  \item
    The game is still in play
  \item
    We can add a valid move by player \emph{p} to board \emph{b}
  \end{itemize}

  Then the result of this move represents a new valid board \emph{b}, with
  swapped player \emph{p}.
\end{enumerate}

This is a denotative way to talk about what it means for a state to be valid.

Note that our ``type safety'' is only as strong as the specification we just
wrote. Type safety using dependent types isn't omnipotent, and it can't read
your mind. However, there is a nice assurance that as long as your
\emph{specification} is right, your program will work as expected. And hey, it's
a step up from the untyped case, where you can have a specification wrong, but
implement it incorrectly. With ``type-safety'', you cut out one huge area where
bugs come from: the implementation.

Alright, let's do this!

\hypertarget{valid-state}{%
\section{Valid State}\label{valid-state}}

First, we'll define the types we need to specify our state:

\begin{Shaded}
\begin{Highlighting}[]
\FunctionTok{$}\NormalTok{(singletons [d|}
\NormalTok{  data Piece = PX | PO}
\NormalTok{    deriving (Eq, Ord)}

\NormalTok{  type Board = [[Maybe Piece]]}
\NormalTok{  |])}
\end{Highlighting}
\end{Shaded}

A \texttt{Piece} will also represent our player -- either \texttt{PX} or
\texttt{PO}. Our \texttt{Board} will be a list of lists of
\texttt{Maybe\ Piece}. If the spot contains \texttt{Nothing}, the spot is
unplayed; if the spot is \texttt{Just\ p}, then it means the spot has been
played by \texttt{p}.

And some values and functions we need to talk about empty boards and state
transformations:

\begin{Shaded}
\begin{Highlighting}[]
\FunctionTok{$}\NormalTok{(singletons [d|}
\NormalTok{  emptyBoard :: Board}
\NormalTok{  emptyBoard = [ [Nothing, Nothing, Nothing]}
\NormalTok{               , [Nothing, Nothing, Nothing]}
\NormalTok{               , [Nothing, Nothing, Nothing]}
\NormalTok{               ]}

\NormalTok{  altP :: Piece -> Piece}
\NormalTok{  altP PX = PO}
\NormalTok{  altP PO = PX}
\NormalTok{  |])}
\end{Highlighting}
\end{Shaded}

Let's just throw in a quick proof as a sanity check:

\begin{Shaded}
\begin{Highlighting}[]
\CommentTok{-- source: https://github.com/mstksg/inCode/tree/master/code-samples/ttt/Part1.hs#L62-L64}

\OtherTok{altP_cyclic ::} \DataTypeTok{Sing}\NormalTok{ p }\OtherTok{->} \DataTypeTok{AltP}\NormalTok{ (}\DataTypeTok{AltP}\NormalTok{ p) }\FunctionTok{:~:}\NormalTok{ p}
\NormalTok{altP_cyclic }\DataTypeTok{SPX} \FunctionTok{=} \DataTypeTok{Refl} \FunctionTok{@}\NormalTok{'}\DataTypeTok{PX}
\NormalTok{altP_cyclic }\DataTypeTok{SPO} \FunctionTok{=} \DataTypeTok{Refl} \FunctionTok{@}\NormalTok{'}\DataTypeTok{PO}
\end{Highlighting}
\end{Shaded}

With that in mind, we can write our valid state constructor. We'll do that with
two helper types that we will implement later. First, we'll use the
\href{https://hackage.haskell.org/package/decidable}{decidable} library to talk
about the kind of a \emph{type-level predicate}.

\begin{Shaded}
\begin{Highlighting}[]
\CommentTok{-- source: https://github.com/mstksg/inCode/tree/master/code-samples/ttt/Part1.hs#L66-L66}

\KeywordTok{data} \DataTypeTok{InPlay}\OtherTok{ ::} \DataTypeTok{Predicate} \DataTypeTok{Board}
\end{Highlighting}
\end{Shaded}

\texttt{InPlay} is a predicate that a given board is in-play; a value of type
\texttt{InPlay\ @@\ b} is a witness or proof that a board is in play.

We also need to define a type for a valid update by a given player onto a given
board:

\begin{Shaded}
\begin{Highlighting}[]
\CommentTok{-- source: https://github.com/mstksg/inCode/tree/master/code-samples/ttt/Part1.hs#L81-L81}

\KeywordTok{data} \DataTypeTok{Update}\OtherTok{ ::} \DataTypeTok{Piece} \OtherTok{->} \DataTypeTok{Board} \OtherTok{->} \DataTypeTok{Board} \OtherTok{->} \DataTypeTok{Type} \KeywordTok{where}
\end{Highlighting}
\end{Shaded}

A value of type \texttt{Update\ p\ b1\ b2} will represent a valid update to
board \texttt{b1} by player \texttt{p} to create a board \texttt{b2}.

And finally, our valid state constructor:

\begin{Shaded}
\begin{Highlighting}[]
\CommentTok{-- source: https://github.com/mstksg/inCode/tree/master/code-samples/ttt/Part1.hs#L68-L79}

\KeywordTok{data} \DataTypeTok{GameState}\OtherTok{ ::} \DataTypeTok{Piece} \OtherTok{->} \DataTypeTok{Board} \OtherTok{->} \DataTypeTok{Type} \KeywordTok{where}
    \CommentTok{-- | The empty board is a valid state}
    \DataTypeTok{GSStart}
\OtherTok{        ::} \DataTypeTok{GameState}\NormalTok{ '}\DataTypeTok{PX} \DataTypeTok{EmptyBoard}
    \CommentTok{-- | We can also construct a valid game state if we have:}
    \DataTypeTok{GSUpdate}
\OtherTok{        ::}\NormalTok{ forall p b1 b2}\FunctionTok{.}\NormalTok{ ()}
        \OtherTok{=>} \DataTypeTok{InPlay}          \FunctionTok{@@}\NormalTok{ b1     }\CommentTok{-- ^ a proof that b1 is in play}
        \OtherTok{->} \DataTypeTok{Update}\NormalTok{    p        b1 b2  }\CommentTok{-- ^ a valid update}
        \OtherTok{->} \DataTypeTok{GameState}\NormalTok{ p        b1     }\CommentTok{-- ^ a proof that p, b1 are a valid state}
        \CommentTok{-- ---------------------------- then}
        \OtherTok{->} \DataTypeTok{GameState}\NormalTok{ (}\DataTypeTok{AltP}\NormalTok{ p)    b2  }\CommentTok{-- ^ `AltP p`, b2 is a valid satte}
\end{Highlighting}
\end{Shaded}

And that's it --- a verified-correct representation of a game state, directly
transcribed from our plain-language denotative specification.

Now we just need to talk about \texttt{InPlay} and \texttt{Update}. In
particular, we need:

\begin{enumerate}
\def\labelenumi{\arabic{enumi}.}
\tightlist
\item
  A definition of \texttt{Update}, and a way to turn user-input into a valid
  \texttt{Update} (or reject it if it isn't valid).
\item
  A definition of \texttt{InPlay}, and a way to decide whether or not a given
  board \texttt{b} is \texttt{InPlay}. This is something that the appropriately
  named \emph{\href{https://hackage.haskell.org/package/decidable}{decidable}}
  library will help us with.
\end{enumerate}

\hypertarget{update}{%
\subsection{Update}\label{update}}

Let's go about what thinking about what defines a valid update. Remember, the
kind we wanted was:

\begin{Shaded}
\begin{Highlighting}[]
\CommentTok{-- source: https://github.com/mstksg/inCode/tree/master/code-samples/ttt/Part1.hs#L81-L81}

\KeywordTok{data} \DataTypeTok{Update}\OtherTok{ ::} \DataTypeTok{Piece} \OtherTok{->} \DataTypeTok{Board} \OtherTok{->} \DataTypeTok{Board} \OtherTok{->} \DataTypeTok{Type} \KeywordTok{where}
\end{Highlighting}
\end{Shaded}

An \texttt{Update\ p\ b1\ b2} will be a valid update of \texttt{b1} by player
\texttt{p} to produce \texttt{b2}. So, we need to:

\begin{enumerate}
\def\labelenumi{\arabic{enumi}.}
\tightlist
\item
  Produce \texttt{b2} from \texttt{b1}
\item
  Be sure that the move is valid --- namely, that it is placed in a clean spot
  so that it doesn't overwrite any previous moves.
\end{enumerate}

Producing \texttt{b2} from \texttt{b1} is simple enough as a type family. In
fact, we can just use the
\emph{\href{https://hackage.haskell.org/package/lens-typelevel}{lens-typelevel}}
library to update our nested list:

\begin{Shaded}
\begin{Highlighting}[]
\FunctionTok{$}\NormalTok{(singletonsOnly [d|}
\NormalTok{  placeBoard :: N -> N -> Piece -> Board -> Board}
\NormalTok{  placeBoard i j p = set (ixList i . ixList j) (Just p)}
\NormalTok{  |])}
\end{Highlighting}
\end{Shaded}

This is just lenses --- \texttt{set\ l\ x} is a function that sets the field
specified by \texttt{l} to \texttt{x}. Here, we set the jth item of the ith list
to be \texttt{Just\ p}. That means we can now produce \texttt{b2} from
\texttt{b1} -- it's just \texttt{PlaceBoard\ i\ j\ p\ b1}.

Here, \texttt{N} is the peano nat type (a lot of libraries define it, but it's
also defined as a uility in \emph{lens-typelevel}). It's essentially
\texttt{{[}(){]}} (which makes it useful as an index type), or:

\begin{Shaded}
\begin{Highlighting}[]
\KeywordTok{data} \DataTypeTok{N} \FunctionTok{=} \DataTypeTok{Z} \FunctionTok{|} \DataTypeTok{S} \DataTypeTok{N}
\end{Highlighting}
\end{Shaded}

A natural number is either zero, or the successor of another natural number.
\texttt{S\ (S\ Z)}, for example, would represent 2.

The trickier part is making sure that the spot at \emph{(i, j)} isn't already
taken. For that, we'll introduce a common helper type to say \emph{what} the
piece at spot \emph{(i, j)} is:

\begin{Shaded}
\begin{Highlighting}[]
\CommentTok{-- source: https://github.com/mstksg/inCode/tree/master/code-samples/ttt/Part1.hs#L95-L95}

\KeywordTok{data} \DataTypeTok{Coord}\OtherTok{ ::}\NormalTok{ (}\DataTypeTok{N}\NormalTok{, }\DataTypeTok{N}\NormalTok{) }\OtherTok{->}\NormalTok{ [[k]] }\OtherTok{->}\NormalTok{ k }\OtherTok{->} \DataTypeTok{Type} \KeywordTok{where}
\end{Highlighting}
\end{Shaded}

A \texttt{Coord\ \textquotesingle{}(i,\ j)\ xss\ x} is a data type that
specifies that the jth item in the ith list in \texttt{b} is \texttt{p}.

And we require \texttt{Update} to only be constructable if the spot at \emph{(i,
j)} is \texttt{Nothing}:

\begin{Shaded}
\begin{Highlighting}[]
\CommentTok{-- source: https://github.com/mstksg/inCode/tree/master/code-samples/ttt/Part1.hs#L81-L86}

\KeywordTok{data} \DataTypeTok{Update}\OtherTok{ ::} \DataTypeTok{Piece} \OtherTok{->} \DataTypeTok{Board} \OtherTok{->} \DataTypeTok{Board} \OtherTok{->} \DataTypeTok{Type} \KeywordTok{where}
    \DataTypeTok{MkUpdate}
\OtherTok{        ::}\NormalTok{ forall i j p b}\FunctionTok{.}\NormalTok{ ()}
        \OtherTok{=>} \DataTypeTok{Coord}\NormalTok{ '(i, j) b '}\DataTypeTok{Nothing}         \CommentTok{-- ^ If the item at (i, j) in b is Nothing}
        \CommentTok{-- ------------------------------------- then}
        \OtherTok{->} \DataTypeTok{Update}\NormalTok{ p b (}\DataTypeTok{PlaceBoard}\NormalTok{ i j p b)  }\CommentTok{-- ^ Placing `Just p` at i, j is a valid update}
\end{Highlighting}
\end{Shaded}

\texttt{Update} is now defined so that, for \texttt{Update\ p\ b1\ b2},
\texttt{b2} is the update via placement of a piece \texttt{p} at some position
in \texttt{b1}, where the placement does not overwrite a previous piece. Note
that our \texttt{MkUpdate} constructor only has four ``free'' variables,
\texttt{i}, \texttt{j}, \texttt{p}, and \texttt{b}. If we use \texttt{MkUpdate},
it means that the ``final board'' is fully determined from only \texttt{i},
\texttt{j}, \texttt{p}, and \texttt{b}.

\hypertarget{coord}{%
\subsubsection{Coord}\label{coord}}

Now we need to define \texttt{Coord}. We're going to do that in terms of a
simpler type that is essentially the same for normal lists --- a type:

\begin{Shaded}
\begin{Highlighting}[]
\CommentTok{-- source: https://github.com/mstksg/inCode/tree/master/code-samples/ttt/Part1.hs#L88-L88}

\KeywordTok{data} \DataTypeTok{Sel}\OtherTok{ ::} \DataTypeTok{N} \OtherTok{->}\NormalTok{ [k] }\OtherTok{->}\NormalTok{ k }\OtherTok{->} \DataTypeTok{Type} \KeywordTok{where}
\end{Highlighting}
\end{Shaded}

A value of type \texttt{Sel\ n\ xs\ x} says that the nth item in \texttt{xs} is
\texttt{x}.

We can define this type inductively, similar to the common
\href{http://hackage.haskell.org/package/type-combinators-0.2.4.3/docs/Data-Type-Index.html}{\texttt{Index}}
data type. We can mention our induction rules:

\begin{enumerate}
\def\labelenumi{\arabic{enumi}.}
\tightlist
\item
  The first item in a list as at index 0 (\texttt{Z})
\item
  If an item is at index \texttt{n} in list \texttt{as}, then it is also at
  index \texttt{S\ n} in list \texttt{b\ \textquotesingle{}:\ as}.
\end{enumerate}

\begin{Shaded}
\begin{Highlighting}[]
\CommentTok{-- source: https://github.com/mstksg/inCode/tree/master/code-samples/ttt/Part1.hs#L88-L93}

\KeywordTok{data} \DataTypeTok{Sel}\OtherTok{ ::} \DataTypeTok{N} \OtherTok{->}\NormalTok{ [k] }\OtherTok{->}\NormalTok{ k }\OtherTok{->} \DataTypeTok{Type} \KeywordTok{where}
    \CommentTok{-- | The first item in a list is at index ''Z'}
    \DataTypeTok{SelZ}\OtherTok{ ::} \DataTypeTok{Sel}\NormalTok{ '}\DataTypeTok{Z}\NormalTok{ (a '}\FunctionTok{:}\NormalTok{ as) a}
    \DataTypeTok{SelS}\OtherTok{ ::} \DataTypeTok{Sel}\NormalTok{     n        as  a  }\CommentTok{-- ^ If item `a` is at index `n` in list `as`}
         \CommentTok{-- ---------------------------- then}
         \OtherTok{->} \DataTypeTok{Sel}\NormalTok{ ('}\DataTypeTok{S}\NormalTok{ n) (b '}\FunctionTok{:}\NormalTok{ as) a  }\CommentTok{-- ^ Item `a` is at index `S n` in list `b : as`}
\end{Highlighting}
\end{Shaded}

For example, for the type-level list \texttt{\textquotesingle{}{[}10,5,2,8{]}},
we can make values:

\begin{Shaded}
\begin{Highlighting}[]
\DataTypeTok{SelZ}\OtherTok{             ::} \DataTypeTok{Sel}\NormalTok{         '}\DataTypeTok{Z}\NormalTok{   '[}\DecValTok{10}\NormalTok{,}\DecValTok{5}\NormalTok{,}\DecValTok{2}\NormalTok{,}\DecValTok{8}\NormalTok{] }\DecValTok{10}
\DataTypeTok{SelS} \DataTypeTok{SelZ}\OtherTok{        ::} \DataTypeTok{Sel}\NormalTok{     ('}\DataTypeTok{S}\NormalTok{ '}\DataTypeTok{Z}\NormalTok{)  '[}\DecValTok{10}\NormalTok{,}\DecValTok{5}\NormalTok{,}\DecValTok{2}\NormalTok{,}\DecValTok{8}\NormalTok{] }\DecValTok{5}
\DataTypeTok{SelS}\NormalTok{ (}\DataTypeTok{SelS} \DataTypeTok{SelZ}\NormalTok{)}\OtherTok{ ::} \DataTypeTok{Sel}\NormalTok{ ('}\DataTypeTok{S}\NormalTok{ ('}\DataTypeTok{S}\NormalTok{ '}\DataTypeTok{Z}\NormalTok{)) '[}\DecValTok{10}\NormalTok{,}\DecValTok{5}\NormalTok{,}\DecValTok{2}\NormalTok{,}\DecValTok{8}\NormalTok{] }\DecValTok{2}
\end{Highlighting}
\end{Shaded}

etc.

We can then use this to define \texttt{Coord}:

\begin{Shaded}
\begin{Highlighting}[]
\CommentTok{-- source: https://github.com/mstksg/inCode/tree/master/code-samples/ttt/Part1.hs#L95-L100}

\KeywordTok{data} \DataTypeTok{Coord}\OtherTok{ ::}\NormalTok{ (}\DataTypeTok{N}\NormalTok{, }\DataTypeTok{N}\NormalTok{) }\OtherTok{->}\NormalTok{ [[k]] }\OtherTok{->}\NormalTok{ k }\OtherTok{->} \DataTypeTok{Type} \KeywordTok{where}
\OtherTok{    (:$:) ::}\NormalTok{ forall i j rows row p}\FunctionTok{.}\NormalTok{ ()}
          \OtherTok{=>} \DataTypeTok{Sel}\NormalTok{ i rows row         }\CommentTok{-- ^ If the ith list in `rows` is `row`}
          \OtherTok{->} \DataTypeTok{Sel}\NormalTok{ j row  p           }\CommentTok{-- ^ And the jth item in `row` is `p`}
          \CommentTok{-- --------------------------- then}
          \OtherTok{->} \DataTypeTok{Coord}\NormalTok{ '(i, j) rows p   }\CommentTok{-- ^ The item at (i, j) is `p`}
\end{Highlighting}
\end{Shaded}

A \texttt{Coord\ \textquotesingle{}(i,\ j)\ rows\ piece} contains a selection
into the ith list in \texttt{rows}, to get \texttt{row}, and a selection into
the jth item in \texttt{row}, to get \texttt{piece}.

\hypertarget{trying-it-out}{%
\subsection{Trying it out}\label{trying-it-out}}

That's it! Let's see if we can generate some sensible \texttt{Update}s, and
maybe even play a sample game.

We'll start with the \texttt{EmptyBoard}, and let's add a piece by \texttt{PX}
at the middle spot, index (1,1). This means we want
\texttt{SelS\ SelZ\ :\$:\ SelS\ SelZ} (a \texttt{Coord} with two indexes into
spots 1 and 1) applied to \texttt{MkUpdate}. We'll use \emph{-XTypeApplications}
to specify the type variables \texttt{p} and \texttt{b}:

\begin{Shaded}
\begin{Highlighting}[]
\NormalTok{ghci}\FunctionTok{>} \FunctionTok{:}\NormalTok{t }\DataTypeTok{MkUpdate} \FunctionTok{@}\NormalTok{_ }\FunctionTok{@}\NormalTok{_ }\FunctionTok{@}\NormalTok{'}\DataTypeTok{PX} \FunctionTok{@}\DataTypeTok{EmptyBoard}\NormalTok{ (}\DataTypeTok{SelS} \DataTypeTok{SelZ} \FunctionTok{:$:} \DataTypeTok{SelS} \DataTypeTok{SelZ}\NormalTok{)}
\DataTypeTok{Update}
\NormalTok{  '}\DataTypeTok{PX}
\NormalTok{  '[ '[ '}\DataTypeTok{Nothing}\NormalTok{, '}\DataTypeTok{Nothing}\NormalTok{ , '}\DataTypeTok{Nothing}\NormalTok{],}
\NormalTok{     '[ '}\DataTypeTok{Nothing}\NormalTok{, '}\DataTypeTok{Nothing}\NormalTok{ , '}\DataTypeTok{Nothing}\NormalTok{],}
\NormalTok{     '[ '}\DataTypeTok{Nothing}\NormalTok{, '}\DataTypeTok{Nothing}\NormalTok{ , '}\DataTypeTok{Nothing}\NormalTok{]}
\NormalTok{   ]}
\NormalTok{  '[ '[ '}\DataTypeTok{Nothing}\NormalTok{, '}\DataTypeTok{Nothing}\NormalTok{ , '}\DataTypeTok{Nothing}\NormalTok{],}
\NormalTok{     '[ '}\DataTypeTok{Nothing}\NormalTok{, '}\DataTypeTok{Just}\NormalTok{ '}\DataTypeTok{PX}\NormalTok{, '}\DataTypeTok{Nothing}\NormalTok{],}
\NormalTok{     '[ '}\DataTypeTok{Nothing}\NormalTok{, '}\DataTypeTok{Nothing}\NormalTok{ , '}\DataTypeTok{Nothing}\NormalTok{]}
\NormalTok{  ]}
\end{Highlighting}
\end{Shaded}

Nice! This update produces exactly he board expected.

Let's see if we can see if this prevents us from creating an illegal board.
We'll take the result board and see if we can place a \texttt{PO} piece there:

\begin{Shaded}
\begin{Highlighting}[]
\NormalTok{ghci}\FunctionTok{>} \KeywordTok{let} \DataTypeTok{NewBoard} \FunctionTok{=}\NormalTok{ '[ '[ '}\DataTypeTok{Nothing}\NormalTok{, '}\DataTypeTok{Nothing}\NormalTok{ , '}\DataTypeTok{Nothing}\NormalTok{ ]}
\NormalTok{                      , '[ '}\DataTypeTok{Nothing}\NormalTok{, '}\DataTypeTok{Just}\NormalTok{ '}\DataTypeTok{PX}\NormalTok{, '}\DataTypeTok{Nothing}\NormalTok{ ]}
\NormalTok{                      , '[ '}\DataTypeTok{Nothing}\NormalTok{, '}\DataTypeTok{Nothing}\NormalTok{ , '}\DataTypeTok{Nothing}\NormalTok{ ]}
\NormalTok{                      ]}
\NormalTok{ghci}\FunctionTok{>} \FunctionTok{:}\NormalTok{k }\DataTypeTok{MkUpdate} \FunctionTok{@}\NormalTok{_ }\FunctionTok{@}\NormalTok{_ }\FunctionTok{@}\NormalTok{'}\DataTypeTok{PO} \FunctionTok{@}\DataTypeTok{NewBoard}\NormalTok{ (}\DataTypeTok{SelS} \DataTypeTok{SelZ} \FunctionTok{:$:} \DataTypeTok{SelS} \DataTypeTok{SelZ}\NormalTok{)}
\NormalTok{    • }\DataTypeTok{Couldn't}\NormalTok{ match }\KeywordTok{type}\NormalTok{ ‘'}\DataTypeTok{Nothing}\NormalTok{’ with ‘'}\DataTypeTok{Just}\NormalTok{ '}\DataTypeTok{PX}\NormalTok{’}
\end{Highlighting}
\end{Shaded}

Right! That's because \texttt{SelS\ SelZ\ :\&:\ SelS\ SellZ}, applied to
\texttt{NewBoard}, gives
\texttt{Coord\ \textquotesingle{}(\textquotesingle{}S\ \textquotesingle{}Z,\ \textquotesingle{}S\ \textquotesingle{}Z)\ NewBoard\ (\textquotesingle{}Just\ \textquotesingle{}PX)}.
However, in order to be used with \texttt{MkUpdate}, the final field has to be
\texttt{\textquotesingle{}Nothing}, not
\texttt{\textquotesingle{}Just\ \textquotesingle{}PX}. So, type error.

\hypertarget{type-safe-play}{%
\subsection{Type-safe Play}\label{type-safe-play}}

At the end of this all, we finally have enough to write a truly type-safe
\texttt{play} function that allows us to play a round of our game!

\begin{Shaded}
\begin{Highlighting}[]
\CommentTok{-- source: https://github.com/mstksg/inCode/tree/master/code-samples/ttt/Part1.hs#L102-L108}

\NormalTok{play}
\OtherTok{    ::}\NormalTok{ forall i j p b}\FunctionTok{.}\NormalTok{ ()}
    \OtherTok{=>} \DataTypeTok{InPlay} \FunctionTok{@@}\NormalTok{ b}
    \OtherTok{->} \DataTypeTok{Coord}\NormalTok{ '(i, j) b '}\DataTypeTok{Nothing}
    \OtherTok{->} \DataTypeTok{GameState}\NormalTok{ p b}
    \OtherTok{->} \DataTypeTok{GameState}\NormalTok{ (}\DataTypeTok{AltP}\NormalTok{ p) (}\DataTypeTok{PlaceBoard}\NormalTok{ i j p b)}
\NormalTok{play r c }\FunctionTok{=} \DataTypeTok{GSUpdate}\NormalTok{ r (}\DataTypeTok{MkUpdate}\NormalTok{ c)}
\end{Highlighting}
\end{Shaded}

\texttt{play} is basically the entirety of our game engine! (Minus defining
\texttt{InPlay}, which we will take care of later). It'll take our new move and
a proof that the game is still in play, and return a updated new game state. Our
entire game is done, and type-safe! It's impossible to play a game in an
incorrect way! (once we define \texttt{InPlay}).

Let's try out a few rounds in ghci, using \texttt{undefined} instead of a proper
\texttt{InPlay} for now:

\begin{Shaded}
\begin{Highlighting}[]
\NormalTok{ghci}\FunctionTok{>}\NormalTok{ g1 }\FunctionTok{=}\NormalTok{ play undefined (}\DataTypeTok{SelS} \DataTypeTok{SelZ} \FunctionTok{:$:} \DataTypeTok{SelS} \DataTypeTok{SelZ}\NormalTok{) }\DataTypeTok{GSStart}   \CommentTok{-- X plays (1,1)}
\NormalTok{ghci}\FunctionTok{>} \FunctionTok{:}\NormalTok{t g1}
\DataTypeTok{GameState}\NormalTok{ '}\DataTypeTok{PO}
\NormalTok{    '[ '[ '}\DataTypeTok{Nothing}\NormalTok{, '}\DataTypeTok{Nothing}\NormalTok{ , '}\DataTypeTok{Nothing}\NormalTok{]}
\NormalTok{     , '[ '}\DataTypeTok{Nothing}\NormalTok{, '}\DataTypeTok{Just}\NormalTok{ '}\DataTypeTok{PX}\NormalTok{, '}\DataTypeTok{Nothing}\NormalTok{]}
\NormalTok{     , '[ '}\DataTypeTok{Nothing}\NormalTok{, '}\DataTypeTok{Nothing}\NormalTok{ , '}\DataTypeTok{Nothing}\NormalTok{]}
\NormalTok{     ]}

\NormalTok{ghci}\FunctionTok{>}\NormalTok{ g2 }\FunctionTok{=}\NormalTok{ play undefined (}\DataTypeTok{SelZ} \FunctionTok{:$:} \DataTypeTok{SelS} \DataTypeTok{SelZ}\NormalTok{) g1   }\CommentTok{-- O plays (0,1)}
\NormalTok{ghci}\FunctionTok{>} \FunctionTok{:}\NormalTok{t g2}
\DataTypeTok{GameState}\NormalTok{ '}\DataTypeTok{PX}
\NormalTok{    '[ '[ '}\DataTypeTok{Nothing}\NormalTok{, '}\DataTypeTok{Just}\NormalTok{ '}\DataTypeTok{PO}\NormalTok{, '}\DataTypeTok{Nothing}\NormalTok{]}
\NormalTok{     , '[ '}\DataTypeTok{Nothing}\NormalTok{, '}\DataTypeTok{Just}\NormalTok{ '}\DataTypeTok{PX}\NormalTok{, '}\DataTypeTok{Nothing}\NormalTok{]}
\NormalTok{     , '[ '}\DataTypeTok{Nothing}\NormalTok{, '}\DataTypeTok{Nothing}\NormalTok{ , '}\DataTypeTok{Nothing}\NormalTok{]}
\NormalTok{     ]}

\NormalTok{ghci}\FunctionTok{>}\NormalTok{ g3 }\FunctionTok{=}\NormalTok{ play undefined (}\DataTypeTok{SelZ} \FunctionTok{:$:} \DataTypeTok{SelS} \DataTypeTok{SelZ}\NormalTok{) g2   }\CommentTok{-- X plays (1,0)}
\NormalTok{ghci}\FunctionTok{>} \FunctionTok{:}\NormalTok{t g3}
\DataTypeTok{GameState}\NormalTok{ '}\DataTypeTok{PO}
\NormalTok{    '[ '[ '}\DataTypeTok{Nothing}\NormalTok{ , '}\DataTypeTok{Just}\NormalTok{ '}\DataTypeTok{PO}\NormalTok{, '}\DataTypeTok{Nothing}\NormalTok{]}
\NormalTok{     , '[ '}\DataTypeTok{Just}\NormalTok{ '}\DataTypeTok{PX}\NormalTok{, '}\DataTypeTok{Just}\NormalTok{ '}\DataTypeTok{PX}\NormalTok{, '}\DataTypeTok{Nothing}\NormalTok{]}
\NormalTok{     , '[ '}\DataTypeTok{Nothing}\NormalTok{ , '}\DataTypeTok{Nothing}\NormalTok{ , '}\DataTypeTok{Nothing}\NormalTok{]}
\NormalTok{     ]}

\NormalTok{ghci}\FunctionTok{>}\NormalTok{ g4 }\FunctionTok{=}\NormalTok{ play undefined (}\DataTypeTok{SelS} \DataTypeTok{SelZ} \FunctionTok{:$:} \DataTypeTok{SelS} \DataTypeTok{SelZ}\NormalTok{) g3   }\CommentTok{-- O plays (1,1)}
\NormalTok{    • }\DataTypeTok{Couldn't}\NormalTok{ match }\KeywordTok{type}\NormalTok{ ‘'}\DataTypeTok{Just}\NormalTok{ '}\DataTypeTok{PX}\NormalTok{’ with ‘'}\DataTypeTok{Nothing}\NormalTok{’}

\NormalTok{ghci}\FunctionTok{>}\NormalTok{ g4 }\FunctionTok{=}\NormalTok{ play undefined (}\DataTypeTok{SelS}\NormalTok{ (}\DataTypeTok{SelS}\NormalTok{ (}\DataTypeTok{SelS} \DataTypeTok{SelZ}\NormalTok{)) }\FunctionTok{:$:} \DataTypeTok{SelZ}\NormalTok{) g3  }\CommentTok{-- O plays (3,0)}
\NormalTok{    • }\DataTypeTok{Couldn't}\NormalTok{ match }\KeywordTok{type}\NormalTok{ ‘'[]’ with ‘'}\DataTypeTok{Nothing}\NormalTok{ '}\FunctionTok{:}\NormalTok{ as’}
\end{Highlighting}
\end{Shaded}

\texttt{play} enforces:

\begin{enumerate}
\def\labelenumi{\arabic{enumi}.}
\tightlist
\item
  Turns are always alternating X, then O
\item
  We cannot place a piece in a previously-played spot
\item
  We cannot place a piece out-of-bounds.
\end{enumerate}

\hypertarget{decision-functions-and-views}{%
\section{Decision Functions and Views}\label{decision-functions-and-views}}

This seems nice, but we're forgetting an important part. \texttt{play} requires
us to only give valid inputs, and enforces that the inputs are valid. However,
how do we \emph{create} valid inputs, in a way that satisfies \texttt{play}?

As we'll see, this is one of the core problems that dependently typed
programming gives us tools to solve.

At this point, we've reached the important part of any ``type-safe''
application: \emph{decision functions} and dependent \emph{views}.
\emph{Decision functions} let you slowly refine your more general values (types)
into more specific valid types. \emph{Views} let you sort out your our values
into more ``useful'' perspectives.

We're going to allow for users to pick to move at any natural number pair
(\texttt{(N,\ N)}), but only \emph{some} of those natural numbers can become
valid updates. In particular, we only allow an \texttt{Update} to be made if
\texttt{(N,\ N)} represent valid updates.

What are two ways this can go wrong? Well, if we allow the user to enter any two
natural numbers, here are all of the potential outcomes:

\begin{enumerate}
\def\labelenumi{\arabic{enumi}.}
\tightlist
\item
  We might get a coordinate that is out of bounds in x
\item
  We might get a coordinate that is in bounds in x, but out of bounds in y
\item
  We might get a coordinate that is in bounds in x, in bounds in y, but
  referencing a position that has already been played.
\item
  We might get a coordinate that is in bounds in x, in bounds in y, and
  references a blank position. This is the only ``success'' case.
\end{enumerate}

Note that we could also just have a ``success or nor success'' situation, but,
because we might want to provide feedback to the user, it is helpful to not be
``\href{https://twitter.com/cattheory/status/887760004622757890}{decision-blind}''
(a cousin of
\href{https://existentialtype.wordpress.com/2011/03/15/boolean-blindness/}{boolean
blindness}).

We'll call these potential ``views'' out of \texttt{(N,\ N)} with respect to
some board \texttt{b}. Let's create a data type representing all of these
possibilities (using \texttt{OutOfBounds} as a placeholder predicate for an
out-of-bounds coordinate):

\begin{Shaded}
\begin{Highlighting}[]
\CommentTok{-- | Placeholder predicate if a given number `n` is out of bounds for a given}
\CommentTok{-- list}
\KeywordTok{data} \DataTypeTok{OutOfBounds}\OtherTok{ n ::} \DataTypeTok{Predicate}\NormalTok{ [k]}

\CommentTok{-- source: https://github.com/mstksg/inCode/tree/master/code-samples/ttt/Part1.hs#L115-L125}

\KeywordTok{data} \DataTypeTok{Pick}\OtherTok{ ::}\NormalTok{ (}\DataTypeTok{N}\NormalTok{, }\DataTypeTok{N}\NormalTok{, }\DataTypeTok{Board}\NormalTok{) }\OtherTok{->} \DataTypeTok{Type} \KeywordTok{where}
    \CommentTok{-- | We are out of bounds in x}
    \DataTypeTok{PickOoBX}\OtherTok{   ::} \DataTypeTok{OutOfBounds}\NormalTok{ i }\FunctionTok{@@}\NormalTok{ b                         }\OtherTok{->} \DataTypeTok{Pick}\NormalTok{ '(i, j, b)}
    \CommentTok{-- | We are in-bounds in x, but out of bounds in y}
    \DataTypeTok{PickOoBY}\OtherTok{   ::} \DataTypeTok{Sel}\NormalTok{ i b row        }\OtherTok{->} \DataTypeTok{OutOfBounds}\NormalTok{ j }\FunctionTok{@@}\NormalTok{ row }\OtherTok{->} \DataTypeTok{Pick}\NormalTok{ '(i, j, b)}
    \CommentTok{-- | We are in-bounds in x, in-bounds in y, but spot is taken by `p`.}
    \CommentTok{-- We include `Sing p` in this constructor to potentially provide}
    \CommentTok{-- feedback to the user on what piece is already in the spot.}
    \DataTypeTok{PickPlayed}\OtherTok{ ::} \DataTypeTok{Coord}\NormalTok{ '(i, j) b ('}\DataTypeTok{Just}\NormalTok{ p) }\OtherTok{->} \DataTypeTok{Sing}\NormalTok{ p        }\OtherTok{->} \DataTypeTok{Pick}\NormalTok{ '(i, j, b)}
    \CommentTok{-- | We are in-bounds in x, in-bounds in y, and spot is clear}
    \DataTypeTok{PickValid}\OtherTok{  ::} \DataTypeTok{Coord}\NormalTok{ '(i, j) b '}\DataTypeTok{Nothing}                   \OtherTok{->} \DataTypeTok{Pick}\NormalTok{ '(i, j, b)}
\end{Highlighting}
\end{Shaded}

So, if we have an \texttt{(N,\ N,\ Board)}, we should be able to categorize it
into one of each of these potential views.

This is the job of a ``decision function''; in this case, actually, a ``proving
function''. We need to be able to write a function:

\begin{Shaded}
\begin{Highlighting}[]
\OtherTok{pick ::}\NormalTok{ forall i j b}\FunctionTok{.}\NormalTok{ ()}
     \OtherTok{=>} \DataTypeTok{Sing}\NormalTok{ '(i, j, b) }\OtherTok{->} \DataTypeTok{Pick}\NormalTok{ '(i, j, b)}
\end{Highlighting}
\end{Shaded}

That is, given any coordinate and board, we should be able to \emph{totally}
categorize it to one of the four categories, without exception.

This can be considered the boundary between the unsafe and the safe world. And,
to me, this is the ``hard part'' about dependently typed programming :)

We can write this by scratch, by hand, but we're going to look at a couple of
useful tools from the \emph{decidable} library to help us.

\hypertarget{the-decidable-library}{%
\section{The Decidable Library}\label{the-decidable-library}}

The \emph{\href{https://hackage.haskell.org/package/decidable}{decidable}}
library offers a couple of conceptual tools to work with views and predicates.
Here's a quick run-down:

The main type that the library works with is \texttt{Predicate}:

\begin{Shaded}
\begin{Highlighting}[]
\KeywordTok{type} \DataTypeTok{Predicate}\NormalTok{ k }\FunctionTok{=}\NormalTok{ k }\FunctionTok{~>} \DataTypeTok{Type}
\end{Highlighting}
\end{Shaded}

\texttt{k\ \textasciitilde{}\textgreater{}\ Type} is the kind of a
\emph{defunctionalization symbol} --- it's a dummy data type that can be passed
around, and represents a function
\texttt{k\ \textasciitilde{}\textgreater{}\ Type} that can be ``applied'' using
\texttt{Apply} or \texttt{@@}. We say that, for predicate \texttt{MyPred}, we
define:

\begin{Shaded}
\begin{Highlighting}[]
\KeywordTok{type} \KeywordTok{instance} \DataTypeTok{Apply} \DataTypeTok{MyPred}\NormalTok{ x }\FunctionTok{=} \DataTypeTok{MyWitness}
\end{Highlighting}
\end{Shaded}

Where \texttt{MyWitness} is the witness for the type-level predicate
\texttt{MyPred}. We can define a predicate from scratch by declaring the above
type family instance, but the library is defined so that you rarely ever have to
define a \texttt{Predicate} by hand. Usually, we can use predicate
``combinators'', to construct predicates from simpler pieces.

For example, we have the \texttt{TyPred} combinator:

\begin{Shaded}
\begin{Highlighting}[]
\DataTypeTok{TyPred}\OtherTok{ ::}\NormalTok{ (k }\OtherTok{->} \DataTypeTok{Type}\NormalTok{) }\OtherTok{->} \DataTypeTok{Predicate}\NormalTok{ k}
\end{Highlighting}
\end{Shaded}

It turns a normal \texttt{k\ -\textgreater{}\ Type} type constructor into a
\texttt{Predicate\ k}. So, we can use
\texttt{Pick\ ::\ (N,\ N,\ Board)\ -\textgreater{}\ Type}

\begin{Shaded}
\begin{Highlighting}[]
\NormalTok{ghci}\FunctionTok{>} \FunctionTok{:}\NormalTok{k }\DataTypeTok{TyPred} \DataTypeTok{Pick}
\DataTypeTok{Predicate}\NormalTok{ (}\DataTypeTok{N}\NormalTok{, }\DataTypeTok{N}\NormalTok{, }\DataTypeTok{Board}\NormalTok{)}
\end{Highlighting}
\end{Shaded}

\texttt{TyPred\ Pick} is a predicate that, given a coordinate and a board, we
can create a valid \texttt{Pick} using one of the \texttt{Pick} constructors.

\hypertarget{provable}{%
\subsection{Provable}\label{provable}}

\emph{decidable} makes this a little nicer to work with by providing a typeclass
for predicates with ``canonical'' viewing functions, called \texttt{Provable}:

\begin{Shaded}
\begin{Highlighting}[]
\CommentTok{-- | Class providing a canonical proving function or view for predicate `p`.}
\KeywordTok{class} \DataTypeTok{Provable}\NormalTok{ p }\KeywordTok{where}
    \CommentTok{-- | Given any `x`, produce the witness `p @@ x`.}
\OtherTok{    prove ::}\NormalTok{ forall x}\FunctionTok{.} \DataTypeTok{Sing}\NormalTok{ x }\OtherTok{->}\NormalTok{ (p }\FunctionTok{@@}\NormalTok{ x)}
\end{Highlighting}
\end{Shaded}

The benefit of using a typeclass is that we can associate a canonical
proving/viewing function with a consistent name, and also so that higher-order
predicate combinators can build proving functions based on proving functions of
the predicates they are parameterized on.

In our case, writing a view function would look like this:

\begin{Shaded}
\begin{Highlighting}[]
\KeywordTok{instance} \DataTypeTok{Provable}\NormalTok{ (}\DataTypeTok{TyPred} \DataTypeTok{Pick}\NormalTok{) }\KeywordTok{where}
\OtherTok{    prove ::} \DataTypeTok{Sing}\NormalTok{ ijb }\OtherTok{->} \DataTypeTok{Pick}\NormalTok{ ijb}
\NormalTok{    prove (}\DataTypeTok{STuple3}\NormalTok{ i j b) }\FunctionTok{=}\NormalTok{ undefined}
        \CommentTok{-- ^ STuple3 is the singleton for three-tuples}
\end{Highlighting}
\end{Shaded}

Then, given any \texttt{(i,\ j,\ b)} combination, we can classify it into one of
the constructors of \texttt{Pick} by just using
\texttt{prove\ @(TyPred\ Pick)\ sIJB}.

Now that we've restated things in the context of \emph{decidable}\ldots{}how do
we actually write \texttt{prove\ @(TyPred\ Pick)}?

Well, remember that a \emph{succcesful} \texttt{Pick} contains a
\texttt{Sel\ i\ b\ row} and a \texttt{Sel\ j\ row\ p}. We need to somehow take
an \texttt{i\ ::\ N} and turn it into a \texttt{Sel\ i\ b\ row}, and take a
\texttt{j\ ::\ N} and turn it into a \texttt{Sel\ j\ row\ p}. We need to
``convert'' a \texttt{N} into some \texttt{Sel}, in a way that could potentially
fail.

\hypertarget{parampred}{%
\subsection{ParamPred}\label{parampred}}

Another useful kind synonym that \emph{decidable} gives is in
\emph{Data.Type.Predicate.Param}, the ``parameterized predicate'':

\begin{Shaded}
\begin{Highlighting}[]
\KeywordTok{type} \DataTypeTok{ParamPred}\NormalTok{ k v }\FunctionTok{=}\NormalTok{ k }\OtherTok{->} \DataTypeTok{Predicate}\NormalTok{ v}
\end{Highlighting}
\end{Shaded}

If \texttt{MyPP\ ::\ ParamPred\ k\ v} is a parameterized predicate, then
\texttt{MyPP\ x} is a \texttt{Predicate\ v}.

The main usage of parameterized predicate is for usage with the \texttt{Found}
predicate combinator:

\begin{Shaded}
\begin{Highlighting}[]
\DataTypeTok{Found}\OtherTok{ ::} \DataTypeTok{ParamPred}\NormalTok{ k v }\OtherTok{->} \DataTypeTok{Predicate}\NormalTok{ k}
\end{Highlighting}
\end{Shaded}

\texttt{Found\ MyPP} is a predicate that, for any \texttt{x\ ::\ k}, we can find
\emph{some} \texttt{y\ ::\ v} that satisfies \texttt{MyPP\ x\ @@\ y}.

Again, the library is constructed so that you shouldn't need to define a
\texttt{ParamPred} by hand; you can just use combinators and constructors.

For example, we have \texttt{TyPP}:

\begin{Shaded}
\begin{Highlighting}[]
\DataTypeTok{TyPP}\OtherTok{ ::}\NormalTok{ (k }\OtherTok{->}\NormalTok{ v }\OtherTok{->} \DataTypeTok{Type}\NormalTok{) }\OtherTok{->} \DataTypeTok{ParamPred}\NormalTok{ k v}
\end{Highlighting}
\end{Shaded}

Which turns any normal type constructor into a \texttt{ParamPred}. For example,
let's look at \texttt{Sel\ \textquotesingle{}Z}:

\begin{Shaded}
\begin{Highlighting}[]
\NormalTok{ghci}\FunctionTok{>} \FunctionTok{:}\NormalTok{k }\DataTypeTok{TyPP}\NormalTok{ (}\DataTypeTok{Sel}\NormalTok{ '}\DataTypeTok{Z}\NormalTok{)}
\DataTypeTok{ParamPred}\NormalTok{ [k] k}
\end{Highlighting}
\end{Shaded}

\texttt{TyPP\ (Sel\ \textquotesingle{}Z)} is the parameterized predicate that,
given a list \texttt{xs\ ::\ {[}k{]}}, we can produce an \texttt{x\ ::\ k} that
is at index \texttt{\textquotesingle{}Z}. That's because its witness is
\texttt{Sel\ \textquotesingle{}Z\ xs\ x} (the witness that \texttt{x} is at
position \texttt{\textquotesingle{}Z} in \texttt{xs}).

What is \texttt{Found\ (TyPP\ (Sel\ \textquotesingle{}Z))}?

\begin{Shaded}
\begin{Highlighting}[]
\NormalTok{ghci}\FunctionTok{>} \FunctionTok{:}\NormalTok{k }\DataTypeTok{Found}\NormalTok{ (}\DataTypeTok{TyPP}\NormalTok{ (}\DataTypeTok{Sel}\NormalTok{ '}\DataTypeTok{Z}\NormalTok{))}
\DataTypeTok{Predicate}\NormalTok{ [k]}
\end{Highlighting}
\end{Shaded}

Judging from the type, it is some predicate on a type level list. And knowing
what we know about \texttt{Found}, we can conclude what it is: It is a predicate
that, given some list \texttt{xs}, there \emph{is some value \texttt{x}} at
position \texttt{\textquotesingle{}Z}. It's essentially a predicate that the
list \emph{has} something at position \texttt{\textquotesingle{}Z}.

We can generalize it further;
\texttt{Found\ (TyPP\ (Sel\ (\textquotesingle{}S\ \textquotesingle{}Z)))} must
be the predicate that some given list \texttt{xs} has a value \texttt{x} at
position \texttt{\textquotesingle{}S\ \textquotesingle{}Z}. It says that there
must be \emph{some} value at \texttt{\textquotesingle{}S\ \textquotesingle{}Z}.

Really, \texttt{Found\ (TyPP\ (Sel\ n))} is a predicate that some list
\texttt{xs} is \emph{at least} \texttt{n\ +\ 1} items long. That's because we
know that the list has to have some item at position \texttt{n}.

There's a better name for this --- we'll call it \texttt{InBounds}

\begin{Shaded}
\begin{Highlighting}[]
\CommentTok{-- source: https://github.com/mstksg/inCode/tree/master/code-samples/ttt/Part1.hs#L110-L110}

\KeywordTok{type} \DataTypeTok{InBounds}\NormalTok{    n }\FunctionTok{=} \DataTypeTok{Found}\NormalTok{ (}\DataTypeTok{TyPP}\NormalTok{ (}\DataTypeTok{Sel}\NormalTok{ n))}
\end{Highlighting}
\end{Shaded}

\texttt{InBounds\ n\ ::\ Predicate\ {[}k{]}} is the predicate that, given some
list \texttt{xs}, \texttt{n} is ``in bounds'' of \texttt{xs}.

And \emph{decidable} is nice because it offers a predicate combinator
\texttt{Not}, which gives the negation of any predicate:

\begin{Shaded}
\begin{Highlighting}[]
\CommentTok{-- source: https://github.com/mstksg/inCode/tree/master/code-samples/ttt/Part1.hs#L112-L112}

\KeywordTok{type} \DataTypeTok{OutOfBounds}\NormalTok{ n }\FunctionTok{=} \DataTypeTok{Not}\NormalTok{ (}\DataTypeTok{InBounds}\NormalTok{ n)}
\end{Highlighting}
\end{Shaded}

\texttt{OutOfBounds\ n\ ::\ Predicate\ {[}k{]}} is the predicate that, given
some list \texttt{xs}, \texttt{n} is \emph{not} in bounds of \texttt{xs}, and
that it is actually \emph{out} of bounds.

\hypertarget{decidable}{%
\subsection{Decidable}\label{decidable}}

Now, is \texttt{InBounds\ n} going to be \texttt{Provable}? No, not quite.
That's because a given list \texttt{xs} might be actually out of bounds. For
example,
\texttt{InBounds\ \textquotesingle{}Z\ @@\ \textquotesingle{}{[}1,2,3{]}} is
satisfiable, but
\texttt{InBounds\ (\textquotesingle{}S\ \textquotesingle{}Z)\ \textquotesingle{}{[}{]}}
is not.

To implement our view of \texttt{Pick}, we would like a function that can
\emph{decide} whether or not \texttt{InBounds\ n} is satisfied by a given list
\texttt{xs}. What we want is a \emph{decision function}:

\begin{Shaded}
\begin{Highlighting}[]
\OtherTok{inBounds ::}\NormalTok{ forall n xs}\FunctionTok{.}\NormalTok{ ()}
         \OtherTok{=>} \DataTypeTok{Sing}\NormalTok{ xs}
         \OtherTok{->} \DataTypeTok{Decision}\NormalTok{ (}\DataTypeTok{InBounds}\NormalTok{ n }\FunctionTok{@@}\NormalTok{ xs)}
\end{Highlighting}
\end{Shaded}

Remember that \texttt{Decision} is a data type that is kind of like
\texttt{Maybe}, but with a ``disproof'' if the input is disprovable:

\begin{Shaded}
\begin{Highlighting}[]
\KeywordTok{data} \DataTypeTok{Decision}\NormalTok{ a}
    \FunctionTok{=} \DataTypeTok{Proved}\NormalTok{     a                }\CommentTok{-- ^ `a` is provably true}
    \FunctionTok{|} \DataTypeTok{Disproved}\NormalTok{ (a }\OtherTok{->} \DataTypeTok{Void}\NormalTok{)       }\CommentTok{-- ^ `a` is provably false}

\CommentTok{-- | The type with no constructors.  If we have a function `a -> Void`, it must}
\CommentTok{-- mean that no value of type `a` exists.}
\KeywordTok{data} \DataTypeTok{Void}
\end{Highlighting}
\end{Shaded}

The \emph{decidable} library offers a typeclass for a \emph{canonical} decision
function for any \texttt{Predicate}:

\begin{Shaded}
\begin{Highlighting}[]
\CommentTok{-- | Class providing a canonical decision function for predicate `p`.}
\KeywordTok{class} \DataTypeTok{Decidable}\NormalTok{ p }\KeywordTok{where}
    \CommentTok{-- | Given any `x`, either prove or disprove the witness `p @@ x`.}
\OtherTok{    decide ::}\NormalTok{ forall x}\FunctionTok{.} \DataTypeTok{Sing}\NormalTok{ x }\OtherTok{->} \DataTypeTok{Decision}\NormalTok{ (p }\FunctionTok{@@}\NormalTok{ x)}
\end{Highlighting}
\end{Shaded}

Of course, we could always just write our decision function \texttt{inBounds}
from scratch, but it's convenient to pull everything into a typeclass instead
for the reasons discussed earlier.

\hypertarget{deciding-inbounds}{%
\subsection{Deciding InBounds}\label{deciding-inbounds}}

Alright, time to write our first bona-fide decision function for
\texttt{InBounds}, which we will use to write our view function for
\texttt{Pick}.

The decision function requires us to produce a witness for
\texttt{InBounds\ n\ @@\ xs}\ldots{}so we need to know what that witness looks
like.

To do this, we could either look at the documentation for \texttt{Found}
(because \texttt{InBounds\ n\ =\ Found\ (TyPP\ (Sel\ n))}) to find its
\texttt{Apply} instance, or we could just ask GHC what this looks like for a
given input, using \texttt{:kind!}:

\begin{Shaded}
\begin{Highlighting}[]
\NormalTok{ghci}\FunctionTok{>} \FunctionTok{:}\NormalTok{kind}\FunctionTok{!} \DataTypeTok{InBounds}\NormalTok{ '}\DataTypeTok{Z} \FunctionTok{@@}\NormalTok{ '[}\DecValTok{1}\NormalTok{,}\DecValTok{2}\NormalTok{,}\DecValTok{3}\NormalTok{]  }\CommentTok{-- what is the type of the witness for `InBounds 'Z1 ?}
\NormalTok{Σ }\DataTypeTok{Nat}\NormalTok{ (}\DataTypeTok{TyPP}\NormalTok{ (}\DataTypeTok{Sel}\NormalTok{ '}\DataTypeTok{Z}\NormalTok{) '[}\DecValTok{1}\NormalTok{,}\DecValTok{2}\NormalTok{,}\DecValTok{3}\NormalTok{])}
\end{Highlighting}
\end{Shaded}

In general, the witness for \texttt{Found\ (p\ ::\ ParamPred\ k\ v)} is:

\begin{Shaded}
\begin{Highlighting}[]
\KeywordTok{type} \KeywordTok{instance} \DataTypeTok{Apply}\NormalTok{ (}\DataTypeTok{Found}\NormalTok{ p) x }\FunctionTok{=}\NormalTok{ Σ v (p x)}
\end{Highlighting}
\end{Shaded}

\texttt{Σ} might seem a little scary, but remember that it's a type synonym for
the dependent pair \texttt{Sigma} type, from \emph{Data.Singletons.Sigma}:

\begin{Shaded}
\begin{Highlighting}[]
\KeywordTok{data} \DataTypeTok{Sigma}\OtherTok{ k ::}\NormalTok{ (k }\FunctionTok{~>} \DataTypeTok{Type}\NormalTok{) }\OtherTok{->} \DataTypeTok{Type} \KeywordTok{where}
\OtherTok{    (:&:) ::} \DataTypeTok{Sing}\NormalTok{ x }\OtherTok{->}\NormalTok{ (f }\FunctionTok{@@}\NormalTok{ x) }\OtherTok{->} \DataTypeTok{Sigma}\NormalTok{ k f}

\KeywordTok{type}\NormalTok{ Σ k }\FunctionTok{=} \DataTypeTok{Sigma}\NormalTok{ k}
\end{Highlighting}
\end{Shaded}

I wrote a small mini-tutorial on \texttt{Sigma}
\href{https://blog.jle.im/entry/introduction-to-singletons-4.html\#sigma}{here},
if you need a refresher. Basically, if we had
\texttt{f\ ::\ k\ \textasciitilde{}\textgreater{}\ Type}, then
\texttt{Sigma\ k\ f} contains an \texttt{f\ @@\ x}, for some \texttt{x}, along
with \texttt{Sing\ x} (to help us recover what \texttt{x} was, once we pattern
match). It's a \emph{dependent pair} or \emph{dependent sum} type. You can think
of it as \texttt{Sigma\ k\ f} existentially \emph{wrapping} \texttt{x\ ::\ k},
to show that there is at least some \texttt{x} somewhere out there such that
\texttt{f\ @@\ x} exists.

This makes a lot of sense as a witness to \texttt{Found\ p}.
\texttt{Found\ p\ @@\ x} says that there is some \texttt{y} such that
\texttt{p\ x\ @@\ y} is satisfied. So, what is the witness of that statement?
The \texttt{y} itself! (wrapped in a \texttt{Σ})

So, the witness for
\texttt{InBounds\ \textquotesingle{}Z\ @@\ \textquotesingle{}{[}\ \textquotesingle{}True,\ \textquotesingle{}False\ {]}}
is the item in the list \texttt{\textquotesingle{}{[}1,2,3{]}} at position
\texttt{\textquotesingle{}Z} --- \texttt{\textquotesingle{}True}. Let's see this
in action:

\begin{Shaded}
\begin{Highlighting}[]
\OtherTok{inBoundsTest1 ::} \DataTypeTok{InBounds}\NormalTok{ '}\DataTypeTok{Z} \FunctionTok{@@}\NormalTok{ '[ '}\DataTypeTok{True}\NormalTok{, '}\DataTypeTok{False}\NormalTok{ ]}
\NormalTok{inBoundsTest1 }\FunctionTok{=} \DataTypeTok{STrue} \FunctionTok{:&:} \DataTypeTok{SelZ}
                       \CommentTok{-- ^ Sel 'Z '[ 'True, 'False ] 'True}
\end{Highlighting}
\end{Shaded}

Note that we can't put \texttt{SFalse} in \texttt{inBoundsTest1}, because the
second half \texttt{SelZ} would be
\texttt{Sel\ ::\ \textquotesingle{}Z\ \textquotesingle{}{[}\ \textquotesingle{}True,\ \textquotesingle{}False\ {]}\ \textquotesingle{}True}
(because \texttt{\textquotesingle{}True} is the 0th item in the list), so we
have to have the first half match \texttt{\textquotesingle{}True}.

And we can write a witness for
\texttt{InBounds\ (\textquotesingle{}S\ \textquotesingle{}Z)\ @@\ \textquotesingle{}{[}\ \textquotesingle{}True,\ \textquotesingle{}False\ {]}},
as well, by giving the value of the list at index 1,
\texttt{\textquotesingle{}False}:

\begin{Shaded}
\begin{Highlighting}[]
\OtherTok{inBoundsTest2 ::} \DataTypeTok{InBounds}\NormalTok{ ('}\DataTypeTok{S}\NormalTok{ '}\DataTypeTok{Z}\NormalTok{) }\FunctionTok{@@}\NormalTok{ '[ '}\DataTypeTok{True}\NormalTok{, '}\DataTypeTok{False}\NormalTok{ ]}
\NormalTok{inBoundsTest2 }\FunctionTok{=} \DataTypeTok{SFalse} \FunctionTok{:&:} \DataTypeTok{SelS} \DataTypeTok{SelZ}
                        \CommentTok{-- ^ Sel ('S 'Z) '[ 'True, 'False ] 'False}
\end{Highlighting}
\end{Shaded}

With that in mind, let's write our decision function for \texttt{InBounds\ n}.
It's going to be our actual first dependently typed function!

For the sake of learning, we're going to write it as a standalone function
\texttt{inBounds}. It's going to take \texttt{Sing\ n} (the index) and
\texttt{Sing\ xs} (the list) and produce a decision on
\texttt{InBounds\ n\ @@\ xs}. Like for any Haskell function on ADTs, we'll start
out by just writing all of our case statement branches (using
\emph{-XLambdaCase} for conciseness). An \texttt{N} can either be \texttt{Z} or
\texttt{S\ n}, so we match on singletons \texttt{SZ} and \texttt{SS}. A
\texttt{{[}a{]}} can either be \texttt{{[}{]}} or \texttt{x\ :\ xs}, so we match
on singletons \texttt{SNil} and
\texttt{x\ \textasciigrave{}SCons\textasciigrave{}\ xs}

\begin{Shaded}
\begin{Highlighting}[]
\OtherTok{inBounds ::} \DataTypeTok{Sing}\NormalTok{ n }\OtherTok{->} \DataTypeTok{Sing}\NormalTok{ xs }\OtherTok{->} \DataTypeTok{Decision}\NormalTok{ (}\DataTypeTok{InBounds}\NormalTok{ n }\FunctionTok{@@}\NormalTok{ xs)}
\NormalTok{inBounds }\FunctionTok{=}\NormalTok{ \textbackslash{}}\KeywordTok{case}
    \DataTypeTok{SZ} \OtherTok{->}\NormalTok{ \textbackslash{}}\KeywordTok{case}
      \DataTypeTok{SNil}         \OtherTok{->}\NormalTok{ _}
\NormalTok{      x }\OtherTok{`SCons`}\NormalTok{ xs }\OtherTok{->}\NormalTok{ _}
    \DataTypeTok{SS}\NormalTok{ n }\OtherTok{->}\NormalTok{ \textbackslash{}}\KeywordTok{case}
      \DataTypeTok{SNil}         \OtherTok{->}\NormalTok{ _}
\NormalTok{      x }\OtherTok{`SCons`}\NormalTok{ xs }\OtherTok{->}\NormalTok{ _}
\end{Highlighting}
\end{Shaded}

Okay, four cases. Initially daunting, but we can just handle this one by one.
Again, for learning's sake, ket's split these branches into four helper
functions --- one for each case.

\begin{Shaded}
\begin{Highlighting}[]
\CommentTok{-- source: https://github.com/mstksg/inCode/tree/master/code-samples/ttt/Part1.hs#L131-L159}

\OtherTok{inBounds ::} \DataTypeTok{Sing}\NormalTok{ n }\OtherTok{->} \DataTypeTok{Sing}\NormalTok{ xs }\OtherTok{->} \DataTypeTok{Decision}\NormalTok{ (}\DataTypeTok{InBounds}\NormalTok{ n }\FunctionTok{@@}\NormalTok{ xs)}
\NormalTok{inBounds }\FunctionTok{=}\NormalTok{ \textbackslash{}}\KeywordTok{case}
    \DataTypeTok{SZ} \OtherTok{->}\NormalTok{ \textbackslash{}}\KeywordTok{case}
      \DataTypeTok{SNil}         \OtherTok{->}\NormalTok{ inBounds_znil}
\NormalTok{      x }\OtherTok{`SCons`}\NormalTok{ xs }\OtherTok{->}\NormalTok{ inBounds_zcons x xs}
    \DataTypeTok{SS}\NormalTok{ n }\OtherTok{->}\NormalTok{ \textbackslash{}}\KeywordTok{case}
      \DataTypeTok{SNil}         \OtherTok{->}\NormalTok{ inBounds_snil n}
\NormalTok{      x }\OtherTok{`SCons`}\NormalTok{ xs }\OtherTok{->}\NormalTok{ inBounds_scons n x xs}

\NormalTok{inBounds_znil}
\OtherTok{    ::} \DataTypeTok{Decision}\NormalTok{ (}\DataTypeTok{InBounds}\NormalTok{ '}\DataTypeTok{Z} \FunctionTok{@@}\NormalTok{ '[])}

\NormalTok{inBounds_zcons}
\OtherTok{    ::} \DataTypeTok{Sing}\NormalTok{ x}
    \OtherTok{->} \DataTypeTok{Sing}\NormalTok{ xs}
    \OtherTok{->} \DataTypeTok{Decision}\NormalTok{ (}\DataTypeTok{InBounds}\NormalTok{ '}\DataTypeTok{Z} \FunctionTok{@@}\NormalTok{ (x '}\FunctionTok{:}\NormalTok{ xs))}

\NormalTok{inBounds_snil}
\OtherTok{    ::} \DataTypeTok{Sing}\NormalTok{ n}
    \OtherTok{->} \DataTypeTok{Decision}\NormalTok{ (}\DataTypeTok{InBounds}\NormalTok{ ('}\DataTypeTok{S}\NormalTok{ n) }\FunctionTok{@@}\NormalTok{ '[])}

\NormalTok{inBounds_scons}
\OtherTok{    ::} \DataTypeTok{Sing}\NormalTok{ n}
    \OtherTok{->} \DataTypeTok{Sing}\NormalTok{ x}
    \OtherTok{->} \DataTypeTok{Sing}\NormalTok{ xs}
    \OtherTok{->} \DataTypeTok{Decision}\NormalTok{ (}\DataTypeTok{InBounds}\NormalTok{ ('}\DataTypeTok{S}\NormalTok{ n) }\FunctionTok{@@}\NormalTok{ (x '}\FunctionTok{:}\NormalTok{ xs))}
\end{Highlighting}
\end{Shaded}

\begin{enumerate}
\def\labelenumi{\arabic{enumi}.}
\item
  For the first branch, we have \texttt{\textquotesingle{}Z} and
  \texttt{\textquotesingle{}{[}{]}}. This should be false, because there is no
  item in the zeroth position in \texttt{{[}{]}}. But, also, there is no way to
  construct the \texttt{Sel} necessary for the witness, since there is no
  constructor for \texttt{Sel} that gives \texttt{\textquotesingle{}{[}{]}}.

  So we can write this as \texttt{Disproved}, which takes a
  \texttt{InBounds\ \textquotesingle{}Z\ @@\ \textquotesingle{}{[}{]}\ -\textgreater{}\ Void}:

\begin{Shaded}
\begin{Highlighting}[]
\CommentTok{-- source: https://github.com/mstksg/inCode/tree/master/code-samples/ttt/Part1.hs#L140-L142}

\NormalTok{inBounds_znil}
\OtherTok{    ::} \DataTypeTok{Decision}\NormalTok{ (}\DataTypeTok{InBounds}\NormalTok{ '}\DataTypeTok{Z} \FunctionTok{@@}\NormalTok{ '[])}
\NormalTok{inBounds_znil }\FunctionTok{=} \DataTypeTok{Disproved} \FunctionTok{$}\NormalTok{ \textbackslash{}(_ }\FunctionTok{:&:}\NormalTok{ s) }\OtherTok{->} \KeywordTok{case}\NormalTok{ s }\KeywordTok{of}\NormalTok{ \{\}}
\end{Highlighting}
\end{Shaded}

  We can satisfy that
  \texttt{InBounds\ \textquotesingle{}Z\ @@\ \textquotesingle{}{[}{]}\ -\textgreater{}\ Void}
  by pattern matching on the \texttt{Sel} it \emph{would} contain. Because there
  is no \texttt{Sel} for an empty list, the empty pattern match is safe.

  Remember to enable \emph{-Werror=incomplete-patterns} to be sure!
\item
  For the second branch, we have \texttt{\textquotesingle{}Z} and
  \texttt{(x\ \textquotesingle{}:\ xs)}. We want to prove that there exists an
  item at position \texttt{\textquotesingle{}Z} in the list
  \texttt{x\ \textquotesingle{}:\ xs}. The answer is \emph{yes}, there does, and
  that item is \texttt{x}, and the \texttt{Sel} is \texttt{SelZ}!

\begin{Shaded}
\begin{Highlighting}[]
\CommentTok{-- source: https://github.com/mstksg/inCode/tree/master/code-samples/ttt/Part1.hs#L144-L148}

\NormalTok{inBounds_zcons}
\OtherTok{    ::} \DataTypeTok{Sing}\NormalTok{ x}
    \OtherTok{->} \DataTypeTok{Sing}\NormalTok{ xs}
    \OtherTok{->} \DataTypeTok{Decision}\NormalTok{ (}\DataTypeTok{InBounds}\NormalTok{ '}\DataTypeTok{Z} \FunctionTok{@@}\NormalTok{ (x '}\FunctionTok{:}\NormalTok{ xs))}
\NormalTok{inBounds_zcons x _ }\FunctionTok{=} \DataTypeTok{Proved}\NormalTok{ (x }\FunctionTok{:&:} \DataTypeTok{SelZ}\NormalTok{)}
\end{Highlighting}
\end{Shaded}
\item
  For the third branch, we have \texttt{\textquotesingle{}S\ n} and
  \texttt{\textquotesingle{}{[}{]}}. Again, this should be false, because there
  is no item in the \texttt{\textquotesingle{}S\ n} position in
  \texttt{\textquotesingle{}{[}{]}}. We should be able to use the same strategy
  for the first branch:

\begin{Shaded}
\begin{Highlighting}[]
\CommentTok{-- source: https://github.com/mstksg/inCode/tree/master/code-samples/ttt/Part1.hs#L150-L153}

\NormalTok{inBounds_snil}
\OtherTok{    ::} \DataTypeTok{Sing}\NormalTok{ n}
    \OtherTok{->} \DataTypeTok{Decision}\NormalTok{ (}\DataTypeTok{InBounds}\NormalTok{ ('}\DataTypeTok{S}\NormalTok{ n) }\FunctionTok{@@}\NormalTok{ '[])}
\NormalTok{inBounds_snil _ }\FunctionTok{=} \DataTypeTok{Disproved} \FunctionTok{$}\NormalTok{ \textbackslash{}(_ }\FunctionTok{:&:}\NormalTok{ s) }\OtherTok{->} \KeywordTok{case}\NormalTok{ s }\KeywordTok{of}\NormalTok{ \{\}}
\end{Highlighting}
\end{Shaded}
\item
  The fourth branch is the most interesting one. We have
  \texttt{\textquotesingle{}S\ n} and \texttt{(x\ \textquotesingle{}:\ xs)}. How
  do we know if the list \texttt{x\ \textquotesingle{}:\ xs} has an item in the
  \texttt{\textquotesingle{}S\ n} spot?

  Well, we can check if the list \texttt{xs} has an item in its \texttt{n} spot.

  \begin{itemize}
  \item
    If it does, then call that item \texttt{y}, and we know that
    \texttt{x\ \textquotesingle{}:\ xs} has \texttt{y} in its
    \texttt{\textquotesingle{}S\ n} spot.
  \item
    If it doesn't, then we can't have an item at \texttt{\textquotesingle{}S\ n}
    spot in \texttt{x\ \textquotesingle{}:\ xs} either! To show why, we can do a
    proof by contradiction.

    Suppose there \emph{was} an item \texttt{y} at the
    \texttt{\textquotesingle{}S\ n} spot in \texttt{x\ \textquotesingle{}:\ xs}.
    If so, then that means that there would be an item \texttt{y} in the
    \texttt{n} spot in \texttt{xs}. However, this was found to be false.
    Therefore, we cannot have an item in the \texttt{\textquotesingle{}S\ n}
    spot in \texttt{x\ \textquotesingle{}:\ xs}.
  \end{itemize}

\begin{Shaded}
\begin{Highlighting}[]
\CommentTok{-- source: https://github.com/mstksg/inCode/tree/master/code-samples/ttt/Part1.hs#L155-L167}

\NormalTok{inBounds_scons}
\OtherTok{    ::} \DataTypeTok{Sing}\NormalTok{ n}
    \OtherTok{->} \DataTypeTok{Sing}\NormalTok{ x}
    \OtherTok{->} \DataTypeTok{Sing}\NormalTok{ xs}
    \OtherTok{->} \DataTypeTok{Decision}\NormalTok{ (}\DataTypeTok{InBounds}\NormalTok{ ('}\DataTypeTok{S}\NormalTok{ n) }\FunctionTok{@@}\NormalTok{ (x '}\FunctionTok{:}\NormalTok{ xs))}
\NormalTok{inBounds_scons n _ xs }\FunctionTok{=} \KeywordTok{case}\NormalTok{ inBounds n xs }\KeywordTok{of}
    \DataTypeTok{Proved}\NormalTok{ (y }\FunctionTok{:&:}\NormalTok{ s) }\OtherTok{->}       \CommentTok{-- if xs has y in its n spot}
      \DataTypeTok{Proved}\NormalTok{ (y }\FunctionTok{:&:} \DataTypeTok{SelS}\NormalTok{ s)   }\CommentTok{-- then (x : xs) has y in its (S n) spot}
    \DataTypeTok{Disproved}\NormalTok{ v      }\OtherTok{->} \DataTypeTok{Disproved} \FunctionTok{$} \CommentTok{-- v is a disproof that an item is in n spot in xs}
\NormalTok{      \textbackslash{}(y }\FunctionTok{:&:}\NormalTok{ s) }\OtherTok{->}      \CommentTok{-- suppose we had item y in (S n) spot in (x : xs)}
        \KeywordTok{case}\NormalTok{ s }\KeywordTok{of}
          \DataTypeTok{SelS}\NormalTok{ s' }\OtherTok{->}     \CommentTok{-- this would mean that item y is in n spot in xs}
\NormalTok{            v (y }\FunctionTok{:&:}\NormalTok{ s') }\CommentTok{-- however, v disproves this.}
\end{Highlighting}
\end{Shaded}

  If you have problems understanding this, try playing around with typed holes
  in GHC, or trying to guess what types everything has in the implementation
  above, until you can figure out what is happening when.
\end{enumerate}

Finally, we can wrap everything up by providing our first ever
\texttt{Decidable} instance. We need to give \texttt{inBounds} a
\texttt{Sing\ n}, so we can do that using \texttt{sing\ ::\ Sing\ n}, provided
that the instance has a \texttt{SingI\ n} constraint.

\begin{Shaded}
\begin{Highlighting}[]
\CommentTok{-- source: https://github.com/mstksg/inCode/tree/master/code-samples/ttt/Part1.hs#L127-L129}

\KeywordTok{instance} \DataTypeTok{SingI}\NormalTok{ n }\OtherTok{=>} \DataTypeTok{Decidable}\NormalTok{ (}\DataTypeTok{InBounds}\NormalTok{ n) }\KeywordTok{where}
\OtherTok{    decide ::} \DataTypeTok{Sing}\NormalTok{ xs }\OtherTok{->} \DataTypeTok{Decision}\NormalTok{ (}\DataTypeTok{InBounds}\NormalTok{ n }\FunctionTok{@@}\NormalTok{ xs)}
\NormalTok{    decide }\FunctionTok{=}\NormalTok{ inBounds sing}
\end{Highlighting}
\end{Shaded}

\hypertarget{proving-pick}{%
\subsection{Proving Pick}\label{proving-pick}}

Now that we can decide \texttt{InBounds}, let's finally prove \texttt{Pick}.

Again, for learning purposes, we'll define \texttt{pick} as its own function and
then write an instance for \texttt{Provable}.

\begin{Shaded}
\begin{Highlighting}[]
\NormalTok{pick}
\OtherTok{    ::}\NormalTok{ forall i j b}\FunctionTok{.}\NormalTok{ ()}
    \OtherTok{=>} \DataTypeTok{Sing}\NormalTok{ i}
    \OtherTok{->} \DataTypeTok{Sing}\NormalTok{ j}
    \OtherTok{->} \DataTypeTok{Sing}\NormalTok{ b}
    \OtherTok{->} \DataTypeTok{Pick}\NormalTok{ '(i, j, b)}
\NormalTok{pick }\DataTypeTok{Sing} \DataTypeTok{Sing}\NormalTok{ b }\FunctionTok{=}
\end{Highlighting}
\end{Shaded}

We'll match with the \texttt{Sing} constructor for \texttt{Sing\ i} and
\texttt{Sing\ j}; the \texttt{Sing} constructor is a pattern synonym that, if
matched on, brings \texttt{SingI\ i} and \texttt{SingI\ j} instances into scope.

Remember, the goal is to try to prove we have a valid pick. We want to create
something with the \texttt{PickValid} constructor if we can:

\begin{Shaded}
\begin{Highlighting}[]
\DataTypeTok{PickValid}\OtherTok{  ::} \DataTypeTok{Coord}\NormalTok{ '(i, j) b '}\DataTypeTok{Nothing} \OtherTok{->} \DataTypeTok{Pick}\NormalTok{ '(i, j, b)}

\OtherTok{(:$:) ::} \DataTypeTok{Sel}\NormalTok{ i rows row}
      \OtherTok{->} \DataTypeTok{Sel}\NormalTok{ j row  p}
      \OtherTok{->} \DataTypeTok{Coord}\NormalTok{ '(i, j) rows p}
\end{Highlighting}
\end{Shaded}

So we need a
\texttt{Coord\ \textquotesingle{}(i,\ j)\ b\ \textquotesingle{}Nothing}, which
means we need a \texttt{Sel\ i\ b\ row} and a
\texttt{Sel\ j\ row\ \textquotesingle{}Nothing}. Let's use our decision
functions we wrote to get these! In particular, we can use
\texttt{decide\ @(InBounds\ i)\ b} to get our \texttt{Sel\ i\ b\ row}, and then
use \texttt{decide\ @(InBounds\ j)\ row} to get our \texttt{Sel\ j\ row\ piece}!

\begin{Shaded}
\begin{Highlighting}[]
\NormalTok{pick}
\OtherTok{    ::}\NormalTok{ forall i j b}\FunctionTok{.}\NormalTok{ ()}
    \OtherTok{=>} \DataTypeTok{Sing}\NormalTok{ i}
    \OtherTok{->} \DataTypeTok{Sing}\NormalTok{ j}
    \OtherTok{->} \DataTypeTok{Sing}\NormalTok{ b}
    \OtherTok{->} \DataTypeTok{Pick}\NormalTok{ '(i, j, b)}
\NormalTok{pick }\DataTypeTok{Sing} \DataTypeTok{Sing}\NormalTok{ b }\FunctionTok{=} \KeywordTok{case}\NormalTok{ decide }\FunctionTok{@}\NormalTok{(}\DataTypeTok{InBounds}\NormalTok{ i) b }\KeywordTok{of}
    \DataTypeTok{Proved}\NormalTok{ (row }\FunctionTok{:&:}\NormalTok{ selX) }\OtherTok{->} \KeywordTok{case}\NormalTok{ decide }\FunctionTok{@}\NormalTok{(}\DataTypeTok{InBounds}\NormalTok{ j) row }\KeywordTok{of}
      \DataTypeTok{Proved}\NormalTok{ (p }\FunctionTok{:&:}\NormalTok{ selY) }\OtherTok{->}
        \KeywordTok{let}\NormalTok{ c }\FunctionTok{=}\NormalTok{ selX }\FunctionTok{:$:}\NormalTok{ selY}
        \KeywordTok{in}  \CommentTok{-- success???}
\end{Highlighting}
\end{Shaded}

Just to clarify what's going on, let's give types to the names above:

\begin{Shaded}
\begin{Highlighting}[]
\OtherTok{b    ::} \DataTypeTok{Sing}\NormalTok{ (}\OtherTok{b   ::}\NormalTok{ board        )}
\OtherTok{row  ::} \DataTypeTok{Sing}\NormalTok{ (}\OtherTok{row ::}\NormalTok{ [}\DataTypeTok{Maybe} \DataTypeTok{Piece}\NormalTok{])}
\OtherTok{selX ::} \DataTypeTok{Sel}\NormalTok{ i b row}
\OtherTok{p    ::} \DataTypeTok{Sing}\NormalTok{ (}\OtherTok{p   ::} \DataTypeTok{Maybe} \DataTypeTok{Piece}\NormalTok{  )}
\OtherTok{selY ::} \DataTypeTok{Sel}\NormalTok{ j row p}
\OtherTok{c    ::} \DataTypeTok{Coord}\NormalTok{ '(i, j) b p}
\end{Highlighting}
\end{Shaded}

\texttt{row} above is the \texttt{Sing} that comes attached with all \texttt{Σ}
constructors, which is why we can give it to \texttt{decide\ @(InBounds\ j)},
which expects a singleton of the list.

So, now we have \texttt{Coord\ \textquotesingle{}(i,\ j)\ b\ p}. We know that
\texttt{i} and \texttt{j} are in-bounds. But, we need to know that \texttt{p} is
\texttt{\textquotesingle{}Nothing} before we can use it with \texttt{PickValid}.
To do that, we can pattern match on \texttt{p}, because it's the singleton that
comes with the \texttt{Σ} constructor:

\begin{Shaded}
\begin{Highlighting}[]
\NormalTok{pick}
\OtherTok{    ::}\NormalTok{ forall i j b}\FunctionTok{.}\NormalTok{ ()}
    \OtherTok{=>} \DataTypeTok{Sing}\NormalTok{ i}
    \OtherTok{->} \DataTypeTok{Sing}\NormalTok{ j}
    \OtherTok{->} \DataTypeTok{Sing}\NormalTok{ b}
    \OtherTok{->} \DataTypeTok{Pick}\NormalTok{ '(i, j, b)}
\NormalTok{pick }\DataTypeTok{Sing} \DataTypeTok{Sing}\NormalTok{ b }\FunctionTok{=} \KeywordTok{case}\NormalTok{ decide }\FunctionTok{@}\NormalTok{(}\DataTypeTok{InBounds}\NormalTok{ i) b }\KeywordTok{of}
    \DataTypeTok{Proved}\NormalTok{ (row }\FunctionTok{:&:}\NormalTok{ selX) }\OtherTok{->} \KeywordTok{case}\NormalTok{ decide }\FunctionTok{@}\NormalTok{(}\DataTypeTok{InBounds}\NormalTok{ j) row }\KeywordTok{of}
      \DataTypeTok{Proved}\NormalTok{ (p }\FunctionTok{:&:}\NormalTok{ selY) }\OtherTok{->}
        \KeywordTok{let}\NormalTok{ c }\FunctionTok{=}\NormalTok{ selX }\FunctionTok{:$:}\NormalTok{ selY}
        \KeywordTok{in}  \KeywordTok{case}\NormalTok{ p }\KeywordTok{of}
              \DataTypeTok{SNothing} \OtherTok{->} \DataTypeTok{PickValid}\NormalTok{   c}
              \DataTypeTok{SJust}\NormalTok{ p' }\OtherTok{->} \DataTypeTok{PickPlayed}\NormalTok{  c p'}
\end{Highlighting}
\end{Shaded}

Finally, knowing that \texttt{p} is \texttt{\textquotesingle{}Nothing}, we can
create \texttt{PickValid}!

As a bonus, if we know that \texttt{p} is \texttt{\textquotesingle{}Just\ p}, we
can create \texttt{PickPlayed}, which is the constructor for an in-bounds pick
but pointing to a spot that is already occupied by piece
\texttt{p\textquotesingle{}}.

\begin{Shaded}
\begin{Highlighting}[]
\DataTypeTok{PickPlayed}\OtherTok{ ::} \DataTypeTok{Coord}\NormalTok{ '(i, j) b ('}\DataTypeTok{Just}\NormalTok{ p)}
           \OtherTok{->} \DataTypeTok{Sing}\NormalTok{ p}
           \OtherTok{->} \DataTypeTok{Pick}\NormalTok{ '(i, j, b)}
\end{Highlighting}
\end{Shaded}

We now have to deal with the situations where things are out of bounds.

\begin{Shaded}
\begin{Highlighting}[]
\DataTypeTok{PickOoBX}\OtherTok{ ::} \DataTypeTok{OutOfBounds}\NormalTok{ i }\FunctionTok{@@}\NormalTok{ b}
         \OtherTok{->} \DataTypeTok{Pick}\NormalTok{ '(i, j, b)}
\DataTypeTok{PickOoBY}\OtherTok{ ::} \DataTypeTok{Sel}\NormalTok{ i b row}
         \OtherTok{->} \DataTypeTok{OutOfBounds}\NormalTok{ j }\FunctionTok{@@}\NormalTok{ row}
         \OtherTok{->} \DataTypeTok{Pick}\NormalTok{ '(i, j, b)}
\end{Highlighting}
\end{Shaded}

However, thanks to the
\emph{\href{https://hackage.haskell.org/package/decidable}{decidable}} library,
things work out nicely. That's because \texttt{OutOfBounds\ n} we defined as:

\begin{Shaded}
\begin{Highlighting}[]
\CommentTok{-- source: https://github.com/mstksg/inCode/tree/master/code-samples/ttt/Part1.hs#L112-L112}

\KeywordTok{type} \DataTypeTok{OutOfBounds}\NormalTok{ n }\FunctionTok{=} \DataTypeTok{Not}\NormalTok{ (}\DataTypeTok{InBounds}\NormalTok{ n)}
\end{Highlighting}
\end{Shaded}

and \texttt{Not}, the predicate combinator, is defined as:

\begin{Shaded}
\begin{Highlighting}[]
\KeywordTok{data} \DataTypeTok{Not}\OtherTok{ ::} \DataTypeTok{Predicate}\NormalTok{ k }\OtherTok{->} \DataTypeTok{Predicate}\NormalTok{ k}

\KeywordTok{type} \KeywordTok{instance} \DataTypeTok{Apply}\NormalTok{ (}\DataTypeTok{Not}\NormalTok{ p) x }\FunctionTok{=}\NormalTok{ (p }\FunctionTok{@@}\NormalTok{ x) }\OtherTok{->} \DataTypeTok{Void}
\end{Highlighting}
\end{Shaded}

That is, a witness of \texttt{Not\ p\ @@\ x} is
\texttt{p\ @@\ x\ -\textgreater{}\ Void}. That means that \texttt{PickOoBX}
expects an \texttt{InBounds\ i\ @@\ b\ -\textgreater{}\ Void}, and
\texttt{PickOoBY} expects an
\texttt{InBounds\ j\ @@\ row\ -\textgreater{}\ Void}. And that's \emph{exactly}
what the \texttt{Disproved} branches give!

\begin{Shaded}
\begin{Highlighting}[]
\CommentTok{-- source: https://github.com/mstksg/inCode/tree/master/code-samples/ttt/Part1.hs#L169-L187}

\NormalTok{pick}
\OtherTok{    ::}\NormalTok{ forall i j b}\FunctionTok{.}\NormalTok{ ()}
    \OtherTok{=>} \DataTypeTok{Sing}\NormalTok{ i}
    \OtherTok{->} \DataTypeTok{Sing}\NormalTok{ j}
    \OtherTok{->} \DataTypeTok{Sing}\NormalTok{ b}
    \OtherTok{->} \DataTypeTok{Pick}\NormalTok{ '(i, j, b)}
\NormalTok{pick }\DataTypeTok{Sing} \DataTypeTok{Sing}\NormalTok{ b }\FunctionTok{=} \KeywordTok{case}\NormalTok{ decide }\FunctionTok{@}\NormalTok{(}\DataTypeTok{InBounds}\NormalTok{ i) b }\KeywordTok{of}
    \DataTypeTok{Proved}\NormalTok{ (row }\FunctionTok{:&:}\NormalTok{ selX) }\OtherTok{->} \KeywordTok{case}\NormalTok{ decide }\FunctionTok{@}\NormalTok{(}\DataTypeTok{InBounds}\NormalTok{ j) row }\KeywordTok{of}
      \DataTypeTok{Proved}\NormalTok{ (p }\FunctionTok{:&:}\NormalTok{ selY) }\OtherTok{->}
        \KeywordTok{let}\NormalTok{ c }\FunctionTok{=}\NormalTok{ selX }\FunctionTok{:$:}\NormalTok{ selY}
        \KeywordTok{in}  \KeywordTok{case}\NormalTok{ p }\KeywordTok{of}
              \DataTypeTok{SNothing} \OtherTok{->} \DataTypeTok{PickValid}\NormalTok{   c}
              \DataTypeTok{SJust}\NormalTok{ p' }\OtherTok{->} \DataTypeTok{PickPlayed}\NormalTok{  c p'}
      \DataTypeTok{Disproved}\NormalTok{ vY }\OtherTok{->} \DataTypeTok{PickOoBY}\NormalTok{ selX vY    }\CommentTok{-- vY :: InBounds j @@ row -> Void}
                                          \CommentTok{-- vY :: Not (InBounds j) @@ row}
                                          \CommentTok{-- vY :: OutOfBounds j @@ row}
    \DataTypeTok{Disproved}\NormalTok{ vX }\OtherTok{->} \DataTypeTok{PickOoBX}\NormalTok{ vX   }\CommentTok{-- vX :: InBounds i @@ b   -> Void}
                                  \CommentTok{-- vX :: Not (InBounds i) @@ b}
                                  \CommentTok{-- vX :: OutOfBounds i @@ b}
\end{Highlighting}
\end{Shaded}

And that's it!

Now to just tie it all together with a \texttt{Provable} instance, using the
\texttt{STuple3} singletons constructor:

\begin{Shaded}
\begin{Highlighting}[]
\CommentTok{-- source: https://github.com/mstksg/inCode/tree/master/code-samples/ttt/Part1.hs#L189-L191}

\KeywordTok{instance} \DataTypeTok{Provable}\NormalTok{ (}\DataTypeTok{TyPred} \DataTypeTok{Pick}\NormalTok{) }\KeywordTok{where}
\OtherTok{    prove ::} \DataTypeTok{Sing}\NormalTok{ ijb }\OtherTok{->} \DataTypeTok{Pick}\NormalTok{ ijb}
\NormalTok{    prove (}\DataTypeTok{STuple3}\NormalTok{ i j b) }\FunctionTok{=}\NormalTok{ pick i j b}
\end{Highlighting}
\end{Shaded}

\hypertarget{play-ball}{%
\section{Play Ball}\label{play-ball}}

Bringing it all together, we can write a simple function to take user input and
\emph{play} it.

\hypertarget{signoff}{%
\section{Signoff}\label{signoff}}

Hi, thanks for reading! You can reach me via email at
\href{mailto:justin@jle.im}{\nolinkurl{justin@jle.im}}, or at twitter at
\href{https://twitter.com/mstk}{@mstk}! This post and all others are published
under the \href{https://creativecommons.org/licenses/by-nc-nd/3.0/}{CC-BY-NC-ND
3.0} license. Corrections and edits via pull request are welcome and encouraged
at \href{https://github.com/mstksg/inCode}{the source repository}.

If you feel inclined, or this post was particularly helpful for you, why not
consider \href{https://www.patreon.com/justinle/overview}{supporting me on
Patreon}, or a \href{bitcoin:3D7rmAYgbDnp4gp4rf22THsGt74fNucPDU}{BTC donation}?
:)

\end{document}
