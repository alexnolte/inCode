\documentclass[]{article}
\usepackage{lmodern}
\usepackage{amssymb,amsmath}
\usepackage{ifxetex,ifluatex}
\usepackage{fixltx2e} % provides \textsubscript
\ifnum 0\ifxetex 1\fi\ifluatex 1\fi=0 % if pdftex
  \usepackage[T1]{fontenc}
  \usepackage[utf8]{inputenc}
\else % if luatex or xelatex
  \ifxetex
    \usepackage{mathspec}
    \usepackage{xltxtra,xunicode}
  \else
    \usepackage{fontspec}
  \fi
  \defaultfontfeatures{Mapping=tex-text,Scale=MatchLowercase}
  \newcommand{\euro}{€}
\fi
% use upquote if available, for straight quotes in verbatim environments
\IfFileExists{upquote.sty}{\usepackage{upquote}}{}
% use microtype if available
\IfFileExists{microtype.sty}{\usepackage{microtype}}{}
\usepackage[margin=1in]{geometry}
\ifxetex
  \usepackage[setpagesize=false, % page size defined by xetex
              unicode=false, % unicode breaks when used with xetex
              xetex]{hyperref}
\else
  \usepackage[unicode=true]{hyperref}
\fi
\hypersetup{breaklinks=true,
            bookmarks=true,
            pdfauthor={Justin Le},
            pdftitle={The Centrifugal Force isn't what you think it is},
            colorlinks=true,
            citecolor=blue,
            urlcolor=blue,
            linkcolor=magenta,
            pdfborder={0 0 0}}
\urlstyle{same}  % don't use monospace font for urls
% Make links footnotes instead of hotlinks:
\renewcommand{\href}[2]{#2\footnote{\url{#1}}}
\setlength{\parindent}{0pt}
\setlength{\parskip}{6pt plus 2pt minus 1pt}
\setlength{\emergencystretch}{3em}  % prevent overfull lines
\setcounter{secnumdepth}{0}

\title{The Centrifugal Force isn't what you think it is}
\author{Justin Le}

\begin{document}
\maketitle

\emph{Originally posted on
\textbf{\href{https://blog.jle.im/entry/fictitious-forces.html}{in Code}}.}

The other day I ran into \href{https://i.imgur.com/tkCsLYM.gifv}{this video} of
a pilot of a jet plane apparently pouring water from a bottle into a cup while
in the middle of a roll. The interesting effect is that the water continues to
flow into the cup, apparently defying gravity, throughout the entirety of the
roll. The camera follows the plane's rotation, so it looks like the water is
being poured normally the entire time, even though the ground is seen rotating
around the plane in the background.

In the ensuing discussion, someone commented this was a good example
demonstrating \emph{centrifugal force}. Someone then replied ``\emph{Actually,
it's the centripetal force.}''

I replied, and in the process of explaining why centripetal force \emph{doesn't}
explain what is happening here, I came to realize a few key misconceptions that
people have when thinking about ``fictitious'' (or pseudo-, or inertial) forces.

\hypertarget{two-sides-of-a-coin}{%
\section{Two Sides Of A Coin?}\label{two-sides-of-a-coin}}

Do you subscribe to this view?

\begin{quote}
The centrifugal force is not a ``real force'', because it's just ``the
centripetal force'' in disguise. Every usage of the centrifugal force can be
re-interpreted as ``the centripetal force'', and vice versa. It's just two ways
of looking at the same thing.
\end{quote}

This sort of view trains you into thinking that, if you see a ``centrifugal
force'', you should find the ``real force'' that is responsible for it. And that
``real force'' is ``the centripetal force''. That's what's \emph{really}
happening\ldots{} right? Newton's third law!

This fallacy is exactly what struck Commenter B. Commenter B saw something that
could be interpreted as a centrifugal force. Their logic then proceeded thusly:

\begin{itemize}
\tightlist
\item
  Ah, everyone knows that the centrifugal force isn't a ``real force''. I have
  to correct it.
\item
  Well, what's the \emph{real} force behind the water falling into the cup?
\item
  Isn't the centrifugal force just ``the centripetal force'', but in a different
  viewpoint?
\item
  So, it must be ``the centripetal force''!
\end{itemize}

Of course, if you do hold the viewpoint above, this all feels justified. And you
feel satisfied that you were able to
\href{https://www.edge.org/response-detail/11730}{give a name to the
phenomenon}. That is, until you think about it a little deeper.

Remember, the original question is ``When the water is in the air, what is
causing it to move towards the cup?''

If your answer is ``the centripetal force'', then you have some
problems\ldots{}because you then have to explain where the force is coming from.
It's not gravity, and it's not the spinning motion of the plane (that's going
``sideways'', the opposite direction of a centripetal force). So, what is it?
What force is acting on the water?

How can we have a centrifugal force without a centripetal force?

And that's where you start to see that, no matter what your views on the
metaphysical reality of fictitious forces are, ``centripetal force'' is
certainly \emph{not} the explanation.

So\ldots{}what is?

\hypertarget{down-to-the-root-of-it}{%
\section{Down to the root of it}\label{down-to-the-root-of-it}}

\hypertarget{the-centripetal-fallacy}{%
\subsection{The Centripetal Fallacy}\label{the-centripetal-fallacy}}

You might have noticed why I have been writing ``the centripetal force'' in
quotes. The reason? There is no such thing as ``the centripetal force''. It's a
lie.

One only needs to check the first words of the Wikipedia articles and compare
the difference to see:

\begin{itemize}
\tightlist
\item
  \textbf{\href{https://en.wikipedia.org/wiki/Centrifugal_force}{Centrifugal
  Force}}: \emph{The centrifugal force is\ldots{}}
\item
  \href{https://en.wikipedia.org/wiki/Centripetal_force}{Centripetal Force}:
  \emph{A centripetal force is\ldots{}}
\end{itemize}

Huh. One says ``the''. The other says ``a''. What's the difference?

The difference is that ``centripetal'' is a \emph{role} that a force plays. It
is an adjective, like dampening, supporting, big, small, little, weak, strong,
upwards-pointing. There is no such thing as ``\emph{the} centripetal force'', in
the sense that we have ``the gravitational force''. There are only forces that
are acting in a centripetal way, or play a centripetal role in our current
context.

We call a force ``centripetal'' when it is directed towards a center of
rotation, and causes the object's trajectory to follow a circular path.

There is no such thing as ``the centripetal force''. There are only forces that
are playing a centripetal role in our system. There might be many, there might
be none, or there might be just one.

\hypertarget{centrifugal-roots}{%
\subsection{Centrifugal Roots}\label{centrifugal-roots}}

\hypertarget{signoff}{%
\section{Signoff}\label{signoff}}

Hi, thanks for reading! You can reach me via email at
\href{mailto:justin@jle.im}{\nolinkurl{justin@jle.im}}, or at twitter at
\href{https://twitter.com/mstk}{@mstk}! This post and all others are published
under the \href{https://creativecommons.org/licenses/by-nc-nd/3.0/}{CC-BY-NC-ND
3.0} license. Corrections and edits via pull request are welcome and encouraged
at \href{https://github.com/mstksg/inCode}{the source repository}.

If you feel inclined, or this post was particularly helpful for you, why not
consider \href{https://www.patreon.com/justinle/overview}{supporting me on
Patreon}, or a \href{bitcoin:3D7rmAYgbDnp4gp4rf22THsGt74fNucPDU}{BTC donation}?
:)

\end{document}
