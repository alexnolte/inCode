\documentclass[]{article}
\usepackage{lmodern}
\usepackage{amssymb,amsmath}
\usepackage{ifxetex,ifluatex}
\usepackage{fixltx2e} % provides \textsubscript
\ifnum 0\ifxetex 1\fi\ifluatex 1\fi=0 % if pdftex
  \usepackage[T1]{fontenc}
  \usepackage[utf8]{inputenc}
\else % if luatex or xelatex
  \ifxetex
    \usepackage{mathspec}
    \usepackage{xltxtra,xunicode}
  \else
    \usepackage{fontspec}
  \fi
  \defaultfontfeatures{Mapping=tex-text,Scale=MatchLowercase}
  \newcommand{\euro}{€}
\fi
% use upquote if available, for straight quotes in verbatim environments
\IfFileExists{upquote.sty}{\usepackage{upquote}}{}
% use microtype if available
\IfFileExists{microtype.sty}{\usepackage{microtype}}{}
\usepackage[margin=1in]{geometry}
\usepackage{graphicx}
\makeatletter
\def\maxwidth{\ifdim\Gin@nat@width>\linewidth\linewidth\else\Gin@nat@width\fi}
\def\maxheight{\ifdim\Gin@nat@height>\textheight\textheight\else\Gin@nat@height\fi}
\makeatother
% Scale images if necessary, so that they will not overflow the page
% margins by default, and it is still possible to overwrite the defaults
% using explicit options in \includegraphics[width, height, ...]{}
\setkeys{Gin}{width=\maxwidth,height=\maxheight,keepaspectratio}
\ifxetex
  \usepackage[setpagesize=false, % page size defined by xetex
              unicode=false, % unicode breaks when used with xetex
              xetex]{hyperref}
\else
  \usepackage[unicode=true]{hyperref}
\fi
\hypersetup{breaklinks=true,
            bookmarks=true,
            pdfauthor={Justin Le},
            pdftitle={Giant Photons},
            colorlinks=true,
            citecolor=blue,
            urlcolor=blue,
            linkcolor=magenta,
            pdfborder={0 0 0}}
\urlstyle{same}  % don't use monospace font for urls
% Make links footnotes instead of hotlinks:
\renewcommand{\href}[2]{#2\footnote{\url{#1}}}
\setlength{\parindent}{0pt}
\setlength{\parskip}{6pt plus 2pt minus 1pt}
\setlength{\emergencystretch}{3em}  % prevent overfull lines
\setcounter{secnumdepth}{0}

\title{Giant Photons}
\author{Justin Le}

\begin{document}
\maketitle

\emph{Originally posted on
\textbf{\href{https://blog.jle.im/entry/giant-photon.html}{in Code}}.}

I was asked a question the other day where someone asked what the arrangement of
photons in a laser beam was as it traveled from the laser. Were they equally
spaced, crystalline, or random?

The question didn't seem to make sense to me at first, so I asked for further
clarification. It turns out that the asker viewed a laser beam as consisting of
a bunch of identical ball-like photons all traveling side-by-side, marching in
parallel lines towards their eventual goal. In their mind, a laser beam was like
a water stream, where the photons were individual water molecules. After all, a
laser beam has width, and a photon is a ``point particle'', right? So it must be
a bunch of photons traveling side-by-side.

This makes sense in the context of pop science. We talk about a beam of light as
consisting of photons, where its color is the frequency of its constituent
photons and its \emph{amplitude}/brightness as corresponding roughly to the
``number of photons'' emitted. If we do the math, then the average commercial
hand-held laser, operating at a power of around 1 mW and emitting 700nm red
light, would appear to have a cross-sectional flux of roughly
\includegraphics{https://latex.codecogs.com/png.latex?3.5\%20\%5Ctimes\%2010\%5E\%7B15\%7D}
(three and a half quadrillion) photons per second. So, in this picture, a
handheld laser would be like a hose spraying a quadrillion little balls out of
its end per second at the speed of light.

In this light, the asker's question begins to make sense. If this laser is
traveling in a straight line, then all of these quadrillion balls must also be
traveling perfectly straight side-by-side. So, are these balls all traveling in
a neat arrangement with some geometry, or is it random? The asker even proposed
a way to test this: place a detector screen in the path of the laser, and detect
the points where each laser hits the screen. And you, the reader\ldots does this
interpretation make sense to you? Can you predict the answer to this person's
question?

Answering this question required getting to the heart of popular
misunderstandings of the ``wave-particle duality'', and what a photon even
\emph{is}.

\hypertarget{what-is-a-photon}{%
\section{What is a Photon?}\label{what-is-a-photon}}

In this light, what \emph{is} a laser, if not a stream of photon balls? Well,
like all Quantum Mechanics problems, we can approach this in terms of a wave
function. In the case of the photon and of light, this wave function is an
excitation of electromagnetic field.

In particular, for a laser, this looks like a very powerful and coherent wave
front in the electromagnetic field. You might be aware of the fact that light is
``just'' waves in the electromagnetic field. Typically we see light as the
combination (``superposition'') of many different waves of different frequency
and phase\ldots and it's very rare that we get a wavefront of light that is just
one large coherent wave with a single frequency and single phase. Think of it as
the difference between a single large ocean wave (tsumani) vs.~many different
ocean waves of different widths all coming at once (like in a rough storm).

If we look at a laser beam as a wavefront\ldots{}``where'' is the photon here?

Consider that this wavefront has frequency and phase (the ups and downs in the
wave are all moving ``together''), but it also has \emph{amplitude} (roughly
corresponding to energy or intensity or brightness).

Now comes the quantum part. Light waves, being quantum concepts, can only ever
take on \emph{quantized} amplitudes, or quantized energies. This means that its
amplitude and energy are ``pixelated'', in a sense: they can only take on
discrete values.

In the case of energy, this discretization means that the energy of the laser
beam exists on a ``staircase'' of possible values, where each step is an
increase of
\includegraphics{https://latex.codecogs.com/png.latex?\%5CDelta\%20E\%20\%3D\%20\%5Cfrac\%7Bh\%20c\%7D\%7B\%5Clambda\%7D},
where \includegraphics{https://latex.codecogs.com/png.latex?h} is the Planck
constant and \includegraphics{https://latex.codecogs.com/png.latex?c} is the
seed of light in a vacuum. So for a 700nm red laser, this step size is about
1.77 electron-Volts of energy. You can have a 8.85eV red laser and a 10.62eV
laser (five steps, or six steps)\ldots but you can't have anything in-between
(no 9.5eV 700nm lasers!)

This means that if this laser beam ever \emph{loses} or \emph{gains} energy, it
can't lose any small amount of energy. Its energy can only ever be integer
multiples of 1.77eV\ldots so it can only ever gain or lose energy in increments
of 1.77eV!

This ``minimal increment'' of energy --- this abstract concept of a minimal
\includegraphics{https://latex.codecogs.com/png.latex?\%5CDelta\%20E} --- is
known as a \emph{photon}.

\emph{That} is a photon. Not a little point-particle ball you shoot out of a
canon, but rather an abstract \emph{name} we give to the minimal amount a wave
energy in the electromagnetic field is allowed to change. The \emph{concept} of
that minimal
\includegraphics{https://latex.codecogs.com/png.latex?\%5CDelta\%20E} is what we
call a photon.

Now, the EM field is neat because it allows for superposition --- two waves can
exist independently ``on top'' of each other, and their contributions to the EM
field will add together linearly. So, in this sense, we can think of a 10.62eV
laser beam as \emph{six identical 1.77eV beams} all existing ``on top of each
other''.

In the same sense, we can think of a commercial 1mW 700nm laser as consisting of
about 3.5 quadrillion \emph{identical} 1.77ev beams, all stacked \emph{on top}
of each other. Each of these beams have identical frequency, phase, position
\ldots{} but also they also all have the same full width of the laser beam.

So, a laser beam doesn't consist of three and a half quadrillion little balls
all flying out side-by-side: a 1mW laser beam can be thought of as being a sum
(superposition) of 3.5 quadrillion \emph{identical} weaker (1.77eV) equally wide
laser beams. And each of these \emph{hypothetical} weaker beams is known as a
photon.

\hypertarget{signoff}{%
\section{Signoff}\label{signoff}}

Hi, thanks for reading! You can reach me via email at
\href{mailto:justin@jle.im}{\nolinkurl{justin@jle.im}}, or at twitter at
\href{https://twitter.com/mstk}{@mstk}! This post and all others are published
under the \href{https://creativecommons.org/licenses/by-nc-nd/3.0/}{CC-BY-NC-ND
3.0} license. Corrections and edits via pull request are welcome and encouraged
at \href{https://github.com/mstksg/inCode}{the source repository}.

If you feel inclined, or this post was particularly helpful for you, why not
consider \href{https://www.patreon.com/justinle/overview}{supporting me on
Patreon}, or a \href{bitcoin:3D7rmAYgbDnp4gp4rf22THsGt74fNucPDU}{BTC donation}?
:)

\end{document}
