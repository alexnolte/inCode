\documentclass[]{article}
\usepackage{lmodern}
\usepackage{amssymb,amsmath}
\usepackage{ifxetex,ifluatex}
\usepackage{fixltx2e} % provides \textsubscript
\ifnum 0\ifxetex 1\fi\ifluatex 1\fi=0 % if pdftex
  \usepackage[T1]{fontenc}
  \usepackage[utf8]{inputenc}
\else % if luatex or xelatex
  \ifxetex
    \usepackage{mathspec}
    \usepackage{xltxtra,xunicode}
  \else
    \usepackage{fontspec}
  \fi
  \defaultfontfeatures{Mapping=tex-text,Scale=MatchLowercase}
  \newcommand{\euro}{€}
\fi
% use upquote if available, for straight quotes in verbatim environments
\IfFileExists{upquote.sty}{\usepackage{upquote}}{}
% use microtype if available
\IfFileExists{microtype.sty}{\usepackage{microtype}}{}
\usepackage[margin=1in]{geometry}
\usepackage{color}
\usepackage{fancyvrb}
\newcommand{\VerbBar}{|}
\newcommand{\VERB}{\Verb[commandchars=\\\{\}]}
\DefineVerbatimEnvironment{Highlighting}{Verbatim}{commandchars=\\\{\}}
% Add ',fontsize=\small' for more characters per line
\newenvironment{Shaded}{}{}
\newcommand{\AlertTok}[1]{\textcolor[rgb]{1.00,0.00,0.00}{\textbf{#1}}}
\newcommand{\AnnotationTok}[1]{\textcolor[rgb]{0.38,0.63,0.69}{\textbf{\textit{#1}}}}
\newcommand{\AttributeTok}[1]{\textcolor[rgb]{0.49,0.56,0.16}{#1}}
\newcommand{\BaseNTok}[1]{\textcolor[rgb]{0.25,0.63,0.44}{#1}}
\newcommand{\BuiltInTok}[1]{#1}
\newcommand{\CharTok}[1]{\textcolor[rgb]{0.25,0.44,0.63}{#1}}
\newcommand{\CommentTok}[1]{\textcolor[rgb]{0.38,0.63,0.69}{\textit{#1}}}
\newcommand{\CommentVarTok}[1]{\textcolor[rgb]{0.38,0.63,0.69}{\textbf{\textit{#1}}}}
\newcommand{\ConstantTok}[1]{\textcolor[rgb]{0.53,0.00,0.00}{#1}}
\newcommand{\ControlFlowTok}[1]{\textcolor[rgb]{0.00,0.44,0.13}{\textbf{#1}}}
\newcommand{\DataTypeTok}[1]{\textcolor[rgb]{0.56,0.13,0.00}{#1}}
\newcommand{\DecValTok}[1]{\textcolor[rgb]{0.25,0.63,0.44}{#1}}
\newcommand{\DocumentationTok}[1]{\textcolor[rgb]{0.73,0.13,0.13}{\textit{#1}}}
\newcommand{\ErrorTok}[1]{\textcolor[rgb]{1.00,0.00,0.00}{\textbf{#1}}}
\newcommand{\ExtensionTok}[1]{#1}
\newcommand{\FloatTok}[1]{\textcolor[rgb]{0.25,0.63,0.44}{#1}}
\newcommand{\FunctionTok}[1]{\textcolor[rgb]{0.02,0.16,0.49}{#1}}
\newcommand{\ImportTok}[1]{#1}
\newcommand{\InformationTok}[1]{\textcolor[rgb]{0.38,0.63,0.69}{\textbf{\textit{#1}}}}
\newcommand{\KeywordTok}[1]{\textcolor[rgb]{0.00,0.44,0.13}{\textbf{#1}}}
\newcommand{\NormalTok}[1]{#1}
\newcommand{\OperatorTok}[1]{\textcolor[rgb]{0.40,0.40,0.40}{#1}}
\newcommand{\OtherTok}[1]{\textcolor[rgb]{0.00,0.44,0.13}{#1}}
\newcommand{\PreprocessorTok}[1]{\textcolor[rgb]{0.74,0.48,0.00}{#1}}
\newcommand{\RegionMarkerTok}[1]{#1}
\newcommand{\SpecialCharTok}[1]{\textcolor[rgb]{0.25,0.44,0.63}{#1}}
\newcommand{\SpecialStringTok}[1]{\textcolor[rgb]{0.73,0.40,0.53}{#1}}
\newcommand{\StringTok}[1]{\textcolor[rgb]{0.25,0.44,0.63}{#1}}
\newcommand{\VariableTok}[1]{\textcolor[rgb]{0.10,0.09,0.49}{#1}}
\newcommand{\VerbatimStringTok}[1]{\textcolor[rgb]{0.25,0.44,0.63}{#1}}
\newcommand{\WarningTok}[1]{\textcolor[rgb]{0.38,0.63,0.69}{\textbf{\textit{#1}}}}
\ifxetex
  \usepackage[setpagesize=false, % page size defined by xetex
              unicode=false, % unicode breaks when used with xetex
              xetex]{hyperref}
\else
  \usepackage[unicode=true]{hyperref}
\fi
\hypersetup{breaklinks=true,
            bookmarks=true,
            pdfauthor={},
            pdftitle={Lazy Patterns, Strict Matches, and All That: My Mental Model of Haskell},
            colorlinks=true,
            citecolor=blue,
            urlcolor=blue,
            linkcolor=magenta,
            pdfborder={0 0 0}}
\urlstyle{same}  % don't use monospace font for urls
% Make links footnotes instead of hotlinks:
\renewcommand{\href}[2]{#2\footnote{\url{#1}}}
\setlength{\parindent}{0pt}
\setlength{\parskip}{6pt plus 2pt minus 1pt}
\setlength{\emergencystretch}{3em}  % prevent overfull lines
\setcounter{secnumdepth}{0}

\title{Lazy Patterns, Strict Matches, and All That: My Mental Model of Haskell}

\begin{document}
\maketitle

Evaluation \% Justin Le

\emph{Originally posted on
\textbf{\href{https://blog.jle.im/entry/lazy-strict-mental-model.html}{in
Code}}.}

Every once in a while I get questions about ``what does this
\texttt{\textasciitilde{}} do in a pattern'', what strict pattern matching does,
etc. And while you can understand code like
\texttt{case\ x\ of\ \textasciitilde{}(a,b)\ -\textgreater{}\ ...} in terms of
one-off special cases, I think that having a good mental model of how evaluation
with ADTs work in Haskell will allow you to answer these questions with
confidence, and without second-guessing yourself.

This is less an article on formal semantics as it as an article trying to
explain some of the heuristics and mental models I've developed over my years of
writing Haskell to help me understand how things work. It's not a textbook, but
an experience report :)

\hypertarget{let-and-case}{%
\section{Let and Case}\label{let-and-case}}

A lot of the distinction between lazy and strict matching can be illustrated by
comparing the semantics of \texttt{let} vs.~\texttt{case}. (We're going to be
ignoring type checking here and assuming all values are nice monomorphic
values).

For example, let's see the distinction between:

\begin{Shaded}
\begin{Highlighting}[]
\KeywordTok{let}\NormalTok{ (x, y) }\OtherTok{=} \FunctionTok{undefined}
\KeywordTok{in}  \DataTypeTok{True}

\KeywordTok{case} \FunctionTok{undefined} \KeywordTok{of}
\NormalTok{    (x, y) }\OtherTok{{-}>} \DataTypeTok{True}
\end{Highlighting}
\end{Shaded}

What will either of these return, when you try to run it?

Small note --- here, we are using \texttt{undefined} as a way to directly
observe evaluation in Haskell. This is a common trick people use when discussing
evaluation in Haskell --- it works because \texttt{undefined\ ::\ a} can take on
any type, and any attempt to evaluate \texttt{undefined} can be immediately
detected:

\begin{Shaded}
\begin{Highlighting}[]
\NormalTok{ghci}\OperatorTok{>} \FunctionTok{sum}\NormalTok{ [}\DecValTok{1}\NormalTok{,}\DecValTok{2}\NormalTok{,}\FunctionTok{undefined}\NormalTok{,}\DecValTok{100}\NormalTok{]}
\OperatorTok{***} \DataTypeTok{Exception}\OperatorTok{:}\NormalTok{ Prelude.undefined}
\NormalTok{ghci}\OperatorTok{>} \FunctionTok{sum}\NormalTok{ (}\FunctionTok{take} \DecValTok{2}\NormalTok{ [}\DecValTok{1}\NormalTok{,}\DecValTok{2}\NormalTok{,}\FunctionTok{undefined}\NormalTok{,}\DecValTok{100}\NormalTok{])}
\DecValTok{3}
\end{Highlighting}
\end{Shaded}

Think of it like laying a mine for poor old GHC to encounter. GHC tries to
evaluate things, but when it tries to evaluate \texttt{undefined}, it
immediately explodes and tells you that it has a boo-boo. Our investigation
becomes reframed as ``is this mine ever tripped, or not?''

\begin{center}\rule{0.5\linewidth}{\linethickness}\end{center}

\textbf{Alternatives to \texttt{undefined}}

Admittedly, the idea that \texttt{undefined} can take on any type can seem a
little spooky -- so often I like to use something within the correct type to
make things feel a little more concrete. For example, instead of
\texttt{undefined}, we can write:

\begin{Shaded}
\begin{Highlighting}[]
\OtherTok{myTuple ::}\NormalTok{ (}\DataTypeTok{Int}\NormalTok{, }\DataTypeTok{String}\NormalTok{)}
\NormalTok{myTuple}
  \OperatorTok{|} \FunctionTok{sum}\NormalTok{ [}\DecValTok{1}\OperatorTok{..}\NormalTok{] }\OperatorTok{<} \DecValTok{1000} \OtherTok{=}\NormalTok{ (}\DecValTok{3}\NormalTok{, }\StringTok{"hi"}\NormalTok{)}
  \OperatorTok{|} \FunctionTok{otherwise}        \OtherTok{=}\NormalTok{ (}\DecValTok{5}\NormalTok{, }\StringTok{"bye"}\NormalTok{)}

\KeywordTok{let}\NormalTok{ (x, y) }\OtherTok{=}\NormalTok{ myTuple}
\KeywordTok{in}  \DataTypeTok{True}

\KeywordTok{case}\NormalTok{ myTuple }\KeywordTok{of}
\NormalTok{    (x, y) }\OtherTok{{-}>} \DataTypeTok{True}
\end{Highlighting}
\end{Shaded}

This example should display all of the same nuances of using \texttt{undefined}
--- the only practical difference is that instead of seeing
\texttt{Exception:\ Prelude.undefined} if the value is ever evaluated, you'll
instead just see a program locking up if a value is evaluated.

\begin{center}\rule{0.5\linewidth}{\linethickness}\end{center}

With that in mind, let's find the answer to our previous question:

\begin{Shaded}
\begin{Highlighting}[]
\NormalTok{ghci}\OperatorTok{>} \KeywordTok{let}\NormalTok{ (x,y) }\OtherTok{=} \FunctionTok{undefined} \KeywordTok{in} \DataTypeTok{True}
\DataTypeTok{True}
\NormalTok{ghci}\OperatorTok{>} \KeywordTok{case} \FunctionTok{undefined} \KeywordTok{of}\NormalTok{ (x,y) }\OtherTok{{-}>} \DataTypeTok{True}
\OperatorTok{***} \DataTypeTok{Exception}\OperatorTok{:}\NormalTok{ Prelude.undefined}
\end{Highlighting}
\end{Shaded}

So, there's a difference! What's going on here?

In my mind, this is how I see things:

\begin{enumerate}
\def\labelenumi{\arabic{enumi}.}
\tightlist
\item
  \emph{let} is \textbf{aliasing}
\item
  \emph{case} is \textbf{control flow}
\end{enumerate}

To apply this to what we just saw:

\begin{enumerate}
\def\labelenumi{\arabic{enumi}.}
\item
  \emph{let} is just aliasing \texttt{x} and \texttt{y} to whatever is in the
  body. If you don't use the alias, it's no big deal, really. It's just like if
  you used \texttt{alias\ abc\ rm\ -rf\ /} in \emph{bash}. Aliasing \texttt{abc}
  to \texttt{rm\ -rf\ /} doesn't actually erase your entire file system, it's
  only something that does it if you decide to use the alias.
\item
  \emph{case} here \emph{decides} what to do based on what the input is.
  \emph{If} you are given \texttt{(x,\ y)}, \emph{then} return \texttt{True}.
  It's used to direct the flow of how a program executes, so it \emph{has} to
  look at the value to see what to do.
\end{enumerate}

So, the reason that the \texttt{case} version explodes when you attempt to
evaluate it is because it has to ask the question ``Does this value match this
pattern?'' --- and in the process of investigating it, it has to touch
\texttt{undefined} and so sets off the mine.

The \emph{let} situation never has to answer the question ``does this value
match this pattern?'', and so never needs to investigate it and so doesn't
explode.

\hypertarget{option-one-of-one}{%
\subsection{Option One of One}\label{option-one-of-one}}

To someone new to Haskell (or unfamiliar with its semantics), it kind of seems
silly that \texttt{case} would ``have'' to check \emph{which} constructor is
being used\ldots since there is really only one option, \texttt{,}. Why can't
\texttt{case} just know that it \emph{has} to be \texttt{,} and move on, without
checking? Surely that would be more sensible.

For example, let's look at a situation that is uncontroversial

\begin{Shaded}
\begin{Highlighting}[]
\NormalTok{ghci}\OperatorTok{>} \KeywordTok{case} \FunctionTok{undefined} \KeywordTok{of}
        \DataTypeTok{Left}\NormalTok{ x }\OtherTok{{-}>} \DataTypeTok{False}
        \DataTypeTok{Right}\NormalTok{ y }\OtherTok{{-}>} \DataTypeTok{True}
\OperatorTok{***} \DataTypeTok{Exception}\OperatorTok{:}\NormalTok{ Prelude.undefined}
\end{Highlighting}
\end{Shaded}

Of \emph{course} that one has to explode, because it has to check if the
\texttt{Either} is \texttt{Left} or \texttt{Right}. But in the case of
\texttt{(x,\ y)}, it's always going to be \texttt{,}, so why can't it just
``know'' you are \texttt{(x,\ y)} since there isn't any other situation?

\begin{Shaded}
\begin{Highlighting}[]
\NormalTok{ghci}\OperatorTok{>} \KeywordTok{case} \FunctionTok{undefined} \KeywordTok{of}
\NormalTok{        (x, y) }\OtherTok{{-}>} \DataTypeTok{False}
\OperatorTok{***} \DataTypeTok{Exception}\OperatorTok{:}\NormalTok{ Prelude.undefined}
\end{Highlighting}
\end{Shaded}

We know the answer will be \texttt{False} no matter what, so why bother
checking? Why can't we just have it skip the check and return \texttt{False}?

There are a couple of ways to approach this answer\ldots and I think all
approaches illustrate something important about the philosophy of writing
Haskell.

\begin{enumerate}
\def\labelenumi{\arabic{enumi}.}
\item
  If you shift your question from ``which constructor is it?'' to ``which
  constructor was it \emph{made} with?'', then it starts to make sense.

  Pattern matching isn't about asking ``which constructor does this value
  match?'', but rather ``which constructor was this value \emph{made} with?''

  So, it would be unwise to just have GHC skip the check. When we are doing

\begin{Shaded}
\begin{Highlighting}[]
\KeywordTok{case}\NormalTok{ blahblah }\KeywordTok{of}
\NormalTok{  (x, y) }\OtherTok{{-}>} \CommentTok{{-}{-} ...}
\end{Highlighting}
\end{Shaded}

  We don't \emph{really} care ``what is \texttt{blahblah}''? We care ``how was
  \texttt{blahblah} made?''

  Under this light, we see that there actually are many ways to make a tuple
  value \emph{without} ever using \texttt{(,)}:

\begin{Shaded}
\begin{Highlighting}[]
\OtherTok{tup1 ::}\NormalTok{ (}\DataTypeTok{Int}\NormalTok{, }\DataTypeTok{Bool}\NormalTok{)}
\NormalTok{tup1 }\OtherTok{=} \FunctionTok{undefined}

\OtherTok{tup2 ::}\NormalTok{ (}\DataTypeTok{Int}\NormalTok{, }\DataTypeTok{Bool}\NormalTok{)}
\NormalTok{tup2 }\OtherTok{=}\NormalTok{ go }\DecValTok{0}
  \KeywordTok{where}
\NormalTok{    go n }\OtherTok{=}\NormalTok{ go (n }\OperatorTok{+} \DecValTok{1}\NormalTok{)}

\OtherTok{tup3 ::}\NormalTok{ (}\DataTypeTok{Int}\NormalTok{, }\DataTypeTok{Bool}\NormalTok{)}
\NormalTok{tup3}
  \OperatorTok{|} \FunctionTok{even}\NormalTok{ (}\FunctionTok{sum}\NormalTok{ [}\DecValTok{1}\OperatorTok{..}\NormalTok{]) }\OtherTok{=}\NormalTok{ (}\DecValTok{3}\NormalTok{, }\DataTypeTok{True}\NormalTok{)}
  \OperatorTok{|} \FunctionTok{otherwise}        \OtherTok{=} \FunctionTok{undefined}
\end{Highlighting}
\end{Shaded}

  We can see here three perfectly valid values of type \texttt{(Int,\ Bool)}
  that are \emph{not} constructed with \texttt{(,)}, or at least where the
  question is ambiguous and up in the air. So \emph{skipping} checking the
  constructor here is skipping the real question -- not ``what pattern does it
  match'', but ``what was it made with''.
\item
  Haskell \texttt{data} declarations with boxed fields all follow the same
  general schema in memory:

\begin{Shaded}
\begin{Highlighting}[]
\KeywordTok{data} \DataTypeTok{MyType} \OtherTok{=} \DataTypeTok{A} \DataTypeTok{Int} \DataTypeTok{Bool}
            \OperatorTok{|} \DataTypeTok{B} \DataTypeTok{String} \DataTypeTok{Double}
\end{Highlighting}
\end{Shaded}

\begin{verbatim}
+--------+------------------------+-------------------------+
| A or B | pointer to first field | pointer to second field |
+--------+------------------------+-------------------------+
\end{verbatim}

  So, for example, a value like \texttt{A\ 3\ True} would look like:

\begin{Shaded}
\begin{Highlighting}[]
    \OperatorTok{+{-}{-}{-}+{-}{-}{-}+{-}{-}{-}+}
\OperatorTok{{-}{-}>} \OperatorTok{|} \DataTypeTok{A} \OperatorTok{|}\NormalTok{ o }\OperatorTok{|}\NormalTok{ o }\OperatorTok{|}
    \OperatorTok{+{-}{-}{-}+{-}|{-}+{-}|{-}+}
      \OperatorTok{|}   \OperatorTok{|}
\NormalTok{      v   v}
      \DecValTok{3}   \DataTypeTok{True}
\end{Highlighting}
\end{Shaded}

  For a type with a single constructor, that initial ``which constructor'' field
  is still there. This just makes things a little more consistent --- there is
  no special-case for a single-constructor situation.

  When considering sematics, there is no fundamental difference between a
  single-constructor type and a multi-constructor type. A single-constructor
  type is just an n-constructor type where n = 1. This means that a type like:

\begin{Shaded}
\begin{Highlighting}[]
\KeywordTok{data} \DataTypeTok{MyType} \OtherTok{=} \DataTypeTok{A} \DataTypeTok{Int} \DataTypeTok{Bool}
\end{Highlighting}
\end{Shaded}

  will still have the constructor byte at the front:

\begin{verbatim}
+---+------------------------+-------------------------+
| A | pointer to first field | pointer to second field |
+---+------------------------+-------------------------+
\end{verbatim}
\end{enumerate}

To me, the underlying reason is \emph{consistency}. A lot of beginners, when
starting Haskell, come to the impression that Haskell is \emph{magical}: it just
does the best thing in every situation. For example, a lot of Haskell learners
somehow all independently arrive at the assumption that using the same function
on the same argument will only compute it once, and cache the result. After all,
it's possible (because all functions are pure, so calling the same function on
the same value will always return the same result), so why not have it magically
be true?

However, as you learn more Haskell, you start to realize that what might have
seemed magical at first is just the straightforward application of a few rules
that apply generally, with rare special-cases. That's because for a language to
be useful, it's not about \emph{magic}, but \emph{predictability}.

So, case matching on a single-constructor type should not be special-cased from
the mechanics of a two-constructor type. To check for a match, you always have
to peek at the constructor --- no matter how many there might be.

Another way you can look at it is that ADTs are stored in haskell as a
\texttt{(ConstructorTag,\ Fields)} tuple --- and all \texttt{data}-defined ADTs
are stored in that way, without special-case exceptions.

(Of course, the exception to this rule is \texttt{newtype}s, which follow
different rules from \texttt{data} in general)

And, of course, it's also possible to show ``All tuples must match the
\texttt{(x,y)} pattern and be constructed with \texttt{(,)}'' as false ---
\texttt{undefined\ ::\ (Int,\ Bool)} is an example of a tuple that isn't
constructed with \texttt{(,)} and so doesn't match the \texttt{(x,\ y)} pattern.

\hypertarget{signoff}{%
\section{Signoff}\label{signoff}}

Hi, thanks for reading! You can reach me via email at
\href{mailto:justin@jle.im}{\nolinkurl{justin@jle.im}}, or at twitter at
\href{https://twitter.com/mstk}{@mstk}! This post and all others are published
under the \href{https://creativecommons.org/licenses/by-nc-nd/3.0/}{CC-BY-NC-ND
3.0} license. Corrections and edits via pull request are welcome and encouraged
at \href{https://github.com/mstksg/inCode}{the source repository}.

If you feel inclined, or this post was particularly helpful for you, why not
consider \href{https://www.patreon.com/justinle/overview}{supporting me on
Patreon}, or a \href{bitcoin:3D7rmAYgbDnp4gp4rf22THsGt74fNucPDU}{BTC donation}?
:)

\end{document}
