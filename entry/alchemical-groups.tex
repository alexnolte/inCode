\documentclass[]{article}
\usepackage{lmodern}
\usepackage{amssymb,amsmath}
\usepackage{ifxetex,ifluatex}
\usepackage{fixltx2e} % provides \textsubscript
\ifnum 0\ifxetex 1\fi\ifluatex 1\fi=0 % if pdftex
  \usepackage[T1]{fontenc}
  \usepackage[utf8]{inputenc}
\else % if luatex or xelatex
  \ifxetex
    \usepackage{mathspec}
    \usepackage{xltxtra,xunicode}
  \else
    \usepackage{fontspec}
  \fi
  \defaultfontfeatures{Mapping=tex-text,Scale=MatchLowercase}
  \newcommand{\euro}{€}
\fi
% use upquote if available, for straight quotes in verbatim environments
\IfFileExists{upquote.sty}{\usepackage{upquote}}{}
% use microtype if available
\IfFileExists{microtype.sty}{\usepackage{microtype}}{}
\usepackage[margin=1in]{geometry}
\usepackage{color}
\usepackage{fancyvrb}
\newcommand{\VerbBar}{|}
\newcommand{\VERB}{\Verb[commandchars=\\\{\}]}
\DefineVerbatimEnvironment{Highlighting}{Verbatim}{commandchars=\\\{\}}
% Add ',fontsize=\small' for more characters per line
\newenvironment{Shaded}{}{}
\newcommand{\AlertTok}[1]{\textcolor[rgb]{1.00,0.00,0.00}{\textbf{#1}}}
\newcommand{\AnnotationTok}[1]{\textcolor[rgb]{0.38,0.63,0.69}{\textbf{\textit{#1}}}}
\newcommand{\AttributeTok}[1]{\textcolor[rgb]{0.49,0.56,0.16}{#1}}
\newcommand{\BaseNTok}[1]{\textcolor[rgb]{0.25,0.63,0.44}{#1}}
\newcommand{\BuiltInTok}[1]{#1}
\newcommand{\CharTok}[1]{\textcolor[rgb]{0.25,0.44,0.63}{#1}}
\newcommand{\CommentTok}[1]{\textcolor[rgb]{0.38,0.63,0.69}{\textit{#1}}}
\newcommand{\CommentVarTok}[1]{\textcolor[rgb]{0.38,0.63,0.69}{\textbf{\textit{#1}}}}
\newcommand{\ConstantTok}[1]{\textcolor[rgb]{0.53,0.00,0.00}{#1}}
\newcommand{\ControlFlowTok}[1]{\textcolor[rgb]{0.00,0.44,0.13}{\textbf{#1}}}
\newcommand{\DataTypeTok}[1]{\textcolor[rgb]{0.56,0.13,0.00}{#1}}
\newcommand{\DecValTok}[1]{\textcolor[rgb]{0.25,0.63,0.44}{#1}}
\newcommand{\DocumentationTok}[1]{\textcolor[rgb]{0.73,0.13,0.13}{\textit{#1}}}
\newcommand{\ErrorTok}[1]{\textcolor[rgb]{1.00,0.00,0.00}{\textbf{#1}}}
\newcommand{\ExtensionTok}[1]{#1}
\newcommand{\FloatTok}[1]{\textcolor[rgb]{0.25,0.63,0.44}{#1}}
\newcommand{\FunctionTok}[1]{\textcolor[rgb]{0.02,0.16,0.49}{#1}}
\newcommand{\ImportTok}[1]{#1}
\newcommand{\InformationTok}[1]{\textcolor[rgb]{0.38,0.63,0.69}{\textbf{\textit{#1}}}}
\newcommand{\KeywordTok}[1]{\textcolor[rgb]{0.00,0.44,0.13}{\textbf{#1}}}
\newcommand{\NormalTok}[1]{#1}
\newcommand{\OperatorTok}[1]{\textcolor[rgb]{0.40,0.40,0.40}{#1}}
\newcommand{\OtherTok}[1]{\textcolor[rgb]{0.00,0.44,0.13}{#1}}
\newcommand{\PreprocessorTok}[1]{\textcolor[rgb]{0.74,0.48,0.00}{#1}}
\newcommand{\RegionMarkerTok}[1]{#1}
\newcommand{\SpecialCharTok}[1]{\textcolor[rgb]{0.25,0.44,0.63}{#1}}
\newcommand{\SpecialStringTok}[1]{\textcolor[rgb]{0.73,0.40,0.53}{#1}}
\newcommand{\StringTok}[1]{\textcolor[rgb]{0.25,0.44,0.63}{#1}}
\newcommand{\VariableTok}[1]{\textcolor[rgb]{0.10,0.09,0.49}{#1}}
\newcommand{\VerbatimStringTok}[1]{\textcolor[rgb]{0.25,0.44,0.63}{#1}}
\newcommand{\WarningTok}[1]{\textcolor[rgb]{0.38,0.63,0.69}{\textbf{\textit{#1}}}}
\usepackage{graphicx}
\makeatletter
\def\maxwidth{\ifdim\Gin@nat@width>\linewidth\linewidth\else\Gin@nat@width\fi}
\def\maxheight{\ifdim\Gin@nat@height>\textheight\textheight\else\Gin@nat@height\fi}
\makeatother
% Scale images if necessary, so that they will not overflow the page
% margins by default, and it is still possible to overwrite the defaults
% using explicit options in \includegraphics[width, height, ...]{}
\setkeys{Gin}{width=\maxwidth,height=\maxheight,keepaspectratio}
\ifxetex
  \usepackage[setpagesize=false, % page size defined by xetex
              unicode=false, % unicode breaks when used with xetex
              xetex]{hyperref}
\else
  \usepackage[unicode=true]{hyperref}
\fi
\hypersetup{breaklinks=true,
            bookmarks=true,
            pdfauthor={Justin Le},
            pdftitle={Alchemical Groups: Advent of Code with Free Groups and Group Homomorphisms},
            colorlinks=true,
            citecolor=blue,
            urlcolor=blue,
            linkcolor=magenta,
            pdfborder={0 0 0}}
\urlstyle{same}  % don't use monospace font for urls
% Make links footnotes instead of hotlinks:
\renewcommand{\href}[2]{#2\footnote{\url{#1}}}
\setlength{\parindent}{0pt}
\setlength{\parskip}{6pt plus 2pt minus 1pt}
\setlength{\emergencystretch}{3em}  % prevent overfull lines
\setcounter{secnumdepth}{0}

\title{Alchemical Groups: Advent of Code with Free Groups and Group Homomorphisms}
\author{Justin Le}
\date{December 5, 2018}

\begin{document}
\maketitle

\emph{Originally posted on
\textbf{\href{https://blog.jle.im/entry/alchemical-groups.html}{in Code}}.}

Hi all! If you don't already know,
\emph{\href{https://adventofcode.com/2018}{Advent of Code}} is in full swing
this year! If you're participating and using Haskell, you're welcome to join us
at \href{https://twitter.com/glguy}{glguy}'s semi-official
\href{https://adventofcode.com/2018/leaderboard/private}{Haskell Leaderboard}
(join code \texttt{43100-84040706})! There are also Haskellers on freenode
\#\#adventofcode, and also \#adventofcode on the Functional Programming slack. I
also wrote a
\href{https://hackage.haskell.org/package/advent-of-code-api}{haskell library to
the API}, if you're looking to streamline your process!

My
\href{https://github.com/mstksg/advent-of-code-2018/blob/master/reflections.md}{daily
reflections} are online, where I talk about how I approach each problem and what
insight purely typed Functional Programming gives us for each problem.

Every once in a while I'll find a challenge that I think can be pulled out as a
short blog post, so I'll bring them onto the blog for a more long-term sort of
longevity!\footnote{These short posts won't be counted as ``paid'' Patreon
  posts.}

In this one, I'll talk about using group theory to solve the \emph{Day 5}
challenge. Spoilers for those who have not solved it yet!

\hypertarget{part-1}{%
\section{Part 1}\label{part-1}}

\begin{quote}
You've managed to sneak in to the prototype suit manufacturing lab. The Elves
are making decent progress, but are still struggling with the suit's size
reduction capabilities.

While the very latest in 1518 alchemical technology might have solved their
problem eventually, you can do better. You scan the chemical composition of the
suit's material and discover that it is formed by extremely long
\href{https://en.wikipedia.org/wiki/Polymer}{polymers} (one of which is
{available} as your puzzle input).

The polymer is formed by smaller \emph{units} which, when triggered, react with
each other such that two adjacent units of the same type and opposite polarity
are destroyed. Units' types are represented by letters; units' polarity is
represented by capitalization. For instance, \texttt{r} and \texttt{R} are units
with the same type but opposite polarity, whereas \texttt{r} and \texttt{s} are
entirely different types and do not react.

For example:

\begin{itemize}
\tightlist
\item
  In \texttt{aA}, \texttt{a} and \texttt{A} react, leaving nothing behind.
\item
  In \texttt{abBA}, \texttt{bB} destroys itself, leaving \texttt{aA}. As above,
  this then destroys itself, leaving nothing.
\item
  In \texttt{abAB}, no two adjacent units are of the same type, and so nothing
  happens.
\item
  In \texttt{aabAAB}, even though \texttt{aa} and \texttt{AA} are of the same
  type, their polarities match, and so nothing happens.
\end{itemize}

Now, consider a larger example, \texttt{dabAcCaCBAcCcaDA}:

\begin{verbatim}
dabAcCaCBAcCcaDA  The first 'cC' is removed.
dabAaCBAcCcaDA    This creates 'Aa', which is removed.
dabCBAcCcaDA      Either 'cC' or 'Cc' are removed (the result is the same).
dabCBAcaDA        No further actions can be taken.
\end{verbatim}

After all possible reactions, the resulting polymer contains \emph{10 units}.

\emph{How many units remain after fully reacting the polymer you scanned?}
\end{quote}

Now, I feel like this problem might have been deliberately written to obscure
this fact, but what it is describing is \emph{exactly} the behavior of a
\href{https://en.wikipedia.org/wiki/Free_group}{Free Group}, from group
theory!\footnote{I can't take credit for this idea myself; I heard about it from
  \href{http://mniip.com/}{mniip}, who mentioned it during a discussion on
  freenode \#\#adventofcode.}

A fully reacted polymer is an element of the free group generated by the set of
26 letters of the alphabet,
\includegraphics{https://latex.codecogs.com/png.latex?F\%2826\%29}. Basically,
we can talk about interpeting the string \texttt{dabAcCaCBAcCcaDA} as
\texttt{d\ \textless{}\textgreater{}\ a\ \textless{}\textgreater{}\ b\ \textless{}\textgreater{}\ A\ \textless{}\textgreater{}\ c\ \textless{}\textgreater{}\ C\ \textless{}\textgreater{}\ a\ \textless{}\textgreater{}\ C\ \textless{}\textgreater{}\ B\ \textless{}\textgreater{}\ etc.},
where \texttt{\textless{}\textgreater{}} is the group action (``mappend'', in
Haskell-speak) and \texttt{A} stands for ``\texttt{a} inverse''.

We can use
\emph{\href{https://hackage.haskell.org/package/free-algebras/docs/Data-Group-Free.html}{Data.Group.Free}}
from the
\emph{\href{https://hackage.haskell.org/package/free-algebras}{free-algebras}}
library\footnote{Note that the current version of \emph{free-algebras} on
  haddocks actually has a performance bug that makes appends
  \includegraphics{https://latex.codecogs.com/png.latex?O\%28n\%5E2\%29}. I've
  made a \href{https://github.com/coot/free-algebras/pull/4}{pull request}
  fixing this, to give us reasonable times for this challenge!}, which offers a
free group type \texttt{FreeGroupL}, to let us write:

\begin{Shaded}
\begin{Highlighting}[]
\KeywordTok{import} \KeywordTok{qualified} \DataTypeTok{Data.Group.Free} \KeywordTok{as} \DataTypeTok{FG}

\OtherTok{interpret ::}\NormalTok{ [}\DataTypeTok{Char}\NormalTok{] }\OtherTok{{-}>} \DataTypeTok{FG.FreeGroupL} \DataTypeTok{Char}
\NormalTok{interpret }\OtherTok{=} \FunctionTok{foldMap}\NormalTok{ inject          }\CommentTok{{-}{-} that\textquotesingle{}s \textasciigrave{}foldMap\textasciigrave{} from Data.Foldable}
\end{Highlighting}
\end{Shaded}

where \texttt{inject\ ::\ Char\ -\textgreater{}\ FreeGroupL\ Char} takes a
\texttt{Char} and turns it into the element of our free group. We can do this by
using the library's \texttt{returnFree} and \texttt{invert} (from the
\texttt{Group} typeclass):

\begin{Shaded}
\begin{Highlighting}[]
\KeywordTok{import}           \DataTypeTok{Data.Algebra.Free}
\KeywordTok{import}           \DataTypeTok{Data.Char}
\KeywordTok{import}           \DataTypeTok{Data.Group}

\OtherTok{inject ::} \DataTypeTok{Char} \OtherTok{{-}>} \DataTypeTok{FG.FreeGroupL} \DataTypeTok{Char}
\NormalTok{inject c}
    \OperatorTok{|} \FunctionTok{isAlpha}\NormalTok{ c }\OperatorTok{\&\&} \FunctionTok{isLower}\NormalTok{ c }\OtherTok{=}\NormalTok{ returnFree c}
    \OperatorTok{|} \FunctionTok{isAlpha}\NormalTok{ c }\OperatorTok{\&\&} \FunctionTok{isUpper}\NormalTok{ c }\OtherTok{=}\NormalTok{ invert }\OperatorTok{$}\NormalTok{ returnFree (}\FunctionTok{toLower}\NormalTok{ c)}
    \OperatorTok{|} \FunctionTok{otherwise}              \OtherTok{=} \FunctionTok{mempty}       \CommentTok{{-}{-} group identity element}
\end{Highlighting}
\end{Shaded}

The question is essentially asking for the length of the Tietze list
representation of the final result. We can get this using \texttt{FG.toList},
and so our entire part 1 is just:

\begin{Shaded}
\begin{Highlighting}[]
\OtherTok{day05a ::}\NormalTok{ [}\DataTypeTok{Char}\NormalTok{] }\OtherTok{{-}>} \DataTypeTok{Int}
\NormalTok{day05a }\OtherTok{=} \FunctionTok{length} \OperatorTok{.}\NormalTok{ FG.toList }\OperatorTok{.} \FunctionTok{foldMap}\NormalTok{ inject}
\end{Highlighting}
\end{Shaded}

Was this problem actually deliberately written to obscure the fact that we have
a group action? We can't say for sure, but it definitely seems to paint an
``imperative'', step-by-step picture. Most implementations might scan the list
for things to replace, and iterate over things until there is no more change.
The problem is also written to maybe imply the fact that ordering of reduction
of different items across the string \emph{might} matter (is it left to right,
or right to left?).

However, we know that ordering of reduction \emph{can't} matter, because we have
a \emph{group action} \texttt{\textless{}\textgreater{}}, which we know from
group theory is, by definition, associative. Knowing that we have a group, we
can immediately throw away any thought of caring about things like sequential,
imperative order, and the complications that might come from it.

I think that these useful properties about the reduction process are \emph{not}
obvious from an initial reading. Being able to recognize that we have a group is
the key to unlocking all of these insights, \emph{for free}!

\hypertarget{part-2}{%
\section{Part 2}\label{part-2}}

\begin{quote}
Time to improve the polymer.

One of the unit types is causing problems; it's preventing the polymer from
collapsing as much as it should. Your goal is to figure out which unit type is
causing the most problems, remove all instances of it (regardless of polarity),
fully react the remaining polymer, and measure its length.

For example, again using the polymer \texttt{dabAcCaCBAcCcaDA} from above:

\begin{itemize}
\tightlist
\item
  Removing all \texttt{A}/\texttt{a} units produces \texttt{dbcCCBcCcD}. Fully
  reacting this polymer produces \texttt{dbCBcD}, which has length 6.
\item
  Removing all \texttt{B}/\texttt{b} units produces \texttt{daAcCaCAcCcaDA}.
  Fully reacting this polymer produces \texttt{daCAcaDA}, which has length 8.
\item
  Removing all \texttt{C}/\texttt{c} units produces \texttt{dabAaBAaDA}. Fully
  reacting this polymer produces \texttt{daDA}, which has length 4.
\item
  Removing all \texttt{D}/\texttt{d} units produces \texttt{abAcCaCBAcCcaA}.
  Fully reacting this polymer produces \texttt{abCBAc}, which has length 6.
\end{itemize}

In this example, removing all \texttt{C}/\texttt{c} units was best, producing
the answer \emph{4}.

\emph{What is the length of the shortest polymer you can produce} by removing
all units of exactly one type and fully reacting the result?
\end{quote}

Even though the problem again seems to be written to obscure this fact, we can
see that Part 2 is describing a \emph{group homomorphism}. That is, it talks
about functions that map
\includegraphics{https://latex.codecogs.com/png.latex?F\%2826\%29} (the free
group on the 26 letters of the alphabet) to
\includegraphics{https://latex.codecogs.com/png.latex?F\%2825\%29} (the free
group on the letters of the alphabet excluding some cleaned letter).

Luckily, the free group
\includegraphics{https://latex.codecogs.com/png.latex?F\%28S\%29} comes equipped
with a handy function create group homomorphisms ``for free'':

\begin{Shaded}
\begin{Highlighting}[]
\NormalTok{foldMapFree}
\OtherTok{    ::} \DataTypeTok{Group}\NormalTok{ b}
    \OtherTok{=>}\NormalTok{ (a }\OtherTok{{-}>}\NormalTok{ b)}
    \OtherTok{{-}>}\NormalTok{ (}\DataTypeTok{FreeGroupL}\NormalTok{ a }\OtherTok{{-}>}\NormalTok{ b)     }\CommentTok{{-}{-} the group homomorphism}
\end{Highlighting}
\end{Shaded}

That is, given any \texttt{a\ -\textgreater{}\ b} for a group \texttt{b}, we get
a \emph{guaranteed} group homormohpsim from \texttt{FreeGroupL\ a} to
\texttt{b}. We can write a function from the \emph{generators} of our group (in
this case, \texttt{Char}), and it'll give us a group homomorphism on the
\emph{free group} on our generators.

We can use this to write our
\includegraphics{https://latex.codecogs.com/png.latex?F\%2826\%29\%20\%5Crightarrow\%20F\%2825\%29}
group homomorphism

\begin{Shaded}
\begin{Highlighting}[]
\NormalTok{foldMapFree}
\OtherTok{    ::}\NormalTok{ (}\DataTypeTok{Char} \OtherTok{{-}>} \DataTypeTok{FreeGroupL} \DataTypeTok{Char}\NormalTok{)  }\CommentTok{{-}{-} map letters to letters{-}minus{-}some{-}letter}
    \OtherTok{{-}>} \DataTypeTok{FreeGroupL} \DataTypeTok{Char}
    \OtherTok{{-}>} \DataTypeTok{FreeGroupL} \DataTypeTok{Char}
\end{Highlighting}
\end{Shaded}

(Note that this type signature looks a lot like
\texttt{=\textless{}\textless{}}, for monads. Coincidence?)

We can now create a ``cleaned'' version of our reaction by using:

\begin{Shaded}
\begin{Highlighting}[]
\NormalTok{clean}
\OtherTok{    ::} \DataTypeTok{Char}                                     \CommentTok{{-}{-} \^{} given a letter to clean}
    \OtherTok{{-}>}\NormalTok{ (}\DataTypeTok{FreeGroupL} \DataTypeTok{Char} \OtherTok{{-}>} \DataTypeTok{FreeGroupL} \DataTypeTok{Char}\NormalTok{)     }\CommentTok{{-}{-} \^{} return a group homomorphism}
\NormalTok{clean c }\OtherTok{=}\NormalTok{ foldMapFree }\OperatorTok{$}\NormalTok{ \textbackslash{}d }\OtherTok{{-}>}
        \KeywordTok{if}\NormalTok{ d }\OperatorTok{==}\NormalTok{ c}
          \KeywordTok{then} \FunctionTok{mempty}
          \KeywordTok{else}\NormalTok{ returnFree d}
\end{Highlighting}
\end{Shaded}

And so that's part 2! We just need \texttt{clean} and this next function:

\begin{Shaded}
\begin{Highlighting}[]
\OtherTok{day05b ::} \DataTypeTok{String} \OtherTok{{-}>} \DataTypeTok{Int}
\NormalTok{day05b rawInput }\OtherTok{=} \FunctionTok{minimum}
\NormalTok{    [ }\FunctionTok{length}\NormalTok{ (FG.toList (clean c polymer))}
    \OperatorTok{|}\NormalTok{ c }\OtherTok{<{-}}\NormalTok{ [}\CharTok{\textquotesingle{}a\textquotesingle{}} \OperatorTok{..} \CharTok{\textquotesingle{}z\textquotesingle{}}\NormalTok{]}
\NormalTok{    ]}
  \KeywordTok{where}
\OtherTok{    polymer ::} \DataTypeTok{FreeGroupL} \DataTypeTok{Char}
\NormalTok{    polymer }\OtherTok{=} \FunctionTok{foldMap}\NormalTok{ inject rawInput}
\end{Highlighting}
\end{Shaded}

Basically, we find the minimum of all of the possible ``cleaned'' lengths.

The best part about this, I think, is that we actually introduced a \emph{large
optimization}, completely \emph{by accident}, thanks to group theory.

Because we recognize that this is a group homomorphism, we know the properties
of group homomorphisms apply. Namely, the most important one: ``Aggregating,
then cleaning'' is the \emph{same} as ``cleaning, then aggregating''.

That's because all group homomorphisms necessarily obey the law:

\begin{Shaded}
\begin{Highlighting}[]
\NormalTok{f x }\OperatorTok{<>}\NormalTok{ f y }\OperatorTok{==}\NormalTok{ f (x }\OperatorTok{<>}\NormalTok{ y)}
\end{Highlighting}
\end{Shaded}

This means that we are free to either ``clean first, then aggregate'', or
``aggregate first, then clean''.

\hypertarget{whats-in-a-group}{%
\section{What's in a Group?}\label{whats-in-a-group}}

Now, I don't know about you, but I definitely feel that this choice (clean,
aggregate vs.~aggregate, clean) we have is \emph{definitely not obvious} just
from reading the problem immediately. Indeed, it seems like the problem might be
written to obscure this choice from us: it's implying that ``cleaning, then
reacting'' is the only correct way, and ``reacting, then cleaning'' is not
something that is even mentioned.

But, thanks to group theory, we know that these are equivalent, so we can
substitute which ever version is more efficient!

This is, I believe, at the heart of what people say is the advantage of ``using
monoids'', ``using monads'', ``using functors'', etc. in Haskell. That's because
if we state our programs in terms of monoids, monads, groups, functors, etc.,
then we get \emph{the entire body of group theory} (or monad theory, or functor
theory, etc.) to help us make program reductions that aren't immediately
obviously legal but that have already been proven to be equivalent by
mathematicians. We hijack their work!

We get program optimizations and reductions and substitutions for free, by
``stealing'' from the large body of such things that mathematicians have spent
centuries collecting.

\hypertarget{epilogue-a-note-on-cleaning}{%
\subsection{Epilogue: A Note on Cleaning}\label{epilogue-a-note-on-cleaning}}

Since I have posted this, I have gotten a few comments on different platforms
debating whether or not ``clean then aggregate'' is truly same as ``aggregate
then clean''.

Just to clarify, \texttt{clean\ c} here does \emph{not} mean ``remove
\texttt{c}''. It means ``re-interpret our group element with the new property
that \texttt{c} is the identity element''. Clean is like a ``re-reaction''. To
draw the chemical analogy, it can be thought of as a redox reaction that doesn't
``remove'' atoms, but rather ``rewrites the rules of the polymer''. The identity
is that ``aggregate the polymer, and then rewrite its roles'' is the same as
``rewrite the rules, then aggregate the polymer''.

Clean is not a ``removal''. It's ``re-interpretation''.

To be more clear, if we have a function \texttt{cleanB} which cleans out the
letter \texttt{b}, we can take \texttt{abAcdD} as an example and
\emph{aggregate} it to get:

\begin{Shaded}
\begin{Highlighting}[]
\NormalTok{a }\OperatorTok{<>}\NormalTok{ b }\OperatorTok{<>} \DataTypeTok{A} \OperatorTok{<>}\NormalTok{ c }\OperatorTok{<>} \DataTypeTok{D}
\end{Highlighting}
\end{Shaded}

Now, we can clean that aggregation to get:

\begin{Shaded}
\begin{Highlighting}[]
\NormalTok{cleanB (a }\OperatorTok{<>}\NormalTok{ b }\OperatorTok{<>} \DataTypeTok{A} \OperatorTok{<>}\NormalTok{ c }\OperatorTok{<>} \DataTypeTok{D}\NormalTok{)}
\end{Highlighting}
\end{Shaded}

That's our ``aggregate, then clean''. Our ``clean, then aggregate'' would be:

\begin{Shaded}
\begin{Highlighting}[]
\NormalTok{cleanB a }\OperatorTok{<>}\NormalTok{ cleanB b }\OperatorTok{<>}\NormalTok{ cleanB }\DataTypeTok{A} \OperatorTok{<>}\NormalTok{ cleanB c }\OperatorTok{<>}\NormalTok{ cleanB }\DataTypeTok{D}
\end{Highlighting}
\end{Shaded}

The library \emph{free-algebras} gives us \texttt{foldMapFree}, which creates
group homomorphisms \emph{by construction}. That means that whatever function we
get out of \texttt{foldMapFree}, it is \emph{mathematically proven} that:

\begin{Shaded}
\begin{Highlighting}[]
\NormalTok{cleanB (a }\OperatorTok{<>}\NormalTok{ b }\OperatorTok{<>} \DataTypeTok{A} \OperatorTok{<>}\NormalTok{ c }\OperatorTok{<>} \DataTypeTok{D}\NormalTok{)}
 \OperatorTok{==}\NormalTok{ cleanB a }\OperatorTok{<>}\NormalTok{ cleanB b }\OperatorTok{<>}\NormalTok{ cleanB }\DataTypeTok{A} \OperatorTok{<>}\NormalTok{ cleanB c }\OperatorTok{<>}\NormalTok{ cleanB }\DataTypeTok{D}
\end{Highlighting}
\end{Shaded}

Or, that ``aggregate then clean'' is the same as ``clean then aggregate''.

This is because the implementation of \texttt{foldMapFree} (and therefore
\texttt{cleanB}) is not to simply ``remove'' \texttt{b}, but rather to move us
into a \emph{new} group where \texttt{b} is the identity. This completely
re-interprets the structure of the polymer, and the implementation is careful to
do this properly so that it \emph{does} commute.

\hypertarget{signoff}{%
\section{Signoff}\label{signoff}}

Hi, thanks for reading! You can reach me via email at
\href{mailto:justin@jle.im}{\nolinkurl{justin@jle.im}}, or at twitter at
\href{https://twitter.com/mstk}{@mstk}! This post and all others are published
under the \href{https://creativecommons.org/licenses/by-nc-nd/3.0/}{CC-BY-NC-ND
3.0} license. Corrections and edits via pull request are welcome and encouraged
at \href{https://github.com/mstksg/inCode}{the source repository}.

If you feel inclined, or this post was particularly helpful for you, why not
consider \href{https://www.patreon.com/justinle/overview}{supporting me on
Patreon}, or a \href{bitcoin:3D7rmAYgbDnp4gp4rf22THsGt74fNucPDU}{BTC donation}?
:)

\end{document}
