\documentclass[]{article}
\usepackage{lmodern}
\usepackage{amssymb,amsmath}
\usepackage{ifxetex,ifluatex}
\usepackage{fixltx2e} % provides \textsubscript
\ifnum 0\ifxetex 1\fi\ifluatex 1\fi=0 % if pdftex
  \usepackage[T1]{fontenc}
  \usepackage[utf8]{inputenc}
\else % if luatex or xelatex
  \ifxetex
    \usepackage{mathspec}
    \usepackage{xltxtra,xunicode}
  \else
    \usepackage{fontspec}
  \fi
  \defaultfontfeatures{Mapping=tex-text,Scale=MatchLowercase}
  \newcommand{\euro}{€}
\fi
% use upquote if available, for straight quotes in verbatim environments
\IfFileExists{upquote.sty}{\usepackage{upquote}}{}
% use microtype if available
\IfFileExists{microtype.sty}{\usepackage{microtype}}{}
\usepackage[margin=1in]{geometry}
\usepackage{color}
\usepackage{fancyvrb}
\newcommand{\VerbBar}{|}
\newcommand{\VERB}{\Verb[commandchars=\\\{\}]}
\DefineVerbatimEnvironment{Highlighting}{Verbatim}{commandchars=\\\{\}}
% Add ',fontsize=\small' for more characters per line
\newenvironment{Shaded}{}{}
\newcommand{\AlertTok}[1]{\textcolor[rgb]{1.00,0.00,0.00}{\textbf{#1}}}
\newcommand{\AnnotationTok}[1]{\textcolor[rgb]{0.38,0.63,0.69}{\textbf{\textit{#1}}}}
\newcommand{\AttributeTok}[1]{\textcolor[rgb]{0.49,0.56,0.16}{#1}}
\newcommand{\BaseNTok}[1]{\textcolor[rgb]{0.25,0.63,0.44}{#1}}
\newcommand{\BuiltInTok}[1]{#1}
\newcommand{\CharTok}[1]{\textcolor[rgb]{0.25,0.44,0.63}{#1}}
\newcommand{\CommentTok}[1]{\textcolor[rgb]{0.38,0.63,0.69}{\textit{#1}}}
\newcommand{\CommentVarTok}[1]{\textcolor[rgb]{0.38,0.63,0.69}{\textbf{\textit{#1}}}}
\newcommand{\ConstantTok}[1]{\textcolor[rgb]{0.53,0.00,0.00}{#1}}
\newcommand{\ControlFlowTok}[1]{\textcolor[rgb]{0.00,0.44,0.13}{\textbf{#1}}}
\newcommand{\DataTypeTok}[1]{\textcolor[rgb]{0.56,0.13,0.00}{#1}}
\newcommand{\DecValTok}[1]{\textcolor[rgb]{0.25,0.63,0.44}{#1}}
\newcommand{\DocumentationTok}[1]{\textcolor[rgb]{0.73,0.13,0.13}{\textit{#1}}}
\newcommand{\ErrorTok}[1]{\textcolor[rgb]{1.00,0.00,0.00}{\textbf{#1}}}
\newcommand{\ExtensionTok}[1]{#1}
\newcommand{\FloatTok}[1]{\textcolor[rgb]{0.25,0.63,0.44}{#1}}
\newcommand{\FunctionTok}[1]{\textcolor[rgb]{0.02,0.16,0.49}{#1}}
\newcommand{\ImportTok}[1]{#1}
\newcommand{\InformationTok}[1]{\textcolor[rgb]{0.38,0.63,0.69}{\textbf{\textit{#1}}}}
\newcommand{\KeywordTok}[1]{\textcolor[rgb]{0.00,0.44,0.13}{\textbf{#1}}}
\newcommand{\NormalTok}[1]{#1}
\newcommand{\OperatorTok}[1]{\textcolor[rgb]{0.40,0.40,0.40}{#1}}
\newcommand{\OtherTok}[1]{\textcolor[rgb]{0.00,0.44,0.13}{#1}}
\newcommand{\PreprocessorTok}[1]{\textcolor[rgb]{0.74,0.48,0.00}{#1}}
\newcommand{\RegionMarkerTok}[1]{#1}
\newcommand{\SpecialCharTok}[1]{\textcolor[rgb]{0.25,0.44,0.63}{#1}}
\newcommand{\SpecialStringTok}[1]{\textcolor[rgb]{0.73,0.40,0.53}{#1}}
\newcommand{\StringTok}[1]{\textcolor[rgb]{0.25,0.44,0.63}{#1}}
\newcommand{\VariableTok}[1]{\textcolor[rgb]{0.10,0.09,0.49}{#1}}
\newcommand{\VerbatimStringTok}[1]{\textcolor[rgb]{0.25,0.44,0.63}{#1}}
\newcommand{\WarningTok}[1]{\textcolor[rgb]{0.38,0.63,0.69}{\textbf{\textit{#1}}}}
\usepackage{fancyvrb}
\ifxetex
  \usepackage[setpagesize=false, % page size defined by xetex
              unicode=false, % unicode breaks when used with xetex
              xetex]{hyperref}
\else
  \usepackage[unicode=true]{hyperref}
\fi
\hypersetup{breaklinks=true,
            bookmarks=true,
            pdfauthor={Justin Le},
            pdftitle={The Functor Combinatorpedia},
            colorlinks=true,
            citecolor=blue,
            urlcolor=blue,
            linkcolor=magenta,
            pdfborder={0 0 0}}
\urlstyle{same}  % don't use monospace font for urls
% Make links footnotes instead of hotlinks:
\renewcommand{\href}[2]{#2\footnote{\url{#1}}}
\setlength{\parindent}{0pt}
\setlength{\parskip}{6pt plus 2pt minus 1pt}
\setlength{\emergencystretch}{3em}  % prevent overfull lines
\setcounter{secnumdepth}{0}
\VerbatimFootnotes % allows verbatim text in footnotes

\title{The Functor Combinatorpedia}
\author{Justin Le}
\date{June 19, 2019}

\begin{document}
\maketitle

\emph{Originally posted on
\textbf{\href{https://blog.jle.im/entry/functor-combinatorpedia.html}{in
Code}}.}

\textbf{functor-combinators}:
\href{https://hackage.haskell.org/package/functor-combinators}{hackage} /
\href{https://github.com/mstksg/functor-combinators}{github}

Recently I've been very productive what I have been calling the ``Functor
Combinator'' design pattern. It is heavily influenced by ideas like
\href{http://www.cs.ru.nl/~W.Swierstra/Publications/DataTypesALaCarte.pdf}{Data
types a la Carte} and
\href{http://oleg.fi/gists/posts/2018-02-21-single-free.html}{unified free
monoidal functors}, but the end goal is slightly different in spirit. The goal
is to represent schemas, DSL's, and computations (things like parsers, things to
execute, things to consume or produce data) by assembling ``self-evident'' basic
primitives and subjecting them to many \emph{different} successive
transformations and combiners (through combinators, free structures, tensors,
and other options). The process of doing so:

\begin{enumerate}
\def\labelenumi{\arabic{enumi}.}
\tightlist
\item
  Forces you to make explicit decisions about the structure of your computation
  type as an ADT.
\item
  Allows you to retain isolation of fundamental parts of your domain as separate
  types
\item
  Lets you manipulate the structure of your final computation type through
  \emph{normal Haskell techniques} like pattern matching. The structure is
  available throughout the entire process, so you can replace individual
  components and values within your structure.
\item
  Allows you to fully \emph{reflect} the structure of your final computation
  through pattern matching and folds, so you can inspect the structure and
  produce useful summaries.
\end{enumerate}

Like ``data types a la carte'' and free monad/applicative/alternative designs,
these techniques allow you to separate the assembly and inspection of your
programs from the ``running'' of them.\footnote{On the surface, this functor
  combinator design pattern might look like it fills a similar space to effects
  systems and libraries like
  \emph{\href{https://hackage.haskell.org/package/mtl}{mtl}},
  \emph{\href{https://hackage.haskell.org/package/polysemy}{polysemy}},
  \emph{\href{https://hackage.haskell.org/package/freer-simple}{freer-simple}},
  or
  \emph{\href{https://hackage.haskell.org/package/fused-effects}{fused-effects}}.
  However, this design pattern actually exists on a different level.

  Functor combinator design patterns can be used to help build the
  \emph{structure} of the \emph{data types} and schemas that define your
  program/DSL. Once you build these nice structures, you then \emph{interpret}
  them into some target context. This ``target context'' is the realm that
  libraries like \emph{mtl} and \emph{polysemy} can fill; functor combinators
  serve to help you define a structure for your program \emph{before} you
  interpret it into whatever Applicative or Monad or effects system you end up
  using.} However, the main difference is that here we focus not just on
products and sums, but many different varied and multi-purpose combinators --- a
``zoo'' of combinators. The fixed point is \emph{not} the end goal.

This post is a run-down on the wide variety of such ``functor combinators''
across the Haskell ecosystem --- a functor combinatorpedia. To speak about them
all with the same language and vocabulary, this post also serves as an overview
of the
\emph{\href{https://hackage.haskell.org/package/functor-combinators}{functor-combinators}}
library, which doesn't really define these functor combinators, but rather pulls
them all together and provides a unified interface for working with them. Most
of these types and typeclasses are exported by
\emph{\href{https://hackage.haskell.org/package/functor-combinators/docs/Data-Functor-Combinator.html}{Data.Functor.Combinator}}.

Right now I already have some posts about this general design pattern,
\href{https://blog.jle.im/entry/interpreters-a-la-carte-duet.html}{``Interpreters
a la Carte'' in Advent of Code 2017 Duet} and
\href{https://blog.jle.im/entry/free-alternative-regexp.html}{Applicative
Regular Expressions using the Free Alternatuve}, but I do have some posts
planned in the future going through projects using this unified interface. In a
way, this post also serves as the ``introduction to free structures'' that I
always wanted to write :)

Please refer to the
\href{https://blog.jle.im/entry/functor-combinatorpedia.html\#title}{table of
contents} if you are using this as a reference!

\hypertarget{preface-what-is-a-functor-combinator}{%
\section{Preface: What is a functor
combinator?}\label{preface-what-is-a-functor-combinator}}

A functor combinator takes ``functors'' (or any other indexed type,
\texttt{k\ -\textgreater{}\ Type}) and returns a new functor, enhances or mixes
them together in some way. That is, they take things of kind
\texttt{k\ -\textgreater{}\ Type} and themselves return a
\texttt{j\ -\textgreater{}\ Type}. This lets us build complex functors/indexed
types out of simpler ``primitive'' ones. This includes many some monad
transformers, free structures, and tensors.

For example, \texttt{ReaderT\ r} is a famous one that takes a functor \texttt{f}
and enhances it with ``access to an \texttt{r} environment'' functionality.
Another famous one is \texttt{Free}, which takes a functor \texttt{f} and
enhances it with ``sequential binding'' capabilities: it turns \texttt{f} into a
\texttt{Monad}.

The main thing that distinguishes these functor combinators from things like
monad transformers is that they are ``natural on \texttt{f}'': they work on
\emph{all} \texttt{f}s, not just monads, and assume no structure (not even
\texttt{Functor}).

Sometimes, we have binary functor combinators, like \texttt{:+:}, which takes
two functors \texttt{f} and \texttt{g} and returns a functor that is ``either''
\texttt{f} or \texttt{g}. Binary functor combinators ``mix together'' the
functionality of different functors in different ways.

\hypertarget{examples}{%
\subsection{Examples}\label{examples}}

If your final DSL/program/schema is some functor, then functor combinators allow
you to construct your final functor by combining simpler ``primitive'' functors,
and take advantage of common functionality.

For example, if you were making a data type/EDSL to describe a command line
argument parser, you might have two primitives: \texttt{data\ Arg\ a}, for
positional arguments parsing \texttt{a}, and \texttt{data\ Option\ a}, for
\texttt{-\/-flag} non-positional options parsing \texttt{a}. From there, you can
\emph{choose} what structure of command line arguments you want to be able to
express.

For instance, a structure that can support multiple arguments and optionally a
single \texttt{Option} would be:

\begin{Shaded}
\begin{Highlighting}[]
\KeywordTok{type} \DataTypeTok{CommandArgs} \OtherTok{=} \DataTypeTok{Ap} \DataTypeTok{Arg} \OperatorTok{:*:} \DataTypeTok{Lift} \DataTypeTok{Option}
\end{Highlighting}
\end{Shaded}

And a structure that supports \emph{multiple named commands} on top of that
would be:

\begin{Shaded}
\begin{Highlighting}[]
\KeywordTok{type} \DataTypeTok{CommandArgs} \OtherTok{=} \DataTypeTok{MapF} \DataTypeTok{String}\NormalTok{ (}\DataTypeTok{Ap} \DataTypeTok{Arg} \OperatorTok{:*:} \DataTypeTok{Lift} \DataTypeTok{Option}\NormalTok{)}
\end{Highlighting}
\end{Shaded}

You can mix or match combinators to decide exactly what sort of structures you
allow in your DSL.

Now, instead of writing one ``giant''
\texttt{runParser\ ::\ MapF\ String\ (Ap\ Arg\ :*:\ Lift\ Option)\ a\ -\textgreater{}\ IO\ a}
function, you can instead just write parsers for your simple primitives
\texttt{Arg\ a\ -\textgreater{}\ IO\ a} and
\texttt{Option\ a\ -\textgreater{}\ IO\ a}, and then use functor combinator
tools to ``promote'' them to being runnable on a full
\texttt{MapF\ String\ (Ap\ Arg\ :*:\ Lift\ Option)} without any extra work.

\hypertarget{common-functionality}{%
\subsection{Common Functionality}\label{common-functionality}}

Most of these functor combinators allow us to ``swap out'' the underlying
functor, retaining all of the ``enhanced'' structure. We abstract over all of
these using \texttt{hmap} for single-argument functor combinators
(``enhancers'') and \texttt{hbimap} for two-argument functor combinators
(``mixers'').

\begin{Shaded}
\begin{Highlighting}[]
\KeywordTok{class} \DataTypeTok{HFunctor}\NormalTok{ t }\KeywordTok{where}
    \CommentTok{{-}{-} | Swap out underlying functor for a single{-}argument functor combinator}
\NormalTok{    hmap}
\OtherTok{        ::}\NormalTok{ t f a }\OtherTok{{-}>}\NormalTok{ t g a}

\KeywordTok{class} \DataTypeTok{HBifunctor}\NormalTok{ t }\KeywordTok{where}
    \CommentTok{{-}{-} | Swap out underlying functors for a two{-}argument functor combinator}
\NormalTok{    hbimap}
\OtherTok{        ::}\NormalTok{ (}\KeywordTok{forall}\NormalTok{ x}\OperatorTok{.}\NormalTok{ f x }\OtherTok{{-}>}\NormalTok{ h x)}
        \OtherTok{{-}>}\NormalTok{ (}\KeywordTok{forall}\NormalTok{ x}\OperatorTok{.}\NormalTok{ g x }\OtherTok{{-}>}\NormalTok{ j x)}
        \OtherTok{{-}>}\NormalTok{ t f g a}
        \OtherTok{{-}>}\NormalTok{ t g j a}
\end{Highlighting}
\end{Shaded}

However, for this post, the concept of a ``natural transformation'' between
\texttt{f} and \texttt{g} --- a function of type
\texttt{forall\ x.\ f\ x\ -\textgreater{}\ g\ x}, is given a type synonym:

\begin{Shaded}
\begin{Highlighting}[]
\KeywordTok{type}\NormalTok{ f }\OperatorTok{\textasciitilde{}>}\NormalTok{ g }\OtherTok{=} \KeywordTok{forall}\NormalTok{ x}\OperatorTok{.}\NormalTok{ f x }\OtherTok{{-}>}\NormalTok{ g x}
\end{Highlighting}
\end{Shaded}

Then the type signatures of \texttt{hmap} and \texttt{hbimap} become:

\begin{Shaded}
\begin{Highlighting}[]
\KeywordTok{class} \DataTypeTok{HFunctor}\NormalTok{ t }\KeywordTok{where}
\NormalTok{    hmap}
\OtherTok{        ::}\NormalTok{ t f }\OperatorTok{\textasciitilde{}>}\NormalTok{ t g}

\KeywordTok{class} \DataTypeTok{HBifunctor}\NormalTok{ t }\KeywordTok{where}
\NormalTok{    hbimap}
\OtherTok{        ::}\NormalTok{ f }\OperatorTok{\textasciitilde{}>}\NormalTok{ h}
        \OtherTok{{-}>}\NormalTok{ g }\OperatorTok{\textasciitilde{}>}\NormalTok{ j}
        \OtherTok{{-}>}\NormalTok{ t f g }\OperatorTok{\textasciitilde{}>}\NormalTok{ t h j}
\end{Highlighting}
\end{Shaded}

What does it mean exactly when we say that \texttt{hmap} and \texttt{hbimap}
``preserve the enhanced structure''? Well, for example, the type
\texttt{newtype\ ListF\ f\ a\ =\ ListF\ {[}f\ a{]}} is essentially a list of
\texttt{f\ a}s. \texttt{hmap} will swap out and replace each \texttt{f\ a}, but
it must \emph{preserve the relative order} between each of the original
\texttt{f\ a}s. It must also preserve the \emph{length} of the list. It's a
complete ``in-place swap''. This is formalizing by requiring
\texttt{hmap\ id\ ==\ id} and \texttt{hbimap\ id\ id\ ==\ id}.

You can also always ``lift'' a functor value into its transformed type. We
abstract over this by using \texttt{inject} (for single-argument functors) and
\texttt{inL} and \texttt{inR} (for two-argument functors):

\begin{Shaded}
\begin{Highlighting}[]
\CommentTok{{-}{-} single argument functor combinators}
\OtherTok{inject ::}\NormalTok{ f }\OperatorTok{\textasciitilde{}>}\NormalTok{ t f}

\CommentTok{{-}{-} two{-}argument functor combinators}
\OtherTok{inL ::}\NormalTok{ (}\DataTypeTok{Monoidal}\NormalTok{ t, }\DataTypeTok{CM}\NormalTok{ t g)     }\CommentTok{{-}{-} more on the \textasciigrave{}CM t\textasciigrave{} later}
    \OtherTok{=>}\NormalTok{ f }\OperatorTok{\textasciitilde{}>}\NormalTok{ t f g}

\OtherTok{inR ::}\NormalTok{ (}\DataTypeTok{Monoidal}\NormalTok{ t, }\DataTypeTok{CM}\NormalTok{ t f)     }\CommentTok{{-}{-} more on the \textasciigrave{}CM t\textasciigrave{} later}
    \OtherTok{=>}\NormalTok{ g }\OperatorTok{\textasciitilde{}>}\NormalTok{ t f g}
\end{Highlighting}
\end{Shaded}

Finally, in order to \emph{use} any functor combinators, you have to
\emph{interpret} them into some target context. The choice of combinator imposes
some constraints on the target context. We abstract over this using
\texttt{interpret} and \texttt{binterpret}:

\begin{Shaded}
\begin{Highlighting}[]
\KeywordTok{class} \DataTypeTok{Interpret}\NormalTok{ t }\KeywordTok{where}
    \KeywordTok{type} \DataTypeTok{C}\OtherTok{ t ::}\NormalTok{ (}\DataTypeTok{Type} \OtherTok{{-}>} \DataTypeTok{Type}\NormalTok{) }\OtherTok{{-}>} \DataTypeTok{Constraint}

    \CommentTok{{-}{-} | Interpret unary functor combinator}
\NormalTok{    interpret}
\OtherTok{        ::} \DataTypeTok{C}\NormalTok{ t g}
        \OtherTok{=>}\NormalTok{ f }\OperatorTok{\textasciitilde{}>}\NormalTok{ g             }\CommentTok{{-}{-} \^{} interpreting function}
        \OtherTok{{-}>}\NormalTok{ t f }\OperatorTok{\textasciitilde{}>}\NormalTok{ g}

\KeywordTok{class} \DataTypeTok{Semigroupoidal}\NormalTok{ t }\KeywordTok{where}
    \KeywordTok{type} \DataTypeTok{CS}\OtherTok{ t ::}\NormalTok{ (}\DataTypeTok{Type} \OtherTok{{-}>} \DataTypeTok{Type}\NormalTok{) }\OtherTok{{-}>} \DataTypeTok{Constraint}

    \CommentTok{{-}{-} | Interpret binary functor combinator}
\NormalTok{    binterpret}
\OtherTok{        ::} \DataTypeTok{CS}\NormalTok{ t h}
        \OtherTok{=>}\NormalTok{ f }\OperatorTok{\textasciitilde{}>}\NormalTok{ h             }\CommentTok{{-}{-} \^{} interpreting function on f}
        \OtherTok{=>}\NormalTok{ g }\OperatorTok{\textasciitilde{}>}\NormalTok{ h             }\CommentTok{{-}{-} \^{} interpreting function on g}
        \OtherTok{{-}>}\NormalTok{ t f g }\OperatorTok{\textasciitilde{}>}\NormalTok{ h}
\end{Highlighting}
\end{Shaded}

Each functor combinator defines a constraint (\texttt{C} for unary functor
combinators, and \texttt{CS} and \texttt{CM} for binary functor combinators)
that allows you to ``exit'', or ``run'' the functor combinator.

One nice consequence of this approach is that for many such schemas/functors you
build, there might be many \emph{useful} target functors. For example, if you
build a command line argument parser schema, you might want to run it in
\texttt{Const\ String} to build up a ``help message'', or you might want to run
it in \texttt{Parser} to parse the actual arguments or run pure tests, or you
might want to run it in \texttt{IO} to do interactive parsing.

For some concrete examples of these functor combinators and their constraints:

\begin{Shaded}
\begin{Highlighting}[]
\KeywordTok{type} \DataTypeTok{C} \DataTypeTok{Free} \OtherTok{=} \DataTypeTok{Monad}

\NormalTok{interpret }\OperatorTok{@}\DataTypeTok{Free}
\OtherTok{    ::} \DataTypeTok{Monad}\NormalTok{ g}
    \OtherTok{=>}\NormalTok{ (f }\OperatorTok{\textasciitilde{}>}\NormalTok{ g)}
    \OtherTok{{-}>} \DataTypeTok{Free}\NormalTok{ f a}
    \OtherTok{{-}>}\NormalTok{ g a}

\KeywordTok{type} \DataTypeTok{CM}\NormalTok{ (}\OperatorTok{:+:}\NormalTok{) }\OtherTok{=} \DataTypeTok{Unconstrained}   \CommentTok{{-}{-} no constraints on exiting}

\NormalTok{binterpret }\OperatorTok{@}\NormalTok{(}\OperatorTok{:+:}\NormalTok{)}
\OtherTok{    ::}\NormalTok{ (f }\OperatorTok{\textasciitilde{}>}\NormalTok{ h)}
    \OtherTok{{-}>}\NormalTok{ (g }\OperatorTok{\textasciitilde{}>}\NormalTok{ h)}
    \OtherTok{{-}>}\NormalTok{ (f }\OperatorTok{:+:}\NormalTok{ g) a}
    \OtherTok{{-}>}\NormalTok{ h a}
\end{Highlighting}
\end{Shaded}

We see that \texttt{interpret} lets you ``run'' a \texttt{Free} in any monad
\texttt{g}, and \texttt{binterpret} lets you ``run'' a function over both
\emph{branches} of an \texttt{f\ :+:\ g} to produce an \texttt{h}.

From these, we can also build a lot of useful utility functions (like
\texttt{retract}, \texttt{biretract}, \texttt{getI}, \texttt{biget}, etc.) for
convenience in actually working on them. These are provided in
\emph{\href{https://hackage.haskell.org/package/functor-combinators}{functor-combinators}}.

Without further ado, let's dive into the zoo of functor combinators!

\hypertarget{two-argument}{%
\section{Two-Argument}\label{two-argument}}

Binary functor combinators ``mix together'' two functors/indexed types in
different ways.

We can finally \emph{interpret} (or ``run'') these into some target context
(like \texttt{Parser}, or \texttt{IO}), provided the target satisfies some
constraints.

It can be useful to separate the functionality associated with each binary
functor combinator into two typeclasses: \texttt{Semigroupoidal} and
\texttt{Monoidal}. \texttt{Semigroupoidal} deals with the ``interpreting'' and
``collapsing'' of the mixed functors, and \texttt{Monoidal} deals with the
``introduction'' and ``creation'' of them. A more detailed run-down is available
in the docs for
\emph{\href{https://hackage.haskell.org/package/functor-combinators/docs/Data-Functor-Combinator.html}{Data.Functor.Combinator}}.

From each, we can identify two associated constraints, \texttt{CS\ t} (for
\texttt{Semigroupoidal}) and \texttt{CM\ t} (for \texttt{Monoidal}):\footnote{As
  it turns out, \texttt{CS} and \texttt{CM} seem to generalize the ``has an
  identity'' property of many typeclasses --- the example used illustrates that
  \texttt{Comp} gives us \texttt{type\ CS\ Comp\ =\ Bind} and
  \texttt{type\ CM\ Comp\ =\ Monad}, which is ``monad without pure'' and
  ``monad''. We will see other examples later.}

\begin{itemize}
\item
  \texttt{CS\ t} is what we will call the constraint on where you can
  \emph{interpret} or \emph{run} values of the enhanced type:

\begin{Shaded}
\begin{Highlighting}[]
\NormalTok{binterpret}
\OtherTok{    ::}\NormalTok{ (}\DataTypeTok{Semigroupoidal}\NormalTok{ t, }\DataTypeTok{CS}\NormalTok{ t h)}
    \OtherTok{=>}\NormalTok{ (f }\OperatorTok{\textasciitilde{}>}\NormalTok{ h)}
    \OtherTok{=>}\NormalTok{ (g }\OperatorTok{\textasciitilde{}>}\NormalTok{ h)}
    \OtherTok{{-}>}\NormalTok{ (t f g }\OperatorTok{\textasciitilde{}>}\NormalTok{ h)}
\end{Highlighting}
\end{Shaded}

  For example, for \texttt{Comp} (functor composition):

\begin{Shaded}
\begin{Highlighting}[]
\KeywordTok{data} \DataTypeTok{Comp}\NormalTok{ f g a }\OtherTok{=} \DataTypeTok{Comp}\NormalTok{ (f (g a))}
\end{Highlighting}
\end{Shaded}

  we have \texttt{type\ CS\ Comp\ =\ Bind} (``\texttt{Monad} without pure'').
\item
  \texttt{CM\ t} is the constraint on where you can \emph{create} values of the
  enhanced type (\texttt{pureT}, \texttt{inL}, \texttt{inR})

\begin{Shaded}
\begin{Highlighting}[]
\NormalTok{pureT}
\OtherTok{    ::}\NormalTok{ (}\DataTypeTok{Monoidal}\NormalTok{ t, }\DataTypeTok{CM}\NormalTok{ t f)}
    \OtherTok{=>} \DataTypeTok{I}\NormalTok{ t }\OperatorTok{\textasciitilde{}>}\NormalTok{ f}

\OtherTok{inL ::}\NormalTok{ (}\DataTypeTok{Monoidal}\NormalTok{ t, }\DataTypeTok{CM}\NormalTok{ t g)}
    \OtherTok{=>}\NormalTok{ f }\OperatorTok{\textasciitilde{}>}\NormalTok{ t f g}

\OtherTok{inR ::}\NormalTok{ (}\DataTypeTok{Monoidal}\NormalTok{ t, }\DataTypeTok{CM}\NormalTok{ t f)}
    \OtherTok{=>}\NormalTok{ g }\OperatorTok{\textasciitilde{}>}\NormalTok{ t f g}
\end{Highlighting}
\end{Shaded}

  For example, for \texttt{Comp}, \texttt{type\ CS\ Free\ =\ Monad}.

  It should always be a subclass of \texttt{CS}.
\end{itemize}

These constraints all depend on the extra structure that the specific functor
combinator imbues.

Most of these (the ones that are ``tensors'') also have an identity functor,
\texttt{I\ t}, where applying \texttt{t\ f\ (I\ t)} leaves \texttt{f} unchanged
(\texttt{t\ f\ (I\ t)} is isomorphic to \texttt{f}) and \texttt{t\ (I\ t)\ f} is
also just \texttt{f}. This is represented by the associated type \texttt{I\ t}.
For example, \texttt{type\ I\ Comp\ =\ Identity}, because
\texttt{Comp\ f\ Identity} (composing any functor with \texttt{Identity},
\texttt{f\ (Identity\ a)}) is just the same as \texttt{f\ a} (the original
functor); also, \texttt{Comp\ Identity\ f} (or \texttt{Identity\ (f\ a)}) is the
same as \texttt{f\ a}.

One interesting property of these is that for tensors, if we have a binary
functor combinator \texttt{*}, we can represent a type
\texttt{f\ \textbar{}\ f\ *\ f\ \textbar{}\ f\ *\ f\ *\ f\ \textbar{}\ f\ *\ f\ *\ f\ *\ f\ \textbar{}\ ...}
(``repeatedly apply to something multiple times''), which essentially forms a
linked list along that functor combinator. We call this the ``induced monoidal
functor combinator'', given by \texttt{MF\ t}. We can also make a ``non-empty
variant'', \texttt{SF\ t}, which contains ``at least one \texttt{f}''.

For example, the type that is either \texttt{a}, \texttt{f\ a},
\texttt{f\ (f\ a)}, \texttt{f\ (f\ (f\ a))}, etc. is \texttt{Free\ f\ a}, so
that \texttt{type\ MF\ Comp\ =\ Free}. The type that is either \texttt{f\ a},
\texttt{f\ (f\ a)}, \texttt{f\ (f\ (f\ a))}, etc. (at least one layer of
\texttt{f}) is \texttt{Free1\ f\ a}, so \texttt{type\ SF\ Comp\ =\ Free1}.

\emph{functor-combinators} provides functions like
\texttt{toMF\ ::\ t\ f\ f\ \textasciitilde{}\textgreater{}\ MF\ f} to abstract
over ``converting'' back and forth between \texttt{t\ f\ f\ a} and the induced
monoidal functor combinator \texttt{MF\ t\ f\ a} (for example, between
\texttt{Comp\ f\ f\ a} and \texttt{Free\ f\ a}).\footnote{In fact, the link
  between the binary functor combinator and its induced monoidal/semigroupoidal
  functor combinators is very deep. In \emph{functor-combinators}, it's actually
  used in the definition of \texttt{CS} and \texttt{CM}:

\begin{Shaded}
\begin{Highlighting}[]
\KeywordTok{type} \DataTypeTok{CS}\NormalTok{ t }\OtherTok{=} \DataTypeTok{C}\NormalTok{ (}\DataTypeTok{SF}\NormalTok{ t)}
\KeywordTok{type} \DataTypeTok{CM}\NormalTok{ t }\OtherTok{=} \DataTypeTok{C}\NormalTok{ (}\DataTypeTok{MF}\NormalTok{ t)}
\end{Highlighting}
\end{Shaded}
}

\hypertarget{sum}{%
\subsection{:+: / Sum}\label{sum}}

\begin{itemize}
\item
  \textbf{Origin}:
  \emph{\href{https://hackage.haskell.org/package/base/docs/GHC-Generics.html}{GHC.Generics}}
  (for \texttt{:+:}) /
  \emph{\href{https://hackage.haskell.org/package/base/docs/Data-Functor-Sum.html}{Data.Functor.Sum}}
  (for \texttt{Sum})
\item
  \textbf{Mixing Strategy}: ``Either-or'': provide either case, and user has to
  handle both possibilities. Basically higher-order \texttt{Either}.

\begin{Shaded}
\begin{Highlighting}[]
\KeywordTok{data}\NormalTok{ (f }\OperatorTok{:+:}\NormalTok{ g) a}
    \OtherTok{=} \DataTypeTok{L1}\NormalTok{ (f a)}
    \OperatorTok{|} \DataTypeTok{R1}\NormalTok{ (g a)}

\KeywordTok{data} \DataTypeTok{Sum}\NormalTok{ f g a}
    \OtherTok{=} \DataTypeTok{InL}\NormalTok{ (f a)}
    \OperatorTok{|} \DataTypeTok{InR}\NormalTok{ (g a)}
\end{Highlighting}
\end{Shaded}

  It can be useful for situations where you can validly use one or the other in
  your schema or functor. For example, if you are describing an HTTP request, we
  could have \texttt{data\ GET\ a} describing a GET request and
  \texttt{data\ POST\ a} describing a POST request; \texttt{(GET\ :+:\ POST)\ a}
  would be a functor that describes either a GET or POST request.

  The person who creates the \texttt{f\ :+:\ g} decides which one to give, and
  the person who consumes/interprets/runs the \texttt{f\ :+:\ g} must provide a
  way of handling \emph{both}

\begin{Shaded}
\begin{Highlighting}[]
\NormalTok{binterpret }\OperatorTok{@}\NormalTok{(}\OperatorTok{:+:}\NormalTok{)}
\OtherTok{    ::}\NormalTok{ (f }\OperatorTok{\textasciitilde{}>}\NormalTok{ h)}
    \OtherTok{{-}>}\NormalTok{ (g }\OperatorTok{\textasciitilde{}>}\NormalTok{ h)}
    \OtherTok{{-}>}\NormalTok{ (f }\OperatorTok{:+:}\NormalTok{ g) a}
    \OtherTok{{-}>}\NormalTok{ h a}
\end{Highlighting}
\end{Shaded}

  \texttt{binterpret} becomes analogous to \texttt{either} from
  \emph{\href{https://hackage.haskell.org/package/base/docs/Data-Either.html}{Data.Either}}
\item
  \textbf{Identity}

\begin{Shaded}
\begin{Highlighting}[]
\KeywordTok{type} \DataTypeTok{I}\NormalTok{ (}\OperatorTok{:+:}\NormalTok{) }\OtherTok{=} \DataTypeTok{Void1}

\CommentTok{{-}{-} | Data type with no inhabitants}
\KeywordTok{data} \DataTypeTok{Void1}\NormalTok{ a}
\end{Highlighting}
\end{Shaded}

  \texttt{f\ :+:\ Void1} is equivalent to just \texttt{f}, because you can never
  have a value of the right branch.
\item
  \textbf{Constraints}

\begin{Shaded}
\begin{Highlighting}[]
\KeywordTok{type} \DataTypeTok{CS}\NormalTok{ (}\OperatorTok{:+:}\NormalTok{) }\OtherTok{=} \DataTypeTok{Unconstrained}
\KeywordTok{type} \DataTypeTok{CM}\NormalTok{ (}\OperatorTok{:+:}\NormalTok{) }\OtherTok{=} \DataTypeTok{Unconstrained}

\NormalTok{binterpret }\OperatorTok{@}\NormalTok{(}\OperatorTok{:+:}\NormalTok{)}
\OtherTok{    ::}\NormalTok{ f }\OperatorTok{\textasciitilde{}>}\NormalTok{ h}
    \OtherTok{{-}>}\NormalTok{ g }\OperatorTok{\textasciitilde{}>}\NormalTok{ h}
    \OtherTok{{-}>}\NormalTok{ (f }\OperatorTok{:+:}\NormalTok{ g) }\OperatorTok{\textasciitilde{}>}\NormalTok{ h}

\NormalTok{inL   }\OperatorTok{@}\OtherTok{(:+:) ::}\NormalTok{ f     }\OperatorTok{\textasciitilde{}>}\NormalTok{ f }\OperatorTok{:+:}\NormalTok{ g}
\NormalTok{inR   }\OperatorTok{@}\OtherTok{(:+:) ::}\NormalTok{ g     }\OperatorTok{\textasciitilde{}>}\NormalTok{ f }\OperatorTok{:+:}\NormalTok{ g}
\NormalTok{pureT }\OperatorTok{@}\OtherTok{(:+:) ::} \DataTypeTok{Void1} \OperatorTok{\textasciitilde{}>}\NormalTok{ h}
\end{Highlighting}
\end{Shaded}

  You don't need any constraint in order to use \texttt{binterpret},
  \texttt{inL}, \texttt{inR}, etc.

  However, note that \texttt{pureT} is effectively impossible to call, because
  no values of type \texttt{Void1\ a} exist.
\item
  \textbf{Induced Monoid}

\begin{Shaded}
\begin{Highlighting}[]
\KeywordTok{type} \DataTypeTok{SF}\NormalTok{ (}\OperatorTok{:+:}\NormalTok{) }\OtherTok{=} \DataTypeTok{Step}
\KeywordTok{type} \DataTypeTok{MF}\NormalTok{ (}\OperatorTok{:+:}\NormalTok{) }\OtherTok{=} \DataTypeTok{Step}
\end{Highlighting}
\end{Shaded}

  \texttt{Step} is the result of an infinite application of \texttt{:+:} to the
  same value:

\begin{Shaded}
\begin{Highlighting}[]
\KeywordTok{type} \DataTypeTok{Step}\NormalTok{ f }\OtherTok{=}\NormalTok{ f }\OperatorTok{:+:}\NormalTok{ f }\OperatorTok{:+:}\NormalTok{ f }\OperatorTok{:+:}\NormalTok{ f }\OperatorTok{:+:}\NormalTok{ f }\OperatorTok{:+:}\NormalTok{ f }\OperatorTok{:+:} \OperatorTok{...}\NormalTok{ etc}\OperatorTok{.}
\end{Highlighting}
\end{Shaded}

  The correspondence is:

\begin{Shaded}
\begin{Highlighting}[]
\DataTypeTok{L1}\NormalTok{ x           }\OperatorTok{<=>} \DataTypeTok{Step} \DecValTok{0}\NormalTok{ x}
\DataTypeTok{R1}\NormalTok{ (}\DataTypeTok{L1}\NormalTok{ y)      }\OperatorTok{<=>} \DataTypeTok{Step} \DecValTok{1}\NormalTok{ y}
\DataTypeTok{R1}\NormalTok{ (}\DataTypeTok{R1}\NormalTok{ (}\DataTypeTok{L1}\NormalTok{ z)) }\OperatorTok{<=>} \DataTypeTok{Step} \DecValTok{2}\NormalTok{ z}
\CommentTok{{-}{-} etc.}
\end{Highlighting}
\end{Shaded}

  It's not a particularly useful type, but it can be useful if you want to
  provide an \texttt{f\ a} alongside ``which position'' it is on the infinite
  list.
\end{itemize}

\hypertarget{product}{%
\subsection{:*: / Product}\label{product}}

\begin{itemize}
\item
  \textbf{Origin}:
  \emph{\href{https://hackage.haskell.org/package/base/docs/GHC-Generics.html}{GHC.Generics}}
  (for \texttt{:*:}) /
  \emph{\href{https://hackage.haskell.org/package/base/docs/Data-Functor-Product.html}{Data.Functor.Product}}
  (for \texttt{Product})
\item
  \textbf{Mixing Strategy}: ``Both, separately'': provide values from
  \emph{both} functors, and the user can choose which one they want to use.
  Basically a higher-order tuple.

\begin{Shaded}
\begin{Highlighting}[]
\KeywordTok{data}\NormalTok{ (f }\OperatorTok{:*:}\NormalTok{ g) a }\OtherTok{=}\NormalTok{ f a }\OperatorTok{:*:}\NormalTok{ g a}

\KeywordTok{data} \DataTypeTok{Product}\NormalTok{ f g a }\OtherTok{=} \DataTypeTok{Pair}\NormalTok{ (f a) (g a)}
\end{Highlighting}
\end{Shaded}

  It can be useful for situations where your schema/functor must be
  \emph{specified} using \emph{both} functors, but the \emph{interpreter} can
  choose to use only one or the other (or both).

\begin{Shaded}
\begin{Highlighting}[]
\OtherTok{prodOutL ::}\NormalTok{ (f }\OperatorTok{:*:}\NormalTok{ g) }\OperatorTok{\textasciitilde{}>}\NormalTok{ f}
\NormalTok{prodOutL (x }\OperatorTok{:*:}\NormalTok{ \_) }\OtherTok{=}\NormalTok{ x}

\OtherTok{prodOutR ::}\NormalTok{ (f }\OperatorTok{:*:}\NormalTok{ g) }\OperatorTok{\textasciitilde{}>}\NormalTok{ g}
\NormalTok{prodOutR (\_ }\OperatorTok{:*:}\NormalTok{ y) }\OtherTok{=}\NormalTok{ y}
\end{Highlighting}
\end{Shaded}
\item
  \textbf{Identity}

\begin{Shaded}
\begin{Highlighting}[]
\KeywordTok{type} \DataTypeTok{I}\NormalTok{ (}\OperatorTok{:*:}\NormalTok{) }\OtherTok{=} \DataTypeTok{Proxy}

\CommentTok{{-}{-} | Data type with only a single constructor and no information}
\KeywordTok{data} \DataTypeTok{Proxy}\NormalTok{ a }\OtherTok{=} \DataTypeTok{Proxy}
\end{Highlighting}
\end{Shaded}

  \texttt{f\ :+:\ Proxy} is equivalent to just \texttt{f}, because the left hand
  side doesn't add anything extra to the pair.
\item
  \textbf{Constraints}

\begin{Shaded}
\begin{Highlighting}[]
\KeywordTok{type} \DataTypeTok{CS}\NormalTok{ (}\OperatorTok{:*:}\NormalTok{) }\OtherTok{=} \DataTypeTok{Alt}
\KeywordTok{type} \DataTypeTok{CM}\NormalTok{ (}\OperatorTok{:*:}\NormalTok{) }\OtherTok{=} \DataTypeTok{Plus}

\NormalTok{binterpret }\OperatorTok{@}\NormalTok{(}\OperatorTok{:*:}\NormalTok{)}
\OtherTok{    ::} \DataTypeTok{Alt}\NormalTok{ h}
    \OtherTok{=>}\NormalTok{ f }\OperatorTok{\textasciitilde{}>}\NormalTok{ h}
    \OtherTok{{-}>}\NormalTok{ g }\OperatorTok{\textasciitilde{}>}\NormalTok{ h}
    \OtherTok{{-}>}\NormalTok{ (f }\OperatorTok{:*:}\NormalTok{ g) }\OperatorTok{\textasciitilde{}>}\NormalTok{ h}

\NormalTok{inL   }\OperatorTok{@}\OtherTok{(:*:) ::} \DataTypeTok{Plus}\NormalTok{ g }\OtherTok{=>}\NormalTok{ f     }\OperatorTok{\textasciitilde{}>}\NormalTok{ f }\OperatorTok{:*:}\NormalTok{ g}
\NormalTok{inR   }\OperatorTok{@}\OtherTok{(:*:) ::} \DataTypeTok{Plus}\NormalTok{ f }\OtherTok{=>}\NormalTok{ g     }\OperatorTok{\textasciitilde{}>}\NormalTok{ f }\OperatorTok{:*:}\NormalTok{ g}
\NormalTok{pureT }\OperatorTok{@}\OtherTok{(:*:) ::} \DataTypeTok{Plus}\NormalTok{ h }\OtherTok{=>} \DataTypeTok{Proxy} \OperatorTok{\textasciitilde{}>}\NormalTok{ h}
\end{Highlighting}
\end{Shaded}

  \texttt{Alt}, from
  \emph{\href{https://hackage.haskell.org/package/semigroupoids/docs/Data-Functor-Alt.html}{Data.Functor.Alt}}
  in \emph{semigroupoids}, can be thought of a ``higher-kinded semigroup'': it's
  like \texttt{Alternative}, but with no \texttt{Applicative} constraint and no
  identity:

\begin{Shaded}
\begin{Highlighting}[]
\KeywordTok{class} \DataTypeTok{Alt}\NormalTok{ f }\KeywordTok{where}
\OtherTok{    (<!>) ::}\NormalTok{ f a }\OtherTok{{-}>}\NormalTok{ f a }\OtherTok{{-}>}\NormalTok{ f a}
\end{Highlighting}
\end{Shaded}

  It is used to combine the results in both branches of the \texttt{:*:}.

  To introduce an ``empty'' branch, we need \texttt{Plus} (in
  \emph{\href{https://hackage.haskell.org/package/semigroupoids/docs/Data-Functor-Plus.html}{Data.Functor.Plus}}),
  which is like a higher-kinded \texttt{Monoid}, or \texttt{Alternative} with no
  \texttt{Applicative}:

\begin{Shaded}
\begin{Highlighting}[]
\KeywordTok{class} \DataTypeTok{Alt}\NormalTok{ f }\OtherTok{=>} \DataTypeTok{Plus}\NormalTok{ f }\KeywordTok{where}
\OtherTok{    zero ::}\NormalTok{ f a}
\end{Highlighting}
\end{Shaded}
\item
  \textbf{Induced Monoid}

\begin{Shaded}
\begin{Highlighting}[]
\KeywordTok{type} \DataTypeTok{SF}\NormalTok{ (}\OperatorTok{:*:}\NormalTok{) }\OtherTok{=} \DataTypeTok{NonEmptyF}
\KeywordTok{type} \DataTypeTok{MF}\NormalTok{ (}\OperatorTok{:*:}\NormalTok{) }\OtherTok{=} \DataTypeTok{ListF}
\end{Highlighting}
\end{Shaded}

  \texttt{ListF\ f\ a} is a ``list of \texttt{f\ a}s''. It represents the
  posibility of having \texttt{Proxy} (zero items), \texttt{x\ ::\ f\ a} (one
  item), \texttt{x\ :*:\ y} (two items), \texttt{x\ :*:\ y\ :*:\ z} (three
  items), etc.

  It's basically an ordered collection of \texttt{f\ a}s \texttt{:*:}d with each
  other.

\begin{Shaded}
\begin{Highlighting}[]
\DataTypeTok{Proxy}         \OperatorTok{<=>} \DataTypeTok{ListF}\NormalTok{ []}
\NormalTok{x             }\OperatorTok{<=>} \DataTypeTok{ListF}\NormalTok{ [x]}
\NormalTok{x }\OperatorTok{:*:}\NormalTok{ y       }\OperatorTok{<=>} \DataTypeTok{ListF}\NormalTok{ [x,y]}
\NormalTok{x }\OperatorTok{:*:}\NormalTok{ y }\OperatorTok{:*:}\NormalTok{ z }\OperatorTok{<=>} \DataTypeTok{ListF}\NormalTok{ [x,y,z]}
\CommentTok{{-}{-} etc.}
\end{Highlighting}
\end{Shaded}

  It is useful if you want to define a schema where you can offer
  \emph{multiple} options for the \texttt{f\ a}, and the interpreter/consumer
  can freely pick any one that they want to use.

  \texttt{NonEmptyF} is the version of \texttt{ListF} that has ``at least one
  \texttt{f\ a}''.

  See the information later on \texttt{ListF} alone (in the single-argument
  functor combinator section) for more information on usage and utility.
\end{itemize}

\hypertarget{day}{%
\subsection{Day}\label{day}}

\begin{itemize}
\item
  \textbf{Origin}:
  \emph{\href{https://hackage.haskell.org/package/kan-extensions/docs/Data-Functor-Day.html}{Data.Functor.Day}}
\item
  \textbf{Mixing Strategy}: ``Both, together forever'': provide values from
  \emph{both} functors, and the user \emph{must} also \emph{use} both.

  It can be useful for situations where your schema/functor must be
  \emph{specified} using \emph{both} functors, and the user must also \emph{use}
  both.

\begin{Shaded}
\begin{Highlighting}[]
\NormalTok{binterpret }\OperatorTok{@}\DataTypeTok{Day}
\OtherTok{    ::} \DataTypeTok{Apply}\NormalTok{ h          }\CommentTok{{-}{-} superclass of Applicative}
    \OtherTok{=>}\NormalTok{ (f }\OperatorTok{\textasciitilde{}>}\NormalTok{ h)}
    \OtherTok{{-}>}\NormalTok{ (g }\OperatorTok{\textasciitilde{}>}\NormalTok{ h)}
    \OtherTok{{-}>} \DataTypeTok{Day}\NormalTok{ f g }\OperatorTok{\textasciitilde{}>}\NormalTok{ h}
\end{Highlighting}
\end{Shaded}

  Unlike for \texttt{:*:}, you always have to interpret \emph{both} functor
  values in order to interpret a \texttt{Day}. It's a ``full mixing''.
\item
  \textbf{Identity}

\begin{Shaded}
\begin{Highlighting}[]
\KeywordTok{type} \DataTypeTok{I} \DataTypeTok{Day} \OtherTok{=} \DataTypeTok{Identity}
\end{Highlighting}
\end{Shaded}

  \texttt{Day\ f\ Identity} is equivalent to just \texttt{f}, because
  \texttt{Identity} adds no extra effects or structure.
\item
  \textbf{Constraints}

\begin{Shaded}
\begin{Highlighting}[]
\KeywordTok{type} \DataTypeTok{CS} \DataTypeTok{Day} \OtherTok{=} \DataTypeTok{Apply}
\KeywordTok{type} \DataTypeTok{CM} \DataTypeTok{Day} \OtherTok{=} \DataTypeTok{Applicative}

\NormalTok{binterpret }\OperatorTok{@}\DataTypeTok{Day}
\OtherTok{    ::} \DataTypeTok{Apply}\NormalTok{ h}
    \OtherTok{=>}\NormalTok{ f }\OperatorTok{\textasciitilde{}>}\NormalTok{ h}
    \OtherTok{{-}>}\NormalTok{ g }\OperatorTok{\textasciitilde{}>}\NormalTok{ h}
    \OtherTok{{-}>} \DataTypeTok{Day}\NormalTok{ f g }\OperatorTok{\textasciitilde{}>}\NormalTok{ h}

\NormalTok{inL   }\OperatorTok{@}\DataTypeTok{Day}\OtherTok{ ::} \DataTypeTok{Applicative}\NormalTok{ g }\OtherTok{=>}\NormalTok{ f        }\OperatorTok{\textasciitilde{}>} \DataTypeTok{Day}\NormalTok{ f g}
\NormalTok{inR   }\OperatorTok{@}\DataTypeTok{Day}\OtherTok{ ::} \DataTypeTok{Applicative}\NormalTok{ f }\OtherTok{=>}\NormalTok{ g        }\OperatorTok{\textasciitilde{}>} \DataTypeTok{Day}\NormalTok{ f g}
\NormalTok{pureT }\OperatorTok{@}\DataTypeTok{Day}\OtherTok{ ::} \DataTypeTok{Applicative}\NormalTok{ h }\OtherTok{=>} \DataTypeTok{Identity} \OperatorTok{\textasciitilde{}>}\NormalTok{ h}
\end{Highlighting}
\end{Shaded}

  \texttt{Apply}, from
  \emph{\href{https://hackage.haskell.org/package/semigroupoids/docs/Data-Functor-Apply.html}{Data.Functor.Apply}}
  in \emph{semigroupoids}, is ``\texttt{Applicative} without \texttt{pure}''; it
  only has \texttt{\textless{}*\textgreater{}} (called
  \texttt{\textless{}.\textgreater{}}).

  \texttt{pureT} is essentially
  \texttt{pure\ ::\ Applicative\ h\ =\textgreater{}\ a\ -\textgreater{}\ h\ a}.
\item
  \textbf{Induced Monoid}

\begin{Shaded}
\begin{Highlighting}[]
\KeywordTok{type} \DataTypeTok{SF} \DataTypeTok{Day} \OtherTok{=} \DataTypeTok{Ap1}
\KeywordTok{type} \DataTypeTok{MF} \DataTypeTok{Day} \OtherTok{=} \DataTypeTok{Ap}
\end{Highlighting}
\end{Shaded}

  \texttt{Ap\ f\ a} is a bunch of \texttt{f\ x}s \texttt{Day}d with each other.
  It is either:

  \begin{itemize}
  \tightlist
  \item
    \texttt{a} (zero \texttt{f}s)
  \item
    \texttt{f\ a} (one \texttt{f})
  \item
    \texttt{Day\ f\ f\ a} (two \texttt{f}s)
  \item
    \texttt{Day\ f\ (Day\ f\ f)\ a} (three \texttt{f}s)
  \item
    .. etc.
  \end{itemize}

  Like \texttt{ListF} this is very useful if you want your schema to provide a
  ``bag'' of \texttt{f\ a}s and your interpreter \emph{must use all of them}.

  For example, if we have a schema for a command line argument parser, each
  \texttt{f} may represent a command line option. To interpret it, we must look
  at \emph{all} command line options.

  \texttt{Ap1} is a version with ``at least one'' \texttt{f\ a}.

  See the information later on \texttt{Ap} alone (in the single-argument functor
  combinator section) for more information on usage and utility.
\end{itemize}

\hypertarget{comp}{%
\subsection{Comp}\label{comp}}

\begin{itemize}
\item
  \textbf{Origin}:
  \emph{\href{https://hackage.haskell.org/package/functor-combinators/docs/Control-Monad-Freer-Church.html}{Control.Monad.Freer.Church}}.
  Note that an equivalent type is also found in
  \emph{\href{https://hackage.haskell.org/package/base/docs/GHC-Generics.html}{GHC.Generics}}
  and
  \emph{\href{https://hackage.haskell.org/package/base/docs/Data-Functor-Compose.html}{Data.Functor.Compose}},
  but they are incompatible with the \texttt{HBifunctor} typeclass because they
  require the second input to have a \texttt{Functor} instance.
\item
  \textbf{Mixing Strategy}: ``Both, together, sequentially'' : provide values
  from \emph{both} functors; the user must \emph{use} both, and \emph{in order}.

\begin{Shaded}
\begin{Highlighting}[]
\KeywordTok{data} \DataTypeTok{Comp}\NormalTok{ f g a }\OtherTok{=} \DataTypeTok{Comp}\NormalTok{ (f (g a))}
\end{Highlighting}
\end{Shaded}

  It can be useful for situations where your schema/functor must be specified
  using both functors, and the user must \emph{use} both, but also enforcing
  that they must use both in the \emph{given order}: that is, for a
  \texttt{Comp\ f\ g}, they interpret \texttt{f} \emph{before} they interpret
  \texttt{g}.

\begin{Shaded}
\begin{Highlighting}[]
\NormalTok{binterpret }\OperatorTok{@}\DataTypeTok{Day}
\OtherTok{    ::} \DataTypeTok{Bind}\NormalTok{ h          }\CommentTok{{-}{-} superclass of Monad}
    \OtherTok{=>}\NormalTok{ (f }\OperatorTok{\textasciitilde{}>}\NormalTok{ h)}
    \OtherTok{{-}>}\NormalTok{ (g }\OperatorTok{\textasciitilde{}>}\NormalTok{ h)}
    \OtherTok{{-}>} \DataTypeTok{Comp}\NormalTok{ f g }\OperatorTok{\textasciitilde{}>}\NormalTok{ h}
\end{Highlighting}
\end{Shaded}

  Unlike for \texttt{:*:}, you always have to interpret \emph{both} functor
  values. And, unlike for \texttt{Day}, you must interpret both functor values
  \emph{in that order}.
\item
  \textbf{Identity}

\begin{Shaded}
\begin{Highlighting}[]
\KeywordTok{type} \DataTypeTok{I} \DataTypeTok{Comp} \OtherTok{=} \DataTypeTok{Identity}
\end{Highlighting}
\end{Shaded}

  \texttt{Comp\ f\ Identity} is equivalent to just \texttt{f}, because
  \texttt{Identity} adds no extra effects or structure.
\item
  \textbf{Constraints}

\begin{Shaded}
\begin{Highlighting}[]
\KeywordTok{type} \DataTypeTok{CS} \DataTypeTok{Comp} \OtherTok{=} \DataTypeTok{Bind}
\KeywordTok{type} \DataTypeTok{CM} \DataTypeTok{Comp} \OtherTok{=} \DataTypeTok{Monad}

\NormalTok{binterpret }\OperatorTok{@}\DataTypeTok{Comp}
\OtherTok{    ::} \DataTypeTok{Bind}\NormalTok{ h}
    \OtherTok{=>}\NormalTok{ f }\OperatorTok{\textasciitilde{}>}\NormalTok{ h}
    \OtherTok{{-}>}\NormalTok{ g }\OperatorTok{\textasciitilde{}>}\NormalTok{ h}
    \OtherTok{{-}>} \DataTypeTok{Comp}\NormalTok{ f g }\OperatorTok{\textasciitilde{}>}\NormalTok{ h}

\NormalTok{inL   }\OperatorTok{@}\DataTypeTok{Comp}\OtherTok{ ::} \DataTypeTok{Monad}\NormalTok{ g }\OtherTok{=>}\NormalTok{ f        }\OperatorTok{\textasciitilde{}>} \DataTypeTok{Comp}\NormalTok{ f g}
\NormalTok{inR   }\OperatorTok{@}\DataTypeTok{Comp}\OtherTok{ ::} \DataTypeTok{Monad}\NormalTok{ f }\OtherTok{=>}\NormalTok{ g        }\OperatorTok{\textasciitilde{}>} \DataTypeTok{Comp}\NormalTok{ f g}
\NormalTok{pureT }\OperatorTok{@}\DataTypeTok{Comp}\OtherTok{ ::} \DataTypeTok{Monad}\NormalTok{ h }\OtherTok{=>} \DataTypeTok{Identity} \OperatorTok{\textasciitilde{}>}\NormalTok{ h}
\end{Highlighting}
\end{Shaded}

  \texttt{Bind}, from \emph{{[}Data.Functor.Bind{]}{[}{]}} in
  \emph{semigroupoids}, is ``\texttt{Monad} without \texttt{return}''; it only
  has \texttt{\textgreater{}\textgreater{}=} (called
  \texttt{\textgreater{}\textgreater{}-}).

  Somewhat serendipitously, the \texttt{CM} constraint associated with
  \texttt{Comp} is the famous \texttt{Monad}. Hopefully this insight also gives
  you some insight on the nature of \texttt{Monad} as an abstraction: it's a way
  to ``interpret'' in and out of \texttt{Comp}, which enforces an ordering in
  interpretation :)
\item
  \textbf{Induced Monoid}

\begin{Shaded}
\begin{Highlighting}[]
\KeywordTok{type} \DataTypeTok{SF} \DataTypeTok{Day} \OtherTok{=} \DataTypeTok{Free1}
\KeywordTok{type} \DataTypeTok{MF} \DataTypeTok{Day} \OtherTok{=} \DataTypeTok{Free}
\end{Highlighting}
\end{Shaded}

  \texttt{Free\ f\ a} is a bunch of \texttt{f\ x}s composed with each other. It
  is either:

  \begin{itemize}
  \tightlist
  \item
    \texttt{a} (zero \texttt{f}s)
  \item
    \texttt{f\ a} (one \texttt{f})
  \item
    \texttt{f\ (f\ a)} (two \texttt{f}s)
  \item
    \texttt{f\ (f\ (f\ a))} (three \texttt{f}s)
  \item
    .. etc.
  \end{itemize}

  \texttt{Free} is very useful because it allows you to specify that your schema
  can have many \texttt{f}s, sequenced one after the other, in which the
  \emph{choice} of ``the next \texttt{f}'' is allowed to depend on the
  \emph{result} of ``the previous \texttt{f}''.

  For example, in an interactive ``wizard'' sort of schema, we can have a
  functor representing a dialog box with its result type:

\begin{Shaded}
\begin{Highlighting}[]
\KeywordTok{data} \DataTypeTok{Dialog}\NormalTok{ a}
\end{Highlighting}
\end{Shaded}

  We can then represent our wizard using \texttt{Free\ Dialog\ a} --- an ordered
  sequence of dialog boxes, where the choice of the next box can depend on
  result of the previous box.

  \texttt{Free1} is a version with ``at least one'' \texttt{f\ a}.

  See the information later on \texttt{Free} alone (in the single-argument
  functor combinator section) for more information on usage and utility.
\end{itemize}

\begin{center}\rule{0.5\linewidth}{\linethickness}\end{center}

\textbf{Aside}

Let us pause for a brief aside to compare and contrast the hierarchy of the
above functor combinators, as there is an interesting progression we can draw
from them.

\begin{enumerate}
\def\labelenumi{\arabic{enumi}.}
\tightlist
\item
  \texttt{:+:}: Provide either, be ready for both.
\item
  \texttt{:*:}: Provide both, be ready for either.
\item
  \texttt{Day}: Provide both, be ready for both.
\item
  \texttt{Comp}: Provide both, be ready for both (in order).
\end{enumerate}

\begin{center}\rule{0.5\linewidth}{\linethickness}\end{center}

\hypertarget{these1}{%
\subsection{These1}\label{these1}}

\begin{itemize}
\item
  \textbf{Origin}:
  \emph{\href{https://hackage.haskell.org/package/these/docs/Data-Functor-These.html}{Data.Functor.These}}.
\item
  \textbf{Mixing Strategy}: ``Either-or, or both'': provide either (or both)
  cases, and user has to handle both possibilities. An ``inclusive either''

\begin{Shaded}
\begin{Highlighting}[]
\KeywordTok{data} \DataTypeTok{These1}\NormalTok{ f g a}
    \OtherTok{=} \DataTypeTok{This1}\NormalTok{  (f a)}
    \OperatorTok{|} \DataTypeTok{That1}\NormalTok{        (g a)}
    \OperatorTok{|} \DataTypeTok{These1}\NormalTok{ (f a) (g a)}
\end{Highlighting}
\end{Shaded}

  This can be useful for situations where your schema/functor can be specified
  using one functor or another, or even both. See description on \texttt{:+:}
  for examples.

  The person who creates the \texttt{These1\ f\ g} decides which one to give,
  and the person who consumes/interprets/runs the \texttt{f\ :+:\ g} must
  provide a way of handling \emph{both} situations.

\begin{Shaded}
\begin{Highlighting}[]
\NormalTok{binterpret }\OperatorTok{@}\DataTypeTok{These}
\OtherTok{    ::} \DataTypeTok{Alt}\NormalTok{ h}
    \OtherTok{=>}\NormalTok{ (f }\OperatorTok{\textasciitilde{}>}\NormalTok{ h)}
    \OtherTok{{-}>}\NormalTok{ (g }\OperatorTok{\textasciitilde{}>}\NormalTok{ h)}
    \OtherTok{{-}>} \DataTypeTok{These}\NormalTok{ f g a}
    \OtherTok{{-}>}\NormalTok{ h a}
\end{Highlighting}
\end{Shaded}

  You can also pattern match on the \texttt{These1} directly to be more explicit
  with how you handle each of the tree cases.
\item
  \textbf{Identity}

\begin{Shaded}
\begin{Highlighting}[]
\KeywordTok{type} \DataTypeTok{I} \DataTypeTok{These1} \OtherTok{=} \DataTypeTok{Void}
\end{Highlighting}
\end{Shaded}

  \texttt{These1\ f\ Void} is equivalent to just \texttt{f}, because it means
  the \texttt{That1} and \texttt{These1} branches will be impossible to
  construct, and you are left with only the \texttt{This1} branch.
\item
  \textbf{Constraints}

\begin{Shaded}
\begin{Highlighting}[]
\KeywordTok{type} \DataTypeTok{CS} \DataTypeTok{These1} \OtherTok{=} \DataTypeTok{Alt}
\KeywordTok{type} \DataTypeTok{CM} \DataTypeTok{These1} \OtherTok{=} \DataTypeTok{Alt}

\NormalTok{binterpret }\OperatorTok{@}\DataTypeTok{These1}
\OtherTok{    ::} \DataTypeTok{Alt}\NormalTok{ h}
    \OtherTok{=>}\NormalTok{ f }\OperatorTok{\textasciitilde{}>}\NormalTok{ h}
    \OtherTok{{-}>}\NormalTok{ g }\OperatorTok{\textasciitilde{}>}\NormalTok{ h}
    \OtherTok{{-}>} \DataTypeTok{These1}\NormalTok{ f g }\OperatorTok{\textasciitilde{}>}\NormalTok{ h}

\NormalTok{inL   }\OperatorTok{@}\DataTypeTok{These1}\OtherTok{ ::} \DataTypeTok{Alt}\NormalTok{ g }\OtherTok{=>}\NormalTok{ f     }\OperatorTok{\textasciitilde{}>} \DataTypeTok{Comp}\NormalTok{ f g}
\NormalTok{inR   }\OperatorTok{@}\DataTypeTok{These1}\OtherTok{ ::} \DataTypeTok{Alt}\NormalTok{ f }\OtherTok{=>}\NormalTok{ g     }\OperatorTok{\textasciitilde{}>} \DataTypeTok{Comp}\NormalTok{ f g}
\NormalTok{pureT }\OperatorTok{@}\DataTypeTok{These1}\OtherTok{ ::} \DataTypeTok{Alt}\NormalTok{ h }\OtherTok{=>} \DataTypeTok{Void1} \OperatorTok{\textasciitilde{}>}\NormalTok{ h}
\end{Highlighting}
\end{Shaded}

  You need at least \texttt{Alt} to be able to interpret out of a
  \texttt{These1}, because you need to be able to handle the case where you have
  \emph{both} \texttt{f} and \texttt{g}, and need to combine the result.
  However, you never need a full \texttt{Plus} because we always have at least
  one value to use.
\item
  \textbf{Induced Monoid}

\begin{Shaded}
\begin{Highlighting}[]
\KeywordTok{type} \DataTypeTok{MF} \DataTypeTok{These1} \OtherTok{=} \DataTypeTok{Steps}
\end{Highlighting}
\end{Shaded}

  \texttt{Steps}, the induced monoidal functor combinator, is the result of an
  infinite application of `These1 to the same value:

\begin{Shaded}
\begin{Highlighting}[]
\KeywordTok{type} \DataTypeTok{Steps}\NormalTok{ f }\OtherTok{=}\NormalTok{ f }\OtherTok{\textasciigrave{}These1\textasciigrave{}}\NormalTok{ f }\OtherTok{\textasciigrave{}These1\textasciigrave{}}\NormalTok{ f }\OtherTok{\textasciigrave{}These1\textasciigrave{}}\NormalTok{ f }\OtherTok{\textasciigrave{}These1\textasciigrave{}} \OperatorTok{...}\NormalTok{ etc}\OperatorTok{.}
\end{Highlighting}
\end{Shaded}

  It essentially represents an infinite \emph{sparse} array of \texttt{f\ a}s,
  where an \texttt{f\ a} might exist at many different positions, with gaps here
  and there. There is always at least \emph{one} \texttt{f\ a}.

  Like \texttt{Step}, it's not particularly useful, but it can be used in
  situations where you want a giant infinite sparse array of \texttt{f\ a}s,
  each at a given position, with many gaps between them.

  I've skipped over the the induced semigroupoidal functor, which is
  \texttt{ComposeT\ Flagged\ Steps}; it requires an extra boolean ``flag''
  because of some of the quirks of nonemptiness. I feel it is even less useful
  than \texttt{Steps}.
\end{itemize}

\hypertarget{leftf-rightf}{%
\subsection{LeftF / RightF}\label{leftf-rightf}}

\begin{itemize}
\item
  \textbf{Origin}:
  \emph{\href{https://hackage.haskell.org/package/functor-combinators/docs/Data-HBifunctor.html}{Data.HBifunctor}}
\item
  \textbf{Mixing Strategy}: ``Ignore the left'' / ``ignore the right''.

\begin{Shaded}
\begin{Highlighting}[]
\KeywordTok{data} \DataTypeTok{LeftF}\NormalTok{  f g a }\OtherTok{=} \DataTypeTok{LeftF}\NormalTok{  \{}\OtherTok{ runLeftF  ::}\NormalTok{ f a \}}

\KeywordTok{data} \DataTypeTok{RightF}\NormalTok{ f g a }\OtherTok{=} \DataTypeTok{RightF}\NormalTok{ \{}\OtherTok{ runRightF ::}\NormalTok{ g a \}}
\end{Highlighting}
\end{Shaded}

  You can think of \texttt{LeftF} as ``\texttt{:+:} without the Right case,
  \texttt{R1}'', or \texttt{RightF} as ``\texttt{:+:} without the Left case,
  \texttt{L1}''. \texttt{RightF} can be considered a higher-order version of
  \emph{Tagged}, which ``tags'' a value with some type-level information.

  This can be useful if you want the second (or first) argument to be ignored,
  and only be used maybe at the type level.

  For example, \texttt{RightF\ IgnoreMe\ MyFunctor} is equivalent to just
  \texttt{MyFunctor}, but you might want to use \texttt{IgnoreMe} as a phantom
  type to help limit what values can be used for what functions.
\item
  \textbf{Identity}

  Unlike the previous functor combinators, these three are only
  \texttt{Semigroupoidal}, not \texttt{Monoidal}: this is because there is no
  functor \texttt{i} such that \texttt{LeftF\ i\ g} is equal to \texttt{g}, for
  all \texttt{g}, and no functor \texttt{i} such that \texttt{RightF\ f\ i} is
  equal to \texttt{f}, for all \texttt{f}.
\item
  \textbf{Constraints}

\begin{Shaded}
\begin{Highlighting}[]
\KeywordTok{type} \DataTypeTok{CS} \DataTypeTok{LeftF}  \OtherTok{=} \DataTypeTok{Unconstrained}
\KeywordTok{type} \DataTypeTok{CS} \DataTypeTok{RightF} \OtherTok{=} \DataTypeTok{Unconstrained}
\end{Highlighting}
\end{Shaded}

  Interpreting out of either of these is unconstrained, and can be done in any
  context.
\item
  \textbf{Induced Semigroup}

\begin{Shaded}
\begin{Highlighting}[]
\KeywordTok{type} \DataTypeTok{SF} \DataTypeTok{LeftF} \OtherTok{=} \DataTypeTok{Flagged}
\end{Highlighting}
\end{Shaded}

  For \texttt{LeftF}, the induced semigroup is \texttt{Flagged}, which is the
  \texttt{f\ a} tupled with a \texttt{Bool}. See the information on
  \texttt{Flagged} for more details. This can be useful as a type that marks if
  an \texttt{f} is made with \texttt{inject}/\texttt{pure} and is ``pure''
  (\texttt{False}), or ``tainted'' (\texttt{True}). The \emph{provider} of a
  \texttt{Flagged} can specify ``pure or tainted'', and the \emph{interpreter}
  can make a decision based on that tag.

\begin{Shaded}
\begin{Highlighting}[]
\KeywordTok{type} \DataTypeTok{SF} \DataTypeTok{RightF} \OtherTok{=} \DataTypeTok{Step}
\end{Highlighting}
\end{Shaded}

  For \texttt{RightF}, the induced semigroup is \texttt{Step}. See \texttt{Step}
  and the information on \texttt{:+:} for more details. This can be useful for
  having a value of \texttt{f\ a} at ``some point'', indexed by a
  \texttt{Natural}.
\end{itemize}

\hypertarget{single-argument}{%
\section{Single-Argument}\label{single-argument}}

Unary functor combinators usually directly ``enhance'' a functor with extra
capabilities --- usually in the form of a typeclass instance, or extra data
fields/constructors.

All of these can be ``lifted into'' with any constraint on \texttt{f}.

\begin{Shaded}
\begin{Highlighting}[]
\KeywordTok{class} \DataTypeTok{HFunctor}\NormalTok{ t }\OtherTok{=>} \DataTypeTok{Inject}\NormalTok{ t }\KeywordTok{where}
\OtherTok{    inject ::}\NormalTok{ f }\OperatorTok{\textasciitilde{}>}\NormalTok{ t f}
\end{Highlighting}
\end{Shaded}

\texttt{Inject} seems very similar to \texttt{MonadTrans}'s \texttt{lift}; the
difference is that \texttt{inject} must be \emph{natural} on \texttt{f}: it can
assume nothing about the structure of \texttt{f}, and must work universally the
same. \texttt{MonadTrans}, in contrast, requires \texttt{Monad\ f}.

Each one has an associated constraint, \texttt{C\ t}, which is the constraint on
where you can \emph{interpret} or \emph{run} values of the enhanced type:

\begin{Shaded}
\begin{Highlighting}[]
\KeywordTok{class} \DataTypeTok{Inject}\NormalTok{ t }\OtherTok{=>} \DataTypeTok{Interpret}\NormalTok{ t }\KeywordTok{where}
    \KeywordTok{type} \DataTypeTok{C}\OtherTok{ t ::}\NormalTok{ (}\DataTypeTok{Type} \OtherTok{{-}>} \DataTypeTok{Type}\NormalTok{) }\OtherTok{{-}>} \DataTypeTok{Constraint}

\NormalTok{    interpret}
\OtherTok{        ::} \DataTypeTok{C}\NormalTok{ t g}
        \OtherTok{=>}\NormalTok{ f }\OperatorTok{\textasciitilde{}>}\NormalTok{ g}
        \OtherTok{{-}>}\NormalTok{ t f }\OperatorTok{\textasciitilde{}>}\NormalTok{ g}
\end{Highlighting}
\end{Shaded}

The constraint depends on the structure that the specific functor combinator
imbues.

An important law is that:

\begin{Shaded}
\begin{Highlighting}[]
\NormalTok{interpret }\FunctionTok{id} \OperatorTok{.}\NormalTok{ inject }\OperatorTok{==} \FunctionTok{id}
\end{Highlighting}
\end{Shaded}

This means that if we inject and immediately interpret out of, we should never
\emph{lose} any information in \texttt{f}. All of the original structure in
\texttt{f} must stay intact: functor combinators only ever \emph{add} structure.

\hypertarget{coyoneda}{%
\subsection{Coyoneda}\label{coyoneda}}

\begin{itemize}
\item
  \textbf{Origin}:
  \emph{\href{https://hackage.haskell.org/package/kan-extensions/docs/Data-Functor-Coyoneda.html}{Data.Functor.Coyoneda}}
\item
  \textbf{Enhancement}: The ability to map over the parameter; it's the free
  \texttt{Functor}.

  Can be useful if \texttt{f} is created using a \texttt{GADT} that cannot be
  given a \texttt{Functor} instance.

  For example, here is an indexed type that represents the type of a ``form
  element'', where the type parameter represents the output result of the form
  element.

\begin{Shaded}
\begin{Highlighting}[]
\KeywordTok{data} \DataTypeTok{FormElem}\OtherTok{ ::} \DataTypeTok{Type} \OtherTok{{-}>} \DataTypeTok{Type} \KeywordTok{where}
    \DataTypeTok{FInput}\OtherTok{    ::} \DataTypeTok{FormElem} \DataTypeTok{String}
    \DataTypeTok{FTextbox}\OtherTok{  ::} \DataTypeTok{FormElem} \DataTypeTok{Text}
    \DataTypeTok{FCheckbox}\OtherTok{ ::} \DataTypeTok{FormElem} \DataTypeTok{Bool}
    \DataTypeTok{FNumber}\OtherTok{   ::} \DataTypeTok{FormElem} \DataTypeTok{Int}
\end{Highlighting}
\end{Shaded}

  Then \texttt{Coyoneda\ FormElem} has a \texttt{Functor} instance. We can now
  fmap over the result type of the form element; for example,
  \texttt{fmap\ ::\ (a\ -\textgreater{}\ b)\ -\textgreater{}\ Coyoneda\ FormElem\ a\ -\textgreater{}\ Coyoneda\ FormElem\ b}
  takes a form element whose result is an \texttt{a} and returns a form element
  whose result is a \texttt{b}.
\item
  \textbf{Constraint}

\begin{Shaded}
\begin{Highlighting}[]
\KeywordTok{type} \DataTypeTok{C} \DataTypeTok{Coyoneda} \OtherTok{=} \DataTypeTok{Functor}

\NormalTok{interpret }\OperatorTok{@}\DataTypeTok{Coyoneda}
\OtherTok{    ::} \DataTypeTok{Functor}\NormalTok{ g}
    \OtherTok{=>}\NormalTok{ f }\OperatorTok{\textasciitilde{}>}\NormalTok{ g}
    \OtherTok{{-}>} \DataTypeTok{Coyoneda}\NormalTok{ f }\OperatorTok{\textasciitilde{}>}\NormalTok{ g}
\end{Highlighting}
\end{Shaded}

  Interpreting out of a \texttt{Coyoneda\ f} requires the target context to
  itself be \texttt{Functor}. Usually, the context is an \texttt{Applicative} or
  \texttt{Monad}, so this is typically always satisfied.

  For example, if we want to ``run'' a \texttt{Coyoneda\ FormElem} into
  \texttt{IO}, this would be
  \texttt{interpret\ ::\ (forall\ x.\ FormElem\ x\ -\textgreater{}\ IO\ x)\ -\textgreater{}\ Coyoneda\ FormElem\ a\ -\textgreater{}\ IO\ a}.
\end{itemize}

\hypertarget{listf-nonemptyf}{%
\subsection{ListF / NonEmptyF}\label{listf-nonemptyf}}

\begin{itemize}
\item
  \textbf{Origin}:
  \emph{\href{https://hackage.haskell.org/package/functor-combinators/docs/Control-Applicative-ListF.html}{Control.Applicative.ListF}}
\item
  \textbf{Enhancement}: The ability to offer multiple options for the
  interpreter to pick from; \texttt{ListF} is the free \texttt{Plus}, and
  \texttt{NonEmptyF} is the free \texttt{Alt}.

\begin{Shaded}
\begin{Highlighting}[]
\KeywordTok{data} \DataTypeTok{ListF}\NormalTok{     f a }\OtherTok{=} \DataTypeTok{ListF}\NormalTok{     \{}\OtherTok{ runListF     ::}\NormalTok{ [f a]          \}}
\KeywordTok{data} \DataTypeTok{NonEmptyF}\NormalTok{ f a }\OtherTok{=} \DataTypeTok{NonEmptyF}\NormalTok{ \{}\OtherTok{ runNonEmptyF ::} \DataTypeTok{NonEmpty}\NormalTok{ (f a) \}}
\end{Highlighting}
\end{Shaded}

  Can be useful if you want to provide the ability when you \emph{define} your
  schema to provide multiple \texttt{f\ a}s that the \emph{interpreter}/consumer
  can freely pick from.

  For example, for a schema specifying a form, you might have multiple ways to
  enter a name. If you had a \texttt{Name} schema \texttt{data\ Name\ a}, then
  you can represent ``many different potential name inputs'' schema as
  \texttt{ListF\ Name\ a}.

  Because this has a \texttt{Plus} instance, you can use
  \texttt{(\textless{}!\textgreater{})\ ::\ ListF\ f\ a\ -\textgreater{}\ ListF\ f\ a\ -\textgreater{}\ ListF\ f\ a}
  to combine multiple option sets, and \texttt{zero\ ::\ ListF\ f\ a} to provide
  the ``choice that always fails/is unusuable''.

  \texttt{NonEmptyF} is a variety of \texttt{ListF} where you always have ``at
  least one \texttt{f\ a}''. Can be useful if you want to ensure, for your
  interpreter's sake, that you always have at least one \texttt{f\ a} option to
  pick from. For example, \texttt{NonEmptyF\ Name\ a} will always have at least
  \emph{one} name schema.

  This is essentially \texttt{f} \texttt{:*:}d with itself multiple times;
  \texttt{ListF} is the monoidal functor combinator induced by \texttt{:*:}, and
  \texttt{NonEmptyF} is the semigroupoidal functor combinator induced by
  \texttt{:*:}.

\begin{Shaded}
\begin{Highlighting}[]
\NormalTok{x             }\OperatorTok{<=>} \DataTypeTok{ListF}\NormalTok{ [x]     }\OperatorTok{<=>} \DataTypeTok{NonEmptyF}\NormalTok{ (x }\OperatorTok{:|}\NormalTok{ [])}
\NormalTok{x }\OperatorTok{:*:}\NormalTok{ y       }\OperatorTok{<=>} \DataTypeTok{ListF}\NormalTok{ [x,y]   }\OperatorTok{<=>} \DataTypeTok{NonEmptyF}\NormalTok{ (x }\OperatorTok{:|}\NormalTok{ [y])}
\NormalTok{x }\OperatorTok{:*:}\NormalTok{ y }\OperatorTok{:*:}\NormalTok{ z }\OperatorTok{<=>} \DataTypeTok{ListF}\NormalTok{ [x,y,z] }\OperatorTok{<=>} \DataTypeTok{NonEmptyF}\NormalTok{ (x }\OperatorTok{:|}\NormalTok{ [y,z])}
\end{Highlighting}
\end{Shaded}
\item
  \textbf{Constraint}

\begin{Shaded}
\begin{Highlighting}[]
\KeywordTok{type} \DataTypeTok{C} \DataTypeTok{ListF}     \OtherTok{=} \DataTypeTok{Plus}
\KeywordTok{type} \DataTypeTok{C} \DataTypeTok{NonEmptyF} \OtherTok{=} \DataTypeTok{Alt}

\NormalTok{interpret }\OperatorTok{@}\DataTypeTok{ListF}
\OtherTok{    ::} \DataTypeTok{Plus}\NormalTok{ g}
    \OtherTok{=>}\NormalTok{ f }\OperatorTok{\textasciitilde{}>}\NormalTok{ g}
    \OtherTok{{-}>} \DataTypeTok{ListF}\NormalTok{ f }\OperatorTok{\textasciitilde{}>}\NormalTok{ g}

\NormalTok{interpret }\OperatorTok{@}\DataTypeTok{NonEmptyF}
\OtherTok{    ::} \DataTypeTok{Alt}\NormalTok{ g}
    \OtherTok{=>}\NormalTok{ f }\OperatorTok{\textasciitilde{}>}\NormalTok{ g}
    \OtherTok{{-}>} \DataTypeTok{NonEmptyF}\NormalTok{ f }\OperatorTok{\textasciitilde{}>}\NormalTok{ g}
\end{Highlighting}
\end{Shaded}

  Interpreting out of a \texttt{ListF\ f} requires the target context to be
  \texttt{Plus}, and interpreting out of a \texttt{NonEmptyF\ f} requires
  \texttt{Alt} (because you will never have the empty case). However, you can
  directly pattern match on the list and pick an item you want directly, which
  requires no constraint.
\end{itemize}

\hypertarget{ap-ap1}{%
\subsection{Ap / Ap1}\label{ap-ap1}}

\begin{itemize}
\item
  \textbf{Origin}:
  \emph{\href{https://hackage.haskell.org/package/free/docs/Control-Applicative-Free.html}{Control.Applicative.Free}}
  /
  \emph{\href{https://hackage.haskell.org/package/functor-combinators/docs/Data-Functor-Apply-Free.html}{Data.Functor.Apply.Free}}
\item
  \textbf{Enhancement}: The ability to provide multiple \texttt{f}s that the
  interpreter \emph{must} consume \emph{all} of; \texttt{Ap} is the free
  \texttt{Applicative}, and \texttt{Ap1} is the free \texttt{Apply}.

  While \texttt{ListF} may be considered ``multiple options \emph{offered}'',
  \texttt{Ap} can be considered ``multiple actions all \emph{required}''. The
  interpreter must consume/interpret \emph{all} of the multiple \texttt{f}s in
  order to interpret an \texttt{Ap}.

  For example, for a form schema, you might want to have multiple form elements.
  If a single form element is \texttt{data\ FormElem\ a}, then you can make a
  multi-form schema with \texttt{Ap\ FormElem\ a}. The consumer of the form
  schema must handle \emph{every} \texttt{FormElem} provided.

  Note that ordering is not enforced: while the consumer must handle each
  \texttt{f} eventually, they are free to handle it in whatever order they
  desire. In fact, they could even all be handled in parallel. See \texttt{Free}
  for a version where ordering is enforced.

  Because this has an \texttt{Applicative} instance, you can use
  \texttt{(\textless{}*\textgreater{})\ ::\ Ap\ f\ (a\ -\textgreater{}\ b)\ -\textgreater{}\ Ap\ f\ a\ -\textgreater{}\ Ap\ f\ b}
  to sequence multiple \texttt{Ap\ f}s together, and
  \texttt{pure\ ::\ a\ -\textgreater{}\ Ap\ f\ a} to produce a ``no-op''
  \texttt{Ap} without any \texttt{f}s.

  \texttt{Ap} has some utility over \texttt{Free} in that you can pattern match
  on the constructors directly and look at each individual sequenced
  \texttt{f\ a}, for static analysis, before anything is ever run or
  interpreted.

  \texttt{Ap1} is a variety of \texttt{Ap} where you always have to have ``at
  least one \texttt{f}''. Can be useful if you want to ensure, for example, that
  your form has at least one element.

  Note that this is essentially \texttt{f} \texttt{Day}d with itself multiple
  times; \texttt{Ap} is the monoidal functor combinator induced by \texttt{Day}
  and \texttt{Ap1} is the semigroupoidal functor combinator induced by
  \texttt{Day}.
\item
  \textbf{Constraint}

\begin{Shaded}
\begin{Highlighting}[]
\KeywordTok{type} \DataTypeTok{C} \DataTypeTok{Ap}  \OtherTok{=} \DataTypeTok{Applicative}
\KeywordTok{type} \DataTypeTok{C} \DataTypeTok{Ap1} \OtherTok{=} \DataTypeTok{Apply}

\NormalTok{interpret }\OperatorTok{@}\DataTypeTok{Ap}
\OtherTok{    ::} \DataTypeTok{Applicative}\NormalTok{ g}
    \OtherTok{=>}\NormalTok{ f }\OperatorTok{\textasciitilde{}>}\NormalTok{ g}
    \OtherTok{{-}>} \DataTypeTok{Ap}\NormalTok{ f }\OperatorTok{\textasciitilde{}>}\NormalTok{ g}

\NormalTok{interpret }\OperatorTok{@}\DataTypeTok{Ap1}
\OtherTok{    ::} \DataTypeTok{Apply}\NormalTok{ g}
    \OtherTok{=>}\NormalTok{ f }\OperatorTok{\textasciitilde{}>}\NormalTok{ g}
    \OtherTok{{-}>} \DataTypeTok{Ap1}\NormalTok{ f }\OperatorTok{\textasciitilde{}>}\NormalTok{ g}
\end{Highlighting}
\end{Shaded}

  Interpreting out of an \texttt{Ap\ f} requires the target context to be
  \texttt{Applicative}, and interpreting out of a \texttt{Ap1\ f} requires
  \texttt{Apply} (because you will never need the pure case).
\end{itemize}

\hypertarget{alt}{%
\subsection{Alt}\label{alt}}

\begin{itemize}
\item
  \textbf{Origin}:
  \emph{\href{https://hackage.haskell.org/package/free/docs/Control-Alternative-Free.html}{Control.Alternative.Free}}
\item
  \textbf{Enhancement}: A combination of both \texttt{ListF} and \texttt{Ap}:
  provide a choice (\texttt{ListF}-style) of sequences (\texttt{Ap}-style) of
  choices of sequences of choices \ldots.; it's the free \texttt{Alternative}.

\begin{Shaded}
\begin{Highlighting}[]
\DataTypeTok{Alt}\NormalTok{ f }\OperatorTok{\textasciitilde{}} \DataTypeTok{ListF}\NormalTok{ (}\DataTypeTok{Ap}\NormalTok{ (}\DataTypeTok{ListF}\NormalTok{ (}\DataTypeTok{Ap}\NormalTok{ (}\DataTypeTok{ListF}\NormalTok{ (}\DataTypeTok{Ap}\NormalTok{ (}\OperatorTok{...}\NormalTok{))))}
      \OperatorTok{\textasciitilde{}} \DataTypeTok{ListF}\NormalTok{ (}\DataTypeTok{Ap}\NormalTok{ (}\DataTypeTok{Alt}\NormalTok{ f))}
\end{Highlighting}
\end{Shaded}

  This type imbues \texttt{f} with both sequential ``must use both'' operations
  (via \texttt{\textless{}*\textgreater{}}) and choice-like ``can use either''
  operations (via \texttt{\textless{}\textbar{}\textgreater{}}).

  It can be useful for implementing parser schemas, which often involve both
  sequential and choice-like combinations. If \texttt{f} is a primitive parsing
  unit, then \texttt{Alt\ f} represents a non-deterministic parser of a bunch of
  \texttt{f}s one after the other, with multiple possible results. I wrote
  \href{https://blog.jle.im/entry/free-alternative-regexp.html}{an entire
  article} on the usage of this combinator alone to implement a version of
  regular expressions.
\item
  \textbf{Constraint}

\begin{Shaded}
\begin{Highlighting}[]
\KeywordTok{type} \DataTypeTok{C} \DataTypeTok{Alt} \OtherTok{=} \DataTypeTok{Alternative}

\NormalTok{interpret }\OperatorTok{@}\DataTypeTok{Alt}
\OtherTok{    ::} \DataTypeTok{Alternative}\NormalTok{ g}
    \OtherTok{=>}\NormalTok{ f }\OperatorTok{\textasciitilde{}>}\NormalTok{ g}
    \OtherTok{{-}>} \DataTypeTok{Alt}\NormalTok{ f }\OperatorTok{\textasciitilde{}>}\NormalTok{ g}
\end{Highlighting}
\end{Shaded}

  Interpreting out of an \texttt{Alt\ f} requires the target context to be
  \texttt{Alternative} --- it uses \texttt{\textless{}*\textgreater{}} for
  sequencing, and \texttt{\textless{}\textbar{}\textgreater{}} for choice.
\end{itemize}

\hypertarget{free-free1}{%
\subsection{Free / Free1}\label{free-free1}}

\begin{itemize}
\item
  \textbf{Origin}:
  \emph{\href{https://hackage.haskell.org/package/functor-combinators/docs/Control-Monad-Freer-Church.html}{Control.Monad.Freer.Church}},
  which is a variant of
  \emph{\href{https://hackage.haskell.org/package/free/docs/Control-Monad-Free.html}{Control.Monad.Free}}
  that is compatible with \texttt{HFunctor}.
\item
  \textbf{Enhancement}: The ability to provide multiple \texttt{f}s that the
  interpreter must consume \emph{in order}, sequentially --- the free
  \texttt{Monad}.

  Contrast with \texttt{Ap}, which also sequences multiple \texttt{f}s together,
  but without any enforced order. It does this by \emph{hiding} the ``next
  \texttt{f\ a}'' until the previous \texttt{f\ a} has already been interpreted.

  Perhaps more importantly, you can sequence \texttt{f}s in a way where the
  \emph{choice of the next \texttt{f}} is allowed to depend on the \emph{result
  of the previous \texttt{f}}.

  For example, in an interactive ``wizard'' sort of schema, we can create a
  functor to represent a dialog box with its result type:

\begin{Shaded}
\begin{Highlighting}[]
\KeywordTok{data} \DataTypeTok{Dialog}\NormalTok{ a}
\end{Highlighting}
\end{Shaded}

  We can then construct a type for a wizard:

\begin{Shaded}
\begin{Highlighting}[]
\KeywordTok{type} \DataTypeTok{Wizard} \OtherTok{=} \DataTypeTok{Free} \DataTypeTok{Diloag}
\end{Highlighting}
\end{Shaded}

  \texttt{Wizard} is now an ordered sequence of dialog boxes, where the choice
  of the next box can depend on result of the previous box. Contrast to
  \texttt{Ap\ Dialog}, where the choice of all dialog boxes must be made in
  advanced, up-front, before reading any input from the user.

  In having this, however, we loose the ability to be able to inspect each
  \texttt{f\ a} before interpreting anything.

  Because this has a \texttt{Monad} instance, you can use
  \texttt{(\textless{}*\textgreater{})\ ::\ Free\ f\ (a\ -\textgreater{}\ b)\ -\textgreater{}\ Free\ f\ a\ -\textgreater{}\ Free\ f\ b}
  and
  \texttt{(\textgreater{}\textgreater{}=)\ ::\ Free\ f\ a\ -\textgreater{}\ (a\ -\textgreater{}\ Free\ f\ b)\ -\textgreater{}\ Free\ f\ b)}
  to sequence multiple \texttt{Free\ f}s together, and
  \texttt{pure\ ::\ a\ -\textgreater{}\ Free\ f\ a} to produce a ``no-op''
  \texttt{Free} without any \texttt{f}s.

  \texttt{Free1} is a variety of \texttt{Free1} where you always have to have
  ``at least one \texttt{f}''. Can be useful if you want to ensure, for example,
  that your wizard has at least one dialog box.

\begin{Shaded}
\begin{Highlighting}[]
\KeywordTok{type} \DataTypeTok{NonEmptyWizard} \OtherTok{=} \DataTypeTok{Free1} \DataTypeTok{Dialog}
\end{Highlighting}
\end{Shaded}

  Note that this is essentially \texttt{f} \texttt{Comp}d with itself multiple
  times; \texttt{Free} is the monoidal functor combinator induced by
  \texttt{Comp} and \texttt{Free1} is the semigroupoidal functor combinator
  induced by \texttt{Comp}.
\item
  \textbf{Constraint}

\begin{Shaded}
\begin{Highlighting}[]
\KeywordTok{type} \DataTypeTok{C} \DataTypeTok{Free}  \OtherTok{=} \DataTypeTok{Monad}
\KeywordTok{type} \DataTypeTok{C} \DataTypeTok{Free1} \OtherTok{=} \DataTypeTok{Bind}

\NormalTok{interpret }\OperatorTok{@}\DataTypeTok{Free}
\OtherTok{    ::} \DataTypeTok{Monad}\NormalTok{ g}
    \OtherTok{=>}\NormalTok{ f }\OperatorTok{\textasciitilde{}>}\NormalTok{ g}
    \OtherTok{{-}>} \DataTypeTok{Free}\NormalTok{ f }\OperatorTok{\textasciitilde{}>}\NormalTok{ g}

\NormalTok{interpret }\OperatorTok{@}\DataTypeTok{Free1}
\OtherTok{    ::} \DataTypeTok{Bind}\NormalTok{ g}
    \OtherTok{=>}\NormalTok{ f }\OperatorTok{\textasciitilde{}>}\NormalTok{ g}
    \OtherTok{{-}>} \DataTypeTok{Free1}\NormalTok{ f }\OperatorTok{\textasciitilde{}>}\NormalTok{ g}
\end{Highlighting}
\end{Shaded}

  Interpreting out of a \texttt{Free\ f} requires the target context to be
  \texttt{Monad}, and interpreting out of a \texttt{Free1\ f} requires
  \texttt{Bind} (because you will never need the pure case).
\end{itemize}

\hypertarget{lift-maybeapply}{%
\subsection{Lift / MaybeApply}\label{lift-maybeapply}}

\begin{itemize}
\item
  \textbf{Origin}:
  \emph{\href{https://hackage.haskell.org/package/transformers/docs/Control-Applicative-Lift.html}{Control.Applicative.Lift}}
  /
  \emph{\href{https://hackage.haskell.org/package/semigroupoids/docs/Data-Functor-Apply.html}{Data.Functor.Apply}}
  (the same type)
\item
  \textbf{Enhancement}: Make \texttt{f} ``optional'' in the schema in a way that
  the interpreter can still work with as if the \texttt{f} was still there; it's
  the free \texttt{Pointed}.

\begin{Shaded}
\begin{Highlighting}[]
\KeywordTok{data} \DataTypeTok{Lift}\NormalTok{ f a }\OtherTok{=} \DataTypeTok{Pure}\NormalTok{  a}
              \OperatorTok{|} \DataTypeTok{Other}\NormalTok{ (f a)}

\KeywordTok{newtype} \DataTypeTok{MaybeApply}\NormalTok{ f a }\OtherTok{=} \DataTypeTok{MaybeApply}\NormalTok{ \{}\OtherTok{ runMaybeApply ::} \DataTypeTok{Either}\NormalTok{ a (f a) \}}
    \CommentTok{{-}{-} \^{} same type, from semigroupoids}
\end{Highlighting}
\end{Shaded}

  Can be useful so that an \texttt{f\ a} is \emph{optional} for the schema
  definition, but in a way where the consumer can still continue from it as if
  they \emph{had} the \texttt{f}.

  It can be used, for example, to turn an required parameter \texttt{Param\ a}
  into an optional paramter \texttt{Lift\ Param\ a}.

  Contrast this to \texttt{MaybeF}: this allows the interpreter to still
  ``continue on'' as normal even if the \texttt{f} is not there. However,
  \texttt{MaybeF} forces the interpreter to abort if the \texttt{f} is not
  there.

  This can be thought of as \texttt{Identity\ :+:\ f}.
\item
  \textbf{Constraint}

\begin{Shaded}
\begin{Highlighting}[]
\KeywordTok{type} \DataTypeTok{C} \DataTypeTok{Lift} \OtherTok{=} \DataTypeTok{Pointed}

\NormalTok{interpret }\OperatorTok{@}\DataTypeTok{Lift}
\OtherTok{    ::} \DataTypeTok{Pointed}\NormalTok{ g}
    \OtherTok{=>}\NormalTok{ f }\OperatorTok{\textasciitilde{}>}\NormalTok{ g}
    \OtherTok{{-}>} \DataTypeTok{Lift}\NormalTok{ f }\OperatorTok{\textasciitilde{}>}\NormalTok{ g}
\end{Highlighting}
\end{Shaded}

  Interpreting out of a \texttt{Lift\ f} requires the target context to be
  \texttt{Pointed}, from
  \emph{\href{https://hackage.haskell.org/package/pointed/docs/Data-Pointed.html}{Data.Pointed}}
  --- it uses
  \texttt{point\ ::\ Pointed\ f\ =\textgreater{}\ a\ -\textgreater{}\ f\ a} to
  handle the case where the \texttt{f} is not there.
\end{itemize}

\hypertarget{maybef}{%
\subsection{MaybeF}\label{maybef}}

\begin{itemize}
\item
  \textbf{Origin}:
  \emph{\href{https://hackage.haskell.org/package/functor-combinators/docs/Control-Applicative-ListF.html}{Control.Applicative.ListF}}
\item
  \textbf{Enhancement}: Make \texttt{f} ``optional'' in the schema in a way that
  the interpreter \emph{must fail} if the \texttt{f} is not present.

\begin{Shaded}
\begin{Highlighting}[]
\KeywordTok{newtype} \DataTypeTok{MaybeF}\NormalTok{ f a }\OtherTok{=} \DataTypeTok{MaybeF}\NormalTok{ \{}\OtherTok{ runMaybeF ::} \DataTypeTok{Maybe}\NormalTok{ (f a) \}}
\end{Highlighting}
\end{Shaded}

  Can be useful so that an \texttt{f\ a} is \emph{optional} for the schema
  definition; if the \texttt{f} is not present, the consumer must abort the
  current branch, or find some other external way to continue onwards.

  Contrast this to \texttt{Lift}, which is an ``optional'' \texttt{f} that the
  consumer may continue on from.
\item
  \textbf{Constraint}

\begin{Shaded}
\begin{Highlighting}[]
\KeywordTok{type} \DataTypeTok{C} \DataTypeTok{MaybeF} \OtherTok{=} \DataTypeTok{Plus}

\NormalTok{interpret }\OperatorTok{@}\DataTypeTok{MaybeF}
\OtherTok{    ::} \DataTypeTok{Plus}\NormalTok{ g}
    \OtherTok{=>}\NormalTok{ f }\OperatorTok{\textasciitilde{}>}\NormalTok{ g}
    \OtherTok{{-}>} \DataTypeTok{MaybeF}\NormalTok{ f }\OperatorTok{\textasciitilde{}>}\NormalTok{ g}
\end{Highlighting}
\end{Shaded}

  Interpreting out of a \texttt{Lift\ f} requires the target context to be
  \texttt{Plus} --- it uses \texttt{zero\ ::\ f\ a} to handle the case where the
  \texttt{f} is not there. Note that this is actually ``over-constrained'': we
  really only need \texttt{zero}, and not all of \texttt{Plus} (which includes
  \texttt{\textless{}!\textgreater{}}). However, there is no common typeclass in
  Haskell that provides this, so this is the most pragmatic choice.
\end{itemize}

\hypertarget{envt}{%
\subsection{EnvT}\label{envt}}

\begin{itemize}
\item
  \textbf{Origin}:
  \emph{\href{https://hackage.haskell.org/package/comonad/docs/Control-Comonad-Trans-Env.html}{Control.Comonad.Trans.Env}}
\item
  \textbf{Enhancement}: Provide extra (monoidal) data alongside \texttt{f\ a}
  that the interpreter can access. Basically tuples extra \texttt{e} alongside
  the \texttt{f\ a}.

\begin{Shaded}
\begin{Highlighting}[]
\KeywordTok{newtype} \DataTypeTok{EnvT}\NormalTok{ e f a }\OtherTok{=} \DataTypeTok{EnvT}\NormalTok{ e (f a)}
\end{Highlighting}
\end{Shaded}

  You can use this to basically tuple some extra data alongside an
  \texttt{f\ a}. It can be useful if you want to provide extra information that
  isn't inside the \texttt{f} for the interpreter use for interpretation.

  When using
  \texttt{inject\ ::\ Monoid\ e\ =\textgreater{}\ f\ a\ -\textgreater{}\ EnvT\ e\ f\ a},
  it uses \texttt{mempty} as the initial \texttt{e} value.

  This can be thought of as \texttt{Const\ e\ :*:\ f}.

  This type exists specialized a few times here, as well:

  \begin{itemize}
  \tightlist
  \item
    \texttt{Step} is \texttt{EnvT\ (Sum\ Natural)}
  \item
    \texttt{Flagged} is \texttt{EnvT\ Any}
  \end{itemize}
\item
  \textbf{Constraint}

\begin{Shaded}
\begin{Highlighting}[]
\KeywordTok{type} \DataTypeTok{C}\NormalTok{ (}\DataTypeTok{EnvT}\NormalTok{ e) }\OtherTok{=} \DataTypeTok{Unconstrained}

\NormalTok{interpret }\OperatorTok{@}\NormalTok{(}\DataTypeTok{EnvT}\NormalTok{ e)}
\OtherTok{    ::}\NormalTok{ f }\OperatorTok{\textasciitilde{}>}\NormalTok{ g}
    \OtherTok{{-}>} \DataTypeTok{EnvT}\NormalTok{ e f }\OperatorTok{\textasciitilde{}>}\NormalTok{ g}
\end{Highlighting}
\end{Shaded}

  Interpreting out of \texttt{EnvT\ e} requires no constraints.
\end{itemize}

\hypertarget{mapf-nemapf}{%
\subsection{MapF / NEMapF}\label{mapf-nemapf}}

\begin{itemize}
\item
  \textbf{Origin}:
  \emph{\href{https://hackage.haskell.org/package/functor-combinators/docs/Control-Applicative-ListF.html}{Control.Applicative.ListF}}
\item
  \textbf{Enhancement}: Contain multiple \texttt{f\ a}s, each indexed at a
  specific \emph{key}.

\begin{Shaded}
\begin{Highlighting}[]
\KeywordTok{newtype} \DataTypeTok{MapF}\NormalTok{   k f a }\OtherTok{=} \DataTypeTok{MapF}\NormalTok{   \{}\OtherTok{ runMapF ::} \DataTypeTok{Map}\NormalTok{   k (f a) \}}
\KeywordTok{newtype} \DataTypeTok{NEMapF}\NormalTok{ k f a }\OtherTok{=} \DataTypeTok{NEMapF}\NormalTok{ \{}\OtherTok{ runMapF ::} \DataTypeTok{NEMap}\NormalTok{ k (f a) \}}
\end{Highlighting}
\end{Shaded}

  This is very similar in functionality to \texttt{ListF} and
  \texttt{NonEmptyF}, except instead of ``positional'' location, each
  \texttt{f\ a} exists at a given index. \texttt{NEMapF\ k} is the ``non-empty''
  variant. You can think of this as a \texttt{ListF} plus \texttt{EnvT}: it's a
  ``container'' of multiple \texttt{f\ a}s, but each one exists with a given
  ``tag'' index \texttt{k}.

  In usage, like for \texttt{ListF}, the \emph{definer} provides multiple
  ``labeled'' \texttt{f\ a}s, and the \emph{interpreter} can choose to interpret
  some or all of them, with accews to each labeled.

  \texttt{inject} creates a singleton \texttt{Map} at key \texttt{mempty}.

  This is very useful in schemas that have sub-schemas indexed at specific keys.
  For example, in a command line argument parser, if we have a functor that
  represents a single command:

\begin{Shaded}
\begin{Highlighting}[]
\KeywordTok{data} \DataTypeTok{Command}\NormalTok{ a}
\end{Highlighting}
\end{Shaded}

  We can immediately promote it to be a functor representing \emph{multiple
  possible} named commands, each at a given string:

\begin{Shaded}
\begin{Highlighting}[]
\KeywordTok{type} \DataTypeTok{Commands} \OtherTok{=} \DataTypeTok{MapF} \DataTypeTok{String} \DataTypeTok{Command}
\end{Highlighting}
\end{Shaded}

  So we can implement ``git push'' and ``git pull'' using:

\begin{Shaded}
\begin{Highlighting}[]
\OtherTok{push ::} \DataTypeTok{Command} \DataTypeTok{Action}
\OtherTok{pull ::} \DataTypeTok{Command} \DataTypeTok{Action}

\OtherTok{gitCommands ::} \DataTypeTok{Commands} \DataTypeTok{Action}
\NormalTok{gitCOmmands }\OtherTok{=} \DataTypeTok{MapF} \OperatorTok{.}\NormalTok{ M.fromList }\OperatorTok{$}
\NormalTok{    [ (}\StringTok{"push"}\NormalTok{, push)}
\NormalTok{    , (}\StringTok{"pull"}\NormalTok{, pull)}
\NormalTok{    ]}
\end{Highlighting}
\end{Shaded}

  This is also useful for specifying things like routes in a server.

  This type exists specialized as \texttt{Steps}, which is
  \texttt{NEMapF\ (Sum\ Natural)}.
\item
  \textbf{Constraint}

\begin{Shaded}
\begin{Highlighting}[]
\KeywordTok{type} \DataTypeTok{C}\NormalTok{ (}\DataTypeTok{MapF}\NormalTok{ k  ) }\OtherTok{=} \DataTypeTok{Plus}
\KeywordTok{type} \DataTypeTok{C}\NormalTok{ (}\DataTypeTok{NEMapF}\NormalTok{ k) }\OtherTok{=} \DataTypeTok{Alt}

\NormalTok{interpret }\OperatorTok{@}\NormalTok{(}\DataTypeTok{MapF}\NormalTok{ k)}
\OtherTok{    ::} \DataTypeTok{Plus}\NormalTok{ g}
    \OtherTok{=>}\NormalTok{ f }\OperatorTok{\textasciitilde{}>}\NormalTok{ g}
    \OtherTok{{-}>} \DataTypeTok{MapF}\NormalTok{ f }\OperatorTok{\textasciitilde{}>}\NormalTok{ g}

\NormalTok{interpret }\OperatorTok{@}\NormalTok{(}\DataTypeTok{NEMapF}\NormalTok{ k)}
\OtherTok{    ::} \DataTypeTok{Alt}\NormalTok{ g}
    \OtherTok{=>}\NormalTok{ f }\OperatorTok{\textasciitilde{}>}\NormalTok{ g}
    \OtherTok{{-}>} \DataTypeTok{NEMapF}\NormalTok{ f }\OperatorTok{\textasciitilde{}>}\NormalTok{ g}
\end{Highlighting}
\end{Shaded}

  Interpreting out of a \texttt{MapF\ f} requires the target context to be
  \texttt{Plus}, and interpreting out of a \texttt{NEMapF\ f} requires
  \texttt{Alt} (because you will never have the empty case). However, you can
  directly \emph{look up} into the \texttt{Map} and pick an item you want
  directly, which requires no constraint.
\end{itemize}

\hypertarget{readert}{%
\subsection{ReaderT}\label{readert}}

\begin{itemize}
\item
  \textbf{Origin}:
  \emph{\href{https://hackage.haskell.org/package/transformers/docs/Control-Monad-Trans-Reader.html}{Control.Monad.Trans.Reader}}
\item
  \textbf{Enhancement}: Provide each \texttt{f\ a} with access to some
  ``environment'' \texttt{r}.

\begin{Shaded}
\begin{Highlighting}[]
\KeywordTok{newtype} \DataTypeTok{ReaderT}\NormalTok{ r f a }\OtherTok{=} \DataTypeTok{ReaderT}\NormalTok{ \{}\OtherTok{ runReaderT ::}\NormalTok{ r }\OtherTok{{-}>}\NormalTok{ f a \}}
\end{Highlighting}
\end{Shaded}

  \texttt{ReaderT\ r} is often used to model some form of
  \href{https://en.wikipedia.org/wiki/Dependency_injection}{dependency
  injection}: it allows you to work ``assuming'' you had an \texttt{r}; later,
  when you \emph{run} it, you provide the \texttt{r}. It delays the evaluation
  of your final result until you provide the missing \texttt{r}.

  Another way of looking at it is that it makes your entire functor have values
  that are \emph{parameterized} with an \texttt{r}.

  For example, if you have a form data type:

\begin{Shaded}
\begin{Highlighting}[]
\KeywordTok{data} \DataTypeTok{FormElem}\NormalTok{ a}
\end{Highlighting}
\end{Shaded}

  you can now make a form data type that is parameterized by the current server
  hostname:

\begin{Shaded}
\begin{Highlighting}[]
\KeywordTok{type} \DataTypeTok{FormElemWithHost} \OtherTok{=} \DataTypeTok{ReaderT} \DataTypeTok{HostName} \DataTypeTok{FormElem}
\end{Highlighting}
\end{Shaded}

  The actual structure of your \texttt{FormElem} is deferred until you provide
  the \texttt{HostName}.

  Note that, unlike \texttt{ReaderT}, most monad transformers from
  \emph{transformers} are actually valid functor combinators under our
  perspective here, because most of them are not \emph{natural} on \texttt{f}:
  they require \texttt{Functor\ f}, at least, to implement \texttt{inject} or
  \texttt{hmap}.
\item
  \textbf{Constraint}

\begin{Shaded}
\begin{Highlighting}[]
\KeywordTok{type} \DataTypeTok{C}\NormalTok{ (}\DataTypeTok{ReaderT}\NormalTok{ r) }\OtherTok{=} \DataTypeTok{MonadReader}\NormalTok{ r}

\NormalTok{interpret }\OperatorTok{@}\NormalTok{(}\DataTypeTok{ReaderT}\NormalTok{ r)}
\OtherTok{    ::} \DataTypeTok{MonadReader}\NormalTok{ r g}
    \OtherTok{=>}\NormalTok{ f }\OperatorTok{\textasciitilde{}>}\NormalTok{ g}
    \OtherTok{{-}>} \DataTypeTok{ReaderT}\NormalTok{ r f }\OperatorTok{\textasciitilde{}>}\NormalTok{ g}
\end{Highlighting}
\end{Shaded}

  Interpreting out of a \texttt{ReaderT\ r} requires requires the target context
  to be \texttt{MonadReader\ r}, which means it must have access to
  \texttt{ask\ ::\ MonadReader\ r\ f\ =\textgreater{}\ f\ r}.
\end{itemize}

\hypertarget{step}{%
\subsection{Step}\label{step}}

\begin{itemize}
\item
  \textbf{Origin}:
  \emph{\href{https://hackage.haskell.org/package/functor-combinators/docs/Control-Applicative-Step.html}{Control.Applicative.Step}}
\item
  \textbf{Enhancement}: Tuples the \texttt{f\ a} with an extra natural number
  index.

\begin{Shaded}
\begin{Highlighting}[]
\KeywordTok{data} \DataTypeTok{Step}\NormalTok{ f a }\OtherTok{=} \DataTypeTok{Step}\NormalTok{ \{}\OtherTok{ stepPos ::} \DataTypeTok{Natural}\NormalTok{,}\OtherTok{ stepVal ::}\NormalTok{ f a \}}
\end{Highlighting}
\end{Shaded}

  This is essentially a specialized \texttt{EnvT}: it's
  \texttt{EnvT\ (Sum\ Natural)}.

  This is a useful type because it can be seen as equivalent to
  \texttt{f\ :+:\ f\ :+:\ f\ :+:\ f\ :+:\ f\ ...} forever: it's an \texttt{f},
  but at some index. In
  \emph{\href{https://hackage.haskell.org/package/functor-combinators/docs/Control-Applicative-Step.html}{Control.Applicative.Step}},
  we have specialized functions \texttt{stepUp} and \texttt{stepDown}, which
  allows you to ``match'' on the ``first'' \texttt{f} in that infinite chain; it
  will increment and decrement the index relatively to make this work properly.
\item
  \textbf{Constraint}

\begin{Shaded}
\begin{Highlighting}[]
\KeywordTok{type} \DataTypeTok{C} \DataTypeTok{Step} \OtherTok{=} \DataTypeTok{Unconstrained}

\NormalTok{interpret }\OperatorTok{@}\DataTypeTok{Step}
\OtherTok{    ::}\NormalTok{ f }\OperatorTok{\textasciitilde{}>}\NormalTok{ g}
    \OtherTok{{-}>} \DataTypeTok{Step}\NormalTok{ f }\OperatorTok{\textasciitilde{}>}\NormalTok{ g}
\end{Highlighting}
\end{Shaded}

  Interpreting out of \texttt{Step} requires no constraints; we just drop the
  \texttt{Natural} data.
\end{itemize}

\hypertarget{steps}{%
\subsection{Steps}\label{steps}}

\begin{itemize}
\item
  \textbf{Origin}:
  \emph{\href{https://hackage.haskell.org/package/functor-combinators/docs/Control-Applicative-Step.html}{Control.Applicative.Step}}
\item
  \textbf{Enhancement}: The ability to offer multiple \emph{indexed} options for
  the interpreter to pick from. Like \texttt{NonEmptyF}, except with each
  \texttt{f\ a} existing at an indexed position that the consumer/interpreter
  can look up or access.

\begin{Shaded}
\begin{Highlighting}[]
\KeywordTok{newtype} \DataTypeTok{Steps}\NormalTok{ f a }\OtherTok{=} \DataTypeTok{Steps}\NormalTok{ \{}\OtherTok{ getSteps ::} \DataTypeTok{NEMap} \DataTypeTok{Natural}\NormalTok{ (f a) \}}
\end{Highlighting}
\end{Shaded}

  This is like a mix between \texttt{NonEmptyF} and \texttt{Step}: multiple
  \texttt{f\ a} options (at least one) for the consumer/interpreter to pick
  from. Unlike \texttt{NonEmptyF}, each \texttt{f\ a} exists at an ``index'' ---
  there might be one at 0, one at 5, one at 100, etc.

  Another way of looking at this is like an infinite \emph{sparse array} of
  \texttt{f\ a}s: it's an inifinitely large collection where each spot may
  potentially have an \texttt{f\ a}.

  Useful for ``provide options that the consumer can pick from, index, or
  access'', like \texttt{ListF}/\texttt{NonEmptyF}.

  This type can be seen as an infinite
  \texttt{f\ \textasciigrave{}These1\textasciigrave{}\ f\ \textasciigrave{}These1\textasciigrave{}\ f\ \textasciigrave{}These1\textasciigrave{}\ f\ ...},
  and along these lines, \texttt{stepsDown} and \texttt{stepsUp} exist inside
  \emph{\href{https://hackage.haskell.org/package/functor-combinators/docs/Control-Applicative-Step.html}{Control.Applicative.Step}}
  analogous to \texttt{stepUp} and \texttt{stepDown} to treat a \texttt{Steps}
  in this manner.
\item
  \textbf{Constraint}

\begin{Shaded}
\begin{Highlighting}[]
\KeywordTok{type} \DataTypeTok{C} \DataTypeTok{Steps} \OtherTok{=} \DataTypeTok{Alt}

\NormalTok{interpret }\OperatorTok{@}\DataTypeTok{Steps}
\OtherTok{    ::} \DataTypeTok{Alt}\NormalTok{ g}
    \OtherTok{=>}\NormalTok{ f }\OperatorTok{\textasciitilde{}>}\NormalTok{ g}
    \OtherTok{{-}>} \DataTypeTok{Steps}\NormalTok{ f }\OperatorTok{\textasciitilde{}>}\NormalTok{ g}
\end{Highlighting}
\end{Shaded}

  Interpreting out of \texttt{Steps} requires an \texttt{Alt} to combine
  different possibilities. It does not require a full \texttt{Plus} constraint
  because we never need \texttt{zero}: a \texttt{Steps\ f\ a} always has at
  least one \texttt{f\ a}.
\end{itemize}

\hypertarget{flagged}{%
\subsection{Flagged}\label{flagged}}

\begin{itemize}
\item
  \textbf{Origin}:
  \emph{\href{https://hackage.haskell.org/package/functor-combinators/docs/Control-Applicative-Step.html}{Control.Applicative.Step}}
\item
  \textbf{Enhancement}: The ability to ``tag'' a functor value with a
  \texttt{True}/\texttt{False} boolean value.

\begin{Shaded}
\begin{Highlighting}[]
\KeywordTok{data} \DataTypeTok{Flagged}\NormalTok{ f a }\OtherTok{=} \DataTypeTok{Flagged}\NormalTok{ \{}\OtherTok{ flaggedFlag ::} \DataTypeTok{Bool}\NormalTok{,}\OtherTok{ flaggedVal ::}\NormalTok{ f a \}}
\end{Highlighting}
\end{Shaded}

  This is essentially a specialized \texttt{EnvT}: it's \texttt{EnvT\ Any}.

  If created with \texttt{inject} or \texttt{pure}, it adds the flag
  \texttt{False}. This is helpful for helping indicate if the value was created
  using a ``pure'' method like \texttt{inject} or \texttt{pure}, or an
  ``impure'' method (any other method, including direct construction).
\item
  \textbf{Constraint}

\begin{Shaded}
\begin{Highlighting}[]
\KeywordTok{type} \DataTypeTok{C} \DataTypeTok{Flagged} \OtherTok{=} \DataTypeTok{Unconstrained}

\NormalTok{interpret }\OperatorTok{@}\DataTypeTok{Flagged}
\OtherTok{    ::}\NormalTok{ f }\OperatorTok{\textasciitilde{}>}\NormalTok{ g}
    \OtherTok{{-}>} \DataTypeTok{Flagged}\NormalTok{ f }\OperatorTok{\textasciitilde{}>}\NormalTok{ g}
\end{Highlighting}
\end{Shaded}

  Interpreting out of \texttt{Flagged} requires no constraints; we just drop the
  boolean flag.
\end{itemize}

\hypertarget{final}{%
\subsection{Final}\label{final}}

\begin{itemize}
\item
  \textbf{Origin}:
  \emph{\href{https://hackage.haskell.org/package/functor-combinators/docs/Data-HFunctor-Final.html}{Data.HFunctor.Final}}
\item
  \textbf{Enhancement}: \texttt{Final\ c} will lift \texttt{f} into a free
  structure of any typeclass \texttt{c}; it will give it all of the actions/API
  of a typeclass for ``free''. \texttt{Final\ c\ f} is the ``free \texttt{c}''
  over \texttt{f}.

\begin{Shaded}
\begin{Highlighting}[]
\KeywordTok{data} \DataTypeTok{Final}\NormalTok{ c f a}
\end{Highlighting}
\end{Shaded}

  In a way, this is the ``ultimate free structure'': it can fully replace all
  other free structures of typeclasses of kind
  \texttt{Type\ -\textgreater{}\ Type}. For example:

\begin{Shaded}
\begin{Highlighting}[]
\DataTypeTok{Coyoneda}  \OperatorTok{\textasciitilde{}} \DataTypeTok{Final} \DataTypeTok{Functor}
\DataTypeTok{ListF}     \OperatorTok{\textasciitilde{}} \DataTypeTok{Final} \DataTypeTok{Plus}
\DataTypeTok{NonEmptyF} \OperatorTok{\textasciitilde{}} \DataTypeTok{Final} \DataTypeTok{Alt}
\DataTypeTok{Ap}        \OperatorTok{\textasciitilde{}} \DataTypeTok{Final} \DataTypeTok{Applicative}
\DataTypeTok{Ap1}       \OperatorTok{\textasciitilde{}} \DataTypeTok{Final} \DataTypeTok{Apply}
\DataTypeTok{Free}      \OperatorTok{\textasciitilde{}} \DataTypeTok{Final} \DataTypeTok{Monad}
\DataTypeTok{Free1}     \OperatorTok{\textasciitilde{}} \DataTypeTok{Final} \DataTypeTok{Bind}
\DataTypeTok{Lift}      \OperatorTok{\textasciitilde{}} \DataTypeTok{Final} \DataTypeTok{Pointed}
\DataTypeTok{IdentityT} \OperatorTok{\textasciitilde{}} \DataTypeTok{Final} \DataTypeTok{Unconstrained}
\end{Highlighting}
\end{Shaded}

  All of these are connections are witnessed by instances of the typeclass
  \texttt{FreeOf} in
  \emph{\href{https://hackage.haskell.org/package/functor-combinators/docs/Data-HFunctor-Final.html}{Data.HFunctor.Final}}.

  In fact, \texttt{Final\ c} is often more performant than the actual concrete
  free structures.

  The main downside is that you cannot directly pattern match on the structure
  of a \texttt{Final\ c} the same way you can pattern match on, say, \texttt{Ap}
  or \texttt{ListF}. However, you can get often around this by using
  \texttt{Final\ Plus} for most of your operations, and then
  \texttt{interpret\ inject}-ing it into \texttt{ListF} when you want to
  actually pattern match.

  You can also think of this as the ``ultimate \texttt{Interpret}'', because
  with \texttt{inject} you can push \texttt{f} into \texttt{Final\ c\ f}, and
  with \texttt{interpret} you only ever need the \texttt{c} constraint to
  ``run''/interpret this.

  So, next time you want to give an \texttt{f} the ability to
  \texttt{\textless{}*\textgreater{}} and \texttt{pure}, you can throw it into
  \texttt{Final\ Applicative}: \texttt{f} now gets ``sequencing'' abilities, and
  is equivalent to \texttt{Ap\ f}.

  If you want the API of a given typeclass \texttt{c}, you can inject \texttt{f}
  into \texttt{Final\ c}, and you get the API of that typeclass for free on
  \texttt{f}.
\item
  \textbf{Constraint}

\begin{Shaded}
\begin{Highlighting}[]
\KeywordTok{type} \DataTypeTok{C}\NormalTok{ (}\DataTypeTok{Final}\NormalTok{ c) }\OtherTok{=}\NormalTok{ c}

\NormalTok{interpret }\OperatorTok{@}\NormalTok{(}\DataTypeTok{Final}\NormalTok{ c)}
\OtherTok{    ::}\NormalTok{ c g}
    \OtherTok{=>}\NormalTok{ f }\OperatorTok{\textasciitilde{}>}\NormalTok{ g}
    \OtherTok{{-}>} \DataTypeTok{Final}\NormalTok{ c f }\OperatorTok{\textasciitilde{}>}\NormalTok{ g}
\end{Highlighting}
\end{Shaded}

  Interpreting out of a \texttt{Final\ c} requires \texttt{c}, since that is the
  extra context that \texttt{f} is lifted into.
\end{itemize}

\hypertarget{chain-chain1}{%
\subsection{Chain / Chain1}\label{chain-chain1}}

\begin{itemize}
\item
  \textbf{Origin}:
  \emph{\href{https://hackage.haskell.org/package/functor-combinators/docs/Data-HFunctor-Chain.html}{Data.HFunctor.Chain}}
\item
  \textbf{Enhancement}: \texttt{Chain\ t} will lift \texttt{f} into a linked
  list of \texttt{f}s chained by \texttt{t}.

\begin{Shaded}
\begin{Highlighting}[]
\CommentTok{{-}{-} i is intended to be the identity of t}
\KeywordTok{data} \DataTypeTok{Chain}\NormalTok{ t i f a }\OtherTok{=} \DataTypeTok{Done}\NormalTok{ (i a)}
                   \OperatorTok{|} \DataTypeTok{More}\NormalTok{ (t f (}\DataTypeTok{Chain}\NormalTok{ t i f a))}
\end{Highlighting}
\end{Shaded}

  For example, for \texttt{:*:}, \texttt{Chain\ (:*:)\ Proxy\ f} is equivalent
  to one of:

\begin{Shaded}
\begin{Highlighting}[]
\DataTypeTok{Proxy}   \OperatorTok{<=>} \DataTypeTok{Done} \DataTypeTok{Proxy}                           \OperatorTok{<=>} \DataTypeTok{ListF}\NormalTok{ []}
\NormalTok{x       }\OperatorTok{<=>} \DataTypeTok{More}\NormalTok{ (x }\OperatorTok{:*:} \DataTypeTok{Done} \DataTypeTok{Proxy}\NormalTok{)              }\OperatorTok{<=>} \DataTypeTok{ListF}\NormalTok{ [x]}
\NormalTok{x }\OperatorTok{:*:}\NormalTok{ y }\OperatorTok{<=>} \DataTypeTok{More}\NormalTok{ (x }\OperatorTok{:*:} \DataTypeTok{More}\NormalTok{ (y }\OperatorTok{:*:} \DataTypeTok{Done} \DataTypeTok{Proxy}\NormalTok{)) }\OperatorTok{<=>} \DataTypeTok{ListF}\NormalTok{ [x,y]}
\CommentTok{{-}{-} etc.}
\end{Highlighting}
\end{Shaded}

  For \texttt{:+:}, \texttt{Chain\ (:+:)\ Void1\ f} is equivalent to one of:

\begin{Shaded}
\begin{Highlighting}[]
\DataTypeTok{L1}\NormalTok{ x           }\OperatorTok{<=>} \DataTypeTok{More}\NormalTok{ (}\DataTypeTok{L1}\NormalTok{ x)                         }\OperatorTok{<=>} \DataTypeTok{Step} \DecValTok{0}\NormalTok{ x}
\DataTypeTok{R1}\NormalTok{ (}\DataTypeTok{L1}\NormalTok{ y)      }\OperatorTok{<=>} \DataTypeTok{More}\NormalTok{ (}\DataTypeTok{R1}\NormalTok{ (}\DataTypeTok{More}\NormalTok{ (}\DataTypeTok{L1}\NormalTok{ y)))             }\OperatorTok{<=>} \DataTypeTok{Step} \DecValTok{1}\NormalTok{ y}
\DataTypeTok{R1}\NormalTok{ (}\DataTypeTok{R1}\NormalTok{ (}\DataTypeTok{L1}\NormalTok{ z)) }\OperatorTok{<=>} \DataTypeTok{More}\NormalTok{ (}\DataTypeTok{R1}\NormalTok{ (}\DataTypeTok{More}\NormalTok{ (}\DataTypeTok{R1}\NormalTok{ (}\DataTypeTok{More}\NormalTok{ (}\DataTypeTok{L1}\NormalTok{ z))))) }\OperatorTok{<=>} \DataTypeTok{Step} \DecValTok{2}\NormalTok{ z}
\CommentTok{{-}{-} etc.}
\end{Highlighting}
\end{Shaded}

  This is useful because it provides a nice uniform way to work with all
  ``induced Monoidal functors''. That's because the following types are all
  isomorphic:

\begin{Shaded}
\begin{Highlighting}[]
\DataTypeTok{ListF} \OperatorTok{\textasciitilde{}} \DataTypeTok{Chain}\NormalTok{ (}\OperatorTok{:*:}\NormalTok{)  }\DataTypeTok{Proxy}
\DataTypeTok{Ap}    \OperatorTok{\textasciitilde{}} \DataTypeTok{Chain} \DataTypeTok{Day}    \DataTypeTok{Identity}
\DataTypeTok{Free}  \OperatorTok{\textasciitilde{}} \DataTypeTok{Chain} \DataTypeTok{Comp}   \DataTypeTok{Identity}
\DataTypeTok{Step}  \OperatorTok{\textasciitilde{}} \DataTypeTok{Chain}\NormalTok{ (}\OperatorTok{:+:}\NormalTok{)  }\DataTypeTok{Void}
\DataTypeTok{Steps} \OperatorTok{\textasciitilde{}} \DataTypeTok{Chain} \DataTypeTok{These1} \DataTypeTok{Void}
\end{Highlighting}
\end{Shaded}

  This isomorphism is witnessed by \texttt{unrollMF} (turn into the
  \texttt{Chain}) and \texttt{rerollMF} (convert back from the \texttt{Chain})
  in
  \emph{\href{https://hackage.haskell.org/package/functor-combinators/docs/Data-HFunctor-Chain.html}{Data.HFunctor.Chain}}.

  We also have a ``non-empty'' version, \texttt{Chain1}, for induced
  semigroupoids:

\begin{Shaded}
\begin{Highlighting}[]
\KeywordTok{data} \DataTypeTok{Chain1}\NormalTok{ t f a }\OtherTok{=} \DataTypeTok{Done1}\NormalTok{ (f a)}
                  \OperatorTok{|} \DataTypeTok{More1}\NormalTok{ (t f (}\DataTypeTok{Chain1}\NormalTok{ t f a))}
\end{Highlighting}
\end{Shaded}

\begin{Shaded}
\begin{Highlighting}[]
\DataTypeTok{NonEmptyF} \OperatorTok{\textasciitilde{}} \DataTypeTok{Chain1}\NormalTok{ (}\OperatorTok{:*:}\NormalTok{)}
\DataTypeTok{Ap1}       \OperatorTok{\textasciitilde{}} \DataTypeTok{Chain1} \DataTypeTok{Day}
\DataTypeTok{Free1}     \OperatorTok{\textasciitilde{}} \DataTypeTok{Chain1} \DataTypeTok{Comp}
\DataTypeTok{Step}      \OperatorTok{\textasciitilde{}} \DataTypeTok{Chain1}\NormalTok{ (}\OperatorTok{:+:}\NormalTok{)}
\DataTypeTok{Steps}     \OperatorTok{\textasciitilde{}} \DataTypeTok{Chain1} \DataTypeTok{These1}
\DataTypeTok{EnvT} \DataTypeTok{Any}  \OperatorTok{\textasciitilde{}} \DataTypeTok{Chain1} \DataTypeTok{LeftF}
\DataTypeTok{Step}      \OperatorTok{\textasciitilde{}} \DataTypeTok{Chain1} \DataTypeTok{RightF}
\end{Highlighting}
\end{Shaded}

  Using \texttt{ListF}, \texttt{Ap}, \texttt{Free}, \texttt{Step},
  \texttt{Steps}, etc. can sometimes feel very different, but with
  \texttt{Chain} you get a uniform interface to pattern match on (and construct)
  all of them in the same way.

  Using \texttt{NonEmptyF}, \texttt{Ap1}, \texttt{Free1}, \texttt{Step},
  \texttt{Steps}, \texttt{EnvT}, etc. can sometimes feel very different, but
  with \texttt{Chain1} you get a uniform interface to pattern match on (and
  construct) all of them in the same way.

  The construction of \texttt{Chain} is inspired by
  \href{http://oleg.fi/gists/posts/2018-02-21-single-free.html}{Oleg Grenrus's
  blog post}, and the construction of \texttt{Chain1} is inspired by
  implementations of finite automata and iteratees.
\item
  \textbf{Constraint}

\begin{Shaded}
\begin{Highlighting}[]
\KeywordTok{type} \DataTypeTok{C}\NormalTok{ (}\DataTypeTok{Chain}\NormalTok{  t (}\DataTypeTok{I}\NormalTok{ t)) }\OtherTok{=} \DataTypeTok{CM}\NormalTok{ t}
\KeywordTok{type} \DataTypeTok{C}\NormalTok{ (}\DataTypeTok{Chain1}\NormalTok{ t      ) }\OtherTok{=} \DataTypeTok{CS}\NormalTok{ t}

\NormalTok{interpret }\OperatorTok{@}\NormalTok{(}\DataTypeTok{Chain}\NormalTok{ t (}\DataTypeTok{I}\NormalTok{ t))}
\OtherTok{    ::} \DataTypeTok{CM}\NormalTok{ t g}
    \OtherTok{=>}\NormalTok{ f }\OperatorTok{\textasciitilde{}>}\NormalTok{ g}
    \OtherTok{{-}>} \DataTypeTok{Chain}\NormalTok{ t (}\DataTypeTok{I}\NormalTok{ t) f }\OperatorTok{\textasciitilde{}>}\NormalTok{ g}

\NormalTok{interpret }\OperatorTok{@}\NormalTok{(}\DataTypeTok{Chain1}\NormalTok{ t)}
\OtherTok{    ::} \DataTypeTok{CS}\NormalTok{ t g}
    \OtherTok{=>}\NormalTok{ f }\OperatorTok{\textasciitilde{}>}\NormalTok{ g}
    \OtherTok{{-}>} \DataTypeTok{Chain1}\NormalTok{ t f }\OperatorTok{\textasciitilde{}>}\NormalTok{ g}
\end{Highlighting}
\end{Shaded}

  Interpreting out of a \texttt{Chain} requires the monoidal constraint on
  \texttt{t}, since we have to ``squish'' all of the layers of \texttt{t}
  together with a potential empty case. Interpreting out of a \texttt{Chain1}
  requires the semigroupoidal constraint on \texttt{t}, since we have to squish
  all of the layers of \texttt{t} together, but we don't have to worry about the
  empty case.

  For example, we have:

\begin{Shaded}
\begin{Highlighting}[]
\KeywordTok{type} \DataTypeTok{C}\NormalTok{ (}\DataTypeTok{Chain}\NormalTok{  (}\OperatorTok{:*:}\NormalTok{) }\DataTypeTok{Proxy}\NormalTok{) }\OtherTok{=} \DataTypeTok{Plus}
\KeywordTok{type} \DataTypeTok{C}\NormalTok{ (}\DataTypeTok{Chain1}\NormalTok{ (}\OperatorTok{:*:}\NormalTok{)      ) }\OtherTok{=} \DataTypeTok{Alt}

\NormalTok{interpret }\OperatorTok{@}\NormalTok{(}\DataTypeTok{Chain}\NormalTok{ (}\OperatorTok{:*:}\NormalTok{) }\DataTypeTok{Proxy}\NormalTok{)}
\OtherTok{    ::} \DataTypeTok{Plus}\NormalTok{ g}
    \OtherTok{=>}\NormalTok{ f }\OperatorTok{\textasciitilde{}>}\NormalTok{ g}
    \OtherTok{{-}>} \DataTypeTok{Chain}\NormalTok{ (}\OperatorTok{:*:}\NormalTok{) }\DataTypeTok{Proxy}\NormalTok{ f }\OperatorTok{\textasciitilde{}>}\NormalTok{ g}

\NormalTok{interpret }\OperatorTok{@}\NormalTok{(}\DataTypeTok{Chain1}\NormalTok{ (}\OperatorTok{:*:}\NormalTok{))}
\OtherTok{    ::} \DataTypeTok{Alt}\NormalTok{ g}
    \OtherTok{=>}\NormalTok{ f }\OperatorTok{\textasciitilde{}>}\NormalTok{ g}
    \OtherTok{{-}>} \DataTypeTok{Chain1}\NormalTok{ (}\OperatorTok{:*:}\NormalTok{) f }\OperatorTok{\textasciitilde{}>}\NormalTok{ g}

\KeywordTok{type} \DataTypeTok{C}\NormalTok{ (}\DataTypeTok{Chain}  \DataTypeTok{Day} \DataTypeTok{Identity}\NormalTok{) }\OtherTok{=} \DataTypeTok{Applicative}
\KeywordTok{type} \DataTypeTok{C}\NormalTok{ (}\DataTypeTok{Chain1} \DataTypeTok{Day}\NormalTok{         ) }\OtherTok{=} \DataTypeTok{Apply}

\KeywordTok{type} \DataTypeTok{C}\NormalTok{ (}\DataTypeTok{Chain}  \DataTypeTok{Comp} \DataTypeTok{Identity}\NormalTok{) }\OtherTok{=} \DataTypeTok{Monad}
\KeywordTok{type} \DataTypeTok{C}\NormalTok{ (}\DataTypeTok{Chain1} \DataTypeTok{Comp}\NormalTok{         ) }\OtherTok{=} \DataTypeTok{Bind}
\end{Highlighting}
\end{Shaded}
\end{itemize}

\hypertarget{identityt}{%
\subsection{IdentityT}\label{identityt}}

\begin{itemize}
\item
  \textbf{Origin}:
  \emph{\href{https://hackage.haskell.org/package/base/docs/Data-Functor-Identity.html}{Data.Functor.Identity}}
\item
  \textbf{Enhancement}: None whatsoever; it adds no extra structure to
  \texttt{f}, and \texttt{IdentityT\ f} is the same as \texttt{f}; it's the
  ``free \texttt{Unconstrained}''

\begin{Shaded}
\begin{Highlighting}[]
\KeywordTok{data} \DataTypeTok{IdentityT}\NormalTok{ f a }\OtherTok{=} \DataTypeTok{IdentityT}\NormalTok{ \{}\OtherTok{ runIdentityT ::}\NormalTok{ f a \}}
\end{Highlighting}
\end{Shaded}

  This isn't too useful on its own, but it can be useful to give to the functor
  combinator combinators as a no-op functor combinator. It can also be used to
  signify ``no structure'', or as a placeholder until you figure out what sort
  of structure you want to have.

  In that sense, it can be thought of as a ``\texttt{ListF} with always one
  item'', a ``\texttt{MaybeF} that's always \texttt{Just}''', an ``\texttt{Ap}
  with always one sequenced item'', a ``\texttt{Free} with always exactly one
  layer of effects'', etc.
\item
  \textbf{Constraint}

\begin{Shaded}
\begin{Highlighting}[]
\KeywordTok{type} \DataTypeTok{C} \DataTypeTok{IdentityT} \OtherTok{=} \DataTypeTok{Unconstrained}

\NormalTok{interpret }\OperatorTok{@}\DataTypeTok{IdentityT}
\OtherTok{    ::}\NormalTok{ f }\OperatorTok{\textasciitilde{}>}\NormalTok{ g}
    \OtherTok{{-}>} \DataTypeTok{IdentityT}\NormalTok{ f }\OperatorTok{\textasciitilde{}>}\NormalTok{ g}
\end{Highlighting}
\end{Shaded}

  Interpreting out of \texttt{IdentityT} requires no constraints --- it
  basically does nothing.
\end{itemize}

\hypertarget{proxyf-constf}{%
\subsection{ProxyF / ConstF}\label{proxyf-constf}}

\begin{itemize}
\item
  \textbf{Origin}:
  \emph{\href{https://hackage.haskell.org/package/base/docs/Data-HFunctor.html}{Data.HFunctor}}
\item
  \textbf{Enhancement}: ``Black holes'' --- they completely forget all the
  structure of \texttt{f}, and are impossible to \texttt{interpret} out of.
  \texttt{Impossible}".

\begin{Shaded}
\begin{Highlighting}[]
\KeywordTok{data} \DataTypeTok{ProxyF}\NormalTok{ f a }\OtherTok{=} \DataTypeTok{ProxyF}
\KeywordTok{data} \DataTypeTok{ConstF}\NormalTok{ e f a }\OtherTok{=} \DataTypeTok{ConstF}\NormalTok{ e}
\end{Highlighting}
\end{Shaded}

  \texttt{ProxyF} is essentially \texttt{ConstF\ ()}.

  These are both valid functor combinators in that you can inject into them, and
  \texttt{interpret\ id\ .\ inject\ ==\ id} is \emph{technically} true (the best
  kind of true).

  You can use them if you want your schema to be impossible to interpret, as a
  placeholder or to signify that one branch is uninterpretable. In this sense,
  this is like a ``\texttt{ListF} that is always empty'' or a ``\texttt{MaybeF}
  that is always \texttt{Nothing}''.

  Because of this, they aren't too useful on their own --- they're more useful
  in the context of swapping out and combining or manipulating with other
  functor combinators or using with functor combinator combinators.
\item
  \textbf{Constraint}

\begin{Shaded}
\begin{Highlighting}[]
\KeywordTok{type} \DataTypeTok{C} \DataTypeTok{ProxyF}     \OtherTok{=} \DataTypeTok{Impossible}
\KeywordTok{type} \DataTypeTok{C}\NormalTok{ (}\DataTypeTok{ConstF}\NormalTok{ e) }\OtherTok{=} \DataTypeTok{Impossible}

\NormalTok{interpret }\OperatorTok{@}\DataTypeTok{ProxyF}
\OtherTok{    ::} \DataTypeTok{Impossible}\NormalTok{ g}
    \OtherTok{=>}\NormalTok{ f }\OperatorTok{\textasciitilde{}>}\NormalTok{ g}
    \OtherTok{{-}>} \DataTypeTok{ProxyF}\NormalTok{ f }\OperatorTok{\textasciitilde{}>}\NormalTok{ g}

\NormalTok{interpret }\OperatorTok{@}\NormalTok{(}\DataTypeTok{ConstF}\NormalTok{ e)}
\OtherTok{    ::} \DataTypeTok{Impossible}\NormalTok{ g}
    \OtherTok{=>}\NormalTok{ f }\OperatorTok{\textasciitilde{}>}\NormalTok{ g}
    \OtherTok{{-}>} \DataTypeTok{ConstF}\NormalTok{ e f }\OperatorTok{\textasciitilde{}>}\NormalTok{ g}
\end{Highlighting}
\end{Shaded}

  Interpreting out of these requires an impossible constraint.
\end{itemize}

\hypertarget{combinator-combinators}{%
\section{Combinator Combinators}\label{combinator-combinators}}

There exist higher-order functor combinator combinators that take functor
combinators and return new ones, too. We can talk about a uniform interface for
them, but they aren't very common, so it is probably not worth the extra
abstraction.

\hypertarget{composet}{%
\subsection{ComposeT}\label{composet}}

\begin{itemize}
\item
  \textbf{Origin}:
  \emph{\href{https://hackage.haskell.org/package/mmorph/docs/Control-Monad-Trans-Compose.html}{Control.Monad.Trans.Compose}}
\item
  \textbf{Enhancement}: Compose enhancements from two different functor
  combinators

\begin{Shaded}
\begin{Highlighting}[]
\KeywordTok{newtype} \DataTypeTok{ComposeT}\NormalTok{ s t f a }\OtherTok{=} \DataTypeTok{ComposeT}\NormalTok{ \{}\OtherTok{ getComposeT ::}\NormalTok{ s (t f) a \}}
\end{Highlighting}
\end{Shaded}

  Can be useful if you want to layer or nest functor combinators to get both
  enhancements as a \emph{single} functor combinator*.

  Usually really only useful in the context of other abstractions that expect
  functor combinators, since this is the best way to turn two functor
  combinators into a third one.
\item
  \textbf{Constraint}

\begin{Shaded}
\begin{Highlighting}[]
\KeywordTok{type} \DataTypeTok{C}\NormalTok{ (}\DataTypeTok{ComposeT}\NormalTok{ s t) }\OtherTok{=} \DataTypeTok{AndC}\NormalTok{ (}\DataTypeTok{C}\NormalTok{ s) (}\DataTypeTok{C}\NormalTok{ t)}

\NormalTok{interpret }\OperatorTok{@}\NormalTok{(}\DataTypeTok{ComposeT}\NormalTok{ s t)}
\OtherTok{    ::}\NormalTok{ (}\DataTypeTok{C}\NormalTok{ s g, }\DataTypeTok{C}\NormalTok{ t g)}
    \OtherTok{=>}\NormalTok{ f }\OperatorTok{\textasciitilde{}>}\NormalTok{ g}
    \OtherTok{{-}>} \DataTypeTok{ComposeT}\NormalTok{ s t f }\OperatorTok{\textasciitilde{}>}\NormalTok{ g}
\end{Highlighting}
\end{Shaded}

  Interpreting out of these requires the constraints on \emph{both} layers.
\end{itemize}

\hypertarget{hlift}{%
\subsection{HLift}\label{hlift}}

\begin{itemize}
\item
  \textbf{Origin}:
  \emph{\href{https://hackage.haskell.org/package/base/docs/Data-HFunctor.html}{Data.HFunctor}}
\item
  \textbf{Enhancement}: \texttt{HLift\ t\ f} lets \texttt{f} exist either
  unchanged, or with the structure of \texttt{t}.

\begin{Shaded}
\begin{Highlighting}[]
\KeywordTok{data} \DataTypeTok{HLift}\NormalTok{ t f a}
    \OtherTok{=} \DataTypeTok{HPure}\NormalTok{  (f a)}
    \OperatorTok{|} \DataTypeTok{HOther}\NormalTok{ (t f a)}
\end{Highlighting}
\end{Shaded}

  Can be useful if you want to ``conditionally enhance'' \texttt{f}. Either
  \texttt{f} can be enhanced by \texttt{t}, or it can exist in its pure
  ``newly-injected'' form.

  If \texttt{t} is \texttt{Identity}, we get \texttt{EnvT\ Any}, or
  \texttt{f\ :+:\ f}: the ``pure or impure'' combinator.
\item
  \textbf{Constraint}

\begin{Shaded}
\begin{Highlighting}[]
\KeywordTok{type} \DataTypeTok{C}\NormalTok{ (}\DataTypeTok{HLift}\NormalTok{ t) }\OtherTok{=} \DataTypeTok{C}\NormalTok{ t}

\NormalTok{interpret }\OperatorTok{@}\NormalTok{(}\DataTypeTok{HLift}\NormalTok{ t)}
\OtherTok{    ::} \DataTypeTok{C}\NormalTok{ t g}
    \OtherTok{=>}\NormalTok{ f }\OperatorTok{\textasciitilde{}>}\NormalTok{ g}
    \OtherTok{{-}>} \DataTypeTok{HLift}\NormalTok{ t f }\OperatorTok{\textasciitilde{}>}\NormalTok{ g}
\end{Highlighting}
\end{Shaded}

  Interpreting out of these requires the constraint on \texttt{t}, to handle the
  \texttt{HOther} case.
\end{itemize}

\hypertarget{hfree}{%
\subsection{HFree}\label{hfree}}

\begin{itemize}
\item
  \textbf{Origin}:
  \emph{\href{https://hackage.haskell.org/package/base/docs/Data-HFunctor.html}{Data.HFunctor}}
\item
  \textbf{Enhancement}: \texttt{HFree\ t\ f} lets \texttt{f} exist either
  unchanged, or with multiple nested enhancements by \texttt{t}.

\begin{Shaded}
\begin{Highlighting}[]
\KeywordTok{data} \DataTypeTok{HFree}\NormalTok{ t f a}
    \OtherTok{=} \DataTypeTok{HReturn}\NormalTok{ (f a)}
    \OperatorTok{|} \DataTypeTok{HJoin}\NormalTok{   (t (}\DataTypeTok{HFree}\NormalTok{ t f) a)}
\end{Highlighting}
\end{Shaded}

  It is related to \texttt{HLift}, but lets you lift over arbitrary many
  compositions of \texttt{t}, enhancing \texttt{f} multiple times. This
  essentially creates a ``tree'' of \texttt{t} branches.

  One particularly useful functor combinator to use is \texttt{MapF}. In our
  earlier examples, if we have

\begin{Shaded}
\begin{Highlighting}[]
\KeywordTok{data} \DataTypeTok{Command}\NormalTok{ a}
\end{Highlighting}
\end{Shaded}

  to represent the structure of a single command line argument parser, we can
  use

\begin{Shaded}
\begin{Highlighting}[]
\KeywordTok{type} \DataTypeTok{Commands} \OtherTok{=} \DataTypeTok{MapF} \DataTypeTok{String} \DataTypeTok{Command}
\end{Highlighting}
\end{Shaded}

  to represent \emph{multiple} potential named commands, each under a different
  \texttt{String} argument. With \texttt{HFree}, we can also use:

\begin{Shaded}
\begin{Highlighting}[]
\KeywordTok{type} \DataTypeTok{CommandTree} \OtherTok{=} \DataTypeTok{HFree}\NormalTok{ (}\DataTypeTok{MapF} \DataTypeTok{String}\NormalTok{) }\DataTypeTok{Command}
\end{Highlighting}
\end{Shaded}

  to represent \emph{nested} named commands, where each nested sub-command is
  descended on by a \texttt{String} key.

  For another example, \texttt{HFree\ IdentityT} is essentially \texttt{Step}.
\item
  \textbf{Constraint}

\begin{Shaded}
\begin{Highlighting}[]
\KeywordTok{type} \DataTypeTok{C}\NormalTok{ (}\DataTypeTok{HFree}\NormalTok{ t) }\OtherTok{=} \DataTypeTok{C}\NormalTok{ t}

\NormalTok{interpret }\OperatorTok{@}\NormalTok{(}\DataTypeTok{HFree}\NormalTok{ t)}
\OtherTok{    ::} \DataTypeTok{C}\NormalTok{ t g}
    \OtherTok{=>}\NormalTok{ f }\OperatorTok{\textasciitilde{}>}\NormalTok{ g}
    \OtherTok{{-}>} \DataTypeTok{HFree}\NormalTok{ t f }\OperatorTok{\textasciitilde{}>}\NormalTok{ g}
\end{Highlighting}
\end{Shaded}

  Interpreting out of these requires the constraint on \texttt{t}, to handle the
  \texttt{HJoin} case.

  However, it is probably usually more useful to directly pattern match on
  \texttt{HReturn} and \texttt{HJoin} and handle the recursion explicitly.

  Alternatively, we can also define a recursive folding function (provided in
  \emph{\href{https://hackage.haskell.org/package/base/docs/Data-HFunctor.html}{Data.HFunctor}})
  to recursively fold down each branch:

\begin{Shaded}
\begin{Highlighting}[]
\NormalTok{foldHFree}
\OtherTok{    ::} \DataTypeTok{HFunctor}\NormalTok{ t}
    \OtherTok{=>}\NormalTok{ (f }\OperatorTok{\textasciitilde{}>}\NormalTok{ g)}
    \OtherTok{{-}>}\NormalTok{ (t g }\OperatorTok{\textasciitilde{}>}\NormalTok{ g)}
    \OtherTok{{-}>} \DataTypeTok{HFree}\NormalTok{ t f }\OperatorTok{\textasciitilde{}>}\NormalTok{ g}
\end{Highlighting}
\end{Shaded}

  This can be useful because it allows you to distinguish between the different
  branches, and also requires no constraint on \texttt{g}.

  Applied to the \texttt{CommandTree} example, this becomes:

\begin{Shaded}
\begin{Highlighting}[]
\NormalTok{foldHFree }\OperatorTok{@}\NormalTok{(}\DataTypeTok{MapF} \DataTypeTok{String}\NormalTok{) }\OperatorTok{@}\DataTypeTok{Command}
\OtherTok{    ::} \DataTypeTok{Command} \OperatorTok{\textasciitilde{}>}\NormalTok{ g}
    \OtherTok{{-}>} \DataTypeTok{MapF} \DataTypeTok{String} \OperatorTok{\textasciitilde{}>}\NormalTok{ g}
    \OtherTok{{-}>} \DataTypeTok{CommandTree} \OperatorTok{\textasciitilde{}>}\NormalTok{ g}
\end{Highlighting}
\end{Shaded}
\end{itemize}

\hypertarget{closing-comments}{%
\section{Closing Comments}\label{closing-comments}}

As I discover more interesting or useful functor combinators (or as the
abstractions in \emph{functor-combinators} change), I will continue to update
this post. And, in the upcoming weeks and months I plan to present specific
programs I have written (and simple examples of usage) that will help show this
design pattern in use within a real program.

For now, I hope you can appreciate this as a reference to help guide your
exploration of unique ``a la carte'' (yet not fixed-point-centric) approach to
building your programs! You can jump right into using these tools to build your
program \emph{today} by importing
\emph{\href{https://hackage.haskell.org/package/functor-combinators/docs/Data-Functor-Combinator.html}{Data.Functor.Combinator}}
or wherever they can be found.

I'd be excited to hear about what programs you are able to write, so please do
let me know! You can leave a comment, find me on {[}twitter at
@mstk{]}\href{https://twitter.com/mstk}{twitter}, or find me on freenode irc
idling on \emph{\#haskell} as \emph{jle`} if you want to share, or have any
questions.

\hypertarget{special-thanks}{%
\section{Special Thanks}\label{special-thanks}}

I am very humbled to be supported by an amazing community, who make it possible
for me to devote time to researching and writing these posts. Very special
thanks to my supporter at the ``Amazing'' level on
\href{https://www.patreon.com/justinle/overview}{patreon}, Josh Vera! :)

\hypertarget{signoff}{%
\section{Signoff}\label{signoff}}

Hi, thanks for reading! You can reach me via email at
\href{mailto:justin@jle.im}{\nolinkurl{justin@jle.im}}, or at twitter at
\href{https://twitter.com/mstk}{@mstk}! This post and all others are published
under the \href{https://creativecommons.org/licenses/by-nc-nd/3.0/}{CC-BY-NC-ND
3.0} license. Corrections and edits via pull request are welcome and encouraged
at \href{https://github.com/mstksg/inCode}{the source repository}.

If you feel inclined, or this post was particularly helpful for you, why not
consider \href{https://www.patreon.com/justinle/overview}{supporting me on
Patreon}, or a \href{bitcoin:3D7rmAYgbDnp4gp4rf22THsGt74fNucPDU}{BTC donation}?
:)

\end{document}
