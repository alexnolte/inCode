\documentclass[]{article}
\usepackage{lmodern}
\usepackage{amssymb,amsmath}
\usepackage{ifxetex,ifluatex}
\usepackage{fixltx2e} % provides \textsubscript
\ifnum 0\ifxetex 1\fi\ifluatex 1\fi=0 % if pdftex
  \usepackage[T1]{fontenc}
  \usepackage[utf8]{inputenc}
\else % if luatex or xelatex
  \ifxetex
    \usepackage{mathspec}
    \usepackage{xltxtra,xunicode}
  \else
    \usepackage{fontspec}
  \fi
  \defaultfontfeatures{Mapping=tex-text,Scale=MatchLowercase}
  \newcommand{\euro}{€}
\fi
% use upquote if available, for straight quotes in verbatim environments
\IfFileExists{upquote.sty}{\usepackage{upquote}}{}
% use microtype if available
\IfFileExists{microtype.sty}{\usepackage{microtype}}{}
\usepackage[margin=1in]{geometry}
\usepackage{color}
\usepackage{fancyvrb}
\newcommand{\VerbBar}{|}
\newcommand{\VERB}{\Verb[commandchars=\\\{\}]}
\DefineVerbatimEnvironment{Highlighting}{Verbatim}{commandchars=\\\{\}}
% Add ',fontsize=\small' for more characters per line
\newenvironment{Shaded}{}{}
\newcommand{\AlertTok}[1]{\textcolor[rgb]{1.00,0.00,0.00}{\textbf{#1}}}
\newcommand{\AnnotationTok}[1]{\textcolor[rgb]{0.38,0.63,0.69}{\textbf{\textit{#1}}}}
\newcommand{\AttributeTok}[1]{\textcolor[rgb]{0.49,0.56,0.16}{#1}}
\newcommand{\BaseNTok}[1]{\textcolor[rgb]{0.25,0.63,0.44}{#1}}
\newcommand{\BuiltInTok}[1]{#1}
\newcommand{\CharTok}[1]{\textcolor[rgb]{0.25,0.44,0.63}{#1}}
\newcommand{\CommentTok}[1]{\textcolor[rgb]{0.38,0.63,0.69}{\textit{#1}}}
\newcommand{\CommentVarTok}[1]{\textcolor[rgb]{0.38,0.63,0.69}{\textbf{\textit{#1}}}}
\newcommand{\ConstantTok}[1]{\textcolor[rgb]{0.53,0.00,0.00}{#1}}
\newcommand{\ControlFlowTok}[1]{\textcolor[rgb]{0.00,0.44,0.13}{\textbf{#1}}}
\newcommand{\DataTypeTok}[1]{\textcolor[rgb]{0.56,0.13,0.00}{#1}}
\newcommand{\DecValTok}[1]{\textcolor[rgb]{0.25,0.63,0.44}{#1}}
\newcommand{\DocumentationTok}[1]{\textcolor[rgb]{0.73,0.13,0.13}{\textit{#1}}}
\newcommand{\ErrorTok}[1]{\textcolor[rgb]{1.00,0.00,0.00}{\textbf{#1}}}
\newcommand{\ExtensionTok}[1]{#1}
\newcommand{\FloatTok}[1]{\textcolor[rgb]{0.25,0.63,0.44}{#1}}
\newcommand{\FunctionTok}[1]{\textcolor[rgb]{0.02,0.16,0.49}{#1}}
\newcommand{\ImportTok}[1]{#1}
\newcommand{\InformationTok}[1]{\textcolor[rgb]{0.38,0.63,0.69}{\textbf{\textit{#1}}}}
\newcommand{\KeywordTok}[1]{\textcolor[rgb]{0.00,0.44,0.13}{\textbf{#1}}}
\newcommand{\NormalTok}[1]{#1}
\newcommand{\OperatorTok}[1]{\textcolor[rgb]{0.40,0.40,0.40}{#1}}
\newcommand{\OtherTok}[1]{\textcolor[rgb]{0.00,0.44,0.13}{#1}}
\newcommand{\PreprocessorTok}[1]{\textcolor[rgb]{0.74,0.48,0.00}{#1}}
\newcommand{\RegionMarkerTok}[1]{#1}
\newcommand{\SpecialCharTok}[1]{\textcolor[rgb]{0.25,0.44,0.63}{#1}}
\newcommand{\SpecialStringTok}[1]{\textcolor[rgb]{0.73,0.40,0.53}{#1}}
\newcommand{\StringTok}[1]{\textcolor[rgb]{0.25,0.44,0.63}{#1}}
\newcommand{\VariableTok}[1]{\textcolor[rgb]{0.10,0.09,0.49}{#1}}
\newcommand{\VerbatimStringTok}[1]{\textcolor[rgb]{0.25,0.44,0.63}{#1}}
\newcommand{\WarningTok}[1]{\textcolor[rgb]{0.38,0.63,0.69}{\textbf{\textit{#1}}}}
\ifxetex
  \usepackage[setpagesize=false, % page size defined by xetex
              unicode=false, % unicode breaks when used with xetex
              xetex]{hyperref}
\else
  \usepackage[unicode=true]{hyperref}
\fi
\hypersetup{breaklinks=true,
            bookmarks=true,
            pdfauthor={Justin Le},
            pdftitle={Dead-simple TCP/IP services using servant},
            colorlinks=true,
            citecolor=blue,
            urlcolor=blue,
            linkcolor=magenta,
            pdfborder={0 0 0}}
\urlstyle{same}  % don't use monospace font for urls
% Make links footnotes instead of hotlinks:
\renewcommand{\href}[2]{#2\footnote{\url{#1}}}
\setlength{\parindent}{0pt}
\setlength{\parskip}{6pt plus 2pt minus 1pt}
\setlength{\emergencystretch}{3em}  % prevent overfull lines
\setcounter{secnumdepth}{0}

\title{Dead-simple TCP/IP services using servant}
\author{Justin Le}
\date{August 5, 2019}

\begin{document}
\maketitle

\emph{Originally posted on
\textbf{\href{https://blog.jle.im/entry/simple-tcpip-services-servant.html}{in
Code}}.}

In my time I've written a lot of throwaway binary TCP/IP services (servers and
services you can interact with over an internet connection, through command line
interface or GUI). For me, this involves designing a protocol from scratch every
time with varying levels of hand-rolled authentication and error detection (Send
this byte for this command, this byte for this other command, etc.). Once I
design the protocol, I then have to write both the command line client and the
server --- something I usually do from scratch over the raw TCP streams.

This process was fun (and informative) the first few times I did it, but
spinning it up from scratch again every time discouraged me from doing it very
often. However, thankfully, with the
\emph{\href{https://hackage.haskell.org/package/servant}{servant}} haskell
library (and
\emph{\href{https://hackage.haskell.org/package/servant-cli}{servant-cli}}, for
command line clients), writing a TCP server/client pair for a TCP service (using
HTTP under the hood) becomes dead-simple --- the barrier for creating one fades
away that designing/writing a service becomes a tool that I reach for
immediately in a lot of cases without second thought.

\emph{servant} is usually advertised as a tool for writing web servers, web
applications, and REST APIs, but it's easily adapted to write non-web things as
well. Let's dive in and write a simple TCP/IP service (a todo list manager) to
see how straightforward the process is!

To goal of this article is to take service/program that you already have planned
out, and \emph{easily provide} it with a networked API that can be used over any
TCP/IP connection (over the internet, or even locally). This won't teach you
\emph{how} to write a todo app, but rather how to \emph{hook up} a todo app over
a TCP/IP connection quickly, with a command line client --- and in such a simple
way that you wouldn't give a second thought based on complexity issues.

This post can also serve as a stepping-stone to a ``microservices
architecture'', if you intend to build towards one (this is explored deeper by
\href{https://github.com/k-bx/owlcloud}{k-bx})\ldots but really it's more
focused for standalone user-facing applications. How you apply these techniques
is up to you :)

All of the code in this article is available online, and the server and client
are available as ``stack executables'': if you download them all, and set the
permissions properly (\texttt{chmod\ u+x}), you can directly run them to launch
the server and client (if they are all download to the same directory).

\begin{itemize}
\tightlist
\item
  \href{https://github.com/mstksg/inCode/tree/master/code-samples/servant-services/Api.hs}{API}
\item
  \href{https://github.com/mstksg/inCode/tree/master/code-samples/servant-services/server.hs}{Server}
\item
  \href{https://github.com/mstksg/inCode/tree/master/code-samples/servant-services/client.hs}{Client}
\end{itemize}

\hypertarget{todo-api}{%
\section{Todo API}\label{todo-api}}

As an example, we'll work through building one of my favorite self-contained
mini-app projects, a \href{http://todomvc.com/}{Todo list manager a la
todo-mvc}. Our service will provide functionality for:

\begin{enumerate}
\def\labelenumi{\arabic{enumi}.}
\tightlist
\item
  Viewing all tasks and their status
\item
  Adding a new task
\item
  Setting a task's completion status
\item
  Deleting a task
\item
  Pruning all completed tasks
\end{enumerate}

To facilitate doing this over an API, we'll assign each task a task ID when it
comes in, and so commands 3 and 4 will require a task ID.

To formally specify our API:

\begin{enumerate}
\def\labelenumi{\arabic{enumi}.}
\tightlist
\item
  \texttt{list}: View all tasks by their ID, status, and description. Optionally
  be able to filter for only incomplete tasks.
\item
  \texttt{add}: Given a new task description, insert a new uncompleted task.
  Return the ID of the new task.
\item
  \texttt{set}: Given a task ID and an updated status, update the task's status.
\item
  \texttt{delete}: Given a task ID, delete the task.
\item
  \texttt{prune}: Remove all completed tasks. Returns all the task IDs that
  where deleted.
\end{enumerate}

We can state this using servant's type level DSL, using an \texttt{IntMap} (from
\emph{containers}) to represent the current tasks and an \texttt{IntSet} to
represent a set of task IDs.

\begin{Shaded}
\begin{Highlighting}[]
\CommentTok{{-}{-} source: https://github.com/mstksg/inCode/tree/master/code{-}samples/servant{-}services/Api.hs}

\OtherTok{\{{-}\# LANGUAGE DeriveGeneric \#{-}\}}
\OtherTok{\{{-}\# LANGUAGE TypeInType    \#{-}\}}
\OtherTok{\{{-}\# LANGUAGE TypeOperators \#{-}\}}
\OtherTok{\{{-}\# OPTIONS\_GHC {-}Wall      \#{-}\}}

\KeywordTok{module} \DataTypeTok{Api} \KeywordTok{where}

\KeywordTok{import}           \DataTypeTok{Data.Aeson}
\KeywordTok{import}           \DataTypeTok{Data.IntMap}\NormalTok{ (}\DataTypeTok{IntMap}\NormalTok{)}
\KeywordTok{import}           \DataTypeTok{Data.IntSet}\NormalTok{ (}\DataTypeTok{IntSet}\NormalTok{)}
\KeywordTok{import}           \DataTypeTok{Data.Proxy}
\KeywordTok{import}           \DataTypeTok{Data.Text}\NormalTok{ (}\DataTypeTok{Text}\NormalTok{)}
\KeywordTok{import}           \DataTypeTok{GHC.Generics}
\KeywordTok{import}           \DataTypeTok{Servant.API}

\KeywordTok{data} \DataTypeTok{Task} \OtherTok{=} \DataTypeTok{Task}
\NormalTok{    \{}\OtherTok{ taskStatus ::} \DataTypeTok{Bool}
\NormalTok{    ,}\OtherTok{ taskDesc   ::} \DataTypeTok{Text}
\NormalTok{    \}}
  \KeywordTok{deriving}\NormalTok{ (}\DataTypeTok{Show}\NormalTok{, }\DataTypeTok{Generic}\NormalTok{)}
\KeywordTok{instance} \DataTypeTok{ToJSON}   \DataTypeTok{Task}
\KeywordTok{instance} \DataTypeTok{FromJSON} \DataTypeTok{Task}

\KeywordTok{type} \DataTypeTok{TodoApi} \OtherTok{=} \StringTok{"list"}   \OperatorTok{:>} \DataTypeTok{QueryFlag} \StringTok{"filtered"}
                        \OperatorTok{:>} \DataTypeTok{Get}\NormalTok{  \textquotesingle{}[}\DataTypeTok{JSON}\NormalTok{] (}\DataTypeTok{IntMap} \DataTypeTok{Task}\NormalTok{)}
          \OperatorTok{:<|>} \StringTok{"add"}    \OperatorTok{:>} \DataTypeTok{QueryParam\textquotesingle{}}\NormalTok{ \textquotesingle{}[}\DataTypeTok{Required}\NormalTok{] }\StringTok{"desc"} \DataTypeTok{Text}
                        \OperatorTok{:>} \DataTypeTok{Post}\NormalTok{ \textquotesingle{}[}\DataTypeTok{JSON}\NormalTok{] }\DataTypeTok{Int}
          \OperatorTok{:<|>} \StringTok{"set"}    \OperatorTok{:>} \DataTypeTok{Capture} \StringTok{"id"} \DataTypeTok{Int}
                        \OperatorTok{:>} \DataTypeTok{QueryParam} \StringTok{"completed"} \DataTypeTok{Bool}
                        \OperatorTok{:>} \DataTypeTok{Post}\NormalTok{ \textquotesingle{}[}\DataTypeTok{JSON}\NormalTok{] ()}
          \OperatorTok{:<|>} \StringTok{"delete"} \OperatorTok{:>} \DataTypeTok{Capture} \StringTok{"id"} \DataTypeTok{Int}
                        \OperatorTok{:>} \DataTypeTok{Post}\NormalTok{ \textquotesingle{}[}\DataTypeTok{JSON}\NormalTok{] ()}
          \OperatorTok{:<|>} \StringTok{"prune"}  \OperatorTok{:>} \DataTypeTok{Post}\NormalTok{ \textquotesingle{}[}\DataTypeTok{JSON}\NormalTok{] }\DataTypeTok{IntSet}

\OtherTok{todoApi ::} \DataTypeTok{Proxy} \DataTypeTok{TodoApi}
\NormalTok{todoApi }\OtherTok{=} \DataTypeTok{Proxy}
\end{Highlighting}
\end{Shaded}

We have five routes, which more or less mirror exactly the five bullet points
listed above, with some minor implementation choices:

\begin{itemize}
\tightlist
\item
  For \texttt{list}, we take ``filtered or not filtered'' as a query flag, and
  return an \texttt{IntMap} of a \texttt{Task} data type (status, description)
  under their integer ID key.
\item
  For \texttt{add}, we take the task description as a query parameter, and
  return the new ID.
\item
  For \texttt{set}, we take the task ID as a capture (path component) and an
  optional boolean query parameter. If the parameter is not given, it will be
  taken as a toggle; otherwise, it will be taken as a setting of the completion
  status.
\item
  For \texttt{delete}, we also take the task ID as a capture.
\item
  For \texttt{prune}, we return the deleted IDs as an \texttt{IntSet} (also from
  \emph{containers}).
\end{itemize}

``Query flag'', ``query parameter'', ``capture'' are all a part of the language
of HTTP and W3C. In our case, since we aren't ever directly programming against
the actual protocol-HTTP (it's only used under the hood) or pretending to write
an actual web-interfacing server, we don't really need to care too much to
distinguish them. However, it can be useful to pick meaningful choices if we
ever do want to expose this API as a web service.

\hypertarget{todo-service-server}{%
\section{Todo Service Server}\label{todo-service-server}}

The logic to implement a todo server is pretty straightforward, which is why we
chose it as an example project. It only really needs one state: the
\texttt{IntMap} of current tasks.

To write a servant server with
\emph{\href{https://hackage.haskell.org/package/servant-server}{servant-server}},
I usually like to just set up a skeleton with each route:

\begin{Shaded}
\begin{Highlighting}[]
\OtherTok{serveTodoApi ::} \DataTypeTok{IORef}\NormalTok{ (}\DataTypeTok{IntMap} \DataTypeTok{Task}\NormalTok{) }\OtherTok{{-}>} \DataTypeTok{Server} \DataTypeTok{TodoApi}
\NormalTok{serveTodoApi taskRef }\OtherTok{=}\NormalTok{ serveList}
                  \OperatorTok{:<|>}\NormalTok{ serveAdd}
                  \OperatorTok{:<|>}\NormalTok{ serveSet}
                  \OperatorTok{:<|>}\NormalTok{ serveDelete}
                  \OperatorTok{:<|>}\NormalTok{ servePrune}
\end{Highlighting}
\end{Shaded}

The corresponding GHC error tells us everything we need:

\begin{verbatim}
server.hs:15:24: error:
    Variable not in scope: serveList :: Bool -> Handler (IntMap Task)
   |
15 | serveTodoApi taskRef = serveList
   |                        ^^^^^^^^^

server.hs:16:24: error:
    Variable not in scope: serveAdd :: Text -> Handler Int
   |
16 |                   :<|> serveAdd
   |                        ^^^^^^^^

server.hs:17:24: error:
    Variable not in scope: serveSet :: Int -> Maybe Bool -> Handler ()
   |
17 |                   :<|> serveSet
   |                        ^^^^^^^^

server.hs:18:24: error:
    Variable not in scope: serveDelete :: Int -> Handler ()
   |
18 |                   :<|> serveDelete
   |                        ^^^^^^^^^^^

server.hs:19:24: error:
    Variable not in scope: servePrune :: Handler IntSet
   |
19 |                   :<|> servePrune
   |                        ^^^^^^^^^^
\end{verbatim}

It tells us exactly the types of each handler we need.

Because \texttt{Handler} has a \texttt{MonadIO} instance, we can now directly
just write every handler in terms of how it manipulates the \texttt{IntMap} in
the \texttt{IORef}, and use \texttt{liftIO}:

\begin{Shaded}
\begin{Highlighting}[]
\CommentTok{{-}{-} source: https://github.com/mstksg/inCode/tree/master/code{-}samples/servant{-}services/server.hs\#L16{-}L46}

\OtherTok{serveTodoApi ::} \DataTypeTok{IORef}\NormalTok{ (}\DataTypeTok{IntMap} \DataTypeTok{Task}\NormalTok{) }\OtherTok{{-}>} \DataTypeTok{Server} \DataTypeTok{TodoApi}
\NormalTok{serveTodoApi taskRef }\OtherTok{=}\NormalTok{ serveList}
                  \OperatorTok{:<|>}\NormalTok{ serveAdd}
                  \OperatorTok{:<|>}\NormalTok{ serveSet}
                  \OperatorTok{:<|>}\NormalTok{ serveDelete}
                  \OperatorTok{:<|>}\NormalTok{ servePrune}
  \KeywordTok{where}
\OtherTok{    serveList ::} \DataTypeTok{Bool} \OtherTok{{-}>} \DataTypeTok{Handler}\NormalTok{ (}\DataTypeTok{IntMap} \DataTypeTok{Task}\NormalTok{)}
\NormalTok{    serveList filt }\OtherTok{=}\NormalTok{ filtFunction }\OperatorTok{<$>}\NormalTok{ liftIO (readIORef taskRef)}
      \KeywordTok{where}
\NormalTok{        filtFunction}
          \OperatorTok{|}\NormalTok{ filt      }\OtherTok{=}\NormalTok{ IM.filter (}\FunctionTok{not} \OperatorTok{.}\NormalTok{ taskStatus)}
          \OperatorTok{|} \FunctionTok{otherwise} \OtherTok{=} \FunctionTok{id}
\OtherTok{    serveAdd ::} \DataTypeTok{Text} \OtherTok{{-}>} \DataTypeTok{Handler} \DataTypeTok{Int}
\NormalTok{    serveAdd t }\OtherTok{=}\NormalTok{ liftIO }\OperatorTok{$}\NormalTok{ atomicModifyIORef\textquotesingle{} taskRef }\OperatorTok{$}\NormalTok{ \textbackslash{}ts }\OtherTok{{-}>}
      \KeywordTok{let}\NormalTok{ newKey }\OtherTok{=} \FunctionTok{maybe} \DecValTok{0}\NormalTok{ ((}\OperatorTok{+} \DecValTok{1}\NormalTok{) }\OperatorTok{.} \FunctionTok{fst}\NormalTok{) (IM.lookupMax ts)}
      \KeywordTok{in}\NormalTok{  ( IM.insert newKey (}\DataTypeTok{Task} \DataTypeTok{False}\NormalTok{ t) ts, newKey )}
\OtherTok{    serveSet ::} \DataTypeTok{Int} \OtherTok{{-}>} \DataTypeTok{Maybe} \DataTypeTok{Bool} \OtherTok{{-}>} \DataTypeTok{Handler}\NormalTok{ ()}
\NormalTok{    serveSet tid s }\OtherTok{=}\NormalTok{ liftIO }\OperatorTok{$}\NormalTok{ atomicModifyIORef\textquotesingle{} taskRef }\OperatorTok{$}\NormalTok{ \textbackslash{}ts }\OtherTok{{-}>}
\NormalTok{        ( IM.adjust adjuster tid ts, () )}
      \KeywordTok{where}
\NormalTok{        adjuster (}\DataTypeTok{Task}\NormalTok{ c d) }\OtherTok{=} \KeywordTok{case}\NormalTok{ s }\KeywordTok{of}
          \DataTypeTok{Nothing} \OtherTok{{-}>} \DataTypeTok{Task}\NormalTok{ (}\FunctionTok{not}\NormalTok{ c) d}
          \DataTypeTok{Just}\NormalTok{ c\textquotesingle{} }\OtherTok{{-}>} \DataTypeTok{Task}\NormalTok{ c\textquotesingle{}      d}
\OtherTok{    serveDelete ::} \DataTypeTok{Int} \OtherTok{{-}>} \DataTypeTok{Handler}\NormalTok{ ()}
\NormalTok{    serveDelete tid }\OtherTok{=}\NormalTok{ liftIO }\OperatorTok{$}\NormalTok{ atomicModifyIORef\textquotesingle{} taskRef }\OperatorTok{$}\NormalTok{ \textbackslash{}ts }\OtherTok{{-}>}
\NormalTok{      ( IM.delete tid ts, () )}
\OtherTok{    servePrune ::} \DataTypeTok{Handler} \DataTypeTok{IntSet}
\NormalTok{    servePrune }\OtherTok{=}\NormalTok{ liftIO }\OperatorTok{$}\NormalTok{ atomicModifyIORef\textquotesingle{} taskRef }\OperatorTok{$}\NormalTok{ \textbackslash{}ts }\OtherTok{{-}>}
      \KeywordTok{let}\NormalTok{ (compl,incompl) }\OtherTok{=}\NormalTok{ IM.partition taskStatus ts}
      \KeywordTok{in}\NormalTok{  (incompl, IM.keysSet compl)}
\end{Highlighting}
\end{Shaded}

And that's it!

To run our server, we can use
\emph{\href{https://hackage.haskell.org/package/warp}{warp}}'s \texttt{run} with
\emph{servant-server}'s \texttt{serve}, after initializing the \texttt{IORef}
that our server will use with an empty map:

\begin{Shaded}
\begin{Highlighting}[]
\CommentTok{{-}{-} source: https://github.com/mstksg/inCode/tree/master/code{-}samples/servant{-}services/server.hs\#L48{-}L53}

\OtherTok{main ::} \DataTypeTok{IO}\NormalTok{ ()}
\NormalTok{main }\OtherTok{=} \KeywordTok{do}
\NormalTok{    taskRef }\OtherTok{<{-}}\NormalTok{ newIORef IM.empty}
    \FunctionTok{putStrLn} \StringTok{"Launching server..."}
\NormalTok{    run }\DecValTok{3434} \OperatorTok{$}
\NormalTok{      serve todoApi (serveTodoApi taskRef)}
\end{Highlighting}
\end{Shaded}

We now have a todo TCP/IP service running on port 3434!

\hypertarget{todo-service-client}{%
\section{Todo Service Client}\label{todo-service-client}}

To write a client, we have a couple of options.

You \emph{could} hand-write a command-line client using either
\emph{\href{https://hackage.haskell.org/package/optparse-applicative}{optparse-applicative}}
(or your favorite command line args parser) for an options-and-arguments style
interface or a readline library like
\emph{\href{https://hackage.haskell.org/package/haskeline}{haskeline}} for an
interactive interface.

Hand-writing one is made pretty simple with
\emph{\href{https://hackage.haskell.org/package/servant-client}{servant-client}},
which allows you to generate all of the HTTP calls using the \texttt{client}
function:

\begin{Shaded}
\begin{Highlighting}[]
\NormalTok{list }\OperatorTok{:<|>}\NormalTok{ add }\OperatorTok{:<|>}\NormalTok{ set }\OperatorTok{:<|>}\NormalTok{ delete }\OperatorTok{:<|>}\NormalTok{ prune }\OtherTok{=}\NormalTok{ client todoApi}
\end{Highlighting}
\end{Shaded}

This will give you the functions
\texttt{list\ ::\ Bool\ -\textgreater{}\ ClientM\ (IntMap\ Task)},
\texttt{add\ ::\ Text\ -\textgreater{}\ ClientM\ Int},
\texttt{set\ ::\ Int\ -\textgreater{}\ Maybe\ Bool\ -\textgreater{}\ ClientM\ ()},
etc., that you can now run whenever you want to dispatch a command or make a
fetch according to your hand-rolled needs.

However, this blog post is about ``dead-simple'' setups that you can roll out
within minutes. For \emph{that}, you can use the library
\emph{\href{https://hackage.haskell.org/package/servant-cli}{servant-cli}} to
automatically generate an \emph{optparse-applicative}-based command line
interface that allows you to directly specify your commands based on command
line arguments

\begin{Shaded}
\begin{Highlighting}[]
\CommentTok{{-}{-} source: https://github.com/mstksg/inCode/tree/master/code{-}samples/servant{-}services/client.hs\#L34{-}L60}

\OtherTok{main ::} \DataTypeTok{IO}\NormalTok{ ()}
\NormalTok{main }\OtherTok{=} \KeywordTok{do}
\NormalTok{    c }\OtherTok{<{-}}\NormalTok{ parseHandleClient todoApi (}\DataTypeTok{Proxy}\OtherTok{ ::} \DataTypeTok{Proxy} \DataTypeTok{ClientM}\NormalTok{)}
\NormalTok{        ( header }\StringTok{"todo"} \OperatorTok{<>}\NormalTok{ progDesc }\StringTok{"Todo TCP/IP service client"}\NormalTok{ )}
\NormalTok{        ( displayList}
     \OperatorTok{:<|>}\NormalTok{ (\textbackslash{}i }\OtherTok{{-}>} \StringTok{"Added with ID "} \OperatorTok{++} \FunctionTok{show}\NormalTok{ i)}
     \OperatorTok{:<|>} \FunctionTok{const} \StringTok{"Set!"}
     \OperatorTok{:<|>} \FunctionTok{const} \StringTok{"Deleted!"}
     \OperatorTok{:<|>}\NormalTok{ (\textbackslash{}ts }\OtherTok{{-}>} \StringTok{"Cleared items: "} \OperatorTok{++}\NormalTok{ intercalate }\StringTok{", "}\NormalTok{ (}\FunctionTok{map} \FunctionTok{show}\NormalTok{ (IS.toList ts)))}
\NormalTok{        )}

\NormalTok{    res }\OtherTok{<{-}}\NormalTok{ newManager defaultManagerSettings }\OperatorTok{>>=}\NormalTok{ \textbackslash{}mgr }\OtherTok{{-}>}
\NormalTok{      runClientM c }\OperatorTok{$}
\NormalTok{        mkClientEnv mgr (}\DataTypeTok{BaseUrl} \DataTypeTok{Http} \StringTok{"localhost"} \DecValTok{3434} \StringTok{""}\NormalTok{)}

    \KeywordTok{case}\NormalTok{ res }\KeywordTok{of}
      \DataTypeTok{Left}\NormalTok{  e }\OtherTok{{-}>}\NormalTok{ throwIO e}
      \DataTypeTok{Right}\NormalTok{ r }\OtherTok{{-}>} \FunctionTok{putStrLn}\NormalTok{ r}

\OtherTok{displayList ::} \DataTypeTok{IntMap} \DataTypeTok{Task} \OtherTok{{-}>} \DataTypeTok{String}
\NormalTok{displayList }\OtherTok{=} \FunctionTok{unlines}
            \OperatorTok{.} \FunctionTok{map}\NormalTok{ (\textbackslash{}(k, t) }\OtherTok{{-}>}\NormalTok{ printf }\StringTok{"\%d) \%s"}\NormalTok{ k (displayTask t))}
            \OperatorTok{.}\NormalTok{ IM.toList}
  \KeywordTok{where}
\NormalTok{    displayTask (}\DataTypeTok{Task}\NormalTok{ c t)}
      \OperatorTok{|}\NormalTok{ c         }\OtherTok{=} \StringTok{"[x] "} \OperatorTok{++}\NormalTok{ T.unpack t}
      \OperatorTok{|} \FunctionTok{otherwise} \OtherTok{=} \StringTok{"[ ] "} \OperatorTok{++}\NormalTok{ T.unpack t}
\end{Highlighting}
\end{Shaded}

The main thing that does the work is \texttt{parseHandleClient}, which takes
(after some proxies specifying the API and client type):

\begin{enumerate}
\def\labelenumi{\arabic{enumi}.}
\item
  Extra arguments modifying the command line help messages
\item
  A way to ``handle'' a response for every command.

  \begin{itemize}
  \tightlist
  \item
    For \texttt{list}, we display it using a nice pretty-printer
  \item
    For \texttt{add}, we show the new ID number.
  \item
    For \texttt{set} and \texttt{delete}, we just display the fact that it was
    successful (remember, these routes returned \texttt{()})
  \item
    For \texttt{prune}, we pretty-print the deleted items.
  \end{itemize}

  We choose to handle each command as a \texttt{String}, but we can choose to
  handle them each into any type we want (even something involving\texttt{IO})
  as long as each handler returns something of the same type.

  The handler is run and is returned as the value in \texttt{Right} when used
  with \texttt{runClientM}.
\end{enumerate}

Again, a nice way to ``write'' our \texttt{parseHandleCleint} function with its
handlers is by writing a skeleton definition and letting GHC tell us what goes
in each hole:

\begin{Shaded}
\begin{Highlighting}[]
\OtherTok{main ::} \DataTypeTok{IO}\NormalTok{ ()}
\NormalTok{main }\OtherTok{=} \KeywordTok{do}
\NormalTok{    c }\OtherTok{<{-}}\NormalTok{ parseHandleClient todoApi (}\DataTypeTok{Proxy}\OtherTok{ ::} \DataTypeTok{Proxy} \DataTypeTok{ClientM}\NormalTok{)}
\NormalTok{        ( header }\StringTok{"todo"} \OperatorTok{<>}\NormalTok{ progDesc }\StringTok{"Todo TCP/IP service client"}\NormalTok{ )}
\NormalTok{        ( handleList}
     \OperatorTok{:<|>}\NormalTok{ handleAdd}
     \OperatorTok{:<|>}\NormalTok{ handleSet}
     \OperatorTok{:<|>}\NormalTok{ handleDelete}
     \OperatorTok{:<|>}\NormalTok{ handlePrune}
\NormalTok{        )}
    \FunctionTok{pure}\NormalTok{ ()}
\end{Highlighting}
\end{Shaded}

\begin{verbatim}
client.hs:36:11: error:
    • Variable not in scope: handleList :: IntMap Task -> [Char]
   |
36 |         ( handleList
   |           ^^^^^^^^^^

client.hs:37:11: error:
    Variable not in scope: handleAdd :: Int -> [Char]
   |
37 |      :<|> handleAdd
   |           ^^^^^^^^^

client.hs:38:11: error:
    Variable not in scope: handleSet :: () -> [Char]
   |
38 |      :<|> handleSet
   |           ^^^^^^^^^

client.hs:39:11: error:
    Variable not in scope: handleDelete :: () -> [Char]
   |
39 |      :<|> handleDelete
   |           ^^^^^^^^^^^^

client.hs:40:11: error:
    Variable not in scope: handlePrune :: IS.IntSet -> [Char]
   |
40 |      :<|> handlePrune
   |           ^^^^^^^^^^^
\end{verbatim}

One last thing --- \emph{servant-cli} requires some instances to provide
``help'' documentation for the command line interfaces:

\begin{Shaded}
\begin{Highlighting}[]
\CommentTok{{-}{-} source: https://github.com/mstksg/inCode/tree/master/code{-}samples/servant{-}services/client.hs\#L25{-}L32}

\KeywordTok{instance} \DataTypeTok{ToParam}\NormalTok{ (}\DataTypeTok{QueryFlag} \StringTok{"filtered"}\NormalTok{) }\KeywordTok{where}
\NormalTok{    toParam \_ }\OtherTok{=} \DataTypeTok{DocQueryParam} \StringTok{"filtered"}\NormalTok{ [] }\StringTok{"Whether or not to filter completed items"} \DataTypeTok{Flag}
\KeywordTok{instance} \DataTypeTok{ToParam}\NormalTok{ (}\DataTypeTok{QueryParam\textquotesingle{}}\NormalTok{ \textquotesingle{}[}\DataTypeTok{Required}\NormalTok{] }\StringTok{"desc"} \DataTypeTok{Text}\NormalTok{) }\KeywordTok{where}
\NormalTok{    toParam \_ }\OtherTok{=} \DataTypeTok{DocQueryParam} \StringTok{"desc"}\NormalTok{ [] }\StringTok{"Task description"} \DataTypeTok{Normal}
\KeywordTok{instance} \DataTypeTok{ToParam}\NormalTok{ (}\DataTypeTok{QueryParam} \StringTok{"completed"} \DataTypeTok{Bool}\NormalTok{) }\KeywordTok{where}
\NormalTok{    toParam \_ }\OtherTok{=} \DataTypeTok{DocQueryParam} \StringTok{"completed"}\NormalTok{ [}\StringTok{"True"}\NormalTok{,}\StringTok{"False"}\NormalTok{] }\StringTok{"Set status to (leave out for toggle)"} \DataTypeTok{Normal}
\KeywordTok{instance} \DataTypeTok{ToCapture}\NormalTok{ (}\DataTypeTok{Capture} \StringTok{"id"} \DataTypeTok{Int}\NormalTok{) }\KeywordTok{where}
\NormalTok{    toCapture \_ }\OtherTok{=} \DataTypeTok{DocCapture} \StringTok{"id"} \StringTok{"ID number of task"}
\end{Highlighting}
\end{Shaded}

And we now get a fully-featured client for our service!

\begin{verbatim}
# make sure you run ./server.hs in a separate window
$ ./client.hs --help
todo

Usage: client.hs COMPONENT
  Todo TCP/IP service client

Available options:
  -h,--help                Show this help text

Path components:
  add
  delete
  list
  prune
  set

$ ./client.hs add --help
add

Usage: client.hs add --desc TEXT


Available options:
  --desc TEXT              Task description (Text)
  -h,--help                Show this help text

$ ./client.hs list --help
list

Usage: client.hs list [--filtered]


Available options:
  --filtered               Whether or not to filter completed items
  -h,--help                Show this help text

$ ./client.hs set --help
set

Usage: client.hs set <id> [--completed BOOL]


Available options:
  <id>                     ID number of task (Int)
  --completed BOOL         Set status to (leave out for toggle) (options: True,
                           False)
  -h,--help                Show this help text
\end{verbatim}

\hypertarget{conclusion}{%
\section{Conclusion}\label{conclusion}}

One major thing I like about this method is that it's very type-safe and allows
for types to \emph{drive} your development. Note how all of the messiness of a
binary protocol like TCP/IP are abstracted away, and you only ever deal with
\texttt{IntMap}s, \texttt{Text}, \texttt{Bool}, \texttt{Int}s. And also note how
in every step of the way, we use types to guide us: in writing our server, we
first used ``blanks'' to ask GHC what the type of each of the handlers needs to
be, which helps us plan our server implementation and ensures that it handles
everything properly. In writing our client, we also used ``blanks'' to ask GHC
the type of each of our response handlers needs to be, which allows us to
quickly and effectively drill down every option. These are things that
\emph{all} servant-based projects get to benefit from.

Hopefully this post serves as a good introduction to the \emph{servant},
\emph{servant-server}, and \emph{servant-cli} libraries, and additionally shows
how easy it is to give any application you want a TCP/IP interface to be usable
as a TCP/IP service. In the real world, your applications might be a little more
complex (and you might even require authentication), but hopefully after reading
this article, the ``network-facing'' part of your service or application becomes
a lot less intimidating :)

I know for me, the main benefit has been to tear down the ``barrier of
entry''/mental startup cost, and I've started writing little services and
clients like this as a first-step in a lot of cases, instead of as something I
dread and only end up adding to a few programs.

Also, fair disclosure: the author of \emph{servant-cli} is me! I'm not
\emph{super} happy with the user experience story at the moment, but it has been
usable for me so far. If you have any suggestions or ideas for improving
\emph{servant-cli}, I'd \href{https://github.com/mstksg/servant-cli}{love to
hear} (and look at any PRs!)

\hypertarget{special-thanks}{%
\section{Special Thanks}\label{special-thanks}}

I am very humbled to be supported by an amazing community, who make it possible
for me to devote time to researching and writing these posts. Very special
thanks to my supporter at the ``Amazing'' level on
\href{https://www.patreon.com/justinle/overview}{patreon}, Josh Vera! :)

\hypertarget{signoff}{%
\section{Signoff}\label{signoff}}

Hi, thanks for reading! You can reach me via email at
\href{mailto:justin@jle.im}{\nolinkurl{justin@jle.im}}, or at twitter at
\href{https://twitter.com/mstk}{@mstk}! This post and all others are published
under the \href{https://creativecommons.org/licenses/by-nc-nd/3.0/}{CC-BY-NC-ND
3.0} license. Corrections and edits via pull request are welcome and encouraged
at \href{https://github.com/mstksg/inCode}{the source repository}.

If you feel inclined, or this post was particularly helpful for you, why not
consider \href{https://www.patreon.com/justinle/overview}{supporting me on
Patreon}, or a \href{bitcoin:3D7rmAYgbDnp4gp4rf22THsGt74fNucPDU}{BTC donation}?
:)

\end{document}
